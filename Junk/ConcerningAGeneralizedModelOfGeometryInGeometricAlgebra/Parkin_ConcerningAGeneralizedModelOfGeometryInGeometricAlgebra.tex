\documentclass{birkjour}

%\usepackage{tikz}
%\usepackage{graphicx}
\usepackage{hyperref}
\usepackage{algpseudocode}

\newtheorem{thm}{Theorem}[section]
\newtheorem{cor}[thm]{Corollary}
\newtheorem{lem}[thm]{Lemma}
\newtheorem{prop}[thm]{Proposition}
\theoremstyle{definition}
\newtheorem{defn}[thm]{Definition}
\theoremstyle{remark}
\newtheorem{rem}[thm]{Remark}
\newtheorem*{ex}{Example}
\numberwithin{equation}{section}

\newcommand{\R}{\mathbb{R}}
\newcommand{\B}{\mathbb{B}}
\newcommand{\G}{\mathbb{G}}
\newcommand{\V}{\mathbb{V}}
\newcommand{\Z}{\mathbb{Z}}
\newcommand{\gd}{\dot{g}}
\newcommand{\gh}{\hat{g}}
\newcommand{\Gd}{\dot{G}}
\newcommand{\Gh}{\hat{G}}
\newcommand{\nvai}{\infty}
\newcommand{\nvao}{o}
\newcommand{\grade}{\mbox{grade}}

\begin{document}

\title{Concerning A Generalized Model\\Of Geometry In Geometric Algebra}

\author{Spencer T. Parkin}
\address{102 W. 500 S., \\
Salt Lake City, UT  84101} \email{spencerparkin@outlook.com}

\subjclass{Primary 14J27; Secondary 14J29}

\dedicatory{To my dear wife Melinda.}

\begin{abstract}
Abstract goes here...
\end{abstract}

\keywords{Geometric Algebra, Conformal Model}

\maketitle

\section{Introduction And Motivation}

In the first section of \cite{Parkin13}, a generalization of the conformal model
of geometric algebra was given, and it was seen that, unlike the conformal model,
such generalizations are not limited in representation to any proper subset of the quadrics.\footnote{All definitions
and notation given in the first section of \cite{Parkin13} are now assumed from this point onward.}
Indeed, through an appropriate definition of the function $p:\R^n\to\V$, an instance of this generalized
model can be found that studies algebraic sets of the form $p(x)\cdot B$, where $B$
is a blade in the geometric algebra generated by $\V$.  It is discovered in this paper, however,
that when we go to a generalized model, we lose, up to scale, uniqueness of representation.
To help remedy this problem, a Strong Nullstellensatz (see \cite{}) equivalent in our generalized
model of geometry is given, but let us first discuss why a loss of uniqueness in representation
is a significant problem for those hoping to work in a model of geometry similar to the conformal model.\footnote{When
working in a polynomial ring over an algebraically closed field, the Strong Nullstellensatz tells us that every algebraic set is uniquely
represented by a radical ideal.  Though we are not working in a geometric algebra over an algebraically closed field,
it is shown in this paper that every algebraic set is uniquely represented by an irreducible blade.}

In a nutshell, the problem stems from a realization that without uniqueness of representation,
one cannot algebraically relate any two given characterizations of the same geometry.
The ability to do this is the key to performing intersection and fitting operations in the conformal model.
For example, suppose we formulate the intersection of two spheres in the conformal model.
To find the circle, (real, degenerate or imaginary), that is the intersection of these two spheres,
we simply let their outer product be a scalar multiple of the circle's canonical form, which is usually
characterized as the intersection between a plane and a sphere centered on that plane.
Similarly, we may let the dual of the outer product of four points be a scalar multiple of the
canonical sphere in the conformal model.  In each of these cases we relied upon the key
property that geometries have, up to scale, unique representations
as blades in the model.\footnote{Points are the only geometries of the conformal model
that do not have unique representations.  Compare the flat point to the round point.}

\section{Working In The Generalized Model}

We begin with the following definition.
\begin{defn}[Ideal Factorization]
For any given $k$-blade $B\in\B$, if there exists a scalar $\lambda\in\R$ and $k$ points $\{x_i\}_{i=1}^k\subset\R^n$
such that
\begin{equation*}
B = \lambda\bigwedge_{i=1}^k p(x_i),
\end{equation*}
then we refer to this as an {\it ideal factorization} of $B$.
\end{defn}

Note that a given blade may have many ideal factorizations, or none at all.  Furthermore, a blade
having an ideal factorization may also have a non-ideal factorization.  For example, with the exception
of flats points, every flat geometry of the conformal model has both an ideal and non-ideal factorization.
The non-ideal factorizations use the point at infinity.  The ideal factorizations do not.

Having formulated a single geometry in two distinct ways $A$ and $B$, the main goal of working in our model is
that of finding a way to algebraically relate them.  In the case that $A$ and $B$ are of the same
grade, the following lemma shows now why the existence of an
ideal factorization of either $A$ or $B$ helps us facilitate this goal.

\begin{lem}
If $A,B\in\B$ are each non-zero blades of grade $k$ with $\gh(A)=\gh(B)$, and one of these
has an ideal factorization, then there exists $\lambda\in\R$ such that $A=\lambda B$.
\end{lem}
\begin{proof}
Supposing $A$ has the ideal factorization $\alpha\bigwedge_{i=1}^k p(x_i)$,
it is easy to see that for all $x\in\{x_i\}_{i=1}^k$, we have $x\in\gh(B)$.
Then, since $\{p(x_i)\}_{i=1}^k$ is a linearly independent set of $k$ vectors,
there must exist a scalar $\beta\in\R$ such that $B=\beta\bigwedge_{i=1}^k p(x_i)$,
showing that $B$ also has an ideal factorization.  Now simply let $\lambda=\alpha/\beta$.
\end{proof}

Unfortunately, having formulated $A$ and $B$, though they are likely to have the same grade,
it is unlikely that they will each have an ideal factorization.  We must therefore establish a relationship
between geometries of our model and ideal factorizations.  To that end, we begin with the following
definition and subsequent lemma.

\begin{defn}[Irreducible Blade]
For a given non-zero blade $B\in\B$, if there does not exist a blade $A\in\B$
with $\grade(A)<\grade(B)$ such that $\gh(A)=\gh(B)$, then we refer
to $B$ as an {\it irreducible blade}.
\end{defn}

\begin{lem}[Existence Of Irreducible Blades]
\end{lem}

\begin{lem}[Uniquness Of Irreducible Blades]
\end{lem}

% both existence and uniqueness lets us apply our first lemma
% find new version of first lemma that lets us equate an irreducible
% blade to a reducible blade representing the same geometry!

% give algorithm for reducing a blade

\begin{thebibliography}{9}

\bibitem{Parkin13}
S. Parkin, {\it Mother Minkowski Algebra Of Order $M$}.
Advances in Applied Clifford Algebras (2013).

\end{thebibliography}

\end{document}