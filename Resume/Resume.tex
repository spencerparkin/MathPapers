% LaTeX resume using res.cls
\documentclass[margin]{res}
%\usepackage{helvetica} % uses helvetica postscript font (download helvetica.sty)
%\usepackage{newcent}   % uses new century schoolbook postscript font 
\setlength{\textwidth}{5.1in} % set width of text portion

\usepackage{hyperref}
\hypersetup{
	colorlinks=true,
	urlcolor=blue,
}

\begin{document}

% Center the name over the entire width of resume:
 \moveleft.5\hoffset\centerline{\large\bf Spencer T. Parkin}
% Draw a horizontal line the whole width of resume:
 \moveleft\hoffset\vbox{\hrule width\resumewidth height 1pt}\smallskip
% address begins here
% Again, the address lines must be centered over entire width of resume:
 \moveleft.5\hoffset\centerline{2113 S. Claremont Drive}
 \moveleft.5\hoffset\centerline{Bountiful, UT  84010}
 \moveleft.5\hoffset\centerline{(801) 970-4578}
 \moveleft.5\hoffset\centerline{spencertparkin@gmail.com}


\begin{resume}
 
\section{OBJECTIVE}  A position in the field of software engineering with special interest in windowed- and web-based applications.

\section{EDUCATION} {\sl Bachelor of Science,} Mathematics \\
                      % \sl will be bold italic in New Century Schoolbook (or
	              % any postscript font) and just slanted in
		      %	Computer Modern (default) font
		Weber State University, Ogden, UT, 
		Graduated 2007 \\
		Major: Mathematics \\
		Minor: Computer Science \\
		Graduated High School 2001
 
 
\section{COMPUTER \\ SKILLS} {\sl Languages \& Software:} C/C++, Lua, C\#, Java, Assembly, SIMD/Intrinsics, HLSL, Perl, Git, OpenGL, MFC, wxWidgets, cURL, DevStudio, boost, JSON/REST-services. \\
                {\sl Operating Systems:} Windows, Linux.
 
\section{EXPERIENCE} {\sl Programmer} \hfill 2012-2016 \\
		Avatar Tools Team, Programming Department, {\bf Disney Interactive}
		\begin{itemize}  \itemsep -2pt %reduce space between items
		\item Single-handedly developed the followng new tools from the ground up to increase team productivity.
			\begin{itemize}\itemsep -2pt %reduce space between items
			\item{\bf Custodian} -- Lua-driven, general purpose wizard tool designed to lead the user through a dependency graph of esoteric steps.
			\item{\bf Emu} -- A tool presenting its users with a visual programming language used to create executables for our engine run-time VM.  It featured full undo-redo, dynamic graph layouts, graph animations, cut-copy-paste, and drag-drop, just to name a few.
			\item{\bf Progression Editor} -- A tool level-designers would use to map out each character's progression potential in game.
			\end{itemize}
		\item Maintained existing tools -- fixed bugs, added new features, optimized performance, improved visuals.  Highlights include the following.
			\begin{itemize}\itemsep -2pt %reduce space between items
			\item{\bf GCM Editor} -- Solved massive graph-layout problem in terms of visuals and performance using my own idea of overlapping spanning trees.  Added break-point features for existing tool-to-engine connectivity.
			\item{\bf AnimTree Editor} -- Added telemetry panel visualization for real-time capturing of instrumented anim-tree events for diagnostic purposes.
			\end{itemize}
		\item Developed new features in our proprietary windowing framework.
			\begin{itemize}\itemsep -2pt %reduce space between items
			\item{\bf Twin-Sliders} -- Implemented twin-slider control for range-picking.
			\item{\bf Word-Sort} -- Implemented word-sort control with nifty drag-drop animation.
			\item{\bf Scintilla} -- Incorporated the Scintilla control into our controls library.
			\end{itemize}
                \end{itemize}
 
                {\sl Associate Programmer} \hfill 2007-2012 \\
		Rendering Team, Programming Department, {\bf Disney Interactive}
                 \begin{itemize}  \itemsep -2pt %reduce space between items
		\item Single-handedly developed particle authoring tool from the ground up with live-authoring features.  Wrote both client and game-side support for live-authoring.
		\item Wrote particle processing command-line tool for LIP engine-file generation.
		\item Wrote shared library for particle authoring tool and the particle processing command-line tool.
		\item Wrote GPU shader math code in HLSL to support all particle system features in the context of the GPU's hardware instancing feature.
		\item Wrote CPU math code to support all particle system features on the Wii.  Wrote CPU-side instancing to make up for the Wii's lack of hardware instancing capability.
		\item Implemented lens-flare system from tools to pipe-line to engine-rendering using DirectX's occlusion culling feature.
		\item Implemented emitter-shapes feature for the particle system so that artists could emit in volumes, on surfaces, and on splines.
		\item Ported old particle system to new renderer.
		\item Wrote GA-compatible matrix math library.
		\item Wrote 2D-GA math library.  Most of the code was generated by a command-line program I wrote.
		\item Optimized existing matrix and quaternion-based math routines using assembly, intrinsics and SIMD registers.
		\item Optimized CPU-side particle-buffer processing using assembly, intrinsics and SIMD registers.
		\item Wrote entire HOG file-archiving system including shared library, command-line utility, and windows-based GUI tool using wxWidgets.
		\item Wrote Perl script to sniff out all asset usages per level for the entire game for pre-caching purposes to eliminate run-time hitches.
		\item Fixed many hard-to-find memory-corruption bugs and graphics glitches during crunch modes.
		\item Wrote tetrahedral-based convex hull-finder algorithm.
		\item Wrote scene-file merging utility.
		%\item Wrote some kind of animation constraint...we didn't use it, but it worked...can't remember what kind of constraint it was.
                 \end{itemize} 

		{\sl Lab Aide}\hfill 2003-2007 \\
		Worked as a lab aide to pay for books during the college years.

                {\sl Level 1 Programmer} \hfill 2001-2002 \\
		UI Programming, {\bf Acclaim Entertainment}
                  \begin{itemize}\itemsep -2pt %reduce space between items
                   \item Worked with artists and designers as sole programmer on main front-end user-interface for Legends of Wrestling II.
                   \end{itemize} 

		{\sl Programming Intern}\hfill 2000-2001 \\
		FX Programming, {\bf Acclaim Entertainment}
		\begin{itemize}\itemsep -2pt %reduce space between items
		\item Worked on the particle system.  Implemented blood/sweat splatter for Legends of Wrestling I and various other particle effects.
		\item Implemented body-part resizing subroutine for create-a-wrestler feature.
		\end{itemize}

\section{PROJECTS}
		I was involved in the following video game projects, sorted by developer; projects sorted chronologically.

		{\sl Acclaim Entertainment}\\
		Legends of Wrestling I, Legends of Wrestling II.
		
		{\sl Disney Interactive}\\
		Bolt, Toy Story 3, Cars 2, Infinity 1, Infinity 2, Infinity 3.

\section{PERSONAL PROJECTS}
		The following are computer programs I've written out of personal interest.  Source code for all of these projects can be found at \url{https://github.com/spencerparkin?tab=repositories}.  Demos can be found at \url{https://sites.google.com/site/spencertparkinresume/}.  My research into web-technology can be found at \url{https://ultimate-app.herokuapp.com/}.
		\begin{itemize}\itemsep -2pt %reduce space between items
		\item \textbf{CalcLib} -- A static library providing numeric and symbolic calculation support.  Unlike most calculators, this one understands geometric algebra.
		\item \textbf{GAVisTool} -- Built upon CalcLib, this is a tool for exploring and experimenting with conformal geometric algebra.  It also utilizes a BSP-tree to do real-time alpha-sorting.
		\item \textbf{GALua} -- A Lua module that exposes the capabilities of CalcLib.
		\item \textbf{CSharpMaze} -- A rectangular and circular maze generator written in C\#.
		\item \textbf{ChineseCheckers} -- A wxWidgets-based application utilizing OpenGL's selection mechanism.  It is also multi-player as it can host and connect to a game session using sockets.
		\item \textbf{RubiksCube} -- Another wxWidgets-based application utilizing OpenGL's selection mechanism.  Any Rubik's Cube of degree 3 or higher can be simulated.  An algorithm is provided that can find a solution to any such scrambled cube.  The solution sequence is animated.
		\item \textbf{ImageGenerator} -- A multi-threaded image generator that can generate fractals and ray-trace scenes specified in XML.  It can also render video clips using FFmpeg.
		\end{itemize}

\section{PUBLICATIONS}
		The following publications are given in chronological order.
		\begin{itemize}\itemsep -2pt %reduce space between items
		\item Parkin, S. (2014).  The Mother Minkowski Algebra of Order M.  {\it Advances in Applied Clifford Algebras}, 24(1), 193-203.
		\item Parkin S.  (2014).  The Intersection of Rays with Algebraic Surfaces.  {\it Advances in Applied Clifford Algebras}, 24(3), 309-815.
		\item Parkin, S.  (2015).  Versors That Give Non-Uniform Scale.  {\it Advances in Applied Clifford Algebras}, 25(1), 219-225.
		\item Parkin, S.  (2015).  An Introduction To Geometric Sets.  {\it Advances in Applied Clifford Algebras}, 25(3), 639-655.
		\end{itemize}

\section{SPECIAL INTERESTS}
		\begin{itemize}\itemsep -2pt %reduce space between items
		\item Group Theory -- An absolutely {\it beautiful} theory with applications ranging from solving Rubik's cubes to quantum mechanics.
		\item Geometric Algebra -- A mathematical framework combining the capabilities of matrices and quaternions into one algebraic system.
		\item The Conform Model -- A model of geometry based upon geometric algebra in which round and flat things become the elements of the algebra.
		\item Rubik's Cube-Type Puzzles -- I can solve many types of such puzzles, and find my own solution-methods to such puzzles.
		\end{itemize}

\section{REFERENCES AVAILABLE UPON REQUEST}

\end{resume}
\end{document}




