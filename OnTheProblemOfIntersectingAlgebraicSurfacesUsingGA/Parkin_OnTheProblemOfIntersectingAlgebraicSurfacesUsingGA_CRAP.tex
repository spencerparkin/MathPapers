\documentclass{birkjour}

%\usepackage{tikz}
%\usepackage{graphicx}
\usepackage{hyperref}

\newtheorem{thm}{Theorem}[section]
\newtheorem{cor}[thm]{Corollary}
\newtheorem{lem}[thm]{Lemma}
\newtheorem{prop}[thm]{Proposition}
\theoremstyle{definition}
\newtheorem{defn}[thm]{Definition}
\theoremstyle{remark}
\newtheorem{rem}[thm]{Remark}
\newtheorem*{ex}{Example}
\numberwithin{equation}{section}

\newcommand{\R}{\mathbb{R}}
\newcommand{\B}{\mathbb{B}}
\newcommand{\G}{\mathbb{G}}
\newcommand{\V}{\mathbb{V}}
\newcommand{\gd}{\dot{g}}
\newcommand{\gh}{\hat{g}}
\newcommand{\Gd}{\dot{G}}
\newcommand{\Gh}{\hat{G}}
\newcommand{\nvai}{\infty}
\newcommand{\nvao}{o}
\newcommand{\grade}{\mbox{grade}}

\begin{document}

\title{On The Problem Of Intersecting\\Algebraic Surfaces Using\\Geometric Algebra}

\author{Spencer T. Parkin}
\address{102 W. 500 S., \\
Salt Lake City, UT  84101} \email{spencerparkin@outlook.com}

%\subjclass{Primary 14J70; Secondary 14J29}

%\dedicatory{To my dear wife Melinda.}

\begin{abstract}
Abstract goes here...
\end{abstract}

\keywords{Keywords go here...}

\maketitle

\section{Introduction}

In the first section of the paper \cite{}, a general model\footnote{The conformal model of geometric algebra
is a special case of this general model.} of geometry based upon geometric algebra
was informally developed.  With the aim of finding intersections of algebraic surfaces,
this paper attempts a formal development of this general model, giving examples along the way where helpful.

The way intersections are found in this general model is by relating one characterization of an intersection
with another.  One of these is formulated as desired, while the other, canonical.  A canonical form of a desired intersection
lends itself well to analysis through decomposition.  After composing a desired intersection, it can then be found
by equating it to a canonical form, and then decomposing it as one would this canonical form.\footnote{In short,
formulated intersections are reinterpreted as canonical forms.}

For example, if we wanted
the planar intersection of a conical surface, (a conic section), and we knew it was an ellipse, we could find
this ellipse by decomposing the given intersection as the intersection of an elliptic cylinder meeting the plane
at right angles.  This latter form would no-doubt have a trivial method of decomposition into the parameters
that characterize such an ellipse, which is what would make it a canonical form of the ellipse.
If such a decomposition was attempted, but failed, the given conic may not be an ellipse, but one of
the other conic sections.

This all sounds nice in theory, but as we're going to see, it does not come out as easily as it does in the conformal
model of geometric algebra as it does in our more general model.  Unlike the conformal model, however, all algebraic
surface have representation in the general model, and so this motivates us to attempt intersections generally.

In the sections to follow, we work in a geometric algebra $\G$ generated by an $m$-dimensional vector space $\V$
spanned by a set of vectors $\{e_i\}_{i=1}^m$.  The signature of this geometric algebra is left open to
consideration except where required by specific examples or results.  We will let $\R^n$ denote
$n$-dimensional euclidean space, and we do not embed $\R^n$ as a vector sub-space of $\V$, because
it is not necessary to do so.

\section{Representing Surfaces}

The basis for our model of geometry over $\G$ is a function $p:\R^n\to\V$ mapping points
in our $n$-dimensional euclidean space $\R^n$ to the vectors of our geometric algebra $\G$.

\begin{defn}[Representing Surfaces]\label{def_surface_representation}
Given a blade $B\in\G$, we say that it directly represents a surface as the set of
all points $x\in\R^n$ such that $p(x)\wedge B=0$.  We say that $B$ dually represents
a surface as the set of all points $x\in\R^n$ such that $p(x)\cdot B=0$.
\end{defn}
From Definition~\ref{def_surface_representation}, we see that a blade $B\in\G$ represents two surfaces, one
directly, the other dually.\footnote{These surfaces may be thought of as duals of one another.}
On the other hand, given a single surface, we will find that it is represented dually and directly
by two distinct blades, each duals of one another.
The dimension of these surfaces will have a relation to the grade of the blades representing them.

That we are justified in referring to these subset of $\R^n$ as surfaces comes from a
general form of the function $p$ in which it is clear that any algebraic surface may arise in the model.
\begin{equation}\label{equ_p_general_form}
p(x) = \sum_{i=1}^m f_i(x)e_i
\end{equation}
Here, each function $f_i:\R^n\to\R$ is a polynomial in the components of $x$.  Then, assuming
any signature of our geometric algebra $\G$, it is clear from equation \eqref{equ_p_general_form}
that when $B\in\V$, that $p(x)\cdot B$ is also such a polynomial.  Our function $p$ may define
a polynomial form while the vector $B$ can specify an instance of this form by housing its coefficients.
To show that blades $B\in\G$ in general represent algebraic surfaces, the following notation is convenient
and will be used through the rest of this paper.
\begin{align*}
\gd(B) &= \{x\in\R^n|p(x)\cdot B=0\} \\
\gh(B) &= \{x\in\R^n|p(x)\wedge B = 0\}
\end{align*}

\begin{lem}[Representing Surface Intersections]
\end{lem}

Knowing that when $B\in\V$, that $\gd(B)$ is an algebraic surfaces, it follows now by by Lemma~\ref{}
and the principle of mathematical induction that for any blade $B\in\G$, that $\gd(B)$ is an algebraic
surface.  Indeed, if $\bigwedge_{i=1}^k b_i$ was a factorization of the $k$-blade $B$, then
\begin{equation*}
\gh(B) = \bigcap_{i=1}^k \gh(b_i).
\end{equation*}

\end{document}
