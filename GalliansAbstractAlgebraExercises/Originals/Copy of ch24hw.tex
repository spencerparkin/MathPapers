\documentclass{article}
\usepackage{amsmath}
\usepackage{amssymb}
\title{Chapter 24 Homework}
\author{Spencer}
\addtolength{\oddsidemargin}{-.575in}
\addtolength{\evensidemargin}{-.575in}
\addtolength{\textwidth}{1.0in}
\addtolength{\topmargin}{-.575in}
\addtolength{\textheight}{1.25in}
\begin{document}
\maketitle

\newcommand{\Z}{\mathbb{Z}}
\newcommand{\R}{\mathbb{R}}
\newcommand{\N}{\mathbb{N}}
\newcommand{\Q}{\mathbb{Q}}
\newcommand{\cl}{\mbox{cl}}

\section*{Problem 2}

Calculate all conjugacy classes for the quaternions.

Some work shows that the conjugacy classes are
\begin{equation*}
\{e\},\{a,a^3\},\{a^2\},\{b,ba^2\},\{ba,ba^3\}.
\end{equation*}

\section*{Problem 8}

Find all Sylow 3-subgroups of $A_4$.

The group $A_4$ has order $4!/2=24=2^3\cdot 3$, so we're looking for
all subgroups of order 3.  Since 3 is prime, all such subgroups are cyclic.
So the set of subgroups is given by
$\{\langle a\rangle|\mbox{$a\in A_4$ and $|a|=3$}\}$.
Using Table 5.1, these groups are given by
\begin{align*}
\{\alpha_1,\alpha_5,\alpha_9\},\\
\{\alpha_1,\alpha_6,\alpha_{11}\},\\
\{\alpha_1,\alpha_7,\alpha_{12}\},\\
\{\alpha_1,\alpha_8,\alpha_{12}\}.
\end{align*}
This was given away in Example 2 of the chapter.

\section*{Problem 12}

Show that every group of order 56 has a proper nontrivial normal subgroup.

By Sylow's First Theorem, there is at least one subgroup of order 7.
By Sylow's Third Theorem, the number of such subgroups is congruent
to 1 modulo 7 and divides 56.  The only possible number of such
subgroups is therefore 1 or 8.  If there is only 1 subgroup of order 7,
then it is a nontrivial normal subgroup by the corollary to Sylow's
Third Theorem.  If there are 8 subgroups of order 7, then because
they are cyclic, they account for $8\cdot(7-1)=48$ distinct elements,
leaving $56-48=8$ remaining elements, including the identity element.
It follows that these 8 elements must form a subgroup of order 8,
By Sylow's First Theorem.
Furthermore, this is the only subgroup of order 8, because every
one of the 48 elements previously mentioned have order 7, and 7 does
not divide 8.

\section*{Problem 13}

What is the smallest composite integer $n$ such that there
is a unique group of order $n$?

For groups of order 4, 8, 9, and 12, they could all be
cyclic or the external direct product of cyclic groups that results
in a non-cyclic group.  This is because they can all be written as
the product of two non-trivial and non-coprime divisors.

For groups of order 6, 10, and 14, blah.

Now notice that all groups of order 15 are cyclic by Theorem 24.6.

\section*{Problem 17}

How many Sylow 3-subgroups of $S_5$ are there?  Exhibit five.

By Sylow's Third Theorem, there are 1, 4, 10, or 40 of them.  Since
we're asked to exhibit five of them, there must be 10 or 40.
If there are 40 subgroups of order 3, then there are 80 elements
of order 3.

\section*{Problem 24}

Prove that a group of order 105 contains a subgroup of order 35.

By Sylow's First Theorem, there is a subgroup of order 5, and a subgroup
of order 7.  By Sylow's Third Theorem, there
may be at most 21 groups of order 5, and 15 groups of order 7.  Suppose there
are more than one of both types of subgroups.  Since these are cyclic
of prime order, this would generate
$21\cdot(5-1)=84$ elements of order 5,
and $15\cdot(7-1)=90$ elements of order 7.  But $84+90>105$, so at least one
of these two types of subgroup is unique, and therefore normal by the corollary to
Sylow's Third Theorem.
It then follows by the result of Exercise 51 on
Page 194, that if $H$ is a subgroup of order 5, and $K$ is a subgroup of order 7,
then $HK$ is a subgroup of $G$, since at least one of $H$ and $K$ is normal.
Now notice that $|HK|=|H||K|/|H\cap K|$, but $H\cap K=\{e\}$ since $H$ and $K$
are cyclic groups of distinct prime orders.  It follows that $|HK|=|H||K|=5\cdot 7=35$,
which is what we wanted.

\section*{Problem 25}

Prove that a subgroup of order 595 has a normal Sylow 17-subgroup.

\section*{Problem 28}

\section*{Problem 40}

Let $p$ be prime.  If the order of every element of a finite group $G$ is a
power of $p$, prove that $|G|$ is a power of $p$.

Suppose $|G|$ has a prime $q$ other than $p$ in its factorization.
Then by Sylow's First Theorem, there must exist a cyclic subgroup of
order $q$, and its generator is an element of order $q$, contradicting
the fact that all elements of $G$ have orders that are powers of $p$.
So $p$ is the only prime in the factorization of $|G|$, and so $|G|$
is a power of $p$.

\section*{Lemma 1 for Problem 41}

Let $H$ and $K$ be subgroups of a group $G$ and conjugate in $G$.
Then $H$ and $K$ are isomorphic.

Proof: By hypothesis, there exists $g\in G$ such that $H=gKg^{-1}$.
Define $\phi:K\to H$ by $\phi(k)=gkg^{-1}$.  Letting $x,y\in K$,
we see that
\begin{equation*}
x=y\iff gxg^{-1}=gyg^{-1}\iff\phi(x)=\phi(y),
\end{equation*}
showing that $\phi$ is well defined and one-to-one.  We then see that
\begin{equation*}
\phi(xy)=gxyg^{-1}=gxg^{-1}gyg^{-1}=\phi(x)\phi(y),
\end{equation*}
showing that $\phi$ is operation preserving.  Now let $h\in H\subseteq gKg^{-1}$.
Then there exists $k\in K$ such that $h=gkg^{-1}$, and so $\phi(k)=h$, showing
that $\phi$ is onto.  We have now shown that $\phi$ is an isomorphism between
$H$ and $K$.

Notice that $H$ and $K$ are not necessarily the same subset of $G$.

\section*{Lemma 2 for Problem 41}

Let $H$ be a subgroup of a group $G$.  Then for any $g\in G$,
the subset $gHg^{-1}$ of $G$ is a conjugate of $H$ in $G$.

Proof:  We first show that $gHg^{-1}$ is a subgroup of $G$.
Clearly, $e\in gHg^{-1}$.  Now let $a,b\in gHg^{-1}$.  Then
there exists $h_a,h_b\in H$ such that $a=gh_ag^{-1}$ and $b=gh_bg^{-1}$.
Then we have
\begin{equation*}
ab^{-1}=gh_ag^{-1}(gh_bg^{-1})^{-1}=gh_ag^{-1}gh_b^{-1}g^{-1}=gh_ah_b^{-1}g^{-1}
\in gHg^{-1},
\end{equation*}
showing that $gHg^{-1}$ is a subgroup of $G$ by the one-step subgroup test.

We now show that the subgroups $H$ and $gHg^{-1}$ of $G$ are conjugate in $G$.
Notice that $H$ and $gHg^{-1}$ have the same cardinality.  If $h_1,h_2\in H$
where $h_1\neq h_2$, and $gh_1g^{-1}=gh_2g^{-1}$, then we have a contradiction.
Choosing $g\in G$, we see that $g^{-1}(gHg^{-1})g=(g^{-1}g)H(g^{-1}g)=H$.

Notice that if $H$ is a normal subgroup of $G$, then $H=gHg^{-1}$, but
in general, we can only say that $H\approx gHg^{-1}$.

\section*{Problem 41}

For each prime $p$, prove that all Sylow $p$-subgroups of a finite group
are isomorphic.

It is easy to show that if two subgroups of a group are conjugates of one
another, then they are isomorphic.  (See Lemma 1 above.)
What is hard to show is that if
$K$ is a Sylow $p$-subgroup of a group $G$, then all other Sylow
$p$-subgroups have the form $gKg^{-1}$, where $g\in G$.  Clearly, by Lemma 2
above, every
member of the set $\{gKg^{-1}|g\in G\}$ is a subgroup of $G$, has
the same order as $K$, and is isomorphic
to all other members, but does this set contain all such subgroups?
I do not know!  It is not obvious to me!

This was proven in the second paragraph of the proof of Sylow's
3rd Theorem, though the fact was used in the proof of Sylow's 2nd Theorem.
Grrr.

\end{document}