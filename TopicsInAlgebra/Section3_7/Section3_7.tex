\documentclass[12pt]{article}

\usepackage{amsmath}
\usepackage{amssymb}
\usepackage{amsthm}

\title{Section 3.7 Exercises\\Herstein's Topics In Algebra}
\author{Spencer T. Parkin}

\newtheorem{theorem}{Theorem}[section]
\newtheorem{definition}{Definition}[section]
\newtheorem{corollary}{Corollary}[section]
\newtheorem{identity}{Identity}[section]
\newtheorem{lemma}{Lemma}[section]
\newtheorem{result}{Result}[section]

%\newcommand{\gcd}{\mbox{gcd}}
\newcommand{\lcm}{\mbox{lcm}}
\newcommand{\Z}{\mathbb{Z}}
\newcommand{\R}{\mathbb{R}}
\newcommand{\cl}{\mbox{Cl}}
\newcommand{\aut}{\mbox{Aut}}

\begin{document}
\maketitle

\section*{Thoughts}

The terms {\it unit} and {\it unit element} are, in my opinion, confusing and unfortunate choices.
If I understand correctly, a unit element $x$ of a ring $R$ is such that for all $r\in R$, $rx=xr=r$.
A unit, on the other hand, is an element (but not to be confused with a {\it unit element}) that
has a multiplicative inverse.  I think that Gallian uses the term {\it unity} instead of {\it unit element}.

In an integral domain, we can show that if there is a unit element,
then it is unique.  Let $x_1,x_2\in D$ be such elements.  Then for
any $0\neq r\in D$, we have $rx_1=rx_2\implies r(x_1-x_2)=0\implies x_1-x_2=0\implies x_1=x_2$.
Something similar may be said about units.  If $x\in D$ is a unit, suppose $y,z\in D$
such that $xy=1$ and $xz=1$, yet $y\neq z$.  But then $xy=xz\implies x(y-z)=0\implies y-z=0\implies y=z$.

It's now worth mentioning that if a ring is said to contain a unit, then this must necessarily imply the
existence of a unit element.  The converse, however, does not necessarily hold, as far as I know.

\section*{Problem 1}

In a commutative ring with unit element prove that the relation $a$ is an associate of $b$ is an equivalence relation.

We say $a\sim b$ if and only if there exists a unit $u\in R$ such that $a=ub$.
Clearly any unit element is a unit, so we can easily conclude that $a\sim a$ since $a=(1)a$.
If $a\sim b$, then $a=ub\implies b=u^{-1}a\implies b\sim a$.
If $a\sim b$ and $b\sim a$, then $a=ub$ and $b=vc$, so $a=uvc\implies a\sim c$,
since clearly $(uv)^{-1}=v^{-1}u^{-1}$ is a unit of $R$.

\section*{Problem 2}

In a Euclidean ring prove that any two greatest common divsors of $a$ and $b$ are associates.

Let $A$ be the ideal of our ring $R$ that is generated by $a$ and $b$ as $\{ra+sb|r,s\in R\}$.
Then, as shown in the proof of Lemma 3.7.1, $A=\langle d\rangle$, where $d$ is any greatest common
divisor of $a$ and $b$.  So, letting $d_1,d_2\in A$ be any two such elements, we must have
$d_1=xd_2$ and $d_2=yd_1$ for some pair of elements $x,y\in R$.  But then $d_1=xyd_1\implies 1=xy$,
since we're in an integral domain.  So $x$ is a unit and therefore, $d_1\sim d_2$.

\section*{Problem 3}

Prove that a necessary and sufficient condition that the element $a$ in the Euclidean ring be a unit is that $d(a)=d(1)$.

Let $d(a)=d(1)$, and suppose $a$ is not a unit.  Then by Lemma 3.7.3, $d(1)<d((1)a)$, which is clearly a contradiction.
So $a$ is a unit.  Now if $a$ is a unit, then $d(a)\leq d(aa^{-1})=d(1)$.  But if $R=\langle 1\rangle$, then there does
not exist $a\in R$ such that $d(a)<d(1)$.  So $d(a)=d(1)$.  (See the proof of Lemma 3.7.3 to get this last part.)

\section*{Problem 5}

Prove that if an ideal $U$ of a ring $R$ contains a unit of $R$, then $U=R$.

Let $u\in U$ be a unit.  Now let $r\in R$ such that $r=u^{-1}$.
Then $x=uu^{-1}=ur\in U$ and $x$ is a unit element of $R$.
But now $r=xr\in U$ for all $r\in R$.  So $U=R$.

\end{document}