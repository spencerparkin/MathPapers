\documentclass{article}
\usepackage{amsmath}
\usepackage{amssymb}
\addtolength{\oddsidemargin}{-.575in}
\addtolength{\evensidemargin}{-.575in}
\addtolength{\textwidth}{1.0in}
\addtolength{\topmargin}{-.575in}
\addtolength{\textheight}{1.25in}
\begin{document}

\newcommand{\C}{\mathbb{C}}
\newcommand{\R}{\mathbb{R}}
\newcommand{\Q}{\mathbb{Q}}
\newcommand{\Z}{\mathbb{Z}}
\newcommand{\N}{\mathbb{N}}

\section*{Problem 1}

Let $G$ denote the multiplicative subgroup
$\{1,-1,i,-i\}$ of $\C^*$.  Define $\theta:\Z\to G$
by $\theta(n)=i^n$, for each $n\in\Z$.
Show that $\theta$ is a homomorphism and determine $\ker(\theta)$.

Suppose for the moment that $\theta(n)=g^n$ for any $g\in G$
where $G$ is any group.  Then $\theta$ is clearly a homomorphism
from $\Z$ to $G$ since $\theta(a+b)=g^{a+b}=g^ag^b=\theta(a)\theta(b)$.

Returning to the originally defined $\theta$,
notice that $\theta(\Z)=G$.
Specifically, $G=\langle i\rangle\approx\Z_4$ under the
mapping $\exp(i\pi z/2)\to z$, where $z\in\Z_4$.
To determine $\ker(\theta)$, notice that all elements in
$G$ are of the from $i^{4q+r}$, where $q,r\in\Z$, and $0\leq r< 4$.
We want $i^{4q+r}=1$.  Since $i^0=1$, we see that
$4q+r\equiv 0\pmod{4}$ by Theorem 4.1.
This requires $r=0$.  So we see that $\ker(\theta)=4\Z=\{4z:z\in\Z\}$.

\section*{Problem 2}

Let $a$ belong to a ring $R$.  Let $S=\{x\in R:ax=0\}$.
Show that $S$ is a subring of $R$.

Note that $0\in S$ since $a0=0$, so $S$ is not empty.
Let $u,v\in S$.  Then $a(u-v)=au-av=0-0=0$ and $a(uv)=(au)v=0v=0$,
showing that $u-v\in S$ and $uv\in S$.  So $S$ is a subring of
$R$ by the subring test.

\section*{Problem 3}

Let $\phi$ be a homomorphism of $U(36)$ into $U(36)$ such that
$\ker(\phi)=\{1,13,25\}$.  Suppose $\phi(5)=17$.
Determine all elements in $U(36)$ that map to 17.

It follows directly from Property 6 of Theorem 10.1 that
$\phi^{-1}(17)=5\ker(\phi)=\{5,29,17\}$.  If further justification is
needed, then I would appeal to my limited intuition of homomorphisms.
Homomorphisms create a bijective map of a factor group of the domain to
a subgroup of the codomain.  So the inverse image of an element in the range
of the homomorphism should be a multiple of the kernel.

\section*{Problem 4}

Let $H$ be a normal subgroup of a group $G$, and let $m=|G:H|$.
Show that $a^m\in H$ for every $a\in G$.

For any $a\in G$, choose $aH\in G/H$.  Notice that $m=|G/H|$.
Then by Corollary 4 of LaGrange's Theorem, $a^mH = (aH)^m = H\implies a^m\in H$.

\pagebreak
\section*{Problem 5 (Short version)}

For a given $n\in\N$, how many elements in the group
$\Q/\Z$ have order $n$?

The case $n=1$ is clear, so let $n>1$.
Let $C_n$ denote the set of all cosets in $\Q/\Z$ having order $n$.
It is proposed that 
$C_{n>1}=\{(a/n)\Z:a\in U(n)\}$.  It would
then follow that $|U(n)|$ is the answer to the question.
To verify this, notice that every coset not
in $C_n$ has one of two forms.  It is either $(a/n)\Z$
where $a\not\in U(n)$ with $\gcd(a,n)=1$, or it is $(a/m)\Z$
where $m\neq n$ and $\gcd(a,m)=1$.  In the former case,
$(a/n)\Z\in C_n$ since $a$ must be congruent modulo $n$
to an element $a'\in U(n)$.  Therefore, $(a'/n)\Z=(a/n)\Z$
since $a'\equiv a\pmod{n}$, showing that $a'/n-a/n=(a'-a)/n=\in\Z$.
In the latter case, $|(a/m)\Z|\neq n$.  This is clear if $m<n$.
If $m>n$, then since $a$ and $m$ are coprime, the first $n$
multiples of $a$ will not produce a complete system of residues
modulo $m$.  Er, this needs some work.

\section*{Problem 6}

Let $E$ be the set of even integers with the usual addition,
but define multiplication $*$ by $a*b=\frac{1}{2}(ab)$,
for all $a,b\in E$ ($ab$ is usual multiplication).
Prove that $E$ with $+,*$ is a ring and find the unity of $E$.

It's clear that $E$ is an Abelian group under addition.
What remains to be shown is that $E$ is closed under the
operation $*$, that $*$ is associative, and that $*$ is
distributive over addition.
Closure follows from the fact that the product of two even
numbers is divisible by 4, so that after dividing out by 2,
the result is still even.  Let $a,b,c\in E$.
Proofs of associativity and distributivity now follow.
\begin{equation*}
a*(b*c)=\frac{1}{2}a\left(\frac{1}{2}bc\right)
 = \frac{1}{4}abc
 = \frac{1}{2}\left(\frac{1}{2}ab\right)c = (a*b)*c
\end{equation*}
\begin{equation*}
a*(b+c)=\frac{1}{2}a(b+c)=\frac{1}{2}(ab+ac)=\frac{1}{2}ab+\frac{1}{2}ac=a*b+a*c
\end{equation*}
Proof of right distributivity is similar to that of left distributivity.
Now notice that $2*a=(1/2)2a=a$, showing that 2 is the unity of $E$.

\section*{Problem 7}

Let $G$ be a group of order $pq$ where $p$ and $q$ are not necessarily distinct primes.
Prove that the order of the center of $G$, $Z(G)$, is 1 or $pq$.

Our only choices for $|Z(G)|$ are 1, $p$, $q$, and $pq$, by LaGrange's Theorem.
If $G$ is Abelian, then $Z(G)=G\implies |Z(G)|=pq$.
If $G$ is not Abelian, then $G/Z(G)$ is not cyclic by Theorem 9.3.
Suppose $|Z(G)|=p$.  Then $|G/Z(G)|=pq/p=q$, and it follows from
Corollary 3 of LaGrange's Theorem that $G/Z(G)$ is cyclic.
By contradiction, $|Z(G)|\neq p$.  This same line of reasoning
can be used to show that $|Z(G)|\neq q$.  So the choices we're left with are
$|Z(G)|=1$ or $|Z(G)|=pq$.

\pagebreak
\section*{Problem 8}

Let $H$ be a subgroup of $G$.  Prove that $H$ is a normal subgroup of $G$
if and only if for all $a,b\in G$, $ab\in H\implies ba\in H$.

Suppose $H$ is a normal subgroup of $G$.
Then $ab\in H\implies H=abH=(aH)(bH)=(bH)(aH)=baH\implies ba\in H$.
Here we used the fact that inverses commute with one another.
We needed the normality of $H$ so that we could work in the group $G/H$.

Now suppose $ab\in H\implies ba\in H$.  
Let $g\in G$ and $h\in H$.  Then $g^{-1}(gh)\in H\implies (gh)g^{-1}\in H$,
showing that $H$ is normal in $G$ by the normal subgroup test.

\section*{Problem 5 (Long version)}

For a given $n\in\N$, how many elements in the group
$\Q/\Z$ have order $n$?

Let $C_n$ denote the set of all cosets in $\Q/\Z$ having order $n$.
It will first be shown that
$C_n=\{(a/n)\Z:\mbox{$a\in\Z^*$ and $\gcd(a,n)=1$}\}$ for all $n\in\N$.
We'll then count the number of elements in $C_n$.

Let $c\in C_n$.  Then $c^n=((a/n)\Z)^n=a\Z=\Z$ since $a\in\Z$.
This shows that $|c|\leq n$.  If $n>1$, let $0<k<n$.
Then clearly, $c^k=(ka/n)\Z\neq\Z$, since $n$ does not divide $ak$.
To see this, realize that $a$ and $n$ are coprime.  Therefore, the first $n$
non-negative multiples of $a$ are needed to generate a complete system
of residues modulo $n$.
The first $n-1$ such multiples are non-zero modulo $n$.
So we now have $|c|=n$.

Now since every rational number can be written in lowest terms,
it is clear that $\Q/\Z=\bigcup_{n\in\N}C_n$, since this explores
every possible fraction in lowest terms.
Let $c\in\Q/\Z-C_n$ and suppose $|c|=n$.
But then this means that $c\in C_m$ for some $m\neq n$,
and therefore $|c|=m$.  By contradiction,
it follows that $C_n$ contains all cosets of order $n$.

For a given $n\in\N$, we now want to count the number of cosets
in $C_n$.  Clearly $C_1=\{\Z\}$ and so $|C_1|=1$.  So let $n>1$.
Notice that if $0<a<n$, then
$(a/n)\Z=((a+kn)/n)\Z$ for all integers $k$ since
$a/n-(a+kn)/n=a/n-a/n-k=-k\in\Z$.
This shows that
$C_n=\{(a/n)\Z:\mbox{$0<a<n$ and $\gcd(a,n)=1$}\}$.
It is now clear that $|C_n|=\phi(n)$, where $\phi$ is
Euler's phi-function.

\end{document}
