\documentclass{ecgd-l}

% Who will be the copy right owner?
%\copyrightinfo{2013}{American Mathematical Society}

\newtheorem{theorem}{Theorem}[section]
\newtheorem{lemma}[theorem]{Lemma}

\theoremstyle{definition}
\newtheorem{definition}[theorem]{Definition}
\newtheorem{example}[theorem]{Example}
\newtheorem{xca}[theorem]{Exercise}

\theoremstyle{remark}
\newtheorem{remark}[theorem]{Remark}

\numberwithin{equation}{section}

\newcommand{\R}{\mathbb{R}}
\newcommand{\B}{\mathbb{B}}
\newcommand{\G}{\mathbb{G}}
\newcommand{\V}{\mathbb{V}}
\newcommand{\gd}{\dot{g}}
\newcommand{\gh}{\hat{g}}

\begin{document}

\title{The Mother Minkowski Algebra of Order $m$}

\author{Spencer T. Parkin}
\address{102 W. 500 S., Salt Lake City, UT  84101}
\email{spencerparkin@outlook.com}

\subjclass[2010]{Primary 14J70}

% Date?
%\date{}

\begin{abstract}
Put abstract here.
\end{abstract}

\maketitle

\section{Motivation}

Before presenting the Mother Minkowski algebra of order $m$, we lead up to it here with
some background and motivation.  We begin by recalling that an algebraic set is any 
subset of an $n$-dimensional euclidean space $\R^n$ that is also the zero set of one
or more polynomials.  Given a geometric algebra $\G$, we can represents such sets
using blades $B\in\G$ as the set of all points $x\in\R^n$ such that
\begin{equation*}
p(x)\cdot B=0,
\end{equation*}
where $p:\R^n\to\V$ maps points in $\R^n$ to a vector space $\V$ generating
our geometric algebra $\G$.  Though not necessary, $\R^n$ is often embedded in $\V$;
but regardless of this, the function $p$ is necessarily defined in such a way that the expression
$p(x)\cdot B$ is a polynomial in the vector components of $x$ when $B\in\V$.

Letting $\B$ denote the set of all blades found in $\G$,
and letting $P(\R^n)$ denote the power set of $\R^n$,
we will find it useful to define the mapping $g:\B\to P(\R^n)$ as
\begin{equation*}
\gd(B) = \{x\in\R^n|p(x)\cdot B=0\}.
\end{equation*}
To see that $\gd(B)$ is an algebraic set, we first observe that when $B\in\V$,
$\gd(B)$ is the zero set of a polynomial in the vector components of $x$.
Secondly, we observe that if $\bigwedge_{i=1}^k b_i$ is a factorization
of the $k$-blade $B$, each $b_i$ being in $\V$, then
\begin{equation*}
p(x)\cdot B = -\sum_{i=1}^k (-1)^i (p(x)\cdot b_i)B_i,
\end{equation*}
where $B_i$ is given by
\begin{equation*}
B_i = \bigwedge_{j=1,j\neq i} b_j,
\end{equation*}
and therefore, since $\{B_i\}_{i=1}^k$ is a linearly independent set, we have
\begin{equation*}
\gd(B) = \bigcap_{i=1}^k \gd(b_i).
\end{equation*}
This model of representing algebraic sets using blades of a geometric algebra presents
some interesting properties.  To begin, if $A,B\in\B$ are blades with $A\wedge B\neq 0$,
then
\begin{equation*}
\gd(A)\cap\gd(B)=\gd(A\wedge B).
\end{equation*}
In this way, the outer product serves to take the intersection of two surfaces.  But we can also
look at the outer product in a different light as an operator that takes at least the union
of its two given surfaces.  To see this, we must consider an alternative interpretation
of blades $B\in\B$ as algebraic sets.  Defining $\gh:\B\to P(\R^n)$ as
\begin{equation*}
\gh(B)=\{x\in\R^n|p(x)\wedge B=0\},
\end{equation*}
we see that $\gh(B)=\gd(BI)$, where $I$ is the unit psuedo-scalar of $\G$, showing
that the image of $\gh$, like $\gd$, consists of algebraic sets.
Under this new interpretation, we find that for blades $A,B\in\B$, we have
\begin{equation*}
\gh(A)\cup\gh(B)\subseteq\gh(A\wedge B).
\end{equation*}
Exactly what surface we get from $A\wedge B$ in terms of $\gh$ can
be deduced by considering the surface $(A\wedge B)I$ in terms of $\gd$.

What's further a benefit of using blades to represent surfaces are the transformations
performable on such geometries through the use of outermorphisms; in particular,
outermorphisms $f:\B\to\B$ of the form
\begin{equation*}
f(B) = VBV^{-1},
\end{equation*}
where $V$ is a versor of $\G$.  Given such a function, we wish to
compare $\gd(B)$ with $\gd(f(B))$.  Interestingly, to understand
the latter in terms of the former, we need only understand the mapping from $\R^n\to\R^n$,
if any, induced by $V$ through $p$ as being each point $x\in\R^n$ mapped to a point
$y\in\R^n$, where
\begin{equation*}
Vp(x)V^{-1}=\lambda p(y),
\end{equation*}
$\lambda$ being some scalar in $\R$.
This is, of course, only a well defined mapping, provided that for every point $x\in\R^n$, there exists
such a point $y\in\R^n$, and that it is unique.
Assuming that $V$ and $p$ meet these requirements, and so do indeed induce
such a mapping $h:\R^n\to\R^n$, we can now show that
\begin{equation*}
\gd(f(B)) = h(\gd(B)).
\end{equation*}
We need only show that
\begin{equation*}
\gd(VBV^{-1})=\{x\in\R^n|V^{-1}p(x)V\cdot B=0\}.
\end{equation*}
Do that here...

\bibliographystyle{amsplain}
% Put the bib here!

\end{document}