\documentclass[12pt]{article}

\usepackage{amsmath}
\usepackage{amssymb}
\usepackage{amsthm}

\title{Section 2.6 Exercises\\Hertein's Topics In Algebra}
\author{Spencer T. Parkin}

\newtheorem{theorem}{Theorem}[section]
\newtheorem{definition}{Definition}[section]
\newtheorem{corollary}{Corollary}[section]
\newtheorem{identity}{Identity}[section]
\newtheorem{lemma}{Lemma}[section]
\newtheorem{result}{Result}[section]

%\newcommand{\gcd}{\mbox{gcd}}
\newcommand{\lcm}{\mbox{lcm}}
\newcommand{\Z}{\mathbb{Z}}

\begin{document}
\maketitle

\section*{Thoughts}

Remembering Gallian's book, he shows that the operation Herstein introduces here is well-defined.
Let $N$ be a normal subgroup of $G$ and define, for any $a,b\in G$,
\begin{equation*}
(Na)(Nb)=N(ab).
\end{equation*}
Is this a well-defined operation?  Well, $Na=Nb$ if and only if $ab^{-1}\in N$.
So let $a',b'\in G$ such that $Na=Na'$ and $Nb=Nb'$ and write
\begin{equation*}
(Na')(Nb') = N(a'b').
\end{equation*}
Can we show that
\begin{equation*}
ab(b')^{-1}(a')^{-1} = ab(a'b')^{-1}\in N?
\end{equation*}
Well, clearly $n_b=b(b')^{-1}\in N$.  Now since $n_a=a(a')^{-1}\in N$,
we have
\begin{equation*}
ab(b')^{-1}(a')^{-1} = an_b(a')^{-1} = n_a[(a')n_b(a')^{-1}]\in N,
\end{equation*}
since $N$ is normal.

\section*{Problem 6}

Show that every subgroup of an abelian group is normal.

Let $H$ be a subgroup of an abelian group $G$.
Then, for any $h\in H$ and $g\in G$, observe that
\begin{equation*}
ghg^{-1}=gg^{-1}h = h\in H,
\end{equation*}
showing that $H$ is normal in $G$.

\section*{Problem 7}

Is the converse of Problem 6 true?

The converse would read: if every subgroup of a group is normal, then the group is abelian.

I can't find a counter-example, but I'm willing to bet the converse is false.

When $gN=Ng$, this does not require that $gn=ng$ for all $n\in N$.

\section*{Problem 9}

Suppse $H$ is the only subgroup of order $|H|$ in the finite group $G$.
Prove that $H$ is a normal subgroup of $G$.

This would follow from proving the following statement.
If $\{H_i\}$ is a finite set of subgroups of $G$, each of order $n$, then
for all $g\in G$, and every integer $i$, there exists an integer $j$, such that
\begin{equation*}
gH_i = H_jg.
\end{equation*}

Interestingly, this presents the idea of two subgroups of $G$ being
co-normal.  Neither is necessarily normal by themselves, but together,
they're co-normal.

Here's an idea.  Let $H$ be a subroup of $G$.
Then, for any $g\in G$, let $K$ be the set given by
\begin{equation*}
K = \{g^{-1}hg|h\in H\}.
\end{equation*}
It is clear that $gK=Hg$.  We now show, whether or not $H$ is
a normal subgroup of $G$, that $K$ is a subgroup of $G$ having the
same order as $H$.

Clearly $e\in K$, so $K$ is non-empty.  Closure is trivial, for
\begin{equation*}
(g^{-1}h_1g)(g^{-1}h_2g)=g^{-1}h_1h_2g\in K
\end{equation*}
since $h_1h_2\in H$.  And then
\begin{equation*}
(g^{-1}hg)^{-1} = g^{-1}h^{-1}g\in K
\end{equation*}
since $h^{-1}\in H$.

To show now that $|K|=|H|$, let $\phi_g(h)=g^{-1}hg$ and
write
\begin{equation*}
g^{-1}h_1g = g^{_1}h_2g\implies h_1=h_2,
\end{equation*}
showing that $\phi_g$ is one-to-one.  Then since $H$ is finite,
$\phi_g$ is also onto $K$.  It follows that $|K|=|H|$.

Returning to the original problem, we see that $H$ must be normal in $G$,
because $H$ is the only subgroup of $G$ of its order.  (That is, we must have $K=H$.)

Now, if two subgroups of co-normal, can we make a group out of the
set of cosets shared between the two subgroups?  I don't see how.
There are two identity elements.

\section*{Problem 12}

Suppose that $N$ and $M$ are two normal subgroups of $G$ and that
$N\cap M=\{e\}$.  Show that for any $n\in N$, $m\in M$, $nm=mn$.

Consider the commutator $nmn^{-1}m^{-1}$.  See that $nmn^{-1}\in M$
since $M$ is normal; and therefore, the commutator is in $M$.  Similarly,
see that $mn^{-1}m^{-1}\in N$ since $N$ is normal; and therefore, the
commutator is in $N$.  It follows that $nmn^{-1}m^{-1}=e$, from which
the result follows.

\section*{Problem 15}

If $N$ is normal in $G$ and $a\in G$ is of order $o(a)$, prove that the order,
$m$, of $Na$ in $G/N$ is a dividsor of $o(a)$.

Clearly $(Na)^{|Na|}=N$ by definition of $|Na|$.
Now observe that $(Na)^{|a|}=Na^{|a|}=Ne=N$.
We then see that $|Na|$ divides $|a|$ since
\begin{equation*}
(Na)^{|Na|}=(Na)^{|a|}=N.
\end{equation*}

\end{document}