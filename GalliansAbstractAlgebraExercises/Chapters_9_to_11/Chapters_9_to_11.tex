\documentclass[12pt]{article}

\usepackage{amsmath}
\usepackage{amssymb}
\usepackage{amsthm}

\title{Chapters 9-11 Supplementary Exercises\\Gallian's Book on Abstract Algebra}
\author{Spencer T. Parkin}

\newtheorem{theorem}{Theorem}[section]
\newtheorem{definition}{Definition}[section]
\newtheorem{corollary}{Corollary}[section]
\newtheorem{identity}{Identity}[section]
\newtheorem{lemma}{Lemma}[section]
\newtheorem{result}{Result}[section]

%\newcommand{\gcd}{\mbox{gcd}}
\newcommand{\lcm}{\mbox{lcm}}
\newcommand{\abs}{\mbox{abs}}
\newcommand{\Z}{\mathbb{Z}}
\newcommand{\R}{\mathbb{R}}
\newcommand{\cl}{\mbox{cl}}
\newcommand{\aut}{\mbox{Aut}}
\newcommand{\inn}{\mbox{Inn}}

\begin{document}
\maketitle

%\section*{Exercise 8}
%
%Let $k$ be a divisor of $n$.  The factor gruop $(Z/\langle n\rangle)/(\langle k\rangle/\langle n\rangle)$
%is isomorphic to some very familiar group.  What is the group?
%
%By Exercise 40 of Chapter 10 (The Third Isomorphism Theorem), we
%see that $(Z/\langle n\rangle)/(\langle k\rangle/\langle n\rangle)\approx Z/\langle k\rangle$.
%What more is there to say?
%
%\section*{Exercise 30}
%
%Let $G$ be a group and let $\phi:G\to G$ be a function such that
%\begin{equation*}
%\phi(g_1)\phi(g_2)\phi(g_3) = \phi(h_1)\phi(h_2)\phi(h_3)
%\end{equation*}
%whenever $g_1g_2g_3=e=h_1h_2h_3$.  Prove that there exists an element of $a$
%in $G$ such that $\Psi(x)=a\phi(x)$ is a homomorphism.
%
%If $a=e\in G$ and $u,v\in G$, then
%\begin{equation*}
%\Psi((uv)^{-1})\Psi(u)\Psi(v) = e\implies \Psi(v^{-1}u^{-1})=\Psi(v)^{-1}\Psi(v)^{-1}.
%\end{equation*}
%
%Hmmm...  Can we somehow show that $\Psi$ is a homomorphism here?
%We cannot use homomorphic properties of $\Psi$ before we know that it's
%a homomorphism.

\section*{Exercise 36}

A proper subgroup $H$ of a group $G$ is called {\it maximal} if there is no
subgroup $K$ such that $H\subset K\subset G$.  Prove that $Q$ under
addition has no maximal subgroups.

Let $H$ be any non-trivial, proper subgroup of $Q$.  Then, if $h\in Q$ is also a
member of $H$, then so is any integer multiple of $h$.  In other words,
\begin{equation*}
hZ=\{zh|z\in Z\}\subseteq H.
\end{equation*}
Notice that $hZ$ is also a subgroup of $H$.

Now, since $H$ is a proper subgroup of $Q$, there exists $r\in Q\backslash H$.
If we wanted to form a subgroup of $Q$ containing $H$ and $r$, then it must
contain at least $H$ and $rZ$.  Letting $H+rZ$ denote the set
\begin{equation*}
H+rZ=\{h+zr|h\in H,z\in Z\},
\end{equation*}
it is not hard to show that $H+rZ$ is a subgroup of $Q$ properly containing $H$.
What remains to be shown, however, is that $H+rZ$ is a proper subgroup of $Q$.

To that end, suppose $Q=H+rZ$ in the hopes of reaching a contradiction.  This then implies that
\begin{equation*}
\langle r+H\rangle = Q/H,
\end{equation*}
which is to say that the factor group $Q/H$ is cyclic, being generated by $r+H$.
But, since $rZ\cap H$ is a non-trivial group, it follows that the order of $r+H$
is finite.  (Then, interestingly, since $Q=H+rZ$ is the smallest subgroup of $Q$
containing $H$ properly, we're also assuming here that $H$ is maximal; and since $H$
is maximal, $Q/H$ must be a cyclic group of prime order, it having no non-trivial
and proper subgroups.)  In any case, let $n=|r+H|$.  (We do not care that $n$ is prime.)
Now realize that for the rational $r/n\in Q$, there must exist $h\in H$ and $z\in Z$
such that $r/n=h+zr$.  But then this implies that
\begin{equation*}
r = nh + nzr\in H,
\end{equation*}
(since $nzr\in H$ by the order of $r+H$), which is a contradiction.
Our assumption, therefore, that $Q=H+rZ$, is false, and we must have
$H+rZ$ a proper subgroup of $Q$.  This completes the proof!





\end{document}