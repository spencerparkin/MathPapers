\documentclass{article}
\usepackage{amsmath}
\usepackage{amssymb}
\title{Chapter 4 Homework}
\author{Spencer}
\addtolength{\oddsidemargin}{-.575in}
\addtolength{\evensidemargin}{-.575in}
\addtolength{\textwidth}{1.0in}
\addtolength{\topmargin}{-.575in}
\addtolength{\textheight}{1.25in}
\begin{document}
\maketitle

\section*{Problem 1}

Find all generators of $Z_6$, $Z_8$, and $Z_{20}$.

The generators of $Z_6$ are 1, and 5 since they are the
only integers in $[0,5]$ coprime with 6.

The generators of $Z_8$ are 1, 3, 5, and 7 for similar reasons.

The generators of $Z_{20}$ are 1, 3, 7, 9, 11, 13, 17, and 19.

The problem was done by a direct application of Corollary 3 of
Theorem 4.2.

\section*{Problem 2}

Suppose that $\langle a\rangle$, $\langle b\rangle$, and $\langle c\rangle$
are cyclic groups of orders 6, 8, and 20, respectively.  Find all generators
of these cyclic groups.

Using the results of Problem 1 and applying Corollary 2 of Theorem
4.2 we see that all generators of $\langle a\rangle$ are $a$ and $a^5$,
all generators of $\langle b\rangle$ are $a$, $a^3$, $a^5$, and $a^7$,
and all generators of $\langle c\rangle$ are $a$, $a^3$, $a^7$, $a^9$,
$a^{11}$, $a^{13}$, $a^{17}$, and $a^{19}$.

\section*{Problem 3}

List the elements of the subgroups $\langle 20\rangle$ and
$\langle 10\rangle$ in $Z_{30}$.

By Corollary 1 of Theorem 4.2, since $\gcd(30,10)=\gcd(30,20)$,
we see that $\langle 10\rangle=\langle 20\rangle$, and we
need only enumerate the elements of one of these.
\begin{equation*}
\langle 10\rangle = \{ 10, 20, 0 \}
\end{equation*}
To see how the corollary applies, let $a=1\in Z_{30}$, then $|a|=30$ and
$\langle a^{10}\rangle=\langle a^{20}\rangle$ since $\gcd(30,10)=\gcd(30,20)$.

\section*{Problem 4}

List the elements of the subgroups $\langle 3\rangle$ and $\langle 15\rangle$
in $Z_{18}$.

Again, since $\gcd(18,15)=\gcd(18,3)$ we see that $\langle 3\rangle=\langle 15\rangle$
and $\langle 3\rangle=\{ 3, 6, 9, 12, 15, 0 \}$.

\section*{Problem 5}

List the elements of the subgroups $\langle 3\rangle$, $\langle 7\rangle$ in $U(20)$.

\begin{equation*}
\langle 3\rangle = \{ 3, 9, 7, 1 \} = \{ 7, 9, 3, 1 \} = \langle 7\rangle
\end{equation*}

\newcommand{\lcm}{\mbox{lcm}}

\section*{Problem 6}

What do Exercises 3, 4, and 5 have in common?  Try to make a generalization
that includes these three cases.

Let $a\in G$ with $|a|<\infty$ where $G$ is a group and let $k$ be a positive integer.
Then if $|a|=|a^k|$, then $\langle a\rangle=\langle a^k\rangle$.

Proof: By Theorem 4.2 we see that $|a^k|=|a|/\gcd(|a|,k)$.  But since
$|a^k|=|a|$, this implies that $|a|$ and $k$ are coprime.  Applying the
theorem again we see that
$\langle a^k\rangle=\langle a^{\gcd(|a|,k)}\rangle=\langle a\rangle$.

This was a stupid conclusion by Corollary 1 of Theorem 4.1.

\section*{Problem 7}

Find an example of a noncyclic group, all of whose proper
subgroups are cyclic.

I think that $U(8)$ works.  It doesn't have a generator, so it's
not a cyclic group.  Then the groups generated by each of its members
are the only subgroups, all of which are proper.

\section*{Problem 9}

How many subgroups does $\mathbb{Z}_{20}$ have?  List a generator for each
of these subgroups.  Suppose that $G=\langle a\rangle$ and $|a|=20$.
How many subgroups does $G$ have?  List a generator for each of these subgroups.

The group $\mathbb{Z}_{20}$ has exactly 6 subgroups by the Corollary
of Theorem 4.3 since the divisors of 20 are $\{1,2,4,5,10,20\}$ and
there is a subgroup generated by each divisor.

By Theorem 4.3, we can use our results about $\mathbb{Z}_{20}$ to deduce that
$G$ has exactly 6 subgroups, and the set of generators is $\{a,a^2,a^4,a^5,a^{10},a^{20}=e\}$.

\section*{Problem 11}

Let $G$ be a group and let $a\in G$.  Prove that $\langle a^{-1}\rangle=\langle a\rangle$.

Let $b\in\langle a\rangle$.  Then $b=a^k$ for any $k\in\mathbb{Z}$.
If $k=0$, then $b=e\in\langle a^{-1}\rangle$.  If $k>0$, then
$b=((a^{-1})^{-1})^k=(a^{-1})^{-k}\in\langle a^{-1}\rangle$.
If $k<0$, then $b=(a^{-1})^{|k|}\in\langle a^{-1}\rangle$.

Let $b\in\langle a^{-1}\rangle$.  Then $b=(a^{-1})^k$ for any $k\in\mathbb{Z}$.
If $k=0$, then $b=e\in\langle a\rangle$.  If $k>0$, then
$b=a^{-k}\in\langle a\rangle$.  If $k<0$, then $b=((a^{-1})^{-1})^{|k|}=a^{|k|}\in\langle a\rangle$.

\section*{Problem 13}

Suppose that $|a|=24$.  Find a generator for $\langle a^{21}\rangle\cap\langle a^{10}\rangle$.
In general, what is a generator for the subgroup $\langle a^m\rangle\cap\langle a^n\rangle$.

Let us first use Theorem 4.2 to say that $\langle a^{21}\rangle=\langle a^3\rangle$,
and $\langle a^{10}\rangle=\langle a^2\rangle$.  We then see that...
\begin{equation*}
\langle a^3\rangle\cap\langle a^2\rangle = \langle a^{\lcm(3,2)}\rangle = \langle a^6\rangle
\end{equation*}
In general we have...
\begin{equation*}
\langle a^m\rangle\cap\langle a^n\rangle=\langle a^{\lcm(m,n)}\rangle
\end{equation*}
Let $x\in\langle a^m\rangle\cap\langle a^n\rangle$.  Then $x=a^{i}$
where $m|i$ and $n|i$.  That is, $x$ is raised to the power of some
common multiple of $m$ and $n$.  But $\lcm(m,n)$ is the least positive of
such multiples, so some multiple of it, say $j\lcm(m,n)$, is equal to $i$, and we
see that $x=a^{j\lcm(m,n)}\in\langle a^{\lcm(m,n)}\rangle$.
Now let $x\in\langle a^{\lcm(m,n)}\rangle$.  Then $x\in\langle a^m\rangle\cap\langle a^n\rangle$
trivialy since $\langle a^{\lcm(m,n)}\rangle$ contains all powers of $a$ raised to all
common multiples of $m$ and $n$.

Every divisor of $m$ and $n$ divides $\gcd(m,n)$.  Similarly, every
common multiple of $m$ and $n$ is divisible by $\lcm(m,n)$.  If that
last statement isn't true, then my proof fails.

\section*{Problem 26}

Find all generators of $\mathbb{Z}$.

$1$ is an obvious generator.  By Problem 11 we see that $-1$ is also a generator.
Also by Problem 11, we can show that no other generators exist if there is no
generator greater than 1.  So consider all multiples of $n$ where $n>1$.
Not every positive integer is a multiple of $n$ and therefore $n$ is not a generator of
$\mathbb{Z}$.  So the only generators of $\mathbb{Z}$ are 1 and -1.

\section*{Problem 29}

List all the elements of order 8 in $Z_{8000000}$.  How do you know your list is complete?

By Theorem 4.4, since $8|8000000$, the
number of elements of order 8 is exactly $\phi(8)=8(1-1/2)=4$.
Notice that 1 is a generator of $Z_{8000000}$.  (We already knew that $Z_{8000000}$ was
cyclic.)  Since all powers of 1 generate
our set, we can apply Theorem 4.2 to find the powers of 1 that have order 8.
According to that theorem, these are the elements $1^k$ where
$8=|1|/\gcd(|1|,k)\implies\gcd(8000000,k)=1000000$.  It is then easy to see that
the elements 1000000, 3000000, 5000000, and 7000000 satisfy this.  Then by
Theorem 4.4 these must be the only such elements with order 8.

\section*{Problem 32}

Determine the subgroup lattice for $Z_{12}$.

\section*{Problem 36}

Prove that a finite group is the union of proper subgroups if and only if the group is not cyclic.

Let $G$ be a finite non-cyclic group.  Choose $a\in G$ and notice that
$G-\langle a\rangle\neq\emptyset$, or else $G$ would be cyclic.
So $\langle a\rangle$ is a proper subgroup by this and Theorem 3.4.  Then,
since we chose any $a\in G$,
the union of all such subgroups will recover all elements in $G$.
Notice that we need not consider any
non-cyclic proper subgroups of $G$.

To prove the other direction, consider the contrapositive.  Let $G=\langle a\rangle$
be a finite cyclic group.
Then, by Theorem 4.3, every subgroup of $G$ is cyclic, and they are generated by
the powers of $a$
that are divisors of $|G|$.  Now notice that the only subgroup of $G$ that is not proper
is $G$ itself.  But this is also the only subgroup of $G$ that contains the element $a$.
Therefore, the union of all proper subgroups of $G$ will not recover all elements of $G$.

\section*{Problem 37}

Show that the group of positive rational numbers under multiplication
is not cyclic.

We need to show that there does not exist any positive rational that generates
the set of all positive rationals using multiplication.  Let $q>1$ be a rational
number.  Now notice that there exists at least one positive
rational between $q$ and $q^2$, for example, $(q+q^2)/2$.  In fact, there are a countably
infinite number of rationals in $(q,q^2)$.
No rational $q>1$ can therefore generate the group in question.
Now suppose $q<1$.  Notice then that all powers of $q$ are less than 1, so that
no $q<1$ can generate the group in question.
Finally, suppose $q=1$.  Then $\langle q\rangle=\{1\}$ which does not even have
the same cardinality as the group in question.

\section*{Problem 39}

Give an example of a group that has exactly 6 subgroups (including the
tirival subgroup and the group itself).  Generalize to exactly $n$
subgroups for any positive integer $n$.

The group $\mathbb{Z}_m$ where $m$ has exactly $n$ divisors (include 1 and itself)
will have exactly $n$ subgroups by the Corollary of Theorem 4.3.  Let $p$ be a prime
number.  Then we could choose $m=p^n$.  Choosing $p=2$ and $n=6$, we get $m=64$
so that $\mathbb{Z}_{64}$ has exactly 6 subgroups.

\section*{Problem 40}

Let $m$ and $n$ be elements of the group $\mathbb{Z}$.  Find a generator for the
group $\langle m\rangle\cap\langle n\rangle$.

Here we can re-use all of the work we did in Problem 13 to conclude that $\lcm(m,n)$
is a generator for $\langle m\rangle\cap\langle n\rangle$.

\section*{Problem 42}

Prove that an infinite group must have an infinite number of subgroups.

I don't know how to prove this for uncountably infinite groups, so let's
assume that we have a countably infinite group $G$, and let $a_k$ be
a sequence containing all elements of $G$.

Suppose that $|a_k|<\infty$ for all $k\in\mathbb{N}$.
Then every cyclic subgroup of $G$ has finite order.
Then since we know that...
\begin{equation*}
G = \bigcup_{k\in\mathbb{N}}\langle a_k\rangle
\end{equation*}
...we know that $\{\langle a_k\rangle:k\in\mathbb{N}\}$ must reduce to
an infinite set of distinct cyclic subgroups of $G$.  This is because
the union of a finite number of finite sets cannot give us an infinite set.

Suppose now that not all elements of $G$ have finite order.
Let $a\in G$ be an element having infinite order.
Then by Theorem 4.1, if $a\in G$, then all
distinct powers of $a$ are distinct group elements of $G$.  We can show that
$G$ has an infinite number of subgroups by constructing a sequence of distinct
cyclic subgroups of $G$.  Notice that for all $k\in\mathbb{N}$, that
$\langle a^k\rangle$ is such a sequence.  To see this, let $i$ and $j$
be integers such that $1\leq i<j$.  Then we can always find $x\in\langle a^i\rangle$
such that $x\not\in\langle a^j\rangle$.  For example, choose $x=a^i$.  From this
we see that each new group in the sequence of cyclic subgroups is different from
all those that came before it in the sequence.
So all of the subgroups are distinct and we have identified
an infinite number of subgroups of $G$.

I probably messed this problem up badly.

\section*{Problem 51}

Prove that no group can have exactly two elements of order 2.

Let $a,b\in G$ where $G$ is a group, $a\neq b$, and $|a|=|b|=2$.  Note that
$a$ and $b$ are their own inverses.
\begin{align*}
e &= a^2b^2 \\
 &= aabb \\
 &= ab^{-1}a^{-1}b \\
 &= abab \\
 &= (ab)^2
\end{align*}
Now notice that $ab\neq a$ since this would implies that $b=e$, and similarly,
$ab\neq b$ since this would imply that $a=e$.  (The identity has order one, not two.)
Then to show that $|ab|=2$, we must show that $a$ is not the inverse of $b$.
But this must be so since the inverse of $b$ is $b$ and we know that $a\neq b$.
So $ab$ is a third element of $G$ having order 2, contradicting our hypothesis that
$G$ has exactly two elements of order two.  So no group can have exactly two elements
of order two.

\section*{Problem 59}

If $|a^5|=12$, what are the possibilities for $|a|$?  If $|a^4|=12$, what
are the possibilities for $|a|$?

By Theorem 4.2, $12=|a^5|=|a|/\gcd(|a|,5)$.  This implies that
$|a|=12\gcd(|a|,5)$.  But since $\gcd(|a|,5)\in\mathbb{N}$, we must
have 12 as a divisor of $|a|$.  Then for some integer $k>0$ we have
$|a|=12\gcd(12k,5)$.  So the possibilities for $|a|$ are $12\cdot 1$ and $12\cdot 5=60$,
if I didn't screw up.

Using the same line of reasoning for the other part of the question,
Theorem 4.2 tells us that $12=|a^4|=|a|/\gcd(|a|,4)$.  This implies that
$|a|=12\gcd(|a|,4)\implies |a|=12\gcd(12k,4)$ for some integer $k>0$.
But since $4|12k$ for all such $k$, we have $|a|=12\cdot 4 = 48$ as our
only possibility.

\section*{Problem 62}

Prove that $H=\left.\left\{\left[\begin{array}{cc}1&n\\0&1\end{array}\right]\right|n\in\mathbb{Z}\right\}$
is a cyclic subgroup of $GL(2,\mathbb{R})$.

Notice that $H\subset GL(2,\mathbb{R})$.
All we need to show now is that $h=\left[\begin{array}{cc}1&1\\0&1\end{array}\right]$ is
a generator of $H$ and then the proof follows by Theorem 3.4.  It's easy to see
that $h\in H$.  Suppose now that $h^k=\left[\begin{array}{cc}1&k\\0&1\end{array}\right]\in H$
for some fixed integer $k$.  Then...
\begin{equation*}
h^{k+1}=h^kh=
\left[\begin{array}{cc}1&k\\0&1\end{array}\right]
\left[\begin{array}{cc}1&1\\0&1\end{array}\right]=
\left[\begin{array}{cc}1&k+1\\0&1\end{array}\right]\in H
\end{equation*}
...and we see also that...
\begin{equation*}
h^{k-1}=h^kh^{-1}=
\left[\begin{array}{cc}1&k\\0&1\end{array}\right]
\left[\begin{array}{cc}1&-1\\0&1\end{array}\right]=
\left[\begin{array}{cc}1&k-1\\0&1\end{array}\right]\in H
\end{equation*}
Therefore, by induction, $h^k\in H$ for all integers $k$.  Then because
$h^k=\left[\begin{array}{cc}1&k\\0&1\end{array}\right]$, (we just proved
this by induction), there are no members of $H$ that are not of the form $h^k$.
So $h$ is a generator of $H$.

\end{document}