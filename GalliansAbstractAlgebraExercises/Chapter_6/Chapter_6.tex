\documentclass[12pt]{article}

\usepackage{amsmath}
\usepackage{amssymb}
\usepackage{amsthm}

\title{Chapter 6 Exercises\\Gallian's Book on Abstract Algebra}
\author{Spencer T. Parkin}

\newtheorem{theorem}{Theorem}[section]
\newtheorem{definition}{Definition}[section]
\newtheorem{corollary}{Corollary}[section]
\newtheorem{identity}{Identity}[section]
\newtheorem{lemma}{Lemma}[section]
\newtheorem{result}{Result}[section]

%\newcommand{\gcd}{\mbox{gcd}}
\newcommand{\lcm}{\mbox{lcm}}
\newcommand{\abs}{\mbox{abs}}
\newcommand{\Z}{\mathbb{Z}}
\newcommand{\R}{\mathbb{R}}
\newcommand{\G}{\mathbb{G}}
\newcommand{\stab}{\mbox{stab}}
\newcommand{\aut}{\mbox{Aut}}
\newcommand{\inn}{\mbox{Inn}}

\begin{document}
\maketitle

\section*{Problem 15}

If $G$ is a group, prove that $\aut(G)$ and $\inn(G)$ are groups.

It is clear that the identity function $\epsilon$ is in $\aut(G)$.
Let $\alpha,\beta\in\aut(G)$.  Notice that if each of $\alpha$ and
$\beta$ are bijections from $G$ to $G$, then so is $\alpha\beta$.
Letting $a,b\in G$, we see that
\begin{equation*}
\alpha(\beta(ab))=\alpha(\beta(a)\beta(b)=\alpha(\beta(a))\alpha(\beta(b)),
\end{equation*}
showing that $\alpha\beta$ is operation preserving as well.
Thus far we have proven closure.  Now notice that if $\alpha$ is
a bijection of $G$, then so is $\alpha^{-1}$.  To show that
$\alpha^{-1}$ is operation preserving, notice that
\begin{equation*}
\alpha(\alpha^{-1}(a)\alpha^{-1}(b))=\alpha(\alpha^{-1}(a))\alpha(\alpha^{-1}(b))=ab=\alpha(\alpha^{-1}(ab)),
\end{equation*}
and then take the $\alpha^{-1}$ of both of the furthest sides of this equation.
We now that if $\alpha\in\aut(G)$, then so is $\alpha^{-1}$.  We can now
clame that $\aut(G)$ is a group by the two-step subgroup test.

Go on...

\section*{Problem 17}

Let $r\in U(n)$.  Prove that the mapping $\alpha:Z_n\to Z_n$ defined by
$\alpha(s)=sr\pmod n$ for all $s\in Z_n$ is an automorphism of $Z_n$.

It is clear that $\alpha$ is onto, because $\langle r\rangle=Z_n$ since $\gcd(r,n)=1$.
Then, since $\alpha$ maps a finite set to a finite set, it is also one-to-one.
To see that $\alpha$ is operation preserving, we write, for any $a,b\in Z_n$,
\begin{equation*}
\alpha(a+b) = s(a+b)=sa+sb=\alpha(a)+\alpha(b).
\end{equation*}
This proof was nothing more than a generalization of what was given
in Example 13.

\section*{Made-Up Problems}

\subsection*{Problem 1}

For any geometric algebra $\G$, show that
the set of all versors $V\in\G$ forms a group under the geometric product.
Does the set of all unit-versors form a subgroup?

The geometric product is associative.
Notice that 1 is the identity versor.  Let $A,B\in\G$ be versors.
There then exists a set of $m$ invertible vectors $\{a_i\}_{i=1}^m$ and
a set of $n$ invertible vectors $\{b_i\}_{i=1}^n$ such that
$A=\prod_{i=1}^m a_i$ and $B=\prod_{i=1}^n b_i$.
It now follows by the definition of what a versor is, that $AB$ is also a versor.
$AB$ is a product of one or more invertible vectors.
Now notice that $A^{-1}=\prod_{i=1}^m a_{m-i+1}^{-1}$ is also, by
definition, a versor.  (It may also be required to say that by the zero-product
property of the geometric property, inverses are unique.)

Seeing that $|A|=\prod_{i=1}^m |a_i|$, (this is not hard to prove), if $|A|=1$, we can, without loss
of generality, assume, for each $a_i$, that $|a_i|=1$.  (Such a set of $m$
vectors can always be found for $A$.)  Do the same for $B$.  Closure then follows.  Note that the identity
has magnitude 1.  Then, seeing that, for any $a_i$, we have $|a_i|=|a_i^{-1}|$,
we also have inverses.  So the unit-versors form a subgroup of the versor group.

\subsection*{Problem 2}

Show that the even versors of a geometric algebra $\G$
form a group under the geometric product?  Do the odd versors form a group?

Clearly the odd versors don't form a group, because we don't have closure.
The even versors give us closure.  The identity versor may be considered even.
And it is clear the the inverse of an even versor is also even.

\subsection*{Problem 3}

For a geometric algebra $\G$, show that
the function $\phi_W(V)=WVW^{-1}$, with $W$ a versor of $\G$,
is an automorphism of the group of versors of $\G$.

Notice that we need only consider the case when $W$ is a vector,
since any composition of two or more isomorphisms is an isomorphism by
Theorem 6.4.

\subsection*{Problem 4}

For a geometric algebra $\G$ and a fixed positive integer $k$, show that the $k$-vectors of $\G$
form a group under addition.  Do the $k$-blades form a group?  (Let zero be a $k$-vector for any $k$.)

$k$-blades don't generally form a group, because we don't always have closure for $k\geq 2$.
(In some geometric algebras, we do have closure for $k=2$, but not all.)

$k$-vectors do give us closure.  Zero is the zero $k$-vector.  $k$-vector addition is associative.
The additive inverse of a $k$-vector is easy to calculate and it is unique.

\subsection*{Problem 5}

For any versor $W$ in a geometric algebra $\G$, define
the function $\phi_W(V)=WVW^{-1}$ on the set of $k$-vectors in $\G$,
and show that it is an automorphism of the group of $k$-vectors.

Notice again that we need only consider the case when $W$ is a vector,
since any composition of two or more isomorphisms is an isomorphism by
Theorem 6.4.

\end{document}