\documentclass[12pt]{article}

\usepackage{amsmath}
\usepackage{amssymb}
\usepackage{amsthm}

\title{Chapter 9 Exercises\\Gallian's Book on Abstract Algebra}
\author{Spencer T. Parkin}

\newtheorem{theorem}{Theorem}[section]
\newtheorem{definition}{Definition}[section]
\newtheorem{corollary}{Corollary}[section]
\newtheorem{identity}{Identity}[section]
\newtheorem{lemma}{Lemma}[section]
\newtheorem{result}{Result}[section]

%\newcommand{\gcd}{\mbox{gcd}}
\newcommand{\lcm}{\mbox{lcm}}
\newcommand{\abs}{\mbox{abs}}
\newcommand{\Z}{\mathbb{Z}}
\newcommand{\R}{\mathbb{R}}
\newcommand{\G}{\mathbb{G}}
\newcommand{\stab}{\mbox{stab}}
\newcommand{\aut}{\mbox{Aut}}
\newcommand{\inn}{\mbox{Inn}}
\newcommand{\orb}{\mbox{orb}}

\begin{document}
\maketitle

\section*{Exercise 1}

Let $H=\{(1),(12)\}$.  Is $H$ normal in $S_3$.

No, $(123)H\neq H(123)$, because $(123)(12)=(13)\neq(23)=(12)(123)$.

\section*{Exercise 2}

Prove that $A_n$ is normal in $S_n$.

Let $\alpha\in A_n$ and let $\beta\in S_n$.  Now notice that $\beta\alpha\beta^{-1}\in A_n$,
in the case that $\beta$ is an even permutation, or an odd permuation.
It then follows by Theorem 9.1 that $A_n$ is a normal subgroup of $S_n$.

\section*{Exercise 3}

Show that if $G$ is the internal direct product of $H_1,H_2,\dots,H_n$ and
$i\neq j$ with $1\leq i\leq n$, $1\leq j\leq n$, then $H_i\cap H_j=\{e\}$.

Without loss of generality, let $i<j$.
Now notice that
\begin{equation*}
H_i\subseteq H_1H_2\dots H_i\dots H_{j-2}H_{j-1}
\end{equation*}
and that $H_1H_2\dots H_i\dots H_{j-2}H_{j-1}\cap H_j=\{e\}$.
It follows that $H_i\cap H_j=\{e\}$.

\section*{Finishing Theorem 9.6}

We are given $\phi(h_1h_2\dots h_n)=(h_1,h_2,\dots,h_n)$.
It is immediately clear that $\phi$ is onto $H_1\oplus H_2\oplus\dots\oplus H_n$.
By the uniqueness of representation of elements in $H_1H_2\dots H_n$ already proven,
it follows that $\phi$ is one-to-one.  That $\phi$ is operation preserving follows
from the commutativity among disjoint subgroups.  For all integers $i\in[1,n]$,
for all $a_i,b_i\in H_i$, we have
\begin{align*}
 & \phi(a_1a_2\dots a_nb_1b_2\dots b_n) \\
 =\,&\phi(a_1b_1a_2b_2\dots a_nb_n) \\
 =\,&(a_1b_1,a_2b_2,\dots,a_nb_n) \\
 =\,&(a_1,a_2,\dots,a_n)(b_1,b_2,\dots,b_n) \\
 =\,&\phi(a_1a_2\dots a_n)\phi(b_1b_2\dots b_n).
\end{align*}

\section*{Exercise 7}

Prove that if $H$ has index 2 in $G$, then $H$ is normal in $G$.

The cosets of $H$ in $G$ are $H$ and $xH$ for any $x\in G-H$.
Now consider $Hx$.  This is $H$ or $xH$.  But it can't be $H$, because $x\not\in H$.
It must, therefore, be $xH$.

\section*{Exercise 10}

Prove that a factor group of a cyclic group is cyclic.

Let $H$ be a normal subgroup of a cyclic group $G=\langle g\rangle$.
(All cyclic groups are Abelian and therefore, all subgroups of a cyclic
group are normal.)  We then see that
\begin{equation*}
G/H = \{aH|a\in G\}=\{g^kH|k\in\Z\}=\langle gH\rangle.
\end{equation*}

\section*{Exercise 11}

Let $H$ be a normal subgroup of $G$.  If $H$ and $G/H$ are Abelian, must $G$
be Abelian?

No.  Consider $G=D_4$ and $H$ as the subgroup of rotations in $D_4$.

\section*{Exercise 12}

Prove that a factor group of a cyclic group is cyclic.

Let $a,b\in G/H$ with $a=xH$ and $b=yH$ for $x,y\in G$.
We then have
\begin{equation*}
ab=xHyH=xyH=yxH=yHxH=ba.
\end{equation*}

\section*{Exercise 44}

If $|G|=pq$, where $p$ and $q$ are primes that are not necessarily distinct,
prove that $|Z(G)|=1$ or $pq$.

If $Z(G)=\{e\}$, we're done.  So suppose $Z(G)\neq\{e\}$.
If $G$ has an element of order $pq$, we're done,
so assume no such element exists.  It follows that
$G$ must have an element of order $p$ or $q$.  Suppose
$Z(G)=\langle z\rangle$ for an element $z\in G$ of
order $p$.  Then, since $q$ is prime, $G/Z(G)$ is cyclic,
and so, by Theorem 9.3, we must have $G$ is Abelian.
But then $Z(G)=G$, which is a contradiction.
A similar contradiction is reached if we suppose $Z(G)=\langle z\rangle$
for an element $z\in G$ of order $q$.  It follows that $Z(G)=G$,
and therefore, $|Z(G)|=pq$.

\section*{Exercise 46}

Let $G$ be an Abelian group and let $H$ be the subgroup consisting
of all elements of $G$ that have finite order.  Prove that every nonidentity
element in $G/H$ has infinite order.

Let $a\in G/H$ be a non-identity element.  Then there exists $g\in G$ such
that $a=gH$.  Clearly $g\not\in H$ by Property 2 of the Lemma for Theorem 7.1.
It follows that $|g|=\infty$.

\section*{Exercise 50}

Show that the intersection of two normal subgroups of $G$ is a normal subgroup of $G$.

Let $H$ and $K$ be normal subgroups of $G$.  We know that $H\cap K$ is a subgroup
of $G$ by a previous problem from a previous chapter.  Let $g\in G$ and let
$x\in H\cap K$.  Then $gxg^{-1}\in H$ by the normality of $H$ and $gxg^{-1}\in K$
by the normality of $K$.  It follows that $gxg^{-1}\in H\cap K$ and so $H\cap K$
is normal in $G$.

\section*{Problem similar to Exercise 50}

Gallian himself helped me with this in relation to another problem.
We prove the following statement.  If $K$ and $N$ are subgroups of a group
$G$ with $N$ normal in $G$, then $K\cap N$ is normal in $G$.  ($K$ is not necessarily
normal in $G$.)

Let $x$ belong to $K\cap N$ and let $k$ belong to $K$.  Then, by closure, $kxk^{-1}\in K$,
and by the normality of $N$, we have $kxk^{-1}\in N$.  It follows now that $N\cap K$
is normal in $K$.

\section*{Exercise 51}

Let $N$ be a normal subgroup of $G$ and let $H$
be any subgroup of $G$.  Prove that $NH$ is a subgroup of $G$.
Give an example to show that $NH$ need not be a subgroup of $G$ if neither
$N$ nor $H$ is normal.

Notice that $e\in NH$.  Let $a,b\in NH$.  Then there exist
elements $n_a,n_b\in N$ and $h_a,h_b\in H$ such that $a=n_ah_a$
and $b=n_bh_b$.  Then, by the normality of $N$, there exists
an element $n_b'\in N$ such that
\begin{equation*}
ab^{-1}=n_ah_a(n_bh_b)^{-1}=n_ah_ah_b^{-1}n_b=n_an_b'h_ah_b^{-1}\in NH.
\end{equation*}

I'm terrible at finding examples.

\section*{Exercise 53}

Let $N$ be a normal subgroup of a group $G$.  If $N$ is cyclic,
prove that every subgroup of $N$ is also normal in $G$.

Let $H$ be a subgroup of $N=\langle n\rangle$.  Then, for some
integer $i$, we have $H=\langle n^i\rangle$.  Then, for all $g\in G$,
and any integer $j$, we have
\begin{equation*}
g(n^i)^jg^{-1}=g(n^j)^ig^{-1}=(gn^jg^{-1})^i=(n^k)^i=(n^i)^k\in N,
\end{equation*}
where here, $gn^jg^{-1}=n^k$ for some integer $k$ by
virtue of $N$ being a normal subgroup of $G$.  It follows
now by Theorem 9.1, the normal subgroup test, that $H$ is normal in $G$.

\section*{Exercise 54}

Without looking at inner automorphisms of $D_n$, determine the number
of such automorphisms.

By Theorem 9.4, we know that $\inn(D_n)\approx D_n/Z(D_n)$.
By Example 11 of Chapter 3, we know that $|Z(D_n)|$ is 2 if
$n$ is even, and 1 if $n$ is odd.  Then, knowing that $|D_n|=2n$,
we have
\begin{equation*}
|\inn(D_n)|=\frac{|D_n|}{|Z(D_n)|}=\left\{\begin{array}{ll}
n & \mbox{if $n$ even,} \\
2n & \mbox{if $n$ odd.}
\end{array}\right.
\end{equation*}

\section*{Exercise 55}

Let $H$ be a normal subgroup of a finite group $G$ and let $x\in G$.
If $\gcd(|x|,|G/H|)=1$, show that $x\in H$.

Consider the cyclic subgroup of $G/H$ generated by $xH$.
It is clear that $|\langle xH\rangle|=|x|$, but we must have
$|\langle xH\rangle|$ dividing $|G/H|$.  This means that
$|x|$ must divide $|G|/|H|$.  Therefore, $|x|=1$,
because $\gcd(|x|,|G|/|H|)=1$, and we see that
$xH=H\implies x\in H$.

\section*{Exercise 61}

Suppose that $H$ is a normal subgroup of a finite group $G$.
If $G/H$ has an element of order $n$, show that $H$ has an element of order $n$.
Show, by example, that the assumption that $G$ is finite is necessary.

The case $n=1$ is trivial, so let $n>1$.  Let $a\in G$ such that $|aH|=n$.  Clearly $a\neq e$.
It follows that the mapping $\phi:H\to H$, given by $\phi(h)=a^nh$ is a non-trivial permutation of
the elements of $H$ and so $\phi$ is a member of the group of permutations of $H$.
We then see that $a^{|\phi|n}=e$.  But it is easy to see that for all
integers $i\in [1,|\phi|n-1]$, we have $a^i\neq e$.  So $|a^{|\phi|}|=n$.

\section*{Exercise 62}

Do it...

\section*{Exercise 65}

If $|G|=30$ and $|Z(G)|=5$, what is the
structure of $G/Z(G)?$.

Note that $|G/Z(G)|=30/5=6=2\cdot 3$.
It follows from Theorem 7.2 that $G/Z(G)$
is isomorphic to $Z_6$ or $D_3$.  Erf...which one?
Think about it.

\section*{Made-up Problems}

\subsection*{Problem 1}

Let $G$ be a group and let $N$ be a normal subgroup of $G$.
Then, if $H_1$ and $H_2$ are subgroups of $G$ containing $N$,
and $H_2/N$ is a subgroup of $H_1/N$, is $H_2$ a subgroup of $H_1$?

It is clear that $H_2$ is a subset of $H_1$.  But this is all we need, right?
We already know that $H_2$ is a group, and that $H_1$ is a group.

\subsection*{Problem 2}

Let $G$ be a group and let $N$ be a normal subgroup of $G$.
If $N$ is a maximal subgroup of $G$, is $G/N$ a cyclic group of prime order?

If true, we would need to show that $G/N$ has no non-trivial
and proper subgroups.  Suppose $G/N$ does have a non-trivial
and proper subgroup.  Call it $K$.  Then $K$ can be written as
$H/N$ where $H$ is a proper subgroup of $G$, right?  (I think so by
Problem 1 above.)
Then, since $N$ is maximal in $G$, we must have $H=N$.
It follows that $K=H/N=N/N=\{N\}$ is a trivial factor subgroup of $G/N$, which is a contradiction.

\subsection*{Problem 3}

How about the converse of the statement given in Problem 2?
Is it true?



\end{document}