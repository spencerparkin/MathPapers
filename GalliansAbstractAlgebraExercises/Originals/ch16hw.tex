\documentclass{article}
\usepackage{amsmath}
\usepackage{amssymb}
\title{Chapter 16 Homework}
\author{Spencer}
\addtolength{\oddsidemargin}{-.575in}
\addtolength{\evensidemargin}{-.575in}
\addtolength{\textwidth}{1.0in}
\addtolength{\topmargin}{-.575in}
\addtolength{\textheight}{1.25in}
\begin{document}
\maketitle

\newcommand{\Z}{\mathbb{Z}}
\newcommand{\R}{\mathbb{R}}
\newcommand{\N}{\mathbb{N}}
\newcommand{\Q}{\mathbb{Q}}
\newcommand{\lcm}{\mbox{lcm}}

\section*{Problem 10}

If the rings $R$ and $S$ are isomorphic, show that $R[x]$ and $S[x]$
are isomorphic.

Let $\phi:R\to S$ be an isomorphism from $R$ to $S$.
Then let $\Phi:R[x]\to S[x]$ be defined as
$\Phi(f(x))=\phi(a_n)x^n+\dots+\phi(a_0)$
where $f(x)=a_nx^n+\dots+a_0$.  The bijectivity of
$\Phi$ follows directly from that of $\phi$, as will as the operation
preserving properties.  For addition we have...
\begin{align*}
\Phi(f(x)+g(x))&=\phi(a_n+b_n)x^n+\dots+\phi(a_0+b_0) \\
 &= (\phi(a_n)+\phi(b_n))x^n+\dots+\phi(a_0)+\phi(b_0) \\
 &= \Phi(f(x))+\Phi(g(x))
\end{align*}
...where $g(x)=b_nx^n+\dots+b_0$.  For multiplication we have...
\begin{equation*}
\Phi(f(x)g(x))=\phi(c_{m+n})x^{m+n}+\dots+\phi(c_0)=\Phi(f(x))\Phi(g(x))
\end{equation*}
...since...
\begin{equation*}
\phi(c_k)=\phi(a_kb_0+\dots+a_0b_k)=\phi(a_k)\phi(b_0)+\dots+\phi(a_0)\phi(b_k)
\end{equation*}

\section*{Problem 16}

Show that Corollary 3 of Theorem 16.2 is false for any commutative ring that
has a zero divisor.

The corollary states that a polynomial of degree $n$ over a field
has at most $n$ zeros.
So let $R$ be a commutative ring with a zero divisor.  Let $a\in R$
be that zero divisor, and find non-zero $b\in R$ such that $ab=0$.
Now choose $bx\in R[x]$.  This has degree 1 while $a$ and $0$
are both zeros of it.

\section*{Problem 18}

Prove that the ideal $\langle x\rangle$ in $\Q[x]$ is maximal.

Consider the factor ring $\Q[x]/\langle x\rangle$.
Let $f(x),g(x)\in\Q[x]$ and recall that
$f(x)+\langle x\rangle=g(x)+\langle x\rangle$ if and only if
$f(x)-g(x)\in\langle x\rangle$, showing that the elements of
$\Q[x]/\langle x\rangle$ are uniquely determined by the
constant term of the coset representative.  In other words,
$f(x)+\langle x\rangle=g(x)+\langle x\rangle$ if and only if
the constant term of $f(x)$ is equal to the constant term of $g(x)$.
It should be
clear now that $\Q[x]/\langle x\rangle\approx\Q$, and since
$\Q$ is a field, so is $\Q[x]/\langle x\rangle$.
It now follows from Theorem 14.4 that $\langle x\rangle$ is maximal.

Alternatively, suppose $I$ is an ideal of $\Q[x]$ that properly
contains $\langle x\rangle$.  Then there must exists a polynomial $f(x)$
in $I$ having a non-zero constant term $q$.  By Theorem 16.2,
there exists $g(x)\in\Q[x]$ such that $f(x)=xg(x)+q$.
Then $q^{-1}f(x)\in I$ since $I$ is an ideal, and clearly $xq^{-1}g(x)\in I$
since $I$ contains $\langle x\rangle$.  It follows that
$1=q^{-1}f(x)-xq^{-1}g(x)\in I$, showing that $I$ absorbs all elements
of $\Q[x]$ and therefore $\Q[x]=I$.  Since the only ideal properly
containing $\langle x\rangle$ is $\Q[x]$, the ideal $\langle x\rangle$
must be maximal in $\Q[x]$.

\section*{Problem 22}

Prove that $\Z[x]$ is not a principle ideal domain.

It suffices to show that the ideal $I=\langle x,2\rangle$ of $\Z[x]$
is not a principle ideal.
Suppose there exists $f(x)\in\Z[x]$ such that $I=\langle f(x)\rangle$.
Then $f(x)$ is a factor or every polynomial in $I$.
Since $2\in I$, we must have $f(x)=1$ or $f(x)=2$.
Suppose the former case.  Then $I=\Z[x]$.  But $I$ is the set of all
polynomials in $\Z[x]$ having an even constant term, so $I\subset\Z[x]$.
By contradiction, we must have $f(x)=2$.  But this doesn't work either
since the non-constant terms of a polynomial in $I$ may be odd.

\section*{Problem 24}

Let $f(x)\in\R[x]$.  Suppose that $f(a)=0$ but $f'(a)\neq 0$.
Show that $a$ is a zero of $f(x)$ of multiplicity 1.

It follows from the hypothesis that $x-a$ is a factor of $f(x)$.  So there exists
$g(x)\in\R[x]$ such that $f(x)=(x-a)g(x)$, showing that
$a$ is a zero of $f(x)$ with a multiplicity of at least 1.
Now write $f'(x)=g(x)+(x-a)g'(x)$ using the product rule.
Then since $g(a)=g(a)+(a-a)g'(a)=f'(a)\neq 0$, we see that $x-a$ is
not a factor of $g(x)$.  It follows that $a$ is a zero of $f(x)$
with a multiplicity of at most 1.  This completes the proof.

\section*{Problem 28}

Let $F$ be a field and let $f(x)=a_nx^n+a_{n-1}x^{n-1}\dots+a_0\in F[x]$.
Prove that $x-1$ is a factor of $f(x)$ if and only if $a_n+a_{n-1}+\dots+a_0=0$.

Dividing $x-1$ into $f(x)$ gives us the following.
\begin{equation*}
\sum_{k=0}^na_kx^k=(x-1)\sum_{i=0}^{n-1}x^i\sum_{j=i+1}^na_j+\sum_{k=0}^na_k
\end{equation*}
It should now be clear that if $x-1$ is a factor of $f(x)$, then
$a_n+a_{n-1}+\dots+a_0=0$, and that if $a_n+a_{n-1}+\dots+a_0=0$,
then $x-1$ is a factor of $f(x)$.  We needed $F$ to be a field in order
to guarentee that the division algorithm gave us a unique quotient and remainder.

\section*{Problem 31}

For every prime $p$, show that $x^{p-1}-1=(x-1)(x-2)\dots(x-(p-1))$ in $\Z_p$.

It suffices to show that $x^{p-1}-1$ is divisible by $x-k$ for all
$k=1,2,\dots,p-1$.  Long division shows us that when we reduce
$x^{p-1}-1$ modulo $x-k$, we get a remainder of $k^{p-1}-1$.
By Fermat's Little Theorem, this is zero in $\Z_p$.

\section*{Problem 32}

For every integer $n>1$, prove that $(n-1)!\equiv n-1\pmod{n}$ if
and only if $n$ is prime.

I cheated and looked up the answer in my number theory text.
Here is an explanation using abstract algebra.
We first show that the only elements of $\Z_n$ that
are there own inverse are $1$ and $n-1$ if $n$ is prime.
If $n$ is prime, then $\Z_n$ is a field and therefore an
integral domain.  So if $z\in\Z_n$
with $z^2=1$, then $(z-1)(z+1)=0$ implies that $z=1$ or $z=n-1$.

Notice that $(n-1)!\equiv n-1\pmod{n}$ when $n=2$.
Now let $n$ be an odd prime and consider the product $(n-2)(n-3)\dots (3)(2)$.
There are an even number of terms in this product and we can break it down
into distinct pairs of inverses since no term is its own inverse and every
term has an inverse.
This shows that the product is unity.  It follows that $(n-1)!=n-1$ in $\Z_n$.

The other direction was not given in my number theory text.
So here is an idea.  If $n=2$, then we're done.  So let $n>2$.
We know that $n$ and $n-1$ have no common factors.
Since $(n-1)!\equiv n-1\pmod{n}$, it follows that $(n-1)!$
and $n$ have no common factors.  This means that $n-k$ does
not divide $n$ for all $k=1,2,\dots,n-2$.  We now see that $n$
is prime by definition.

To make the preceding argument clearer, suppose $a\equiv b\pmod{n}$,
and that $a$ and $n$ have no common divisor.  Now assume there exists an
integer $d$ dividing $b$ and $n$.  Then there is an integer $k$ such that
$a=b+kn$, but then $d|(b+kn)\implies d|a$.  By contradiction, $b$ and $n$
are also relatively prime.

\section*{Problem 34}

Find the remainder upon dividing $98!$ by 101.

Let us first show that for every prime $p$, we have $(p-2)!\equiv 1\pmod{p}$.
By problem 31, $(p-2)!=(p-1)(p^{p-1}-1)=p^p-p-p^{p-1}+1=1$ in $\Z_p$.

So since 101 is prime,
$(99)(98!)=(101-2)!\equiv 1\pmod{101}$, and we see that the
answer to our question is the multiplicative inverse of 99 in $\Z_{101}$.
Fermat's Little Theorem tells us that $99\cdot 99^{101-2}\equiv 1\pmod{101}$,
so the answer is $50=99^{99}$ in $\Z_{101}$.  I needed my calculator to
compute this, unfortunately.  Modular exponentiation can help if no
calculator is available.

\section*{Problem 40}

Prove that $\Q[x]/\langle x^2-2\rangle$ is ring-isomorphic to
$\Q[\sqrt{2}]=\{a+b\sqrt{2}:a,b\in\Q\}$.

Let $I=\langle x^2-2\rangle$ and let $f(x)\in\Q[x]$ and consider the tipical
coset $f(x)+I$ in $\Q[x]/I$.
Since $\Q$ is a field, we may apply the division algorithm to see
that $f(x)+I=(x^2-2)q(x)+r(x)+I$,
where $q(x),r(x)\in\Q[x]$ and $\deg r(x)<2$.
But $(x^2-2)q(x)\in I$, so
$f(x)+I=r(x)+I$, showing
that elements of the factor ring have the form $bx+a+I$,
where $a,b\in\Q$.  Does a pair of rational numbers $(a,b)$ uniquely determine
an element of the factor ring?
Well, if $bx+a+I=dx+c+I$, then
$(b-d)x+a-c\in I$ which implies that $b=d$ and $a=c$, since all elements
of $I$ are divisible by $x^2-2$.

Define the mapping
$\phi:\Q[x]/I\to\Q[\sqrt{2}]$ by
$\phi(f(x)+I)=b\sqrt{2}+a$.  It is clear that this function
is well defined and injective.
Let us show that $\phi$ is surjective.
If we want $\phi(f(x)+I)=b\sqrt{2}+a$, then choose
$f(x)=x^2+bx+a-2$.
Let $g(x)=dx+c$ with $c,d\in\Q$.
We now show that $\phi$ is operation preserving.
\begin{align*}
\phi(f(x)+g(x)+I)
&=\phi((b+d)x+a+c+I)\\
&=(b+d)\sqrt{2}+a+c \\
&=b\sqrt{2}+a+d\sqrt{2}+c \\
&=\phi(f(x)+I)+\phi(g(x)+I)
\end{align*}
Before showing the preservation of multiplication, notice that
$0+I=x^2-2+I\implies
2+I=x^2+I$.  Then...
\begin{align*}
\phi(f(x)g(x)+I) &= \phi((bx+a)(dx+c)+I) \\
 &= \phi(bdx^2+ac+(bc+ad)x+I) \\
 &= \phi((bc+ad)x+ac+2bd+I) \\
 &= (bc+ad)\sqrt{2}+ac+2bd \\
 &= (b\sqrt{2}+a)(d\sqrt{2}+c) \\
 &= \phi(f(x)+I)\phi(g(x)+I)
\end{align*}

\pagebreak
\section*{Problem 42}

Let $F$ be a field and let $I=\{f(x)\in F[x]:\mbox{$f(a)=0$ for all $a\in F$}\}$.
Prove that $I$ is an ideal of $F[x]$.  Prove that $I$ is infinite when $F$ is finite
and $I=\{0\}$ when $F$ is infinite.

Notice that $\{0\}\subseteq I$, so that $I$ is non-empty.
Let $f(x),g(x)\in I$.  Then for all $a\in F$, clearly $f(a)-g(a)=0-0=0$, showing
that $f(x)-g(x)\in I$.  Now let $f(x)\in I$ and $g(x)\in F[x]$.
Then for all $a\in F$, clearly $f(a)g(a)=0g(a)=0$, showing that $I$ absorbs
elements from $F[x]$.  We can now conclude that $I$ is an ideal of $F[x]$.

Let $F=\{a_0,a_1,\dots,a_{n-1}\}$ be a finite field.  Then it's
easy to see that $f(x)\in I$ where
$f(x)=\prod_{k=0}^{n-1}(x-a_k)$.  Further more, $g_k(x)=(f(x))^k$ is a sequence
of distinct polynomials in $I$, showing that $I$ is infinite.

Let $F$ be an infinite field.  Notice that $f(x)=0$ is in $I$, showing that
$I$ is non-empty.  Now let $f(x)\in I$ and suppose $f$ has a degree.
Given any sequence $a_k$
taken from $F$, we may factor $f(x)$ as $g(x)\prod_{k=0}^{n-1}(x-a_k)$, where
$g(x)\in F[x]$ and $n$ is arbitrarily large.
This shows that the degree of $f$ is indeterminant (or infinite), which
contradicts the assumption that $f$ has a degree.  Therefore, there are
no polynomials in $I$ with a degree, and $I=\{0\}$.

\end{document}