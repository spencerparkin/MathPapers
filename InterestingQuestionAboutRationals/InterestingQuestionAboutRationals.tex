\documentclass[12pt]{article}

\usepackage{amsmath}
\usepackage{amssymb}
\usepackage{amsthm}

\newcommand{\Q}{\mathbb{Q}}
\newcommand{\Z}{\mathbb{Z}}
\newcommand{\spn}{\mbox{span}\;}

\title{An Interesting Question About\\The Rationals $\Q$}
\author{Spencer T. Parkin}

\begin{document}
\maketitle

\section*{The Question}

Let $\Q$ denote the set of rational numbers and $\Z$ the integers.
For any set $S$ of rationals numbers, let the span of $S$ be defined as the set
\begin{equation*}
\spn S = \{z_1q_1+z_2q_2+\dots|z_i\in\Z\},
\end{equation*}
where $S=\{q_1,q_2,\dots\}$ is any enumeration of $S$, a finite
or countably infinite set of rational numbers.
If the cardinality of $S$ is greater than 1, say that such a set $S$ is
linearly independent if and only if for all
integers $k>1$, $q_k\not\in\spn\{q_1,q_2,\dots,q_{k-1}\}$.

Now here's the question: is $\Q$ the span of every or any linearly
independent and countably infinite set of rational numbers?

\section*{Motivation For The Question}

This question came about while trying to show that no proper subgroup
of $\Q$ under addition is maximal.  Given any proper subgroup $H$ of $\Q$,
it is easy to find a subgroup $K$ of $\Q$ containing $H$ properly, but it is
not at all obvious that $K$ is a proper subgroup of $\Q$.

Letting $H$ be a proper subgroup of $\Q$, if $q\in\Q-H$, it is easy to show that $H+Z(q)$
properly contains $H$ and is a subgroup of $\Q$, where
$Z(q)=\{zq|z\in\Z\}$ and $H+Z(q)=\{h+q'|\mbox{$h\in H$ and $q'\in Z(q)$}\}$.
Now notice that $q/2\not\in H$ and $q/2\not\in Z(q)$.
But it is not at all clear whether $q/2\in H+Z(q)$.  Suppose it is.
Then there exists $h\in H$ and $z\in\Z$ such that $q/2=h+zq$.
Rearranging, we find that $2h=(1-2z)q$.  Since $1-2z$ is never zero, this would imply that
the intersection of $H$ and $Z(q)$ is a non-trivial group.

Taking a step back for a moment, realize that if $q\in H$,
then $Z(q)$ is a subgroup of $H$.  Now suppose that $q\not\in H$,
but that there exists $q'\in H\cap Z(q)$ where $q'\neq 0$.
Then $Z(q')$ is a subgroup of $H\cap Z(q)$ and it follows that
some multiple of $q$, (namely, $q'=zq$ for some $|z|>1$), is in $H$.

Taking a step back again, can we always find $q\in\Q-H$ such that
$Z(q)\cap H$ is empty?  I have no idea, but if we could always find
such a $q$, then a proof would go through.

\end{document}