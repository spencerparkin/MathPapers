\documentclass{birkjour}

\usepackage{float}

\newtheorem{thm}{Theorem}[section]
\newtheorem{cor}[thm]{Corollary}
\newtheorem{lem}[thm]{Lemma}
\newtheorem{prop}[thm]{Proposition}
\theoremstyle{definition}
\newtheorem{defn}[thm]{Definition}
\theoremstyle{remark}
\newtheorem{rem}[thm]{Remark}
\newtheorem*{ex}{Example}
\numberwithin{equation}{section}

\newcommand{\R}{\mathbb{R}}
\newcommand{\B}{\mathbb{B}}
\newcommand{\G}{\mathbb{G}}
\newcommand{\V}{\mathbb{V}}
\newcommand{\gd}{\dot{g}}
\newcommand{\gh}{\hat{g}}
\newcommand{\Gd}{\dot{G}}
\newcommand{\Gh}{\hat{G}}
\newcommand{\nvai}{\infty}
\newcommand{\nvao}{o}
\newcommand{\grade}{\mbox{grade}}

%\received{}\accepted{}

\begin{document}

\title{On The Expansion Of Algebraic Expressions In Geometric Algebra}

\author{Spencer T. Parkin}
%\address{102 W. 500 S., \\
%Salt Lake City, UT  84101}
\email{spencerparkin@outlook.com}

%\subjclass{Primary ; Secondary }

%\dedicatory{To my dear wife Melinda.}

\begin{abstract}
Abstract goes here...
\end{abstract}

\keywords{Key words go here...}

\maketitle

\section{Introduction}

While the expansion of algebraic expressions taken from, say, a polynomial ring, are found as a trivial matter
of applying the associative and distributive properties, and combining like-terms, it is interesting to note
that this is certainly not true of expressions taken from a geometric algebra.  In this paper, a general
stratagy, or algorithm, if you will, is given for the expansion of such expressions, and it is shown that
it is perhaps just as natural to write an element of a geometric algebra as a sum of ``cursors'' as it is to
write such an element as a sum of blades.  The term ``cursor'' is introduced in Table~\ref{tbl_terms} below,
along with similar, traditional terms found in geometric algebra.

\begin{table}[H]\label{tbl_terms}\caption{Terms used in GA}
\begin{tabular}{p{1cm}p{9cm}}
Term & Definition \\
\hline
Blade & The outer product of zero or more linearly-independent vectors. \\
Versor & The geometric product of zero or more invertible vectors. \\
Cursor & The geometric product of zero or more vectors, not necessarily invertible.
\end{tabular}
\end{table}

From these it is clear that every versor is a cursor, but the converse is not generally true.

The concept of grade will be carried forward in this paper from blades to cursors.  As an $n$-blade
refers to a blade of grade $n$, we will let an $n$-cursor refer to a cursor of grade $n$; that is, a geometric
product of precisely $n$ vectors, none of which need be invertible.  Note that blades of grade zero
are indistinguishable from cursors of grade zero, each denoting the set of all scalars.

\section{Symmetry Between The Outer And Geometric Products}

As will be shown by the various identities established in this section, there is perhaps a lot more in
common between the outer and geometric products than one might think.  Certainly the outer and
inner products play a complementary role in the building up or tearing down of blades, respectively, but from a
purely algebraic perspective, consider the following well-known definition of the geometric product
between two vectors.
\begin{equation}
ab = a\cdot b + a\wedge b
\end{equation}
The right-hand side of equation \eqref{} is a sum of blades, while the left-hand side is a sum of cursors;
in this case, exactly one; namely, $ab$.  Thus, the element $ab$ appears naturally in blade and cursor form,
but what of the element $a\wedge b$?  Rearranging \eqref{}, we simply find that
\begin{equation}
a\wedge b = -a\cdot b + ab,
\end{equation}
showing that it too may be written as a sum of blades or that of cursors.

\subsection{From Inner Product To Sum Of Blades}

Letting $v$ denote a vector, and $B_r$ a blade of grade $r$ having factorization
\begin{equation}
B_r = \bigwedge_{i=1}^r b_i,
\end{equation}
recall that
\begin{align}
v\cdot B_r &= \frac{1}{2}(vB_r-(-1)^rB_rv), \\
v\wedge B_r &= \frac{1}{2}(vB_r+(-1)^rB_rv).
\end{align}
From these it is clear that
\begin{equation}
vB_r = v\cdot B_r + v\wedge B_r.
\end{equation}
Now letting $a,b$ denote vectors, it is not hard to show that
\begin{equation}
a\cdot(b\wedge B_r) - b\cdot(a\wedge B_r) = (a\cdot b)B_r.
\end{equation}
Similarly, it can be shown (prove it) that
\begin{equation}
a\wedge(b\cdot B_r) - b\wedge(a\cdot B_r) = (a\cdot b)B_r.
\end{equation}

Wishing now to express $v\cdot B_r$ as a sum of blades, we will find, using
the princple of mathematical induction, that for all integers $r\geq 1$, we have
\begin{equation}
a\cdot B_r = -\sum_{i=1}^r(-1)^r(a\cdot b_i)\bigwedge_{j=1,j\neq i}^rb_j.
\end{equation}
Notice that this equation holds in the case $r=1$ if we let an empty wedge product
denote the multiplicative identity.  Assuming now our inductive hypothesis, we find that

%(v\cdot B_{r-1})\wedge b_r - (-1)^r(a\cdot b_r)B_{r-1}.
%\end{equation}

\subsection{From Inner Product To Sum Of Cursors}

\section{The Expansion Algorithm}

% The algorithm will be trivial, even if not very efficient in practice.
% For any outer product, we convert its operands to sums of blades, then distribute.
% For any geometric product, we convert its operands to sums of cursors, then distribute.
% For any inner product, we apply an identity as formulated above.

\end{document}