\documentclass{article}
\usepackage{amsmath}
\usepackage{amssymb}
\title{Chapter 10 Homework}
\author{Spencer}
\addtolength{\oddsidemargin}{-.575in}
\addtolength{\evensidemargin}{-.575in}
\addtolength{\textwidth}{1.0in}
\addtolength{\topmargin}{-.575in}
\addtolength{\textheight}{1.25in}
\begin{document}
\maketitle

\section*{Problem 8}

\newcommand{\sgn}{\mbox{sgn}}
\newcommand{\Z}{\mathbb{Z}}
%\newcommand{\ker}{\mbox{Ker}\;}
\newcommand{\lcm}{\mbox{lcm}}

Let $G$ be a group of permutations.  For each $\sigma\in G$, define...
\begin{equation*}
\sgn(\sigma)=\left\{\begin{array}{ll}+1&\mbox{if $\sigma$ is an even permutation,}\\
-1&\mbox{if $\sigma$ is an odd permutation.}\end{array}\right.
\end{equation*}
Prove that $\sgn$ is a homomorphism from $G$ to the multiplicative group $\{+1,-1\}$.
What is the kernel?

Let $\alpha:\Z_2\to\{+1,-1\}$ be the isomorphism defined by $\alpha(0)=1$ and $\alpha(1)=-1$,
and let $\beta:G\to\Z_2$ be a homomorphism defined by...
\begin{equation*}
\beta(\sigma) = \left\{\begin{array}{ll}0&\mbox{if $\sigma$ is an even permutation,}\\
1&\mbox{if $\sigma$ is an odd permutation.}\end{array}\right.
\end{equation*}
Clearly, $\beta(xy)=\beta(x)\beta(y)$ by the concatination of 2-cycles and Theorem 5.5.
Then $\sgn(\sigma)=\alpha\beta(\sigma)$, and we see that
$\sgn(xy)=\alpha\beta(xy)=\alpha(\beta(x)\beta(y))=\alpha\beta(x)\alpha\beta(y)=\sgn(x)\sgn(y)$,
showing that $\sgn$ is the homomorphism we wanted.

The kernel of $\sgn$ is the set of all even permutations of $G$.  We know this is a group
having half the order of $G$.

\section*{Problem 10}

Let $G$ be a subgroup of some dihedral group.  For each $x\in G$, define...
\begin{equation*}
\phi(x)=\left\{\begin{array}{ll}+1&\mbox{if $x$ is a rotation,}\\-1&\mbox{if $x$ is a reflection.}\end{array}\right.
\end{equation*}
Prove that $\phi$ is a homomorphism from $G$ to the multiplicative group $\{+1,-1\}$.
What is the kernel of $\phi$?

Obviously, the composition of two rotations is a rotation.  Define the orientation of an $n$-gon as the
clock-wise or counter-clock-wise ordering of the labelings of its vertices.  Then a rotation
will not change the orientation, but a reflection will.  Therefore, the composition of a
rotation and a reflection, (or vice-versa), must be a reflection, and the composition of
two reflections must be a rotation.
The homomorphism is now easy to see from the following.
\begin{align*}
\phi(RR')=\phi(R'')=+1=(+1)(+1)=\phi(R)\phi(R') \\
\phi(RF)=\phi(F')=-1=(+1)(-1)=\phi(R)\phi(F) \\
\phi(FR)=\phi(F')=-1=(-1)(+1)=\phi(F)\phi(R) \\
\phi(FF')=\phi(R)=+1=(-1)(-1)=\phi(F)\phi(F')
\end{align*}

The kernel of $\phi$ would be the subgroup of all rotations in $G$.

\section*{Problem 12}

Suppose that $k$ is a divisor of $n$.  Prove that $\Z_n/\langle k\rangle\approx\Z_k$.

Define $\phi:\Z_n\to\Z_k$ by $\phi(z)=z\mod k$.
Notice that $\phi(Z_n)=\Z_k$ since $k\leq n$.
Let $x,a,b\in \Z_n$.
Notice that $k$ divides $x-(a+b)$ if $n$ divides $x-(a+b)$ since $k|n$.
Therefore, if $x\equiv a+b\pmod{n}$, then $x\equiv a+b\pmod{k}$.
We then see that $\phi(ab)=\phi(a)\phi(b)$ since $ab\in Z_n$ is congruent modulo $k$ to
$ab\in Z_k$.
This shows that $\phi$ is a homomorphism.  Then since $\ker\phi=\langle k\rangle$,
it follows from Theorem 10.3 that $\Z_n/\langle k\rangle\approx\Z_k$, which is what we wanted.

I believe that cyclic groups of the same order are isomorphic.  So $\langle n/k\rangle\approx\Z_k$.
We then see that $\Z_n/\langle k\rangle\approx\langle n/k\rangle$.
%To back the claim, let $A$ and $B$ be cyclic groups of the same order, and let $a$ and $b$
%be their generators, respectively.  The mapping $a^n\to b^n$, where $n\in\Z$, is then an isomorphism.
%It's clearly one-to-one and onto.  It's operation preserving since $a^ia^j=a^{i+j}\to b^{i+j}=b^ib^j$.

\section*{Problem 14}

Explain why the correspondence $x\to 3x$ from $\Z_{12}$ to $\Z_{10}$ is not a homorphism.

There are many proofs of this, but I think the best is to show that the mapping fails to
be operation preserving.  This must be the case if any mapping is not a homomorphism.
Let $\phi(x)=3x\mod 10$.  Then $\phi((5+7)\mod 12)=\phi(0)=0$, but
$\phi(5)+\phi(7)=(15+21)\mod 10=6$.

We could also assume that it is a homomorphism.  Then $\ker\phi=\{0,10\}$, but $|\Z_{12}|/2\neq|\Z_{10}|$ when
they should be equal since $10<12$ and $\gcd(3,10)=1$ showing that $\phi$ is onto $\Z_{10}$.

\section*{Problem 18}

Can there be a homomorphism from $\Z_4\oplus \Z_4$ onto $\Z_8$?
Can there be a homomorphism from $\Z_{16}$ onto $\Z_2\oplus\Z_2$?

In the first case, no.  There are 4 elements in $\Z_4\oplus \Z_4$ that are their
own inverses.  If we define $\phi((a,b))=z$, then we must define $\phi((4-a,4-b))=8-z$,
since $\phi((0,0))$ must be 0.  The only element in $\Z_8$ that is it's own inverse
is 4.  So we would have to map $(0,0)$, $(0,2)$, $(2,0)$, and $(2,2)$ all to 4,
but this contradicts the need for $\phi$ to be a 2-to-1 mapping, since $(4\cdot 4)/8=2$.

For the second case, no.  Since $\Z_{16}$ is cyclic, a homomorphism that is onto $\Z_2\oplus\Z_2$ must
be defined in terms of a generator of $\Z_2\oplus\Z_2$, since $\phi(x)=\phi(1^x)=\phi(1)^x$.
But $\Z_2\oplus\Z_2$ has no generator, so it's not possible.

\section*{Problem 22}

Suppose that $\phi$ is a homomorphism from a finite group $G$ onto $\bar{G}$ and
that $\bar{G}$ has an element of order $n$.  Prove that $G$ has an element of
order $n$.

Since $\phi$ is onto, there exists $g\in G$ such that $\phi(g)=\bar{g}$
for any $\bar{g}$ in $\bar{G}$.  Let $n$ be the order of $\bar{g}\in\bar{G}$.
Then $n$ divides $|g|$ by property 3 of by Theorem 10.1.
So $\langle g\rangle$ is a finite cyclic subgroup of $G$ of an order that is a multiple of $n$.
It therefore has an element of order $n$ by Theorem 4.3.  Of course, this element has the same
order in $G$.

\section*{Problem 24}

Suppose that $\phi:\Z_{50}\to\Z_{15}$ is a group homomorphism with $\phi(7)=6$.
Determine $\phi(x)$, the image of $\phi$, the kernel of $\phi$, and $\phi^{-1}(3)$.

Notice that $\phi(x)=\phi(1^x)=\phi(1)^x$, showing that $\phi(x)$ is defined in terms
of what it maps 1 to.  Since $6=\phi(7)=\phi(1^7)=\phi(1)^7$, we see that
$7\phi(1)\equiv 6\pmod{15}$.  Since $\gcd(7,15)=1$, $\phi(1)=3$ is the only solution.
So we have $\phi(x)=3x\mod 15$.  Then $\phi(\Z_{50})=\{0,3,6,9,12\}$, since 50 is large enough,
and $\ker\phi=\{z\in\Z_{15}:z\equiv 0\pmod{|\phi(\Z_{50})|}\}=\{0,5,10,\dots,40,45\}$.
Then since $\phi(1)=3$, we have $\phi^{-1}(3)=1+\ker\phi=\{1,4,7,10,13\}$ by property 6 of Theorem 10.1.

\section*{Problem 26}

Determine all homomorphisms from $\Z_4$ to $\Z_2\oplus\Z_2$.

Let us first find all homomorphisms from $Z_n$ to $G$, where $G$ is any group.
Since we require $\phi(x)=\phi(1^x)=\phi(1)^x$, any homomorphism of the desired type is
defined in terms of what it maps 1 to.  Our choices for this are the elements of $G$.
For which elements of $G$ is $\phi$ a homomorphism?
Well, notice that $\phi(ab)=\phi(1)^{ab}=\phi(1)^a\phi(1)^b=\phi(a)\phi(b)$.
So any element of $G$ will work!  Define $\phi_g(x)=g^x$.
Then the set of all homomorphisms is $\{\phi_g:g\in G\}$.

So to answer the original question, the set of all homomorphisms
is $\{\phi_g:g\in\Z_2\oplus\Z_2\}$.
Hopefully I didn't mess up here.

\section*{Problem 40}

If $M$ and $N$ are normal subgroups of $G$
and $N\leq M$, prove that $(G/N)/(M/N)\approx G/M$.

Define the function $\phi:G/N\to G/M$ as $\phi(gN)=gM$, where $g\in G$.
Let $a,b\in G$.
Since $aN=bN\implies a^{-1}b\in N\implies a^{-1}b\in M\implies aM=bM$, we
see that $\phi$ is well defined.  Then
$\phi(aNbN)=\phi(abN)=abM=aMbM=\phi(aN)\phi(bN)$
shows that $\phi$ is operation preserving and therefore a homormophism.
We can then see that $\ker\phi=\phi^{-1}(M)=\{gN:g\in M\}=M/N$, so that
our result now follows from The First Isomorphism Theorem.

\section*{Problem 46}

Suppose that $\Z_{10}$ and $\Z_{15}$ are both homomorphic images of a finite group $G$.
What can we say about $|G|$?

By the corollary of Theorem 10.3, we can say that 10 and 15 divide $|G|$ so that
$|G|$ is at least $\lcm(10,15)=30$.

\section*{Problem 50}

Show that a homomorphism defined on a cyclic group is completely
determined by its action on a generator of the group.

Let $G=\langle g\rangle$ be a cyclic group generated by $g$.  Then if $x\in G$,
then there exists $n\in\Z$ such that $x=g^n$, and $\phi(x)=\phi(g^n)=\phi(g)^n$,
showing that $\phi$ is defined in terms of what it maps the generator to.
Refering back to problem 26, we might go on to find all possible homomorphisms
from a cyclic group to some other group.

\section*{Problem 52}

Let $\alpha$ and $\beta$ be group homomorphisms from $G$ to $\bar{G}$ and let
$H=\{g\in G:\alpha(g)=\beta(g)\}$.  Prove or disprove that $H$ is a subgroup of $G$.

By property 1 of Theorem 10.2, the images of $\alpha$ and $\beta$ are subgroups of $\bar{G}$.
In a previous exercise we proved that the intersection of two subgroups is a subgroup.
So we see that $\alpha(G)\cap\beta(G)$ is a subgroup of $\bar{G}$.
Now notice that $H=\phi^{-1}(\alpha(G)\cap\beta(G))$ is a subgroup of $G$
by property 7 of Theorem 10.2.

Where the hell did you get $\phi$ from?!  Idiot!

\section*{Problem 54}

If $H$ and $K$ are normal subgroups of $G$ and $H\cap K=\{e\}$, prove that $G$
is isomorphic to a subgroup of $G/H\oplus G/K$.

Define $\phi:G\to G/H\oplus G/K$ as $\phi(g)=(gH,gK)$.  Clearly this is a well defined function.
The following shows that $\phi$ is a homomorphism.
\begin{equation*}
\phi(ab) = (abH,abK) = (aHbH,aKbK) = (aH,aK)(aH,bK) = \phi(a)\phi(b)
\end{equation*}
It now follows from property 1 of Theorem 10.2 that the image of $\phi$ is
a subgroup of $G/H\oplus G/K$.  Clearly, $\phi$ is onto its image.  To show
that $\phi$ is an isomorphism from $G$ to its image, we must show that $\phi$ is one-to-one.
Suppose $\phi(a)=\phi(b)$.  Then $(aH,aK)=(bH,bK)$, which implies that $a^{-1}b\in H$
and $a^{-1}b\in K$.  It follows that $a^{-1}b\in H\cap K\implies a^{-1}b=e\implies a=b$.
We can now say that $G\approx\phi(G)\leq G/H\oplus G/K$.

\end{document}
