\documentclass[12pt]{article}

\usepackage{amsmath}
\usepackage{amssymb}
\usepackage{amsthm}
\usepackage{graphicx}
\usepackage{float}

\title{Nailing Down The Directed Integral}
\author{Spencer T. Parkin}

\numberwithin{equation}{section}

\newcommand{\G}{\mathbb{G}}
\newcommand{\V}{\mathbb{V}}
\newcommand{\R}{\mathbb{R}}

\newtheorem{theorem}{Theorem}[section]
\newtheorem{definition}{Definition}[section]
\newtheorem{corollary}{Corollary}[section]
\newtheorem{identity}{Identity}[section]
\newtheorem{lemma}{Lemma}[section]
\newtheorem{result}{Result}[section]

\begin{document}
\maketitle

\section{Motivation}

%Rant here.

\section{Defining The Directed Integral}

We shall let $\R^n$ denote $n$-dimensional euclidean space and let this space be
represented by a vector space that is denoted by the same symbol $\R^n$.
We will let $\G$ denote the geometric algebra that is generated by $\R^n$.
The euclidean metric shall be assumed on $\R^n$,
which, for any pair of vectors (points) $a,b\in\R^n$, may be taken as $|a-b|$.
The whole of $\R^n$ then becomes a metric space under this measure of distance
between points.  As important as it is that $\R^n$ be a metric space, we assume
a metric on the whole of $\G$ that turns it into a metric space.  For any two
multivectors $A,B\in\G$, the norm $|A-B|$ is taken as a measure of the
distance between $A$ and $B$.

We shall assume the usual topology on $\R^n$ for open sets.

\begin{definition}[Tangent Vector]\label{def_tangent_vector}
Given any subset $S$ of $\R^n$ and a point $x\in S$, we call a vector $t\in\R^n$
a tangent vector of $S$ at $x$ if there exists a sequence of points $\{x_i\}_{i=1}^\infty\subseteq S$
such that for any real number $\epsilon>0$, there exists an integer $j>0$ such that
for all $i\geq j$, we have $|x_i-x|<\epsilon$ and
\begin{equation*}
\left|\frac{t}{|t|} - \frac{x_i-x}{|x_i-x|}\right|<\epsilon.
\end{equation*}
\end{definition}

In light of Definition~\ref{def_tangent_vector}, we shall let
$T(x)$ denote the set of all tangent vectors of $S$ at the point $x$.

\begin{definition}[Surface]\label{def_surface}
A subset $S$ of $\R^n$ is a $k$-dimensional surface if for all points $x\in S$ that
are, under the subspace topology of $S$, interior to $S$, the set
$T(x)\cup\{0\}$ is a vector space of dimension $k$.
\end{definition}

% Does this definition need to be recursive so that the boundry points
% form a (k-1)-dim surface, and the boundry points of that surf form
% a (k-2)-dim surface, and so on?

% If all boundry points of S have a tangent space of dimension k, then the k-dimensional
% surface is a closed surface?

With Definition~\ref{def_surface} in place, it is easy to imagine examples of surfaces in $\R^n$,
such as a hollow sphere or plane,
although the typical surface may not really be anything like what we would or could imagine.

Given a surface $S\subseteq\R^n$, we will, for any point $x\in S$, let $G(x)$ denote the geometric
algebra generated by the tangent space $T(x)$ at $x$.

If $S$ is an orientable surface, then there exists a function $v:S\to\G$ giving, for
each point $x\in S$, a consistent unit psuedo-scalar for the tangent algebra $G(x)$.
The unit psuedo-scalar $v(x)$ is referred to as the tangent of $S$ at $x$, while its
principle dual, the normal of $S$ at $x$.

\begin{definition}[Minimal Surface Covering]\label{def_surface_covering}
Given a surface $S$, a minimal surface covering of $S$ of radius $r$ is a set $C$ of least
possible cardinality of open
balls centered on points of $S$, each of radius $r$, with the property that for any point
$x\in S$, there exists an open ball $b\in C$ such that $x\in b$.
\end{definition}

Letting $\mbox{ball}(x,r)$ denote an open ball of radius $r$
centered at a point $x$, notice that if a surface $S$ is compact, then, by the Heine-Borel property,
(see \cite{}), we can always take the covering $\{\mbox{ball}(x,r)|x\in S\}$ and
reduce it to a finite sub-cover.  That is, find a finite subset of this cover that is
also a cover of $S$.  A minimal surface cover of $S$ is then a cover of this form
of smallest possible cardinality.

If $C$ is a minimal surface covering of $S$, then we are going to let $C'$ denote
the set of open ball centers of all open balls in $C$.

\begin{definition}[Directed Integral]\label{def_directed_integral}
Let $S$ be a compact surface upon which is defined a multivector field $f$.
Then the directed integral of $f$ over $S$, if it exists, is a multivector $L\in\G$, and we write
\begin{equation*}
L = \int_S dv f(x),
\end{equation*}
if for every real number $\epsilon>0$, there exists a real number $\delta>0$ such
that if $C$ is a minimal surface covering of $S$ of radius $r<\delta$, then
\begin{equation*}
\left|L - \sum_{x\in C'} r v(x)f(x)\right|<\epsilon.
\end{equation*}
\end{definition}

We will characterize the set of all integrable functions on a compact surface $S$ as those defined
on such a surface, and for which
the integral of Definition~\ref{def_directed_integral} exists over that surface.

\begin{lemma}
The directed integral of Definition~\ref{def_directed_integral}, as a function, is well defined.
\end{lemma}
\begin{proof}
Letting $f$ be an integrable function on $S$, we must show here that there
are no two multivectors $L_0\neq L_1$ of $\G$ that are both integrals of $f$ over $S$.
To that end, we begin by letting $D=|L_0-L_1|$, choose $\epsilon = \frac{D}{2}$,
and define the function
\begin{equation*}
F(C) = \sum_{x\in C'}rv(x)f(x).
\end{equation*}
Since $L_0$ is an integral of $f$ over $S$, there exists $\delta_0>0$ such that
if $C$ is a minimal surface covering of $S$ of radius $r<\delta_0$, we have $|L_0-F(C)|<\epsilon$.
Similarly, since $L_1$ is an integral of $f$ over $S$, there exists $\delta_1>0$ such that
if $C$ is a minimal surface covering of $S$ of radius $r<\delta_1$, we have $|L_1-F(C)|<\epsilon$.
Now letting $\delta=\mbox{min}\{\delta_0,\delta_1\}$, we see that if $C$ is a minimal surface
covering of $S$ of radius $r<\delta$, we have $|L_0-F(C)|<\epsilon$ and $|L_1-F(C)|<\epsilon$.
We then see that
\begin{equation*}
D = |L_0-L_1|\leq |L_0-F(C)|+|F(C)-L_1|<2\epsilon = D,
\end{equation*}
which is an impossibility.  Having reached this contradiction, we can conclude that there
does not exist a pair of multivectors $L_0\neq L_1$ that are both integrals of $f$ over $S$.
\end{proof}

Having defined our integral only over compact surfaces, notice that we can
integrate over hollow spheres, but not planes.  Also note that not all closed
and bounded subsets of $\R^n$ are compact.

At this point, we should also try to find some sort of geometric interpretation
of the directed integral.  The most fundamental interpretation, perhaps,
is that of surface area (or length or volume, etc.) when integrating the
function $f(x)=1$ over any compact surface, but it is not entirely obvious
that what we get from the integral, as we have defined it, is such a result
in that case.
\begin{lemma}
The directed integral of $f(x)=1$ over a compact $k$-dimensional surface $S$
is the $k$-dimensional hyper-volume of the surface.
\end{lemma}
\begin{proof}
How could this be proven?
\end{proof}

% I am not at all convinced of Hestenes claim that the integral over
% a closed surface of f(x)=1 is 0.  In some cases, the directed
% volume elements do not come in pairs to cancel one another out.

\section{Using The Directed Integral}

The directed integral becomes useful to us when we can find a relationship between it
and an anti-derivative of the function it integrates.  Without this, there is no clear way
to evaluate the integral for a given integrable function.  The goal of this section, therefore,
is to find such a relationship.

\begin{definition}[Surface Boundry]\label{def_surface_boundry}
Assuming the subspace topology on a given surface $S$, the set of
all boundry points of $S$, denoted $\partial S$, is the set of all points
in $S$ that are not interior to $S$.
\end{definition}

Recall that if $x\in S$ is interior to $S$, then there exists an open
set $O$ such that $x\in O\subset S$.

\begin{lemma}
Given a $k$-dimensional surface $S$, the set of points $\partial S$, as given by
Definition~\ref{def_surface_boundry}, form a $(k-1)$-dimensional surface by
Definition~\ref{def_surface}.
\end{lemma}

\begin{lemma}
If $S$ is a compact surface, then $\partial S$ is also compact.
\end{lemma}

% What is the vector derivative?!?!  It is not at all obvious to me.

If $f$ is an integrable function over a surface $S$, what function $g$ is there, if any, such that
\begin{equation*}
\int_S dv f(x) = \int_{\partial S} dv g(x),
\end{equation*}
and what relationship, if any, might it have with $f$?

\end{document}