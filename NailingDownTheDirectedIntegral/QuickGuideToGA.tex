\documentclass[12pt]{article}

\usepackage{amsmath}
\usepackage{amssymb}
\usepackage{amsthm}
\usepackage{graphicx}
\usepackage{float}

\title{Nailing Down The Directed Integral}
\author{Spencer T. Parkin}

\numberwithin{equation}{section}

\newcommand{\G}{\mathbb{G}}
\newcommand{\V}{\mathbb{V}}
\newcommand{\R}{\mathbb{R}}

\newtheorem{theorem}{Theorem}[section]
\newtheorem{definition}{Definition}[section]
\newtheorem{corollary}{Corollary}[section]
\newtheorem{identity}{Identity}[section]
\newtheorem{lemma}{Lemma}[section]
\newtheorem{result}{Result}[section]

\begin{document}
\maketitle

\section{Motivation}

%Rant here.

\section{Definitions}

We shall let $\R^n$ denote $n$-dimensional euclidean space and let this space be
represented by a vector space that is denoted by the same symbol $\R^n$.
We will let $\G$ denote the geometric algebra that is generated by $\R^n$.
To add structure to our space $\R^n$, we shall assume the euclidean metric
which, for any pair of vectors (points) $a,b\in\R^n$, may be taken as $|a-b|$.
It can then be shown that $\R^n$ is a metric space.  To add further structure,
we shall assume the usual topology on $\R^n$ for open sets.

\begin{definition}[Tangent Vector]\label{def_tangent_vector}
Given any subset $S$ of $\R^n$ and a point $x\in S$, we call a vector $t\in\R^n$
a tangent vector of $S$ at $x$ if there exists a sequence of points $\{x_i\}_{i=1}^\infty\subseteq S$
such that for any real number $\epsilon>0$, there exists an integer $j>0$ such that
for all $i\geq j$, we have $|x_i-x|<\epsilon$ and
\begin{equation*}
\left|\frac{t}{|t|} - \frac{x_i-x}{|x_i-x|}\right|<\epsilon.
\end{equation*}
\end{definition}

In light of Definition~\ref{def_tangent_vector}, we shall let
$T(x)$ denote the set of all tangent vectors of $S$ at the point $x$.

\begin{definition}[Surface]\label{def_surface}
A subset $S$ of $\R^n$ is a $k$-dimensional surface if for all points $x\in S$, the set
$T(x)\cup\{0\}$ is a vector space of dimension $k$.
\end{definition}

With Definition~\ref{def_surface} in place, it is easy to imagine examples of surfaces in $\R^n$,
such as a hollow sphere or plane,
although the typical surface may not really be anything like what we would or could imagine.

Given a surface $S\subseteq\R^n$, we will, for any point $x\in S$, let $G(x)$ denote the geometric
algebra generated by the tangent space $T(x)$ at $x$.

If $S$ is an orientable surface, then there exists a function $v:S\to\G$ giving, for
each point $x\in S$, a consistent unit psuedo-scalar for the tangent algebra $G(x)$.
The unit psuedo-scalar $v(x)$ is referred to as the tangent of $S$ at $x$, while its
principle dual, the normal of $S$ at $x$.

\begin{definition}[Surface Covering]\label{def_surface_covering}
Given a surface $S$, a surface covering of $S$ of radius $r$ is a set $C$ of open
balls centered on points of $S$, each of radius $r$, with the property that for any point
$x\in S$, there exists an open ball $b\in C$ such that $x\in b$.
\end{definition}

Letting $\mbox{ball}(x,r)$ denote an open ball of radius $r$
centered at a point $x$, notice that if a surface $S$ is compact, then, by the Heine-Borel property,
(see \cite{}), we can always take the surface covering $\{\mbox{ball}(x,r)|x\in S\}$ and
reduce it to a finite sub-cover.  That is, find a finite subset of this cover that is
also a surface cover of $S$.

If $C$ is a surface covering of $S$, then we are going to let $C'$ denote
the set of open ball centers of all open balls in $C$.

\begin{definition}[Directed Integral]\label{def_directed_integral}
Let $S$ be a compact surface upon which is defined a multivector field $f$.
Then the directed integral of $f$ over $S$, if it exists, is an element $L\in\G$, and we write
\begin{equation*}
L = \int_S dv f(x),
\end{equation*}
if for every real number $\epsilon>0$, there exists a real number $\delta>0$ such
that if $C$ is a finite surface covering of $S$ of radius $r<\delta$, then
\begin{equation*}
\left|L - \sum_{x\in C'} r v(x)f(x)\right|<\epsilon.
\end{equation*}
\end{definition}

We will characterize the set of all integrable functions on a compact surface $S$ as those defined
on such a surface, and for which
the integral of Definition~\ref{def_directed_integral} exists over that surface.

\begin{lemma}
The directed integral of Definition~\ref{def_directed_integral}, as a function, is well defined.
\end{lemma}
\begin{proof}
Letting $f$ be an integrable function on $S$, we must show here that there
are no two multivectors $L_0\neq L_1$ of $\G$ that are both integrals of $f$ over $S$.
To that end, let $D=|L_0-L_1|$ and choose $\epsilon = \frac{D}{2}$.  (Use triangle inequality,
if you can make it work.)
\end{proof}

\end{document}