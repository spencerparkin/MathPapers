\documentclass[12pt]{article}

\usepackage{amsmath}
\usepackage{amssymb}
\usepackage{amsthm}

\title{Chapter 14 Exercises\\Gallian's Book on Abstract Algebra}
\author{Spencer T. Parkin}

\newtheorem{theorem}{Theorem}[section]
\newtheorem{definition}{Definition}[section]
\newtheorem{corollary}{Corollary}[section]
\newtheorem{identity}{Identity}[section]
\newtheorem{lemma}{Lemma}[section]
\newtheorem{result}{Result}[section]

%\newcommand{\gcd}{\mbox{gcd}}
\newcommand{\lcm}{\mbox{lcm}}
\newcommand{\abs}{\mbox{abs}}
\newcommand{\Z}{\mathbb{Z}}
\newcommand{\R}{\mathbb{R}}
\newcommand{\G}{\mathbb{G}}
\newcommand{\stab}{\mbox{stab}}
\newcommand{\aut}{\mbox{Aut}}
\newcommand{\inn}{\mbox{Inn}}
\newcommand{\orb}{\mbox{orb}}
\newcommand{\chr}{\mbox{char}\,}

\begin{document}
\maketitle

\section*{Understanding Example 7}

Let $R$ be the ring of all real-valued functions of a real variable.
The subset $S$ of all differentiable functions is a subgring of $R$
but not an ideal of $R$.

Let $f$ be any real-valued function of a real variable that is not differential and let $g$
be such a function that is differentiable.  Now notice that the
function $h(x)=f(x)g(x)$ is not differentiable.

\section*{Understanding Example 15}

If it can be shown that $A$ contains a non-zero real number $c$, then,
by virtue of being an ideal, it absorbs all elements of $R[x]$,
so all $f(x)c$ with $f(x)\in R[x]$ is all of $R[x]$, showing that $A=R[x]$.

\section*{Exercise 3}

Verify that the set $I$ in Example 5 is an ideal and that if $J$ is any ideal
of $R$ that contains $a_1,a_2,\dots,a_n$, then $I\subseteq J$.
(Hence, $\langle a_1,a_2,\dots,a_n\rangle$ is the smallest ideal of $R$
that contains $a_1,a_2,\dots,a_n$.)

For reference, note that
\begin{equation*}
I=\langle a_1,a_2,\dots,a_n\rangle=\{r_1a_1+r_2a_2+\dots+r_na_n|r_i\in R\}.
\end{equation*}

Clearly $0\in I$.  If $x,y\in I$, then for elements $x_1,x_2,\dots,x_n\in R$
and elements $y_1,y_2,\dots,y_n\in R$, we have
\begin{align*}
x-y&=x_1a_1+x_2a_2+\dots+x_na_n-(y_1a_1+y_2a_2+\dots+y_na_n) \\
 &= (x_1-y_1)a_1+(x_2-y_2)a_2+\dots+(x_n-y_n)a_n \in I,
\end{align*}
since each $x_i-y_i\in R$.  Letting $r\in R$, we have
\begin{equation*}
rx = rx_1a_1+rx_2a_2+\dots+rx_na_n\in I,
\end{equation*}
since each $rx_i\in R$.

Now let $J$ be an ideal of $R$ containing $a_1,a_2,\dots,a_n$,
and let $x$ be any element of $I$.  As before, let $x=x_1a_1+x_2a_2+\dots+x_na_n$.
Now by the definition of what an ideal is, it is clear that each $x_ia_i\in J$, because
$x_i\in R$ and $a_i\in J$.  Furthermore, $x\in J$, because each $x_ia_i\in J$ and $J$ is a group.

\section*{Exercise 7}

Let $a$ belong to a commutative ring $R$.  Show that $aR=\{ar|r\in R\}$
is an ideal of $R$.  If $R$ is the ring of even integers, list the elements of $4R$.

Clearly the additive identity is in $aR$, since $0=a\cdot 0$.
Let $x,y\in aR$.  Then there exist elements $r_x,r_y\in R$ such that $x=ar_x$ and $y=ar_y$.
We then have $x-y=ar_x-ar_y=a(r_x-r_y)\in aR$ since $r_x-r_y\in R$.
Now let $r\in R$ and see that $rx=rar_x=arr_x\in aR$ since $rr_x\in R$.
It follows that $aR$ is an ideal of $R$.

In that case, $4R=\{0, \pm 8, \pm 16, \pm 24, \pm 32, \dots\}$, I think.

\section*{Exercise 9}

If $n$ is an integer greater than 1, show that $\langle n\rangle=nZ$ is a prime ideal of $Z$
if and only if $n$ is prime.

Notice that $nZ$ is an ideal of $Z$ by Exercise 7.

Suppose $n$ is prime.  Let $a,b\in Z$ such that $ab\in nZ$.
Then there exists $z\in Z$ such that $ab=nz$.  It follows
that $n|ab$ which implies that $n|a$ or $n|b$ by Euclid's Lemma.
So there exists $z'\in Z$ such that $a=nz'\in nZ$ or $b=nz'\in nZ$,
showing that $nZ$ is a prime ideal of $Z$.

Now suppose $nZ$ is a prime ideal of $Z$.
Then if $a,b\in Z$ such that $ab=nz$ for some $z\in Z$,
we must have, for some $z'\in Z$, $a=nz'$ or $b=nz'$.
In other words, if $n|ab$, we must have $n|a$ or $n|b$
in every case.  There is no composite number that can do this,
so $n$ must be prime.  (We can also conclude $n$ is prime
by continually factoring what $n$ divides, and then know that $n$
divides one of the factors.  Repeating, we're eventually left with only one prime factor.)

\section*{Exercise 10}

If $A$ and $B$ are ideals of a ring $R$, show that the sum of $A$ and $B$,
$A+B=\{a+b|a\in A,b\in B\}$, is an ideal.

Notice that $0\in A+B$.  Then, if $x,y\in A+B$, then there exist
elements $a_x,a_y\in A$ and $b_x,b_y\in B$ such that
$x=a_x+b_x$ and $y=a_y+b_y$.  Then $x-y=a_x-a_y+b_x-b_y\in A+B$
since $a_x-a_y\in A$ and $b_x-b_y\in B$.  Now letting $r\in R$,
we see that $rx=ra_x+rb_x\in A+B$, since each one of $A$ and $B$
is a left ideal so that $ra_x\in A$ and $rb_x\in B$.
Similarly, $xr=a_xr+b_xr\in A+B$, since each one of $A$ and $B$
is a right ideal so that $a_xr\in A$ and $b_xr\in B$.  We can now claim
that $A+B$ is an ideal by Theorem 14.1.

\section*{Exercise 12}

If $A$ and $B$ are ideals of a ring $R$, show that the product of $A$ and $B$,
$AB=\{a_1b_1+a_2b_2+\dots+a_nb_n|a_i\in A, b_i\in B,\mbox{$n$ a positive integer}\}$,
is an ideal.

Clearly $0\in AB$.  Let $x,y\in AB$.  Then there exist elements $a_i,u_i\in A$
and $b_i,v_i\in B$ such that $x=a_1b_1+\dots+a_mb_m$ and
$y=u_1v_1+\dots+u_nv_n$ for positive integers $m$ and $n$.
Now realize that $x-y\in AB$, since $-u_i\in A$ and $m+n$ is a positive integer.
Now for any $r\in R$, notice that $rx\in AB$ since $A$ is an ideal, and
$xr\in AB$ since $B$ is an ideal.  So $AB$ is an ideal by the Ideal Test
(Theorem 14.1).

\section*{Exercise 14}

Let $A$ and $B$ be ideals of a ring.  Prove that $AB\subseteq A\cap B$.

Let $x\in AB$.  Then for $a_i\in A$, $b_i\in B$ and a positive integer $n$,
we have $x=a_1b_1+\dots+a_nb_n$.  Now notice that for each integer
$i$, $a_ib_i\in A$ since $A$ is an ideal, and $a_ib_i\in B$ since $B$ is an ideal.
Then since $A$ and $B$ are groups, $x\in A\cap B$.

\section*{Exercise 15}

If $A$ is an ideal of a ring $R$ and 1 belongs to $A$, prove that $A=R$.

Since $A$ is an ideal of $R$ and $1\in A$, we have, for all $r\in R$, $r=1r\in A$,
showing that $A=R$.

\section*{Exercise 21}

Verify the claim made in Example 10 about the size of $R/I$.

For reference,
\begin{equation*}
R=\left\{\left.\left[\begin{array}{cc}a_1&a_2\\a_3&a_4\end{array}\right]\right|a_i\in Z\right\}
\end{equation*}
and $I$ is the subset of $R$ consisting of matrices with even entries.

The example in the text helps make the verification easy.  Let $M\in R$.
Then the coset $M+I=B+I$, where $B$ is a matrix consisting of just ones
and zeros.  Since the matrices have 4 possible entries, there are $2^4=16$
possible elements in $R/I$.

\section*{Exercise 23}

Show that the set $B$ in the latter half of the proof of Theorem 14.4 is an
ideal of $R$.

For reference, $B=\{br+c|r\in R, a\in A\}$ with $b\in R-A$.  The subset $A$ is an ideal
of $R$, and $R$ is a commutative ring with unity.

Letting $x\in R$, we must show that for any $y\in B$, that $xy\in B$ and $yx\in B$.
Let $y=br+a$ for elements $r\in R$ and $a\in A$.
Then $xy=bxr+xa\in B$ since $xr\in R$ and $xa\in A$.  (Remember that $A$ is
an ideal of $R$.)  And we have $yx=bry+ay\in B$ since $ry\in R$ and $ay\in A$.

\section*{Exercise 30}

Let $R=Z_8\oplus Z_{30}$.  Find all maximal ideals of $R$, and for each
maximal ideal $I$, identify the size of the field $R/I$.

Let's start by looking for all maximal subgroups of $R$.
By inspection, but mostly by reasoning, I believe these are $2Z_8\oplus Z_{30}$, $Z_8\oplus 2Z_{30}$,
$Z_8\oplus 3Z_{30}$ and $Z_8\oplus 5Z_{30}$.  Which of these
are ideals?  In part, Exercise 7 can be used to say that all of them are, so
they're all maximal ideals.
(The other part is realizing that the other cyclic group taken in the
product is just ``along for the ride'' when performing the ideal test.)
Being maximal, the factor rings generated by these ideas are fields
of orders $8/2\cdot 30$, $8\cdot 30/2$, $8\cdot 30/3$ and $8\cdot 30/5$,
respectively.  Have I missed something?  Alas, no one but me will ever view this.

\section*{Exercise 34}

Let $R$ be a ring and let $I$ be an ideal of $R$.  Prove that the
factor ring $R/I$ is commutative if and only if $rs-sr\in I$ for all $r$ and $s$ in $R$.

For any two elements $a,b\in R$, two elements of $R/I$ are $a+I$ and $b+I$.
Now realize that $ab+I=(a+I)(b+I)=(b+I)(a+I)=ba+I$ if and only if $ab-ba\in I$
by Property 4 on Page 138, (6th Ed.)

\section*{Exercise 38}

Let $R$ be a ring and let $p$ be a fixed prime.  Show that
$I_p=\{r\in R|$ additive order of $r$ is a power of $p\}$ is an
ideal of $R$.

Notice that the additive identity is in $I_p$.  Now for any pair
of elements $x,y\in I_p$, let them have additive orders $p^m$
and $p^n$, respectively.  Now notice that if $k=mn$,
then $p^k\cdot (x-y)=p^k\cdot x-p^k\cdot y=0$, showing
that the additive order of $x-y$ divides $p^k$ by Corollary 2
of Theorem 4.1.  So since $p$ is prime, we have $x-y\in I_p$.
Similarly, letting $r\in R$ be any element,
notice that $p^m\cdot xr=(p^m\cdot x)r=0\cdot r=0$,
showing that the additive order of $xr$ divides $p^m$.
Also, $p^m\cdot rx=r(p^m\cdot x)=r\cdot 0=0$, showing
that the additive order of $rx$ divides $p^m$ also.
So $xr$ and $rx$ are in $I_p$.

\end{document}