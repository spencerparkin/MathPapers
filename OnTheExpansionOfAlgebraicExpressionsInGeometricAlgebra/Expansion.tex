\documentclass{birkjour}

\usepackage{float}

\newtheorem{thm}{Theorem}[section]
\newtheorem{cor}[thm]{Corollary}
\newtheorem{lem}[thm]{Lemma}
\newtheorem{prop}[thm]{Proposition}
\theoremstyle{definition}
\newtheorem{defn}[thm]{Definition}
\theoremstyle{remark}
\newtheorem{rem}[thm]{Remark}
\newtheorem*{ex}{Example}
\numberwithin{equation}{section}

\newcommand{\R}{\mathbb{R}}
\newcommand{\B}{\mathbb{B}}
\newcommand{\G}{\mathbb{G}}
\newcommand{\V}{\mathbb{V}}
\newcommand{\gd}{\dot{g}}
\newcommand{\gh}{\hat{g}}
\newcommand{\Gd}{\dot{G}}
\newcommand{\Gh}{\hat{G}}
\newcommand{\nvai}{\infty}
\newcommand{\nvao}{o}
\newcommand{\grade}{\mbox{grade}}

%\received{}\accepted{}

\begin{document}

\title{On The Expansion Of Algebraic Expressions In Geometric Algebra}

\author{Spencer T. Parkin}
%\address{102 W. 500 S., \\
%Salt Lake City, UT  84101}
\email{spencerparkin@outlook.com}

%\subjclass{Primary ; Secondary }

%\dedicatory{To my dear wife Melinda.}

\begin{abstract}
Abstract goes here...
\end{abstract}

\keywords{Key words go here...}

\maketitle

\section{Introduction}

While the expansion of algebraic expressions taken from, say, a polynomial ring, are found as a trivial matter
of applying the associative and distributive properties, and combining like-terms, it is interesting to note
that this is certainly not true of expressions taken from a geometric algebra.  In this paper, a general
stratagy, or algorithm, if you will, is given for the expansion of such expressions, and it is shown that
it is perhaps just as natural to write an element of a geometric algebra as a sum of ``mercers'' as it is to
write such an element as a sum of blades.  The term ``mercer'' is introduced in Table~\ref{tbl_terms} below,
along with similar, traditional terms found in geometric algebra.\footnote{The term ``versor'' was avoided in
this paper in favour of ``mercer'' as a matter of rigour.  Not knowing a term for the algebraic form in question, and not finding
one in the literature, one was made up.}

\begin{table}[H]\label{tbl_terms}\caption{Terms used in GA}
\begin{tabular}{p{1cm}p{9cm}}
Term & Definition \\
\hline
Blade & The outer product of zero or more linearly-independent vectors. \\
Versor & The geometric product of zero or more invertible vectors, not necessarily forming a linearly-independent set. \\
Mercer & The geometric product of zero or more vectors, not necessarily invertible and not necessarily forming a linearly-independent set.
\end{tabular}
\end{table}

From these it is clear that every versor is a mercer, but not every mercer is a versor.

Similar to the concept of grade, that of rank will be introduced in this paper with respect to mercers.  As an $n$-blade
refers to a blade of grade $n$, we will let an $n$-mercer refer to a mercer of rank $n$; that is, a geometric
product of precisely $n$ vectors.  Note that blades of grade zero
are indistinguishable from mercers of the same rank as each denotes the set of all scalars.

Unlike versors, note that mercers do not form a group by simple reason that not every
mercer is invertible with respect to the geometric product.  They are important to study, none-the-less,
because they appear more often in consideration of the typical expression taken from a geometric algebra.

\section{Symmetry Between The Outer And Geometric Products}

As will be shown by the results established in this section, there is perhaps a lot more in
common between the outer and geometric products than one might think.  Certainly the outer and
inner products play a complementary role in the building up or tearing down of blades, respectively, but from a
purely algebraic perspective, consider the following well-known definition of the geometric product
between two vectors $a$ and $b$.
\begin{equation}\label{equ_ab_is_a_dot_b_and_a_wedge_b}
ab = a\cdot b + a\wedge b
\end{equation}
The right-hand side of equation \eqref{equ_ab_is_a_dot_b_and_a_wedge_b} is a sum of blades, while the left-hand side is a sum of mercers;
in this case, exactly one; namely, $ab$.  Thus, the element $ab$ appears naturally in a sum-of-blades and
that-of-mercers form, but what of the element $a\wedge b$?  Rearranging \eqref{equ_ab_is_a_dot_b_and_a_wedge_b}, we simply find that
\begin{equation}
a\wedge b = -a\cdot b + ab,
\end{equation}
showing that it too may be written as a sum of blades or that of mercers.  In fine, one aim of this paper
is to show that while every element has a sum-of-blades form, it too has a sum-of-mercers form.

\subsection{From Inner Product To Sum Of Blades}

Letting $a$ denote a vector and $B_r$ a blade of grade $r$ having the factorization
given in equation \eqref{equ_B_r}, we wish here to express the inner product $a\cdot B_r$ as a sum of blades.
Since the case $r=1$ is trivial, we begin by writing, for all $r>1$,
\begin{align}
a\cdot B_r
 &= a\cdot(B_{r-1}\wedge b_r)\nonumber \\
 &= (-1)^{r-1}a\cdot(b_r\wedge B_{r-1})\label{a_dot_Br_stepA} \\
 &= -(-1)^r\left(-b_r\wedge(a\cdot B_{r-1})+(a\cdot b_r)B_{r-1}\right)\label{a_dot_Br_stepB} \\
 &= -(-1)^r\left(-(-1)^r(a\cdot B_{r-1})\wedge b_r+(a\cdot b_r)B_{r-1}\right)\nonumber \\
 &= (a\cdot B_{r-1})\wedge b_r - (-1)^r(a\cdot b_r)B_{r-1}.\label{equ_a_dot_Br_recursive}
\end{align}
Here, we've gone from equation \eqref{a_dot_Br_stepA} to that of \eqref{a_dot_Br_stepB} by
applying the identity given in equation \eqref{equ_a_dot_b_Br_identity}.

Applied recursively, it is easy to see here from equation \eqref{equ_a_dot_Br_recursive} that the expansion of
$a\cdot B_r$ as a sum of blades is given by
\begin{equation}\label{equ_a_dot_Br_sum_of_blades}
a\cdot B_r = -\sum_{i=1}^r(-1)^i(a\cdot b_i)\bigwedge_{j=1,j\neq i}^r b_j.
\end{equation}
One might also simply use equation \eqref{equ_a_dot_Br_recursive} to given an inductive
argument of equation \eqref{equ_a_dot_Br_sum_of_blades}.

% We're not done here.  We need to address (a^b^c^...).B_r = (a.(b.(c. ... B_r)...)).

\subsection{From Inner Product To Sum Of Mercers}

Letting $a$ denote a vector and $M_r$ a mercer of rank $r$ having the factorization
given in equation \eqref{equ_M_r},
we wish here to express the inner product $v\cdot M_r$ as a sum of mercers.

\section{The Expansion Algorithm}

% Although, with the exception of the inner-product, the products of GA are generally associative,
% we can impose an arbitrary order on each operation so that any expression becomes a binary tree.

% Note that one of the hurtles in learning GA that the new-comer has is often being confronted
% with expression for which he or she is at a loss to deal with.  Until one builds up a repretwar of
% of useful identities and propreties of the various products in various contexts, ...

% Here we also prove that every element has a sum-of-mercers form through the correctness of the algorithm.

% The algorithm will be trivial, even if not very efficient in practice.
% For any outer product, we convert its operands to sums of blades, then distribute.
% For any geometric product, we convert its operands to sums of mercers, then distribute.
% For any inner product, we apply an identity as formulated above.

% The tree just collapses where at each stage, the leaf nodes are either sums of mersors or sums of blades.
% Since the initial leaf nodes are clearly one or the other, and we can always collaps the tree down to a
% single node, this shows that every element can be written in one form or the other.  Any element can
% be written in any of the two forms, because we can convert from one form to the other, and this operation
% is also a basis for the collapsing algorithm.

% Write psuedo-code?  Reduce to binary tree case.

% It might be noted that the reverse can be now fully justifiably applied to any element of GA.

\section{Appendix Of Identities}

Identities used in this paper are thrown into this appendix so as not to encumber the main body of the paper.

\subsection{Identities Involving Blades}

Letting $a$ denote a vector, and $B_r$ a blade of grade $r$ having factorization
\begin{equation}\label{equ_B_r}
B_r = \bigwedge_{i=1}^r b_i,
\end{equation}
recall that
\begin{equation}
aB_r = a\cdot B_r + a\wedge B_r.
\end{equation}
Recalling also the commutativities of $a$ with $B_r$ in the inner and outer products as
\begin{align}
a\cdot B_r &= -(-1)^r B_r\cdot a,\label{equ_a_dot_Br_commutativity} \\
a\wedge B_r &= (-1)^r B_r\wedge a,\label{equ_a_wedge_Br_commutativity}
\end{align}
we find that
\begin{align}
a\cdot B_r &= \frac{1}{2}a\cdot B_r - \frac{1}{2}(-1)^r B_r\cdot a\nonumber \\
 &= \frac{1}{2}(aB_r - a\wedge B_r - (-1)^r(B_ra - B_r\wedge a))\nonumber \\
 &= \frac{1}{2}(aB_r-(-1)^rB_ra),\label{equ_a_dot_Br}
\end{align}
and that
\begin{align}
a\wedge B_r &= \frac{1}{2}a\wedge B_r + \frac{1}{2}(-1)^r B_r\wedge a\nonumber \\
 &= \frac{1}{2}(aB_r - a\cdot B_r + (-1)^r(B_ra - B_r\cdot a))\nonumber \\
 &= \frac{1}{2}(aB_r+(-1)^rB_ra).\label{equ_a_wedge_Br}
\end{align}
Now letting $a$ and $b$ each denote a vector, it is not hard to show that
\begin{equation}\label{equ_a_dot_b_Br_identity}
a\cdot(b\wedge B_r) + b\wedge(a\cdot B_r) = (a\cdot b)B_r.
\end{equation}
To that end, we apply equations \eqref{equ_a_dot_Br} and \eqref{equ_a_wedge_Br} in writing
\begin{align*}
a\cdot(b\wedge B_r)
 &= \frac{1}{2}\left(a\frac{1}{2}\left(bB_r + (-1)^rB_rb\right)-(-1)^{r+1}\frac{1}{2}\left(bB_r+(-1)^rB_rb\right)a\right) \\
 &= \frac{1}{4}\left(baB_r + (-1)^raB_rb + (-1)^rbB_ra + B_rba\right), \\
b\wedge(a\cdot B_r)
 &= \frac{1}{2}\left(b\frac{1}{2}\left(aB_r-(-1)^rB_ra\right)+(-1)^{r-1}\frac{1}{2}\left(aB_r-(-1)^rB_ra\right)b\right) \\
 &= \frac{1}{4}\left(baB_r - (-1)^rbB_ra - (-1)^raB_rb + B_rab\right),
\end{align*}
from which it is easy to see that
\begin{align*}
a\cdot(b\wedge B_r)+b\wedge(a\cdot B_r) &= \frac{1}{4}(ab+ba)B_r + \frac{1}{4}B_r(ba+ab) \\
 &= \frac{1}{2}(a\cdot b)B_r + \frac{1}{2}B_r(b\cdot a) = (a\cdot b)B_r.
\end{align*}
Similarly, we must note that
\begin{equation}
a\cdot(b\cdot B) = -b\cdot(a\cdot B).
\end{equation}
To see this, we apply equation \eqref{equ_a_dot_Br} in writing
\begin{align*}
a\cdot(b\cdot B_r)
 &= \frac{1}{2}\left(a\frac{1}{2}\left(bB_r-(-1)^rB_rb\right)-(-1)^{r-1}\frac{1}{2}\left(bB_r-(-1)^rB_rb\right)a\right) \\
 &= \frac{1}{4}\left(abB_r - (-1)^raB_rb + (-1)^rbB_ra - B_rba\right),
\end{align*}
Then, by substitution, we can immediately write
\begin{equation*}
b\cdot(a\cdot B_r) = \frac{1}{4}\left(baB_r - (-1)^rbB_ra + (-1)^raB_rb - B_rab\right).
\end{equation*}
Adding these, we then see that
\begin{align*}
a\cdot (b\cdot B)+b\cdot(a\cdot B)
 &= \frac{1}{4}\left(abB_r+baB_r\right)-\frac{1}{4}\left(B_rba+B_rab\right) \\
 &= \frac{1}{4}\left(ab+ba\right)B_r-\frac{1}{4}B_r\left(ba+ab\right) \\
 &= \frac{1}{2}(a\cdot b)B_r - \frac{1}{2}B_r(b\cdot a) = 0.
\end{align*}
Note that we may have arrived at this conclusion sooner had we written
\begin{equation*}
a\cdot(b\cdot B_r) = (a\wedge b)\cdot B_r = -(b\wedge a)\cdot B_r = -b\cdot(a\cdot B_r),
\end{equation*}
but the justification for some intermediate steps is not immediately clear.

\subsection{Identities Involving Mercers}

Letting $M_r$ denote a mercer of rank $r$ having factorization
\begin{equation}\label{equ_M_r}
M_r = \prod_{i=1}^r m_i,
\end{equation}
recall that
\begin{equation*}
M_r = \sum_{i=1}^r\left\langle M_r\right\rangle_i,
\end{equation*}
where here we're making use of the angled-brackets notation $\langle\cdot\rangle_i$ which takes the grade $i$ part of
what it encloses.  (Note that this requires us to visualize the expansion of the enclosure as a sum of blades.)  To be more precise,
if $M_r$ is a mercer of even rank, (if $r$ is even), then
\begin{equation}\label{equ_Mr_even}
M_r = \sum_{i=0}^{r/2}\left\langle M_r\right\rangle_{2i},
\end{equation}
while if $M_r$ is a mercer of odd rank, we have
\begin{equation}\label{equ_Mr_odd}
M_r = \sum_{i=1}^{(r+1)/2}\left\langle M_r\right\rangle_{2i-1}.
\end{equation}
To see this, consider first the trivial case of $r=0$; then, for any $r>0$, the equation
\begin{equation}\label{equ_Mr_split}
M_r = M_{r-1}m_r = \langle M_{r-1}\rangle_1^r\cdot m_r + \langle M_{r-1}\rangle_1^r\wedge m_r + \langle M_{r-1}\rangle_0 m_r.
\end{equation}
Here we have extended our notation $\langle\cdot\rangle_i^j$ to mean a culling of all enclosed blades not of a grade falling
in the interval $[i,j]$.

An inductive hypothesis can now be stated that equations \eqref{equ_Mr_even} and \eqref{equ_Mr_odd} hold for $r-1$.
If $r$ is even, then, by our inductive hypothesis, $M_{r-1}$, when expanded as a sum of blades, consists only of blades of odd grade,
and it is clear that equation \eqref{equ_Mr_split} becomes \eqref{equ_Mr_even}.  If $r$ is odd, then, by our inductive hypothesis, $M_{r-1}$, when expanded as
a sum of blades, consists only of blades of even grade, and it is clear that equation \eqref{equ_Mr_split} becomes \eqref{equ_Mr_odd}.

Notice that we might also have written equation \eqref{equ_Mr_split} as
\begin{equation*}
M_r = M_{r-1}m_r = M_{r-1}\cdot m_r + M_{r-1}\wedge m_r - \langle M_r\rangle_0 m_r.
\end{equation*}

Now let $a$ be a vector, and convince yourself that
\begin{equation}
a\wedge M_r = \left\{\begin{array}{ll}
-(-1)^r M_r\wedge a, & \mbox{if $r$ even,} \\
(-1)^r M_r\wedge a, & \mbox{if $r$ odd,}
\end{array}\right.
\end{equation}
and that
\begin{equation}
a\cdot M_r = \left\{\begin{array}{ll}
(-1)^r M_r\cdot a, & \mbox{if $r$ even,} \\
-(-1)^r M_r\cdot a, & \mbox{if $r$ odd.}
\end{array}\right.
\end{equation}
Refer to equations \eqref{equ_a_dot_Br_commutativity} and \eqref{equ_a_wedge_Br_commutativity} to see this.

\end{document}