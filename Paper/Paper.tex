\documentclass{birkjour}

\usepackage{hyperref}

\newtheorem{thm}{Theorem}[section]
\newtheorem{cor}[thm]{Corollary}
\newtheorem{lem}[thm]{Lemma}
\newtheorem{prop}[thm]{Proposition}
\theoremstyle{definition}
\newtheorem{defn}[thm]{Definition}
\theoremstyle{remark}
\newtheorem{rem}[thm]{Remark}
\newtheorem*{ex}{Example}
\numberwithin{equation}{section}

\newcommand{\R}{\mathbb{R}}
\newcommand{\B}{\mathbb{B}}
\newcommand{\G}{\mathbb{G}}
\newcommand{\V}{\mathbb{V}}
\newcommand{\gd}{\dot{g}}
\newcommand{\gh}{\hat{g}}
\newcommand{\Gd}{\dot{G}}
\newcommand{\Gh}{\hat{G}}
\newcommand{\nvai}{\infty}
\newcommand{\nvao}{o}
\newcommand{\grade}{\mbox{grade}}
\newcommand{\rank}{\mbox{rank}}

%\received{}\accepted{}

\begin{document}

\title{Untitled}

\author{Spencer T. Parkin}
\email{spencerparkin@outlook.com}

%\subjclass{Primary 14J70; Secondary 14J29}

%\dedicatory{To my dear wife Melinda.}

\begin{abstract}
Abstract...
\end{abstract}

\keywords{Key words...}

\maketitle

\section{Introduction And Motivation}

It is well-known that \emph{blades} taken from a geometric algebra are, under a given definition (such as
those given in \cite{Parkin15}), apt representatives of geometry.  The set of all blades being closed under the
meet and join operations, one must consider what this means for the geometries represented by them.
As it turns out, these operations allow us to intersect geometries and build them up as fitting the points of others.
Furthermore, blades are subject to the outermorphic property of the sandwitch-product between a versor and its inverse.
All sorts of wonderful transformations on the geometries represented by blades have been discovered when taken
in this product with versors (see \cite{}).  % Ref paper that finds all classic transformations in versors.

Thus seeing that a particular means of geometric representation lends itself to desirable algebraic proprties, and
therefore desirable geometric properties, alternative means of geometric representation are worth exploring.
Indeed, in this paper, we find that \emph{spades} taken from a geometric algebra are, under a similar definition,
also apt representative of geometry. The set of all spades is closed under the geometric product.  Unlike
the meet and join operations, the geometric product enjoys the potential of invertability.  Furthermore, we'll find
out that spades, like blades, are also subject to the outermorphic proprety of the sandwitch-product between a versor
and its inverse.

Throughout this paper we simply let $\V$ denote a vector space generating the geometric algebra $\G$ within
which we work.

\section{A Treatment Of Spades}

\begin{defn}[Spade]\label{def_spade}
An element $M\in\G$ is a \emph{spade} if it can be written as the geometric
product of zero or more vectors.
\end{defn}

It follows from Definition~\ref{def_spade} that all versors are spades, but not all spades are versors.
Furthermore, while the set of all spades of $\G$ enjoys closure under the geometric product, this set,
unlike the set of versors of $\G$, does not form a group.  We can also say that all null-versors are spades,
but not all spades are null-versors.

\begin{defn}[Spade Rank]\label{def_spader_rank}
The \emph{rank} of a spade $M_r\in\G$, denoted $\rank(M_r)$, is the smallest number\footnote{This smallest number exists
by the well-ordering principle.  See \cite{}.} of vectors for which $M_r$ can
be written as a geometric product of such.  If a spade $M_r\in\G$ has factorization
\begin{equation}\label{equ_M_r_factorization}
M_r=\prod_{i=1}^r m_i,
\end{equation}
then it is not necessarily the case that $\rank(M_r)=r$.  However, such a factorization does exist.  Clearly, it
would not be unique.
\end{defn}

Many identities involving a spade $M_r$ hold whether or not $\rank(M_r)=r$.
In any case, we will become interested in precisely what we can say about the
vectors in $\{m_i\}_{i=1}^r$ when $\rank(M_r)$ is $r$.

\begin{defn}[Dull/Sharp Spade]
For any given spade $M_r\in\G$, we say that $M_r$ is \emph{sharp} if and only if
$\langle M_r\rangle_s\neq 0$, where $s=\rank(M_r)$, and \emph{dull} otherwise.
\end{defn}
The following proposition is the claim that all spades are sharp.

\begin{prop}\label{prop_spade_lin_indep}
For any given non-zero spade $M_r\in\G$ with $r>0$,
\begin{equation*}
\mbox{$\rank(M_r)=r$ if and only if $0\neq\bigwedge_{i=1}^r m_i$.}
\end{equation*}
\end{prop}

While one direction of Proposition~\ref{prop_spade_lin_indep} is trivial to prove, the other, most decidedly, is not!  An outline of a proof follows.

Given a spade $M_r\in\G$ with factorization \eqref{equ_M_r_factorization}, the set of $r$ vectors $\{m_i\}_{i=1}^r$
is either a linearly independent set, or a linearly dependent set.  In the former case, it is clear that $\rank(M_r)=r$,
because $0\neq\bigwedge_{i=1}^r m_i=\langle M_r\rangle_r$.
In the latter case, we must show that $\rank(M_r)<r$.  To that end, let $s$ be the largest integer with $1\leq s<r$ such that
$\{m_i\}_{i=1}^s$ is a linearly independent set, and write
\begin{equation*}
\left\langle\prod_{i=1}^{s+1} m_i\right\rangle_{s-1}=\left\langle\left(\prod_{i=1}^s m_i\right)\sum_{i=1}^s\alpha_i m_i\right\rangle_{s-1} = \sum_{i=1}^s\beta_i\bigwedge_{\substack{j=1\\j\neq i}}^s m_j.
\end{equation*}
Just as not all scalars $\alpha_i$ are necessarily non-zero, we may not have all scalars $\beta_i$ non-zero.
Assuming for the moment that each $\beta_i$ is non-zero, we may write the grade $s-1$ part as the following $(s-1)$-blade.
\begin{equation*}
\sum_{i=1}^s\beta_i\bigwedge_{\substack{j=1\\j\neq i}}^s m_j = \bigwedge_{i=1}^{s-1} n_i,
\end{equation*}
where $n_1=\beta_1m_2 + \beta_2m_1$, and each of the remaining $n_i$ are given by
\begin{equation*}
n_i = m_{i+1}+\frac{\beta_{i+1}}{\beta_i}m_i.
\end{equation*}
In any case, even if some $\beta_i$ are zero, we can still come up with the $(s-1)$-blade that is the grade $s-1$ part of $\prod_{i=1}^{s+1} m_i$
in terms of vectors $n_i$.

With this $(s-1)$-blade in hand, we now must solve, for each integer $k$, the system of equations given by
\begin{equation}\label{equ_system}
\left\langle\prod_{i=1}^{s+1} m_i\right\rangle_{s-1-2k} = \left\langle\prod_{i=1}^{s-1}\left(n_i+\sum_{j=1}^{i-1}\gamma_{i,j} n_j\right)\right\rangle_{s-1-2k}.
\end{equation}
It is clear that any choice for the scalars $\gamma_{i,j}$ is acceptable in the case $k=0$.

If a solution in the variables $\gamma_{i,j}$ can be found, then we have shown that our geometric product of $r$ vectors can be rewritten as a geometric product of $r-2$ vectors.
We then continue this process until the set of vectors taken in the geometric product becomes a linearly independent set.

The proof outlined above is an algorithm for finding one of the smallest possible factorizations of the spade $M_r$; and consequently, its rank.  The truthfulness of Proposition~\ref{prop_spade_lin_indep} hinges on the idea
that a solution to the system of equations \eqref{equ_system} can always be found.  The presence of null-vectors may be trouble-some.

\begin{lem}\label{lem_no_dup_in_factorization}
For any given spade $M_r\in\G$, if there exist integers $1\leq i<j\leq r$ such that $m_i=m_j$, and $m_i$ is invertible, then $\rank(M_r)<r$.
\end{lem}
\begin{proof}
This is trivial in the case that $j=i+1$.  In the case that $j=i+2$, simply notice that
\begin{equation*}
m_im_{i+1}m_j = m_im_{i+1}m_i = 2(m_i\cdot m_{i+1})m_i-m_i^2 m_{i+1}.
\end{equation*}
In the case that $j>i+2$, we see that
\begin{equation*}
m_i\left(\prod_{k=i+1}^{j-1}m_k\right)m_j = m_i^2\prod_{k=i+1}^{j-1}m_im_km_i^{-1}.
\end{equation*}
\end{proof}

Lemma~\ref{lem_no_dup_in_factorization} is an interesting observation and implies that every spade written in the most compact form is square-free in the
sense that the square of no spade will appear in the factorization.  It is still, however, a far cry from helping us prove Proposition~\ref{prop_spade_lin_indep}.

\begin{lem}\label{lem_lin_indep_subblades}
For every non-zero $r$-blade $B_r\in\G$ with $r>1$, and having factorization
\begin{equation*}
B_r = \bigwedge_{i=1}^r b_i,
\end{equation*}
the set of $(r-1)$-blades $\{B_r^{(i)}\}_{i=1}^r$, where the notation $B_r^{(i)}$ is given by
\begin{equation*}
B_r^{(i)}=\bigwedge_{\substack{j=1\\j\neq i}}^r b_j,
\end{equation*}
is a linearly independent set.
\end{lem}
\begin{proof}
Supposing to the contrary, and without loss of generality, let
\begin{equation*}
B_{r-1} = B_r^{(r)} = \sum_{i=1}^{r-1}\alpha_i B_r^{(i)} = \left(\sum_{i=1}^{r-1}\alpha_i B_{r-1}^{(i)}\right)\wedge b_r.
\end{equation*}
Now notice that
\begin{equation*}
0\neq B_r = B_{r-1}\wedge b_r = B_r^{(r)}\wedge b_r = \left(\sum_{i=1}^{r-1}\alpha_i B_r^{(i)}\right)\wedge b_r = 0,
\end{equation*}
which is clearly a contradiction.
\end{proof}

\begin{lem}\label{lem_solution_intersection}
Given any spade $M_r$, the set of all solution sets $\{\alpha_i\}_{i=1}^r$ of the equation
\begin{equation*}
0 = \sum_{i=1}^r\alpha_i M_r^{(i)}
\end{equation*}
is, for all integers $j\in[0,r]$, the intersection of all sets of solution sets of the equations
\begin{equation*}
0 = \sum_{i=1}^r\alpha_i\langle M_r^{(i)}\rangle_j,
\end{equation*}
where the notation $M_r^{(i)}$ is given by
\begin{equation*}
M_r^{(i)}=\prod_{\substack{j=1\\j\neq i}}^r m_j.
\end{equation*}
\end{lem}
\begin{proof}
This is a simple consequence of there being no possibility of interference between elements of differing grade.
\end{proof}

\begin{lem}
For any given spade $M_r\in\G$ with $r>1$, if $\rank(M_r)=r$, then
the set of spades $\{M_r^{(i)}\}_{i=1}^r$ is a linearly independent set.
\end{lem}
\begin{proof}
This follows easily in consideration of Proposition~\ref{prop_spade_lin_indep} with Lemma~\ref{lem_lin_indep_subblades} and Lemma~\ref{lem_solution_intersection}.
\end{proof}

% Make use of <*>_a^b = sum_i <*>_i notation later on.

\begin{thebibliography}{9}

\bibitem{Parkin15}
S. Parkin,
\emph{An Introduction To Geometric Sets}.
Advances in Applied Clifford Algebras, Volume 25, Issue Unknown, pp. 639-655, 2015.

% Ref Geo Alg with App in Eng
% Ref New Foundations For Classical Mechanics
% Ref GA For Physists
% Ref GA For CS

\end{thebibliography}

\end{document}