\documentclass[12pt]{article}

\usepackage{amsmath}
\usepackage{amssymb}
\usepackage{amsthm}

\title{Chapters 5-8 Supplementary Exercises\\Gallian's Book on Abstract Algebra}
\author{Spencer T. Parkin}

\newtheorem{theorem}{Theorem}[section]
\newtheorem{definition}{Definition}[section]
\newtheorem{corollary}{Corollary}[section]
\newtheorem{identity}{Identity}[section]
\newtheorem{lemma}{Lemma}[section]
\newtheorem{result}{Result}[section]

%\newcommand{\gcd}{\mbox{gcd}}
\newcommand{\lcm}{\mbox{lcm}}
\newcommand{\abs}{\mbox{abs}}
\newcommand{\Z}{\mathbb{Z}}
\newcommand{\R}{\mathbb{R}}
\newcommand{\cl}{\mbox{cl}}
\newcommand{\aut}{\mbox{Aut}}

\begin{document}
\maketitle

\section*{Problem 1}

A subgroup $N$ of a group $G$ is called a {\it chracteristic subgroup} if $\phi(N)=N$
for all automorphisms $\phi$ of $G$.  Prove that every subgroup of a cyclic group is characteristic.

Let $\phi$ be an automorphism of $G$, a cyclic group.
Let $a$ be an element of $G$.  Now see that
\begin{equation*}
\phi(\langle a\rangle)=\{\phi(a^k)|k\in\Z\}=\{\phi^k(a)|k\in\Z\}=\langle\phi(a)\rangle.
\end{equation*}
Clearly, $|a|=|\phi(a)|$, so $|\langle a\rangle|=|\langle\phi(a)\rangle|$.  We can
now claim that $\langle a\rangle=\langle \phi(a)\rangle$ by the fundamental
theorem of cyclic groups, because $G$ has one and only one subgroup of each
possible order.

\section*{Problem 2}

Prove that the center of a group is characteristic.

Let $\phi$ be any automorphism of a group $G$.
Letting $a$ be an element in $\phi(Z(G))$ and $g$ an element in $G$,
there must exist an element $a'\in Z(G)$ and an element $g'\in G$ such
that $\phi(a')=a$ and $\phi(g')=g$.  It then follows that
\begin{equation*}
ag=\phi(a')\phi(g')=\phi(a'g')=\phi(g'a')=\phi(g')\phi(a')=ag,
\end{equation*}
showing that $a\in Z(G)$.
Thus far we have shown that $\phi(Z(G))\subseteq Z(G)$.
But $\phi$ is one-to-one, so $\phi(Z(G))$ cannot be a proper
subset of $Z(G)$, and therefore, we must have $\phi(Z(G))=Z(G)$.
Oops, what if $Z(G)$ is infinite?

I'm stumped...

\section*{Problem 4}

Prove that the property of being a characterstic subgroup is transitive.
That is, if $N$ is a characteristic subgroup of $K$ and $K$ is a characteristic
subgroup of $G$, then $N$ is a characteristic subgroup of $G$.

Let $\phi\in\aut(G)$.  If $\phi(K)=K$, then $\phi$, when restricted in
domain to $K$, is an automorphism of $K$.  It follows that
$\phi(N)=N$, showing that $N$ is a characteristic subgroup of $G$.

\section*{Problem 6}

Let $H$ and $K$ be subgroups of a group $G$ and let $HK=\{hk|h\in H,k\in K\}$
and $KH=\{kh|k\in K,h\in H\}$.  Prove that $HK$ is a group if and only if
$HK=KH$.

Suppose $HK=KH$.  Clearly $e\in HK$.  Let $a,b\in HK$.
Then there exists $h,h'\in H$ and $k,k'\in K$ such that
$a=hk$ and $b=h'k'$, and we have
\begin{equation*}
ab^{-1}=hk(h'k')^{-1}=hk(k')^{-1}(h')^{-1} = hh''k''\in HK,
\end{equation*}
for some element $h''\in H$ and another $k''\in K$, because
$HK=KH$.

Now suppose $HK$ is a subgroup of $G$.  If $a\in HK$,
then $a^{-1}=hk$ for some $h\in H$ and $k\in K$.
It follows that $a=k^{-1}h^{-1}\in KH$.
If $a\in KH$, then $a=hk$ for some $h\in H$ and $k\in K$.
It follows that $a^{-1}=h^{-1}k^{-1}\in HK\implies (a^{-1})^{-1}=a\in HK$.

\pagebreak
\section*{Problem 7}

Let $H$ and $K$ be subgroups of a finite group $G$.  Prove that
\begin{equation*}
|HK|=\frac{|H||K|}{|H\cap K|}.
\end{equation*}

Consider the group $H\oplus K$, and define an equivilance
relation on it as follows.  For all $a,b\in H\oplus K$,
let $a\sim b$ if and only if $a_ha_k=b_hb_k$,
where $a=(a_h,a_k)$ and $b=(b_h,b_k)$.  It is not
hard to see that this is an equivilance relation
on $H\oplus K$ that partitions it into $|HK|$
distinct equivilance classes.  Now consider
$[(h,k)]$, the equivilance class containing $(h,k)$.
If $(h',k')\in[(h,k)]$, then $hk=h'k'\implies (h')^{-1}h=k'k^{-1}=x\in H\cap K$,
showing that
\begin{equation*}
[(h,k)]=\{(hx^{-1},xk)|x\in H\cap K\}.
\end{equation*}
Furthermore, for any $x,y\in H\cap K$, if $x\neq y$, then
$(hx^{-1},xk)\neq (hy^{-1},yk)$, showing that
$|[(h,k)]|=|H\cap K|$.  It now follows that
\begin{equation*}
|H||K|=|H\oplus K|=|HK||H\cap K|.
\end{equation*}

%\section*{Problem 17}
%
%Let $p$ be an odd prime.  Show that 1 is the only solution of $x^{p-2}=1$
%in $U(p)$.
%
% This is wrong.  
%A solution to this equation is an element of $U(p)$ having an
%order that divides $p-2$, which is odd.  But all elements of
%$U(p)$, except 1, have even order, because their orders must divide $|U(p)|=\phi(p)=p-1$,
%which is even.  The only solution, therefore, is 1.

\section*{Progblem 50}

Suppose that $H$ and $K$ are subgroups of a group and that $|H|$ and
$|K|$ are relatively prime.  Show that $H\cap K=\{e\}$.

Let $a\in H\cap K$.  Then $|a|$ divides $|H\cap K|$, but
since $|H\cap K|$ divides $|H|$ and $|K|$, we must have $|a|$
dividing $|H|$ and $|K|$.  But if $\gcd(|H|,|K|)=1$, then
we must have $|a|=1\implies a=e$.

\end{document}