\documentclass{birkjour}

\usepackage{tikz}
\usepackage{graphicx}
\usepackage{hyperref}

\newtheorem{thm}{Theorem}[section]
\newtheorem{cor}[thm]{Corollary}
\newtheorem{lem}[thm]{Lemma}
\newtheorem{prop}[thm]{Proposition}
\theoremstyle{definition}
\newtheorem{defn}[thm]{Definition}
\theoremstyle{remark}
\newtheorem{rem}[thm]{Remark}
\newtheorem*{ex}{Example}
\numberwithin{equation}{section}

\newcommand{\R}{\mathbb{R}}
\newcommand{\B}{\mathbb{B}}
\newcommand{\G}{\mathbb{G}}
\newcommand{\V}{\mathbb{V}}
\newcommand{\gd}{\dot{g}}
\newcommand{\gh}{\hat{g}}
\newcommand{\Gd}{\dot{G}}
\newcommand{\Gh}{\hat{G}}
\newcommand{\nvai}{\infty}
\newcommand{\nvao}{o}
\newcommand{\grade}{\mbox{grade}}

\begin{document}

%-------------------------------------------------------------------------
% editorial commands: to be inserted by the editorial office
%
%\firstpage{1} \volume{228} \Copyrightyear{2004} \DOI{003-0001}
%
%
%\seriesextra{Just an add-on}
%\seriesextraline{This is the Concrete Title of this Book\br H.E. R and S.T.C. W, Eds.}
%
% for journals:
%
%\firstpage{1}
%\issuenumber{1}
%\Volumeandyear{1 (2004)}
%\Copyrightyear{2004}
%\DOI{003-xxxx-y}
%\Signet
%\commby{inhouse}
%\submitted{March 14, 2003}
%\received{March 16, 2000}
%\revised{June 1, 2000}
%\accepted{July 22, 2000}
%
%
%
%---------------------------------------------------------------------------

\title{The Intersection Of Rays And\\Algebraic Surfaces}

\author{Spencer T. Parkin}
\address{102 W. 500 S., Salt Lake City, UT  84101}
\email{spencerparkin@outlook.com}

\subjclass{Primary 14J70; Secondary 14J29}

%\date{October 22, 2013}

%\dedicatory{To my dear wife Melinda.}

\begin{abstract}
It is shown that for any multi-variable polynomial defined over the
real numbers that the image of a line through the domain of such a function
is determined entirely by all orders of the directional derivatives of this function at
any one point along the line and in a direction of the line.  This result has an
application in the problem of casting rays through algebraic surfaces as it
shows that such a problem, in all cases, reduces to the problem of finding
the roots of a single-variable polynomial having an explicit formulation
in terms of the multi-variable polynomial and ray in question.
\end{abstract}

\maketitle

\section{Introduction}

Letting $f:\R^n\to\R$ be any polynomial equation in $n$ variables and up to degree $m$, it
was shown in \cite{} that the function $\bigwedge_{i=1}^m p_i(x)$ may be factored out
of this polynomial in terms of the inner product as
\begin{equation}\label{equ_polynomial_func}
f(x) = \bigwedge_{i=1}^m p_i(x)\cdot B,
\end{equation}
where $B$ is an $m$-vector of our geometric algebra with $\nvai_i\cdot B=0$ for all
integers $i\in[1,m]$, and where the function
$p_i:\R^n\to\V_i$ is given by
\begin{equation}\label{equ_point_func}
p_i(x) = \nvao_i + x_i + \frac{1}{2}x_i^2\nvai_i,
\end{equation}
having its origins in the paper \cite{}.  Given a point $x\in\R^n$ and a direction vector $v\in\R^n$,
we wish to find the set of all scalars $\lambda\in\R$ such that $f(x+\lambda v)=0$.
Utilizing equation \eqref{equ_polynomial_func} for this purpose, we easily find that
\begin{equation}\label{equ_ray_polynomial_unexpanded}
f(x+\lambda v) = \bigwedge_{i=1}^m(p_i(x)+\lambda v_i)\cdot B,
\end{equation}
because we can ignore the $\frac{1}{2}x_i^2\nvai_i$ term in equation \eqref{equ_point_func}.
Looking at equation \eqref{equ_ray_polynomial_unexpanded},
it is immediately clear that its expansion is that of a polynomial in $\lambda$ of up to degree $m$.
What we're going to show in this paper is that an explicit formula for this polynomial can
be found in terms of all orders of directional derivatives of $f$ at $x$ and in the
direction of $v$.

\section{The Result}

We begin by rewriting equation \eqref{equ_ray_polynomial_unexpanded} as
\begin{equation}\label{equ_ray_polynomial_expanded_but_unknown}
f(x+\lambda v) = \sum_{i=0}^m T_i(x),
\end{equation}
where $T_i(x)$ will denote the $i^{th}$ term involving $\lambda^i$ in the series expansion of
\eqref{equ_ray_polynomial_unexpanded}.  Carefully formulating this term, we get
\begin{equation*}
T_i(x) = \lambda^i \sum_{j=1}^{\binom{m}{i}} W_{j,i}(x)\cdot B,
\end{equation*}
where $W_{j,i}$ is the $j^{th}$ way to write an outer product involving
$i$ vectors taken from $\{v_k\}_{k=1}^m$ and $m-i$ vectors taken from
$\{p_k(x)\}_{k=1}^m$ in an order having ascending sub-scripts.  The following
examples help clarify this in the case $m=3$.
\begin{align*}
W_{1,0} &= p_1\wedge p_2\wedge p_3 \\
W_{1,1} &= p_1\wedge p_2\wedge v_3 \\
W_{2,1} &= p_1\wedge v_2\wedge p_3 \\
W_{3,1} &= v_1\wedge p_2\wedge p_3 \\
W_{1,2} &= p_1\wedge v_2\wedge v_3 \\
W_{2,2} &= v_1\wedge p_2\wedge v_3 \\
W_{3,2} &= v_1\wedge v_2\wedge v_3 \\
W_{1,3} &= v_1\wedge v_2\wedge v_3
\end{align*}
Having now come to terms, (no pun intended), with the general expansion of
equation \eqref{equ_ray_polynomial_unexpanded}, we proceed now to
fearlessly take the directional derivative of $T_i$ at $x$ and in the direction of $v$.
Doing so, we get
\begin{align*}
\nabla_v T_i(x) &= \lambda^i\sum_{j=1}^{\binom{m}{i}}\lim_{\delta\to 0}
\frac{W_{j,i}(x+\delta v)-W_{j,i}(x)}{\delta}\cdot B,
\end{align*}
knowing that each individual limit will exist.  What we must realize now is that
the term $W_{j,i}(x)$ will get canceled in the expansion
of $W_{j,i}(x+\delta v)$, leaving only terms that are multiples of positive powers of $\delta$.
Furthermore, it is only those remaining terms that are multiples of $\delta$ itself
that will survive the limit process.  We are therefore left to deduce these terms in an
evaluation of the limit.  What we find is that all such terms are of the form
$\delta W_{j,i+1}(x)$, but we need to determine just how many we have.
Realizing that $\binom{m}{i}$ old terms will each contribute $m-i$ new terms of
this form, of which there should be $\binom{m}{i+1}$, but that no type of term
will be produced any more or less than any other, we see that
\begin{equation*}
\frac{(m-i)\binom{m}{i}}{\binom{m}{i+1}}=i+1
\end{equation*}
is the number of such terms of the form $\delta W_{j,i+1}(x)$, and we may write
\begin{align}
\nabla_v T_i(x) &= \lambda^i\sum_{j=1}^{\binom{m}{i+1}}\lim_{\delta\to 0}
\frac{(i+1)\delta W_{j,i+1}(x)}{\delta}\cdot B\nonumber \\
 &= \lambda^i(i+1)\sum_{j=1}^{\binom{m}{i+1}} W_{j,i+1}(x)\cdot B\nonumber \\
 &= \frac{i+1}{\lambda}T_{i+1}(x).\label{equ_recurrence_relation}
\end{align}
Returning to equation \eqref{equ_ray_polynomial_expanded_but_unknown},
and realizing that $T_0(x)=f(x)$, we can now finally deduce the expansion of
\eqref{equ_ray_polynomial_unexpanded} using the recurrence relation
of equation \eqref{equ_recurrence_relation} as
\begin{equation*}
f(x+\lambda v) = \sum_{i=0}^m\frac{\lambda^i}{i!}\nabla_v^i f(x),
\end{equation*}
where $\nabla_v^i f(x)$ is the $i^{th}$ order directional derivative of $f$ at $x$
in the direction of $v$ with $\nabla_v^0 f(x)=f(x)$.

\end{document}