\documentclass[12pt]{article}

\usepackage{amsmath}
\usepackage{amssymb}
\usepackage{amsthm}

\title{Chapters 9-11 Supplementary Exercises\\Gallian's Book on Abstract Algebra}
\author{Spencer T. Parkin}

\newtheorem{theorem}{Theorem}[section]
\newtheorem{definition}{Definition}[section]
\newtheorem{corollary}{Corollary}[section]
\newtheorem{identity}{Identity}[section]
\newtheorem{lemma}{Lemma}[section]
\newtheorem{result}{Result}[section]

%\newcommand{\gcd}{\mbox{gcd}}
\newcommand{\lcm}{\mbox{lcm}}
\newcommand{\abs}{\mbox{abs}}
\newcommand{\Z}{\mathbb{Z}}
\newcommand{\R}{\mathbb{R}}
\newcommand{\cl}{\mbox{cl}}
\newcommand{\aut}{\mbox{Aut}}
\newcommand{\inn}{\mbox{Inn}}

\begin{document}
\maketitle

\section*{Exercise 8}

Let $k$ be a divisor of $n$.  The factor gruop $(Z/\langle n\rangle)/(\langle k\rangle/\langle n\rangle)$
is isomorphic to some very familiar group.  What is the group?

By Exercise 40 of Chapter 10 (The Third Isomorphism Theorem), we
see that $(Z/\langle n\rangle)/(\langle k\rangle/\langle n\rangle)\approx Z/\langle k\rangle$.
What more is there to say?

\section*{Exercise 30}

Let $G$ be a group and let $\phi:G\to G$ be a function such that
\begin{equation*}
\phi(g_1)\phi(g_2)\phi(g_3) = \phi(h_1)\phi(h_2)\phi(h_3)
\end{equation*}
whenever $g_1g_2g_3=e=h_1h_2h_3$.  Prove that there exists an element of $a$
in $G$ such that $\Psi(x)=a\phi(x)$ is a homomorphism.

If $a=e\in G$ and $u,v\in G$, then
\begin{equation*}
\Psi((uv)^{-1})\Psi(u)\Psi(v) = e\implies \Psi(v^{-1}u^{-1})=\Psi(v)^{-1}\Psi(v)^{-1}.
\end{equation*}

Hmmm...  Can we somehow show that $\Psi$ is a homomorphism here?
We cannot use homomorphic properties of $\Psi$ before we know that it's
a homomorphism.

\section*{Exercise 36}

A proper subgroup $H$ of a group $G$ is called {\it maximal} if there is no
subgroup $K$ such that $H\subset K\subset G$.  Prove that $Q$ under
addition has no maximal subgroups.

For any $q\in Q$, let $Z(q)$ denote the set $\{zq|z\in\Z\}$.
Then for any proper subgroup $H$ of the rationals $Q$ under addition,
we will assume that there exists $q\in Q-H$ such that $Z(q)\cap H$ is the trivial
subgroup of $Q$.  (How might I prove that this is true, if it's true?  It is
easy to show that $H$ has no upper bound on the set of its non-members in $Q$.
We can then find a finite sequence of any length where the elements are evenly spaced
and the sequence misses $H$ altogether.  But none of these has the form $Z(q)$ for some $q\in Q-H$.)

Now if $q\in Q-H$, it is easy to show that $H+Z(q)$
properly contains $H$ and is a subgroup of $Q$.
Let $q\in Q-H$ be an element of $Q$ such that $Z(q)\cap H$
is the trivial group.  This can be done by our assumption above.
What remains to be shown is that $H+Z(q)$ is a proper
subgroup of $Q$.  Suppose $Q=H+Z(q)$.
Notice that $q/2\not\in H$, (since $q/2\in H$ would imply that $q\in H$), and $q/2\not\in Z(q)$.
Yet we must have $q/2=zq+h$ for some $h\in H$ and $z\in\Z$.
Rearranging, we have $2h=(1-2z)q$.  Now since $2h\in H$
and $(1-2z)q\in Z(q)$, we must have $2h=(1-2z)q\in Z(q)\cap H$ which
implies that $h=0$ and $q=0$.  But this contradicts the facts that
$q\not\in H$ and $q/2\not\in Z(q)$.  So $H+Z(q)$ is a proper subgroup of $Q$.

\end{document}