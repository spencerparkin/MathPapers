\documentclass[12pt]{article}

\usepackage{amsmath}
\usepackage{amssymb}
\usepackage{amsthm}

\title{Chapters 9-11 Supplementary Exercises\\Gallian's Book on Abstract Algebra}
\author{Spencer T. Parkin}

\newtheorem{theorem}{Theorem}[section]
\newtheorem{definition}{Definition}[section]
\newtheorem{corollary}{Corollary}[section]
\newtheorem{identity}{Identity}[section]
\newtheorem{lemma}{Lemma}[section]
\newtheorem{result}{Result}[section]

%\newcommand{\gcd}{\mbox{gcd}}
\newcommand{\lcm}{\mbox{lcm}}
\newcommand{\abs}{\mbox{abs}}
\newcommand{\Z}{\mathbb{Z}}
\newcommand{\R}{\mathbb{R}}
\newcommand{\cl}{\mbox{cl}}
\newcommand{\aut}{\mbox{Aut}}
\newcommand{\inn}{\mbox{Inn}}

\begin{document}
\maketitle

\section*{Exercise 8}

Let $k$ be a divisor of $n$.  The factor gruop $(Z/\langle n\rangle)/(\langle k\rangle/\langle n\rangle)$
is isomorphic to some very familiar group.  What is the group?

By Exercise 40 of Chapter 10 (The Third Isomorphism Theorem), we
see that $(Z/\langle n\rangle)/(\langle k\rangle/\langle n\rangle)\approx Z/\langle k\rangle$.
What more is there to say?

\section*{Exercise 30}

Let $G$ be a group and let $\phi:G\to G$ be a function such that
\begin{equation*}
\phi(g_1)\phi(g_2)\phi(g_3) = \phi(h_1)\phi(h_2)\phi(h_3)
\end{equation*}
whenever $g_1g_2g_3=e=h_1h_2h_3$.  Prove that there exists an element of $a$
in $G$ such that $\Psi(x)=a\phi(x)$ is a homomorphism.

If $a=e\in G$ and $u,v\in G$, then
\begin{equation*}
\Psi((uv)^{-1})\Psi(u)\Psi(v) = e\implies \Psi(v^{-1}u^{-1})=\Psi(v)^{-1}\Psi(v)^{-1}.
\end{equation*}

Hmmm...  Can we somehow show that $\Psi$ is a homomorphism here?
We cannot use homomorphic properties of $\Psi$ before we know that it's
a homomorphism.

\section*{Exercise 36}

A proper subgroup $H$ of a group $G$ is called {\it maximal} if there is no
subgroup $K$ such that $H\subset K\subset G$.  Prove that $Q$ under
addition has no maximal subgroups.

This is a very difficult problem, and after giving it a great deal of thought,
the best I can come up with so far is the following hypothesis.

Let $H$ be any subgroup of $Q$ and
let $B$ be a subset of $H$ of smallest possible cardinality such that
\begin{equation*}
H=\{z_1b_1+\dots+z_kb_k|z_i\in Z, b_i\in B,k\in Z^{+}\}.
\end{equation*}
Call such a subset $B$ of $H$ a basis for $H$.  If $B$ is of finite
cardinality $k$, then we may write
\begin{equation*}
H = \langle b_1,\dots,b_k\rangle.
\end{equation*}
It is easy to show that if $B$ is of finite cardinality, then $H$ is a proper
subgroup of $Q$.  Also, if $B$ is a basis for $Q$, then $B$ is infinite.
The converse of either of these statements, however, is not obvious.
Let's suppose for the moment, however, that $H$ is a proper subgroup
of $Q$ if and only if $B$ is of finite cardinality.
If then $H$ is a proper subgroup of $Q$ and we let $q\in Q-H$, it is clear that
\begin{equation*}
\langle b_1,\dots,b_k\rangle+\langle q\rangle=\langle b_1,\dots,b_k,q\rangle
\end{equation*}
is a subgroup of $Q$ that properly contains $H$.  That it is a proper subgroup
of $Q$ follows from our assumption above and a realization that a basis
for $H+\langle q\rangle$ is $B\cup\{q\}$, which is clearly finite.

Notice that all subgroups of $Q$ of the form $\langle q\rangle$ for some non-zero $q\in Q$
are minimal subgroups of $Q$ isomoprhic to $Z$.

Can an example be found that disproves the assumption?  Can the assumption
be proved?  One approach to proving it is to take a subgroup $H$ of $Q$
having an infinite basis $B$ and showing that for any $q\in Q$, we have $q\in H$.
This would show that $H=Q$.  It may be easy to show that $q$ is always a limit
point of $H$, but this does not imply membership in $H$.

After some thought, I see that this approach doesn't work at all.
I can think of many proper subgroups of $Q$ for which no basis
can exist.  Consider, for example,
\begin{equation*}
\langle q\rangle+\langle q/2\rangle+\langle q/2^2\rangle+\langle q/2^3\rangle+\dots,
\end{equation*}
where $q\in Q$.  There is no basis for this subgroup of $Q$.

\end{document}