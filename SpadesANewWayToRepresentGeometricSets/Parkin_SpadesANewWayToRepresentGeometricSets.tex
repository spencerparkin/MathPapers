\documentclass{birkjour}

\usepackage{float}
\usepackage{hyperref}

\newtheorem{thm}{Theorem}[section]
\newtheorem{cor}[thm]{Corollary}
\newtheorem{lem}[thm]{Lemma}
\newtheorem{prop}[thm]{Proposition}
\theoremstyle{definition}
\newtheorem{defn}[thm]{Definition}
\theoremstyle{remark}
\newtheorem{rem}[thm]{Remark}
\newtheorem*{ex}{Example}
\numberwithin{equation}{section}

\newcommand{\F}{\mathbb{F}}
\newcommand{\R}{\mathbb{R}}
\newcommand{\C}{\mathbb{C}}
\newcommand{\B}{\mathbb{B}}
\newcommand{\G}{\mathbb{G}}
\newcommand{\V}{\mathbb{V}}
\newcommand{\gd}{\dot{g}}
\newcommand{\gh}{\hat{g}}
\newcommand{\Gd}{\dot{G}}
\newcommand{\Gh}{\hat{G}}
\newcommand{\nvai}{\infty}
\newcommand{\nvao}{o}
\newcommand{\grade}{\mbox{grade}}
\newcommand{\rank}{\mbox{rank}}

%\received{}\accepted{}

\begin{document}

\title{Spades -- A New Way To Represent Geometric Sets}

\author{Spencer T. Parkin}
\email{spencerparkin@outlook.com}

%\subjclass{Primary 14J70; Secondary 14J29}

%\dedicatory{To my dear wife Melinda.}

\begin{abstract}
Abstract...
\end{abstract}

\keywords{Key words...}

\maketitle

\section{Introduction And Motivation}

Traditionally, \emph{blades} are used, in places such as the conformal model, to represent geometric sets in geometric algebra.\footnote{A formal
treatment of geometric sets in an abstract setting is given in \cite{Parkin15}.  Geometric sets are a generalization of algebraic sets.}  Doing so,
the meet and join operations become the principle
means by which geometries are combined or intersected to form new geometries.  In this paper we show that \emph{spades} may also be used
to represent geometric sets; and in so doing, the geometric product becomes the principle means by which geometries are combined or intersected
to form new geometries.

Of course, the term ``spade'' requires some explanation. See Table~\ref{tbl_terms} below.

\begin{table}[H]\label{tbl_terms}\caption{A few terms used in GA}
\begin{tabular}{p{2cm}p{9cm}}
Term & Definition \\
\hline
Blade & An outer product of zero or more linearly-independent vectors. \\
Versor & A geometric product of zero or more \emph{invertible} vectors, not necessarily forming a linearly-independent set. \\
Spade & A geometric product of zero or more vectors, not necessarily forming a linearly-independent set.\\
Null Versor & A geometric product of one or more vectors where at least one of them is null.\\
\end{tabular}
\end{table}

From these definitions it is clear that every versor is a spade, but not every spade is a versor.  The definition of a versor,
which can be found in \cite[p. 90]{Perwass09}, is well established and engrained in the literature just as it is written in Table~\ref{tbl_terms}.  Not requiring
that each vector in the factorization of a spade be invertible, this justifies the new term.

\section{Geometric Sets}

We must begin with a review of geometric sets.  Given an $n$-dimensional space $\F^n$, we let $p:\F^n\to\V$ be a non-zero, vector-valued
function mapping points in $\F^n$ to vectors in a vector space $\V$ generating our geometric algebra $\G$.  With this in hand,
we are ready for the following definition.
\begin{defn}[Geometric Set]\label{def_geo_set}
A subset $S$ of $\F^n$ is a \emph{geometric set} if and only if there exists a set of vectors $\{v_i\}\subseteq\V$, such that
\begin{equation}\label{equ_geo_set_def}
S = G(\{v_i\}) = \bigcap_i \{x\in\F^n|p(x)\cdot v_i=0\}.
\end{equation}
\end{defn}
Notice that the subset $\{v_i\}$ of $\V$ may be of finite or infinite cardinality.  It should also be immediately clear
from Definition~\ref{def_geo_set} that the intersection of any two geometric sets is geometric.
If each expression $p(x)\cdot v_i$ is a polynomial in the components of $x$, then every geometric set is algebraic.

\begin{lem}\label{lem_geo_set_lin_indep}
If $\{E_i\}_{i=1}^r$ is any linearly independent set of elements taken from $\G$, then the set of all
solutions in $\F^n$ to the equation
\begin{equation}\label{equ_geo_set}
0 = \sum_{i=1}^r (p(x)\cdot v_i)E_i
\end{equation}
is a geometric set.
\end{lem}
\begin{proof}
Being a linearly independent set of elements, the only linear combination of these elements that vanishes is the trivial linear combination.
It then follows that for each integer $i\in[1,r]$, we must have $p(x)\cdot v_i=0$.
\end{proof}

\begin{lem}\label{lem_geo_set_lin_dep}
If $\{E_i\}_{i=1}^r$ is any sequence of elements taken from $\G$, then the set of all solutions in $\F^n$ to equation \eqref{equ_geo_set}
is a geometric set.
\end{lem}
\begin{proof}
If $\{E_i\}_{i=1}^r$ is a linearly independent set, then we're done by Lemma~\ref{lem_geo_set_lin_indep}.
Supposing to the contrary, and without loss of generality, we can let $s$ be an integer with $1\leq s<r$ such that
$\{E_i\}_{i=1}^s$ is a linearly independent set, and
\begin{equation*}
\mbox{span}\{E_i\}_{i=1}^r = \mbox{span}\{E_i\}_{i=1}^s.
\end{equation*}
Now for each integer $i\in[s+1,r]$, write $E_i$ as a linear combination of the elements in $\{E_i\}_{i=1}^s$ as
\begin{equation*}
E_i = \sum_{j=1}^s\alpha_{i,j} E_j.
\end{equation*}
Having done so, we see that equation \eqref{lem_geo_set_lin_dep} becomes
\begin{align*}
0 &= \sum_{i=1}^r(p(x)\cdot v_i)E_i \\
 &= \sum_{i=1}^s(p(x)\cdot v_i)E_i + \sum_{i=s+1}^r(p(x)\cdot v_i)\sum_{j=1}^s\alpha_{i,j}E_j \\
 &= \sum_{i=1}^s\left[p(x)\cdot v_i-\sum_{j=s+1}^r\alpha_{j,i}(p(x)\cdot v_j)\right]E_i \\
 &= \sum_{i=1}^s\left[p(x)\cdot\left(v_i-\sum_{j=s+1}^r\alpha_{j,i}v_i\right)\right]E_i.
\end{align*}
We see now that the set of all solutions to equation \eqref{lem_geo_set_lin_dep} is given by
\begin{equation*}
\bigcap_{i=1}^s\left\{x\in\F^n\left|p(x)\cdot\left(v_i-\sum_{j=s+1}^r\alpha_{j,i}v_i\right)=0\right.\right\},
\end{equation*}
which is clearly a geometric set by Definition~\ref{def_geo_set}.
\end{proof}

\begin{lem}\label{lem_reduce_vec_set}
For any set of $r$ vectors $\{v_i\}_{i=1}^r$ taken from $\V$, if $S$ is the geometric set generated by
this set of vectors, then there exists a linearly independent subset of $\{v_i\}_{i=1}^r$ that also generates $S$.
\end{lem}
\begin{proof}
If $\{v_i\}_{i=1}^r$ is a linearly independent set, we're done.  Supposing otherwise, and without loss of generality,
we may let $s$ be an integer with $1\leq s<r$ such that $\{v_i\}_{i=1}^s$ is a linearly independent set, and
\begin{equation*}
\mbox{span}\{v_i\}_{i=1}^r = \mbox{span}\{v_i\}_{i=1}^s.
\end{equation*}
Clearly $G(\{v_i\}_{i=1}^s)\subseteq G(\{v_i\}_{i=1}^r)$ since $s<r$.
Now if $x\in G(\{v_i\}_{i=1}^r)$, then for all integers $i\in[1,s]$, we have $p(x)\cdot v_i=0$.
It then follows that for all integers $i\in[s+1,r]$, we have
\begin{equation*}
p(x)\cdot v_i = p(x)\cdot\sum_{j=1}^s\alpha_{i,j}v_j = 0.
\end{equation*}
Therefore, $x\in G(\{v_i\}_{i=1}^s)$.
\end{proof}

\begin{lem}\label{lem_solution_intersection}
If $\{E_i\}_{i=1}^r$ is any set of $r$ elements taken from our geometric algebra $\G$, then the set $A$ of all solutions in each $\alpha_i$
to the equation
\begin{equation*}
0 = \sum_{i=1}^r\alpha_i E_i
\end{equation*}
is given by
\begin{equation*}
A = \bigcap_{k=1}^r A_k,
\end{equation*}
where each $A_k$ is the set of all solutions in each $\alpha_i$ to equation $k\in[0,\dim(\V)]$, given by
\begin{equation*}
0 = \sum_{i=1}^r\alpha_i\langle E_i\rangle_k.
\end{equation*}
\end{lem}
\begin{proof}
Show it.
\end{proof}

\section{Perliminary Material}

Before we can show how blades and spades can represent geometric sets, we need to lay some ground work
with the following definitions, lemmas, and identities.

Though already given in Table~\ref{tbl_terms}, the term spade deserves its own formal definition as follows.
\begin{defn}[Spade]
An element $M_r\in\G$ is called a \emph{spade} if and only if there exists a set of $r$ vectors $\{m_i\}_{i=1}^r$
such that it may be written as
\begin{equation}\label{equ_M_r}
M_r = \prod_{i=1}^r m_i.
\end{equation}
\end{defn}
It is easy to show that spades, like blades, to not have unique factorizations.  Unlike blades, however, the
size of a spade's factorization can very.  This leads us to the following definition.
\begin{defn}[Spade Rank]
Given any spade $M_r\in\G$, the rank of the spade $M_r$, denoted $\rank(M_r)$, is the
smallest integer $s\in[0,r]$ such that $M_r$ may be rewritten as a geometric product of $s$ vectors.
\end{defn}
Clearly, if $0\neq\bigwedge_{i=1}^r m_i=\langle M_r\rangle_r$, then $\rank(M_r)=r$.
The converse of this statement, however is not immediately clear, to say the least.
In other words, if $0=\bigwedge_{i=1}^r m_i=\langle M_r\rangle_r$, then there does not
appear to be any easy proof that $\rank(M_r)<r$.  For now, we will have to make due with the following lemma.
\begin{lem}\label{lem_spade_no_dup_in_factorization}
For any given invertible spade $M_r\in\G$, if there exist integers $1\leq i<j\leq r$ such that $m_i=m_j$, and $m_i$ is invertible, then $\rank(M_r)\leq r-2$.
\end{lem}
\begin{proof}
This is trivial in the case that $j=i+1$.  In the case that $j=i+2$, simply notice that
\begin{equation*}
m_im_{i+1}m_j = m_im_{i+1}m_i = 2(m_i\cdot m_{i+1})m_i-m_i^2 m_{i+1}.
\end{equation*}
In the case that $j>i+2$, we see that
\begin{equation*}
m_i\left(\prod_{k=i+1}^{j-1}m_k\right)m_j = m_i^2\prod_{k=i+1}^{j-1}m_im_km_i^{-1}.
\end{equation*}
\end{proof}

For completeness, we now give a formal definition of a blade.
\begin{defn}[Blade]
An element $B_r\in\G$ is called an $r$-\emph{blade} if and only if there exists a linearly independent set of $r$
vectors $\{b_i\}_{i=1}^r$ such that
\begin{equation}\label{equ_B_r}
B_r = \bigwedge_{i=1}^r b_i.
\end{equation}
\end{defn}

\begin{lem}\label{lem_lin_indep_subblades}
Letting $B_r^{(i)}$ denote the $(r-1)$-blade
\begin{equation*}
B_r^{(i)} = \bigwedge_{\substack{j=1\\j\neq i}}^r b_i,
\end{equation*}
the set of $r$ blades $\{B_r^{(i)}\}_{i=1}^r$ is linearly independent.
\end{lem}
\begin{proof}
Supposing to the contrary, and without loss of generality, let
\begin{equation*}
B_{r-1} = B_r^{(r)} = \sum_{i=1}^{r-1}\alpha_i B_r^{(i)} = \left(\sum_{i=1}^{r-1}\alpha_i B_{r-1}^{(i)}\right)\wedge b_r.
\end{equation*}
Now notice that
\begin{equation*}
0\neq B_r = B_{r-1}\wedge b_r = B_r^{(r)}\wedge b_r = \left(\sum_{i=1}^{r-1}\alpha_i B_r^{(i)}\right)\wedge b_r = 0,
\end{equation*}
which is clearly a contradiction.
\end{proof}

We will need a result similar to Lemma~\ref{lem_lin_indep_subblades} as concerning spades.  It is as follows.
\begin{lem}\label{lem_lin_indep_subspades}
Letting $M_r^{(i)}$ denote the spade
\begin{equation*}
M_r^{(i)} = \prod_{\substack{j=1\\j\neq i}}^r m_i,
\end{equation*}
if $0\neq\bigwedge_{i=1}^r m_i$, then the set $\{M_r^{(i)}\}_{i=1}^r$ is a linearly independent set.
\end{lem}
\begin{proof}
By Lemma~\ref{lem_solution_intersection}, it suffices to show that the set $\{\langle M_r^{(i)}\rangle_{r-1}\}_{i=1}^r$ is a linearly independent set.
Now since $0\neq\bigwedge_{i=1}^r m_i$, it is clear that
\begin{equation*}
\langle M_r^{(i)}\rangle_{r-1} = \bigwedge_{\substack{j=1\\j\neq i}}^r m_i.
\end{equation*}
Seeing this, the linear independence of the set $\{\langle M_r^{(i)}\rangle_{r-1}\}_{i=1}^r$ follows immediately
from Lemma~\ref{lem_lin_indep_subblades}.
\end{proof}

We turn now to the establishment of some identities that will be important to our cause.

\subsection{Identities Involving Blades}

Letting $a$ denote a vector, and $B_r$ a blade of grade $r$ having the factorization
given in equation \eqref{equ_B_r}, recall that
\begin{equation}\label{equ_aBr_is_a_dot_Br_and_a_wedge_Br}
aB_r = a\cdot B_r + a\wedge B_r.
\end{equation}
Recalling also the commutativities of $a$ with $B_r$ in the inner and outer products as
\begin{align}
a\cdot B_r &= -(-1)^r B_r\cdot a,\label{equ_a_dot_Br_commutativity} \\
a\wedge B_r &= (-1)^r B_r\wedge a,\label{equ_a_wedge_Br_commutativity}
\end{align}
we find that
\begin{align}
a\cdot B_r &= \frac{1}{2}a\cdot B_r - \frac{1}{2}(-1)^r B_r\cdot a\nonumber \\
 &= \frac{1}{2}(aB_r - a\wedge B_r - (-1)^r(B_ra - B_r\wedge a))\nonumber \\
 &= \frac{1}{2}(aB_r-(-1)^rB_ra),\label{equ_a_dot_Br}
\end{align}
and that
\begin{align}
a\wedge B_r &= \frac{1}{2}a\wedge B_r + \frac{1}{2}(-1)^r B_r\wedge a\nonumber \\
 &= \frac{1}{2}(aB_r - a\cdot B_r + (-1)^r(B_ra - B_r\cdot a))\nonumber \\
 &= \frac{1}{2}(aB_r+(-1)^rB_ra).\label{equ_a_wedge_Br}
\end{align}
Now letting $a$ and $b$ each denote a vector, it is not hard to show that for all $r\geq 1$, we have
\begin{equation}\label{equ_a_dot_b_wedge_Br_identity}
a\cdot(b\wedge B_r) + b\wedge(a\cdot B_r) = (a\cdot b)B_r.
\end{equation}
To that end, we apply equations \eqref{equ_a_dot_Br} and \eqref{equ_a_wedge_Br} in writing
\begin{align*}
a\cdot(b\wedge B_r)
 &= \frac{1}{2}\left(a\frac{1}{2}\left(bB_r + (-1)^rB_rb\right)-(-1)^{r+1}\frac{1}{2}\left(bB_r+(-1)^rB_rb\right)a\right) \\
 &= \frac{1}{4}\left(baB_r + (-1)^raB_rb + (-1)^rbB_ra + B_rba\right), \\
b\wedge(a\cdot B_r)
 &= \frac{1}{2}\left(b\frac{1}{2}\left(aB_r-(-1)^rB_ra\right)+(-1)^{r-1}\frac{1}{2}\left(aB_r-(-1)^rB_ra\right)b\right) \\
 &= \frac{1}{4}\left(baB_r - (-1)^rbB_ra - (-1)^raB_rb + B_rab\right),
\end{align*}
from which it is easy to see that
\begin{align*}
a\cdot(b\wedge B_r)+b\wedge(a\cdot B_r) &= \frac{1}{4}(ab+ba)B_r + \frac{1}{4}B_r(ba+ab) \\
 &= \frac{1}{2}(a\cdot b)B_r + \frac{1}{2}B_r(b\cdot a) = (a\cdot b)B_r.
\end{align*}
Similarly, we must note that for all $r>1$, we have
\begin{equation}\label{equ_a_dot_b_dot_Br_identity}
a\cdot(b\cdot B_r) = -b\cdot(a\cdot B_r).
\end{equation}
To see this, we apply equation \eqref{equ_a_dot_Br} in writing
\begin{align*}
a\cdot(b\cdot B_r)
 &= \frac{1}{2}\left(a\frac{1}{2}\left(bB_r-(-1)^rB_rb\right)-(-1)^{r-1}\frac{1}{2}\left(bB_r-(-1)^rB_rb\right)a\right) \\
 &= \frac{1}{4}\left(abB_r - (-1)^raB_rb + (-1)^rbB_ra - B_rba\right),
\end{align*}
Then, by substitution, we can immediately write
\begin{equation*}
b\cdot(a\cdot B_r) = \frac{1}{4}\left(baB_r - (-1)^rbB_ra + (-1)^raB_rb - B_rab\right).
\end{equation*}
Adding these, we then see that
\begin{align*}
a\cdot (b\cdot B)+b\cdot(a\cdot B)
 &= \frac{1}{4}\left(abB_r+baB_r\right)-\frac{1}{4}\left(B_rba+B_rab\right) \\
 &= \frac{1}{4}\left(ab+ba\right)B_r-\frac{1}{4}B_r\left(ba+ab\right) \\
 &= \frac{1}{2}(a\cdot b)B_r - \frac{1}{2}B_r(b\cdot a) = 0.
\end{align*}
Note that we may have arrived at this conclusion sooner had we written
\begin{equation*}
a\cdot(b\cdot B_r) = (a\wedge b)\cdot B_r = -(b\wedge a)\cdot B_r = -b\cdot(a\cdot B_r).
\end{equation*}

We now wish to express the inner product $a\cdot B_r$ as a sum of blades.
Since the case $r=1$ is trivial, we begin by writing, for all $r>1$,
\begin{align}
a\cdot B_r
 &= a\cdot(B_{r-1}\wedge b_r)\nonumber \\
 &= (-1)^{r-1}a\cdot(b_r\wedge B_{r-1})\label{a_dot_Br_stepA} \\
 &= -(-1)^r\left(-b_r\wedge(a\cdot B_{r-1})+(a\cdot b_r)B_{r-1}\right)\label{a_dot_Br_stepB} \\
 &= -(-1)^r\left(-(-1)^r(a\cdot B_{r-1})\wedge b_r+(a\cdot b_r)B_{r-1}\right)\nonumber \\
 &= (a\cdot B_{r-1})\wedge b_r - (-1)^r(a\cdot b_r)B_{r-1}.\label{equ_a_dot_Br_recursive}
\end{align}
Here, we've gone from equation \eqref{a_dot_Br_stepA} to that of \eqref{a_dot_Br_stepB} by
applying the identity given in equation \eqref{equ_a_dot_b_wedge_Br_identity}.

Applied recursively, it is easy to see here from equation \eqref{equ_a_dot_Br_recursive} that an expansion of
$a\cdot B_r$ as a sum of blades is given by
\begin{equation}\label{equ_a_dot_Br_sum_of_blades}
a\cdot B_r = \langle B_r\rangle_0a - \sum_{i=1}^r(-1)^i(a\cdot b_i)\bigwedge_{\substack{j=1\\j\neq i}}^r b_j.
\end{equation}
One might also simply use equation \eqref{equ_a_dot_Br_recursive} to give an inductive
argument of equation \eqref{equ_a_dot_Br_sum_of_blades}.

Notice that for all $r>0$, the term $\langle B_r\rangle_0a$ vanishes in equation \eqref{equ_a_dot_Br_sum_of_blades},
yet its presence allows us the case $r=0$ if we define the summation to be zero in the vacuous case.

Having established equation \eqref{equ_a_dot_Br_sum_of_blades}, it is instructive to show that $a\cdot B_r$ is,
although it is certainly not immediately obvious, a blade of grade $r-1$.  To that end, we write, for all $r>1$,
\begin{equation*}
a\cdot B_r = (a\cdot B_{r-1})\wedge\left(b_r - \frac{a\cdot b_r}{a\cdot b_{r-1}}b_{r-1}\right),
\end{equation*}
with the understanding that if $a\cdot b_{r-1}$ is zero, we can anti-commute vector factors in equation \eqref{equ_a_dot_Br_sum_of_blades}
until this is the case, or else $a\cdot B_r$ is zero anyway.
An inductive argument can now be easily made that $a\cdot B_r$ is indeed a blade of grade $r-1$.
Notice that this proof works in any geometric algebra, regardless of the associated bilinear form.  In a euclidean geometric
algebra, an easier proof is had by writing
\begin{equation*}
a\cdot B_r = (a_{\perp} + a_{\parallel})\cdot B_r = a_{\parallel}\cdot B_r,
\end{equation*}
where $a_{\perp}$ is the orthogonal rejection $a$ from $B_r$, while $a_{\parallel}$ is the
orthogonal projection of $a$ down onto $B_r$.  The blade $B_r$ can now be orthogonalized,
with $a_{\parallel}$ as a principle factor, using the Gram-Schmidt orthogonalization process.\footnote{This process
cannot always be performed on blades taken from a non-euclidean geometric algebra.  To see this, consider rewriting
$a\wedge b$ as $a\wedge(b+\lambda a)$ where $a\cdot(b+\lambda a)=0$.  In a non-euclidean geometric algebra,
no such scalar $\lambda$ may exist due to $a$ being null.  For a description of the Gram-Schmidt process, see \cite{}.}
This factor then falls out quite easily, and we're left with a blade of grade $r-1$.

\subsection{Identities Involving Spades}

Letting $M_r$ denote a spade having the factorization given in equation \eqref{equ_M_r}, recall that
\begin{equation*}
M_r = \sum_{i=1}^r\left\langle M_r\right\rangle_i,
\end{equation*}
To be more precise, if $r$ is even,
\begin{equation}\label{equ_Mr_even}
M_r = \sum_{i=0}^{r/2}\left\langle M_r\right\rangle_{2i},
\end{equation}
while if $r$ is odd, we have
\begin{equation}\label{equ_Mr_odd}
M_r = \sum_{i=1}^{(r+1)/2}\left\langle M_r\right\rangle_{2i-1}.
\end{equation}
To see this, consider first the trivial case of $r=0$; then, for any $r>0$, the equation
\begin{equation}\label{equ_Mr_split}
M_r = M_{r-1}m_r = \langle M_{r-1}\rangle_1^r\cdot m_r + \langle M_{r-1}\rangle_1^r\wedge m_r + \langle M_{r-1}\rangle_0 m_r.
\end{equation}
Here we have extended our notation $\langle\cdot\rangle_i^j$ to mean a culling of all enclosed blades not of a grade falling
in the interval $[i,j]$.  Put another way, we have
\begin{equation*}
\langle M_r\rangle_i^j = \sum_{k=i}^j\langle M_r\rangle_k.
\end{equation*}

An inductive hypothesis can now be stated that equations \eqref{equ_Mr_even} and \eqref{equ_Mr_odd} hold for $r-1$.
If $r$ is even, then, by our inductive hypothesis, $M_{r-1}$, when expanded as a sum of blades, consists only of blades of odd grade,
and it is clear that equation \eqref{equ_Mr_split} becomes \eqref{equ_Mr_even}.  If $r$ is odd, then, by our inductive hypothesis, $M_{r-1}$, when expanded as
a sum of blades, consists only of blades of even grade, and it is clear that equation \eqref{equ_Mr_split} becomes \eqref{equ_Mr_odd}.

Now let $a$ be a vector, and convince yourself that
\begin{align}
a\cdot M_r &= -(-1)^r M_r\cdot a,\label{equ_a_dot_Mr_commutativity} \\
a\wedge M_r &= (-1)^r M_r\wedge a.\label{equ_a_wedge_Mr_commutativity}
\end{align}
Refer to equations \eqref{equ_a_dot_Br_commutativity} and \eqref{equ_a_wedge_Br_commutativity} to see this.

We now turn our attention to the following identity.
\begin{equation}\label{equ_gr_zero_part_of_Mr}
\langle M_r\rangle_0 = \langle M_{r-1}\rangle_1\cdot m_r
\end{equation}
Note that this is trivial in the case that $r$ is odd, since neither $M_r$ nor $M_{r-1}$ have parts of grade zero nor one, respectively.
Letting $r$ be even, we write
\begin{equation*}
M_r = M_{r-1}m_r = M_{r-1}\cdot m_r + M_{r-1}\wedge m_r - \langle M_r\rangle_0 m_r.
\end{equation*}
Now taking the grade zero part of both sides, we get
\begin{equation*}
\langle M_r\rangle_0 = \langle M_{r-1}\cdot m_r\rangle_0 = \langle M_{r-1}\rangle_1\cdot m_r.
\end{equation*}

We now wish to express the inner product $a\cdot M_r$ as a sum of spades.
Since the case $r=1$ is trivial, we begin by writing, for all $r>1$,
\begin{align}
a\cdot M_r &= a\cdot (M_{r-1}m_r)\nonumber \\
 &= a\cdot((\langle M_{r-1}\rangle_0 + \langle M_{r-1}\rangle_1 + \langle M_{r-1}\rangle_2^r)m_r)\nonumber \\
 &= \langle M_{r-1}\rangle_0a\cdot m_r + (\langle M_{r-1}\rangle_1\cdot m_r)a\nonumber \\
 &+ (a\cdot\langle M_{r-1}\rangle_1)m_r - (a\cdot m_r)\langle M_{r-1}\rangle_1 + a\cdot(\langle M_{r-1}\rangle_2^rm_r).\label{equ_a_dot_Mr_stepA}
\end{align}
We will return to this equation momentarily.  Until then, to ease notation, let us write $M=\langle M_{r-1}\rangle_2^r$ and see that
\begin{align}
a\cdot(Mm_r)
 &= a\cdot(M\cdot m_r + M\wedge m_r)\nonumber \\
 &= -(-1)^{r-1}a\cdot(m_r\cdot M) + (-1)^{r-1}a\cdot(m_r\wedge M)\label{equ_a_dot_Mr_stepB} \\
 &= (-1)^rm_r\cdot(a\cdot M) - (-1)^r\left[-m_r\wedge(a\cdot M)+(a\cdot m_r)M\right]\label{equ_a_dot_Mr_stepC} \\
 &= (a\cdot M)\cdot m_r + (a\cdot M)\wedge m_r - (-1)^r(a\cdot m_r)M\nonumber \\
 &= (a\cdot M)m_r - (-1)^r(a\cdot m_r)M.\label{equ_a_dot_Mr_stepD}
\end{align}
Note here our use of equations \eqref{equ_a_dot_b_dot_Br_identity} and \eqref{equ_a_dot_b_wedge_Br_identity} to
arrive at equation \eqref{equ_a_dot_Mr_stepC} from \eqref{equ_a_dot_Mr_stepB}.

Returning now to equation \eqref{equ_a_dot_Mr_stepA}, if we plug equation \eqref{equ_a_dot_Mr_stepD} into it
under the assumption that $r$ is odd, we get
\begin{equation}
a\cdot M_r = (a\cdot M_{r-1})m_r + (a\cdot m_r)M_{r-1} - \langle M_{r-1}\rangle_0am_r.
\end{equation}
And if we plug equation \eqref{equ_a_dot_Mr_stepD} into equation \eqref{equ_a_dot_Mr_stepA} under the assumption
that $r$ is even, we get
\begin{equation}
a\cdot M_r = (a\cdot M_{r-1})m_r - (a\cdot m_r)M_{r-1} + (\langle M_{r-1}\rangle_1\cdot m_r)a.
\end{equation}
It then follows, despite the parity of $r$, that
\begin{align}
a\cdot M_r &= (a\cdot M_{r-1})m_r - (-1)^r(a\cdot m_r)M_{r-1}\nonumber \\
 &- \langle M_{r-1}\rangle_0am_r + \langle M_r\rangle_0a.\label{equ_a_dot_Mr_recursive}
\end{align}
Note the use of equation \eqref{equ_gr_zero_part_of_Mr} here in our arrival at equation \eqref{equ_a_dot_Mr_recursive}.

Applied recursively, it is now easy to see from equation \eqref{equ_a_dot_Mr_recursive} that an expansion of
$a\cdot M_r$ as a sum of spades is given by
\begin{equation}\label{equ_a_dot_Mr_sum_of_spades}
a\cdot M_r = \langle M_r\rangle_0a - \sum_{i=1}^r(-1)^i(a\cdot m_i)\prod_{\substack{j=1\\j\neq i}}^rm_j.
\end{equation}
To see this, consider an inductive argument.  The cases $r=0$ and $r=1$ follow trivially by inspection.
Now make the inductive hypothesis that equation \eqref{equ_a_dot_Mr_sum_of_spades} holds for a fixed case $r-1$.
Then, applying the recursive formula \eqref{equ_a_dot_Mr_recursive} to the equation in \eqref{equ_a_dot_Mr_sum_of_spades},
adjusted for the case $a\cdot M_{r-1}$, we get equation \eqref{equ_a_dot_Mr_sum_of_spades}, thereby completing
our proof by induction.

It is very interesting now to compare this equation \eqref{equ_a_dot_Mr_sum_of_spades} with that of \eqref{equ_a_dot_Br_sum_of_blades}.
One equation is had by the other by a replacement of all outer products with geometric products, or vice-versa.

Having shown that $a\cdot B_r$ was a blade of grade $r-1$, we must consider here whether $a\cdot M_r$ can be written
as a product of $r-1$ vectors.  With that in mind, we write
\begin{align}
a\cdot M_r - \langle M_r\rangle_0a &= \sum_{i=1}^r\alpha_i M_r^{(i)}\nonumber \\
 &= \left[\sum_{i=1}^{r-1}\alpha_iM_{r-1}^{(i)}\right]\left(m_r + \alpha_r\left[\sum_{i=1}^{r-1}\alpha_iM_{r-1}^{(i)}\right]^{-1}M_{r-1}\right),\label{equ_factor_a_dot_M_r}
\end{align}
where $\alpha_i=-(-1)^i(a\cdot m_i)$.  Now, if an inverse of $a\cdot M_{r-1}-\langle M_{r-1}\rangle_0a$ does exist, then
it is probably of the form
\begin{equation*}
\left[\sum_{i=1}^{r-1}\alpha_iM_{r-1}^{(i)}\right]^{-1} = \sum_{i=1}^{r-1}\beta_i\left(M_{r-1}^{(i)}\right)^{\sim}.
\end{equation*}
Assuming a solution to this equation in each $\beta_i$ exists, we can go on to write
\begin{equation*}
\sum_{i=1}^{r-1}\beta_i\left(M_{r-1}^{(i)}\right)^{\sim}M_{r-1} = \sum_{i=1}^{r-1}\beta_i \left(\prod_{j=1}^{i-1}m_i^2\right)\tilde{V}_{i+1}m_iV_{i+1},
\end{equation*}
where $V_i$ is given by
\begin{equation*}
V_i = \sum_{j=i}^{r-1} m_i.
\end{equation*}
Looking back at equation \eqref{equ_factor_a_dot_M_r}, we can see now how a vector could be factored
out of $a\cdot M_r - \langle M_r\rangle_0a$ in terms of the geometric product.

\section{Blades And Spades As Representatives Of Geometric Sets}

At last we have now enough ground covered to begin a treatment of geometric set representation by blades and spades.
We start by showing that any element $E$ of $\G$ is representative of a geometric set as follows.

\begin{defn}
Letting the function $\gd:\G\to P(\F^n)$ be defined as
\begin{equation*}
\gd(E) = \{x\in\F^n|p(x)\cdot E=\langle E\rangle_0p(x)\},
\end{equation*}
we call $\gd(E)$ the geometric set represented by $E$.
\end{defn}

\begin{lem}
For any element $E\in\G$, the set $\gd(E)$ is a geometric set.
\end{lem}
\begin{proof}
If it can be shown, for every integer $k\in[0,\dim(\V)]$, that $\gd(\langle E\rangle_k)$ is a geometric set,
then $\gd(E)$ is a geometric set by Lemma~\ref{lem_solution_intersection}, since
\begin{equation*}
\gd(E) = \bigcap_{k=1}^{\dim{\V}}\gd(\langle E\rangle_k).
\end{equation*}
The case $k=0$ is trivial.  Letting $k>0$, it is clear by equation \eqref{equ_a_dot_Br_sum_of_blades} that
\begin{equation*}
0 = p(x)\cdot\langle E\rangle_k = \sum_i (p(x)\cdot v_i)B_i,
\end{equation*}
where each $B_i$ is a blade of grade $k-1$.  We now see that the set of all solutions $x$
to this equation gives us a geometric set by Lemma~\ref{lem_geo_set_lin_dep}.
\end{proof}

\begin{lem}\label{lem_all_geo_sets_rep_by_blades}
For every geometric set that can be represented by an element $E\in\G$,
there exists an $r$-blade $B_r\in\G$ such that $\gd(E)=\gd(B_r)$.
\end{lem}
\begin{proof}
By virtue of being generated by $E$, it is not too far a stretch to conclude that
there must exist a finite set of vectors $\{v_i\}_{i=1}^r\subset\V$ such that
$\gd(E)=G(\{v_i\}_{i=1}^r)$.  Then, by Lemma~\ref{lem_reduce_vec_set}, a linearly independent subset $\{v_i\}_{i=1}^s$
of $\{v_i\}_{i=1}^r$ can be found such that $G(\{v_i\}_{i=1}^s)=G(\{v_i\}_{i=1}^r)$.
Lastly, we see that $G(\{v_i\}_{i=1}^s)=\gd(\bigwedge_{i=1}^s v_i)$.
\end{proof}

\begin{lem}
If for every vector $v\in\V$, the expression $p(x)\cdot v$ is a polynomial in the components of $x$,
then for every algebraic set $G(\{v_i\})$, where $\{v_i\}\subseteq\V$, there exists an $r$-blade $B_r\in\G$ such that
$G(\{v_i\})=\gd(B_r)$.
\end{lem}
\begin{proof}
By the Hilbert Basis Theorem (see \cite[p. 204]{Garrity13}), there exists a finite subset $\{v_i\}_{i=1}^r\subset\{v_i\}$
such that $G(\{v_i\}_{i=1}^r)=G(\{v_i\})$.  Then, by Lemma~\ref{lem_reduce_vec_set}, a linearly independent subset $\{v_i\}_{i=1}^s$
of $\{v_i\}_{i=1}^r$ can be found such that $G(\{v_i\}_{i=1}^s)=G(\{v_i\}_{i=1}^r)$.
Lastly, we see that $G(\{v_i\}_{i=1}^s)=\gd(\bigwedge_{i=1}^s v_i)$.
\end{proof}

What Lemma~\ref{lem_all_geo_sets_rep_by_blades} tells us is that if we only use blades to represent geometric sets, then we're not missing out on any geometric sets
that we could have otherwise represented using any other type of element of $\G$.  Seeing by our next
lemma that the same holds true for spades, the question of which type we use (blades or spades?) depends
on what each representation can do for us.

\begin{lem}\label{lem_convert_blade_to_spade}
For every blade $B_r\in\G$, there exists a spade $M_r\in\G$ such that $\gd(M_r)=\gd(B_r)$.
\end{lem}
\begin{proof}
In a euclidean geometric algebra, the Gram-Schmidt process allows us to find an orthogonal factoriztion
of $B_r$ so that
\begin{equation*}
B_r = \bigwedge_{i=1}^r b_i = \prod_{i=1}^r b_i.
\end{equation*}
Now simply let each $m_i=b_i$.

In a non-euclidean geometric algebra, an orthogonal factorization can still be found.
See \cite[p. 88]{Doran03}.
\end{proof}

Lemma~\ref{lem_convert_blade_to_spade} now begs the question of how we might convert a spade representation into that of a blade.

\begin{lem}
If $M_r\in\G$ is a spade and $\langle M_r\rangle_s$ is the highest, non-zero grade part of $M_r$, then
\begin{equation*}
\gd(M_r) = \gd(\langle M_r\rangle_s).
\end{equation*}
\end{lem}

\begin{equation*}
\gd(M_r) = \bigcap_{i=0}^r \gd(\langle M_r\rangle_i).
\end{equation*}

What if $\gd(\langle M_r\rangle_i)\subset\gd(\langle M_r\rangle_j)$ where $i>j$?

If $a\cdot B=0$ and $C$ is a subspace of $B$, then $a\cdot C=0$.

If $i<j$, is $\langle M_r\rangle_i$ a subspace of $\langle M_r\rangle_j$?

Easy to show that all parts of spade are blades.  Show nested subspace sequence.  That may turn out to be harder than I thought.

\subsection{What Blades Can Do For Us}

Talk about meet and join...

\subsection{What Spades Can Do For Us}

Can we learn anything new here?!

\section{Examples In The Conformal Model}

\section{Closing Remarks}

\begin{thebibliography}{9}

\bibitem{Parkin15}
S. Parkin,
\emph{An Introduction To Geometric Sets}.
Advances in Applied Clifford Algebras, Volume 25, Issue Unknown, pp. 639-655, 2015.

\bibitem{Perwass09}
C. Perwass,
\emph{Geometric Algebra with Applications in Engineering}
Springer-Verlag Berlin Heidelberg, 2009.

\bibitem{Garrity13}
T. Garrity, et. al.,
\emph{Algebraic Geometry, A Problem Solving Approach}
American Mathematical Society, Institute for Advanced Study

\bibitem{Hestenes99}
D. Hestenes,
\emph{New Foundations for Classical Mechanics}
Kluwer Academic Publishers, 1999.

\bibitem{Doran03}
C. Doran, et. al.,
\emph{Geometric Algebra for Physicists}
Cambridge University Press, 2003.

\end{thebibliography}

\end{document}