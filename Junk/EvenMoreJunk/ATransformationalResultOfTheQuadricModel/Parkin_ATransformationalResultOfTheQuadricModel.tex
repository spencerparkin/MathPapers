\documentclass{birkjour}

\usepackage{amsmath}
\usepackage{amssymb}
\usepackage{amsthm}
\usepackage{graphicx}
\usepackage{float}

\newtheorem{thm}{Theorem}[section]
 \newtheorem{cor}[thm]{Corollary}
 \newtheorem{lem}[thm]{Lemma}
 \newtheorem{prop}[thm]{Proposition}
 \theoremstyle{definition}
 \newtheorem{defn}[thm]{Definition}
 \theoremstyle{remark}
 \newtheorem{rem}[thm]{Remark}
 \newtheorem*{ex}{Example}
 \numberwithin{equation}{section}

\newcommand{\G}{\mathbb{G}}
\newcommand{\V}{\mathbb{V}}
\newcommand{\Vb}{\mathbb{\overline{V}}}
\newcommand{\W}{\mathbb{W}}
\newcommand{\R}{\mathbb{R}}
\newcommand{\Alpha}{A}
\newcommand{\nvao}{o}
\newcommand{\nvai}{\infty}

\begin{document}

\title{A Transformational Result Of The Quadric Model}

\author{Spencer T. Parkin}
\address{%
2113 S. Claremont Dr.\\
Bountiful, Utah  84010\\
USA}
\email{spencer.parkin@gmail.com}

\numberwithin{equation}{section}

\subjclass{Primary 14J70; Secondary 14J29}

\keywords{Quadric Surface, Geometric Algebra, Quadric Model}

%\dedicatory{To Melinda and Naomi}

\begin{abstract}
An important feature of the conformal model is found
to be possessed by the quadric model.  Specifically, it is
shown in this paper that the action of a versor on a point
reveals the action of this versor on any geometry of the model.
This leads to a possible direction in which we might look for a better model of quadric surfaces.
\end{abstract}

\maketitle

\section{Introduction}

It is well known that, with the exceptoin of flat points, all geometries of
the conformal model may be expressed directly as the outer product
of one or more vectors representative of points.  It then immediately
follows by the outermorphic property of versor conjugation, (see
equation \eqref{equ_versor_identity_wedge} below), that the
action of a versor on a conformal point reveals the action of this versor
on any point on the surface of a conformal geometry, and therefore
the conformal geometry itself.  This feature of the conformal model
facilitates the search for versors performing desired actions, and the
analysis of what action a given versor performs.  In this paper we
will find that the quadric model possess its own version of this very feature.
We'll also find that this feature may help point us in a direction of where
we might look for a better model of quadric geometry.

The definitions and results of \cite{Parkin12} will be assumed for
the remainder of this paper so that no background will need be
given before we can dive into the new material.

\section{The Versor Identities Of Geometric Algebra}

There are two well known identities in geometric algebra involving versors.
For any two vectors $p,b\in\W$ and any versor $V\in\G$, we have
\begin{equation}\label{equ_versor_identity_wedge}
V(p\wedge b)V^{-1} = VpV^{-1}\wedge VbV^{-1},
\end{equation}
as well as
\begin{equation}\label{equ_versor_identity_dot}
p\cdot b = VpV^{-1}\cdot VbV^{-1}.
\end{equation}
Proofs of these identities may be found in \cite{Parkin12_intro}.
A perhaps less known identity, however, is the following.
\begin{equation}\label{equ_versor_identity}
VpV^{-1}\cdot b = p\cdot V^{-1}bV
\end{equation}
Let us give a proof of it now.  We will proceed by induction.
Letting $v\in\W$ be a vector, it is easy to see that
\begin{equation}\label{equ_induction_start}
vpv^{-1}\cdot b = \frac{2(v\cdot p)(v\cdot b)}{v^2} - p\cdot b = p\cdot v^{-1}bv.
\end{equation}
Assuming now that the identity \eqref{equ_versor_identity} holds for a versor
composed as the geometric product of some fixed number of vectors, the proof of identity
\eqref{equ_versor_identity} follows by induction with
\begin{align}
& vVp(vV)^{-1}\cdot b \\
=\;& vVpV^{-1}v^{-1}\cdot b \\
=\;& VpV^{-1}\cdot v^{-1}bv & \mbox{by equation \eqref{equ_induction_start},} \\
=\;& p\cdot V^{-1}v^{-1}bvV & \mbox{by our inductive hypothesis,} \\
=\;& p\cdot (vV)^{-1}bvV.
\end{align}

In the next section we'll make use of identity \eqref{equ_versor_identity} as well as
\eqref{equ_versor_identity_wedge} to prove the main result.  The identity
\eqref{equ_versor_identity_dot} has many use cases while working in $\G$
of \cite{Parkin12}, but will not be needed to prove the main result.
Looking back, however, it is not hard to see that \eqref{equ_versor_identity_dot}
implies \eqref{equ_versor_identity} in an easier proof than what has just been given.

\section{Relating The Action Of Versors On Quadrics To That Of Points}

It was established in \cite{Parkin12} that quadrics $E\in\G$ are bivectors
of the form
\begin{equation}\label{equ_quadric_form}
E = \sum_{i=1}^k a_i\wedge\overline{b_i}
\end{equation}
where for each integer $i\in[1,k]$, each of $a_i$ and $b_i$ are
taken from $\V$.  Being a quadric, the set of all projective points
$p\in\V$ on $E$ is given by the set of all projective points $p\in\V$
such that
\begin{equation}\label{equ_quadric_representation}
0 = p\wedge\overline{p}\cdot E.
\end{equation}
Clearly now, if we can visualize the quadric $E$, and if we can understand the action
of a versor $V$ on a projective point $p$, then our imaginations are likely able
to visualize the geometry that is the set of all projective points $p\in\V$ such that
\begin{equation}\label{equ_quadric_transformed}
0 = VpV^{-1}\wedge\overline{VpV^{-1}}\cdot E.
\end{equation}
For example, if $V$ translates $p$ by a direction vector $t$, then $E$ must be
translated by the direction vector $-t$.  Similarly, if $V$ rotates $p$ on
an axis $a$ by an angle $\theta$, then $E$ must be rotated by an
angle $-\theta$ about the axis $a$.  Of course, no
claim is being made here that either of such versor exists.  (A translation
versor is not known to exist, but it has been shown in \cite{Parkin12} that the
rotation versor does exist.)  The idea, however,
that the action of $V$ on $p$ translates into the inverse
action of $V$ on $E$, should be well understood.

We will now proceed to show that the geometry represented in
equation \eqref{equ_quadric_transformed} is the very geometry
represented by the quadric $(V\overline{V})^{-1}EV\overline{V}$,
provided that $V$ has the property that for all vectors $v\in\V$,
we have
\begin{equation}\label{equ_versor_invariant_property}
\begin{array}{cl}
v=\overline{V}v\overline{V^{-1},} \\
\overline{v}=V\overline{v}V^{-1}, & \mbox{(which follows from $v=\overline{V}v\overline{V^{-1}}$),} \\
\end{array}
\end{equation}
as well as
\begin{equation}\label{equ_versor_closure_property}
\begin{array}{cl}
VvV^{-1}\in\V, \\
\overline{V}\overline{v}\overline{V^{-1}}\in\overline{\V}, & \mbox{(which follows from $VvV^{-1}\in\V$),}
\end{array}
\end{equation}
which is to say that $V$ leaves vectors in $\overline{\V}$ invariant under versor
conjugation as $\overline{V}$ leaves vectors in $\V$ invariant under versor conjugation,
as well as that conjugation of a vector in $\V$ by the versor $V$ is an operation closed in $\V$.
It will then immediately follow
that if we understand the action of $V^{-1}$ on a projective point $p$, then
we understand the action of $V$ on $E$ as
\begin{equation}
V\overline{V}E(V\overline{V})^{-1}.
\end{equation}

The proof is straight forward as it follows from the equality of \eqref{equ_start} with \eqref{equ_finish}.
{\allowdisplaybreaks
\begin{align}
 & VpV^{-1}\wedge\overline{VpV^{-1}}\cdot E\label{equ_start} \\
=\;& \sum_{i=1}^k VpV^{-1}\wedge\overline{VpV^{1}}\cdot a_i\wedge\overline{b_i}\label{equ_second} \\
=\;& -\sum_{i=1}^k (VpV^{-1}\cdot a_i)(\overline{VpV^{-1}}\cdot\overline{b_i}) & \mbox{by property \eqref{equ_versor_closure_property},} \\
=\;& -\sum_{i=1}^k (p\cdot V^{-1}a_iV)(\overline{p}\cdot\overline{V^{-1}b_iV}) & \mbox{by identity \eqref{equ_versor_identity},} \\
=\;& \sum_{i=1}^k p\wedge\overline{p}\cdot V^{-1}a_iV\wedge\overline{V^{-1}b_iV} &\mbox{by property \eqref{equ_versor_closure_property},} \\
=\;& \sum_{i=1}^k p\wedge\overline{p}\cdot (V\overline{V})^{-1}a_iV\overline{V}\wedge
(V\overline{V})^{-1}\overline{b_i}V\overline{V} &\mbox{by property \eqref{equ_versor_invariant_property},}\label{equ_third_to_last} \\
=\;& \sum_{i=1}^k p\wedge\overline{p}\cdot (V\overline{V})^{-1}(a_i\wedge\overline{b_i})V\overline{V} & \mbox{by identity \eqref{equ_versor_identity_wedge},} \\
=\;& p\wedge\overline{p}\cdot (V\overline{V})^{-1}EV\overline{V}.\label{equ_finish}
\end{align}}

Of course, this is just one of perhaps many algebraic routes one could take to prove the
identity that \eqref{equ_start} is \eqref{equ_finish}.  In fact, it is not hard to see that
a shorter route can be found from \eqref{equ_second} to \eqref{equ_third_to_last} using only
\eqref{equ_versor_identity_dot} and \eqref{equ_versor_invariant_property}.  Nevertheless,
the route shown above illustrates algebraic techniques that are useful as their need is frequently encountered.

The property \eqref{equ_versor_invariant_property} is not unreasonable at all
since a versor providing any action on a projective point $p\in\V$ must come from $\G(\V)$ anyway,
and by so doing, naturally leaves vectors in $\overline{\V}$ untouched, up to scale.
In fact, the condition of \eqref{equ_versor_invariant_property} may be relaxed to allow
for a sign change, as such a change leaves the geometry represented by a bivector invariant.

\section{The Search For A Better Model}

The main result having now been established, we may harness it as a tool in the search
for a better model of quadric surfaces.  By this it is meant a model that offers
more of the desired types of transformations by versors.  One possible approach
that is motivated by the main result is that of replacing each of $\G(\V)$ and
$\G(\overline{\V})$ as isomorphic Minkowski sub-algebras of $\G(\W)$.
The conformal model could then be imposed on each of these sub-algebras.
Then, ideally, the model we impose on $\G(\W)$ will be able to benefit from
what we already know about the conformal model.

Our first task is to find a versor $S\in\G(\W)$ that, when taken with any
vector $v\in\V$ as the conjugation of $v$ by $S$, produces in this manner
an outermorphic isomorphism between the sub-algebras $\G(\V)$ and $\G(\overline{\V})$.
For any vector $v\in\V$, it has the form
\begin{equation}
v = \alpha\nvao + x + \beta\nvai,
\end{equation}
where $x$ is an $n$-dimensional Euclidean vector, and each of $\nvao$ and $\nvai$
are null-vectors having the relationship $-1=\nvao\cdot\nvai$.  We will name the
counter-parts in $\overline{\V}$ of these vectors as $\overline{\nvao}$ and $\overline{\nvai}$, respectively.
As usual, for any pair of vectors $a,b\in\V$, we will define $a\cdot\overline{b}=0$.

Refering to \cite{LiRockwood},
we can rewrite $v$ in terms of $x$ and a different pair of basis vectors $e_{-}$ and $e_{+}$,
the first anti-Euclidean and the second Euclidean.  Doing so, $v$ becomes
\begin{equation}
v = \left(\frac{1}{2}\alpha + \beta\right)e_{-} + x + \left(\beta - \frac{1}{2}\alpha\right)e_{+}.
\end{equation}
A versor $S$ with the desired property named above is now easily found as
\begin{equation}
S = 2^{-(n+2)/2}(1+e_{-}\overline{e_{-}})(1-e_{+}\overline{e_{+}})\prod_{i=1}^n(1-e_ie_{i+n}),
\end{equation}
where here we have chosen to name the counter parts in $\overline{\V}$ of $e_{-}$ and $e_{+}$
as $\overline{e_{-}}$ and $\overline{e_{+}}$, respectively.

It now follows by the main result of this paper that if a bivector $E\in\G(\W)$
is of the form \eqref{equ_quadric_form}, then the geometry represented
by $E$ using equation \eqref{equ_quadric_representation}, where $p\in\V$ is a conformal point,
may be properly transformed by any transformation supported by the conformal model.
The question that then remains is: does there exist such a bivector $E$ that,
under this new model, is representative of an $n$-dimensional quadric surface?
Well, to find out, begin by noticing that a homogenized conformal point $p\in\V$ is of the form
\begin{equation}
p = \alpha e_{-} + x + \beta e_{+},
\end{equation}
where $\alpha=\frac{1}{2}(x^2+1)$ and $\beta=(\frac{1}{2}x^2-1)$.
Right away we now see what may be a problem with equation \eqref{equ_quadric_representation}.
There does not exist a 2-blade $a_i\wedge\overline{b_i}$ in the sum of $E$ giving
us a constant term in the resulting polynomial equation.  That is, there does not
exist such a 2-blade $a_i\wedge\overline{b_i}$ and a scalar $\lambda$ such that for
all $n$-dimensional Euclidean vectors $x$, we have
\begin{equation}
\lambda=p\wedge\overline{p}\cdot a_i\wedge\overline{b_i},
\end{equation}
a value that remains invarient despite any change in $x$.  This suggests that
it may not be possible to represent quadric surfaces in the newly proposed model.

Though this effort has failed, there must surely be a model for quadric surfaces
that is as nice as the conformal model.  Whether any of \cite{Parkin12} or
this paper is anywhere near close to finding the answer remains highly questionable.

% Use a computer algebra system to find a sample polynomial equation.

\bibliographystyle{amsplain}
\bibliography{Parkin_ATransformationalResultOfTheQuadricModel}

\end{document}