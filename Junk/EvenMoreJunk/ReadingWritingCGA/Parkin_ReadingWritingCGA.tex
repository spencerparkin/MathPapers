%-----------------------------------------------------------------------
% Beginning of ecgd-l-template.tex
%-----------------------------------------------------------------------
%
%     This is a topmatter template file for ECGD for use with AMS-LaTeX.
%
%     Templates for various common text, math and figure elements are
%     given following the \end{document} line.
%
%%%%%%%%%%%%%%%%%%%%%%%%%%%%%%%%%%%%%%%%%%%%%%%%%%%%%%%%%%%%%%%%%%%%%%%%

%     Remove any commented or uncommented macros you do not use.

\documentclass{ecgd-l}

%     If you need symbols beyond the basic set, uncomment this command.
%\usepackage{amssymb}

%     If your article includes graphics, uncomment this command.
%\usepackage{graphicx}

%     If the article includes commutative diagrams, ...
%\usepackage[cmtip,all]{xy}


%     Update the information and uncomment if AMS is not the copyright
%     holder.
%\copyrightinfo{2009}{American Mathematical Society}

\newtheorem{theorem}{Theorem}[section]
\newtheorem{lemma}[theorem]{Lemma}

\theoremstyle{definition}
\newtheorem{definition}[theorem]{Definition}
\newtheorem{example}[theorem]{Example}
\newtheorem{xca}[theorem]{Exercise}

\theoremstyle{remark}
\newtheorem{remark}[theorem]{Remark}

\numberwithin{equation}{section}

\newcommand{\G}{\mathbb{G}}
\newcommand{\V}{\mathbb{V}}
\newcommand{\R}{\mathbb{R}}
\newcommand{\B}{\mathbb{B}}
\newcommand{\nvao}{o}
\newcommand{\nvai}{\infty}

\begin{document}

% \title[short text for running head]{full title}
\title{A System For Reading And Writing The Language Of Conformal Geometric Algebra}

%    Only \author and \address are required; other information is
%    optional.  Remove any unused author tags.

%    author one information
% \author[short version for running head]{name for top of paper}
\author{Spencer T. Parkin}
\address{}
\curraddr{}
\email{spencer.parkin@gmail.com}
\thanks{}

%    author two information
%\author{}
%\address{}
%\curraddr{}
%\email{}
%\thanks{}

%    \subjclass is required.
\subjclass[2010]{Primary }

\date{}

\dedicatory{}

%    Abstract is required.
\begin{abstract}
This paper is the first of its kind to treat, not the development or use of
conformal geometric algebra (CGA), but a systematic method for
reading and writing in the language of CGA.
A consensus on the terminology and
semantics used in such a system is needed by anyone writing
or reading in this language so that analytical ideas can be
properly and universally communicated in the language.  Different
ways of looking at the language and different interpretations have
led to confusion between those who are knowledgable on
the subject, the present author among them.  The main result
of this paper is a system for communicating in the language of
CGA that brings together and disambiguates the current and common phrases
and terminologies used when communicating ideas in CGA.  Of particular
significance is a rigorous treatment of what is meant by an imaginary
geometry in the conformal model.
\end{abstract}

\maketitle

%    Text of article.

\section{The Three Principle Geometric Representations}

We begin by setting forth the three principle geometric representations used by
the conformal model.
\begin{enumerate}
\item A subset of $n$-dimensional Euclidean space $\R^n$.
\item A blade taken from a geometric algebra $\G$ upon which the conformal model may be imposed.
\item A set of parameters consisting of scalars and Euclidean vectors taken from $\G$.
\end{enumerate}
To be able to communicate effectly in the language of CGA, one must master the
relationships between all three of these fundamental representations of geometry.
The first of these (1) is generated by the second (2).  The third (3) generates representations
of the form (2) through composition as well as is generated by representations of the form (2)
through decomposition.

Working with geometry in the representation (1) is often impractical and does not easily lend itself to
geometric analysis.  This is where geometric algebra goes to work for us in the representation (2).
The representation (3) is what ultimately makes CGA a useful analytical tool as it is a set
of parameters that collectively and completely characterize a certain type of geometry.
For example, a center, radius and normal are enough to uniquely characterize a 2-dimensional
circle in 3-dimensional space.

We concern ourselves now with rigorously defining how these three geometric representations
are related to one another in the conformal model.

\section{The Direct And Dual Representations}

For the remainder of this paper, we will let $\G$ denote our conformal geometric algebra,
$\V$ the $(n+2)$-dimensional vector space generating this algebra, and $\V^n$ the
$n$-dimensional Euclidean vector sub-space of $\V$ from which representations of the form (1)
are produced.

% direct rep, dual rep (explain why redundant but needed) we could
% develop our system of interpreting the language without the dual rep, but...
% the intersection and union like ops are more natural using both interps?

% def: unordered set rep -- the compound representation?
% result: what's reped by a blade remains invariant under the taking of duals.

% fixed rep/blade -> free to do two diff interp
% fixed geo -> free to rep dually or directly

% re-interpretation from dual to direct or vice-versa is a "free operation"

% correct vs. incorrect dual/direct interpretations -- def each.
% treat algebraic identification -- the recognition of which interpretation may
% be applied to yield an applicable characterization.  def characterization.

% imaginaries must be rigorously addressed.

% def: interpretation -- we are free to app of any decomposition sequence on a blade
% def: real reps and imag reps -- def rig. how we can apply a geo as being reped imaginatively by a blade

% result?: if dual rep is empty, direct rep is non-empty and vice versa?

% result: all blades dir rep of n-dim round are also dual rep of (m-n)-dim round
% what can we learn about flats with this idea?

% result: something about imaginary intersections -- contour of sphere from point example

% often over-looked by those given an expositin of CGA:
% re-interpretation theorem: common basis for two blades reping the same thing

% all blades can be factored as points (even imaginary) except flat points.
% this leads to a fundamental result about versor derivation and action on geos/blades.
% conjugation by versors is outermorphic

%    Bibliographies can be prepared with BibTeX using amsplain,
%    amsalpha, or (for "historical" overviews) natbib style.
\bibliographystyle{amsplain}
%    Insert the bibliography data here.

\end{document}

%%%%%%%%%%%%%%%%%%%%%%%%%%%%%%%%%%%%%%%%%%%%%%%%%%%%%%%%%%%%%%%%%%%%%%%%

%    Templates for common elements of a journal article; for additional
%    information, see the AMS-LaTeX instructions manual, instr-l.pdf,
%    included in the ECGD author package, and the amsthm user's guide,
%    linked from http://www.ams.org/tex/amslatex.html .

%    Section headings
\section{}
\subsection{}

%    Ordinary theorem and proof
\begin{theorem}[Optional addition to theorem head]
% text of theorem
\end{theorem}

\begin{proof}[Optional replacement proof heading]
% text of proof
\end{proof}

%    Figure insertion; default placement is top; if the figure occupies
%    more than 75% of a page, the [p] option should be specified.
\begin{figure}
\includegraphics{filename}
\caption{text of caption}
\label{}
\end{figure}

%    Mathematical displays; for additional information, see the amsmath
%    user's guide, linked from http://www.ams.org/tex/amslatex.html .

% Numbered equation
\begin{equation}
\end{equation}

% Unnumbered equation
\begin{equation*}
\end{equation*}

% Aligned equations
\begin{align}
  &  \\
  &
\end{align}

%-----------------------------------------------------------------------
% End of ecgd-l-template.tex
%-----------------------------------------------------------------------
