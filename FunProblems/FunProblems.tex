\documentclass[12pt]{article}

\usepackage{amsmath}
\usepackage{amssymb}
\usepackage{amsthm}
\usepackage{graphicx}
\usepackage{float}

\title{Fun Problems}
\author{Spencer T. Parkin}

\begin{document}
\maketitle

\section{Rotating One Vector Into Another}

Let $a$ and $b$ be two non-zero vectors with the property that if there
exists a scalar $\lambda$ such that $a=\lambda b$, then $\lambda>0$.
(These vectors are not necessarily parallel, but if they are parallel,
they point in the same direction.)  Given two such vectors $a$ and $b$,
find a rotor $R$ such that
\begin{equation}
b = RaR^{-1}.
\end{equation}

\section{The Magnitude Of A Rotor}

For any two vectors $a$ and $b$, show that
\begin{equation}
|a||b| = |ab|.
\end{equation}
Hint: You need to know that for any element $E$ of a geometric algebra generated
by an $n$-dimensional vector space, we define
\begin{equation}
|E|^2 = \sum_{i=0}^n|\langle E\rangle_i|^2.
\end{equation}

\section{The Rotor $ab$}

Given any two unit-length vectors $a$ and $b$ with $a\neq b$, and any non-zero
vector $c$, show that the vector $c'$, given by
\begin{equation}
c' = RcR^{-1},
\end{equation}
where $R=ab$, is the rotation of $c$ through an angle $2\theta$
in a plane parallel to the plane containing $a$ and $b$, where $\theta$
is the angle between $a$ and $b$.

Bonus: Find the polar decomposition of $R$.  That is, write $R$
in terms of $e$, the base of the natural logarithm.

\end{document}