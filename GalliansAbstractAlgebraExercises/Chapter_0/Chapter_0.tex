\documentclass[12pt]{article}

\usepackage{amsmath}
\usepackage{amssymb}
\usepackage{amsthm}

\title{Chapter 0 Exercises\\Gallian's Book on Abstract Algebra}
\author{Spencer T. Parkin}

\newtheorem{theorem}{Theorem}[section]
\newtheorem{definition}{Definition}[section]
\newtheorem{corollary}{Corollary}[section]
\newtheorem{identity}{Identity}[section]
\newtheorem{lemma}{Lemma}[section]
\newtheorem{result}{Result}[section]

%\newcommand{\gcd}{\mbox{gcd}}
\newcommand{\lcm}{\mbox{lcm}}
\newcommand{\Z}{\mathbb{Z}}

\begin{document}
\maketitle

\section*{Problem 7}

Show that if $a$ and $b$ are positive integers, then $ab=\lcm(a,b)\cdot\gcd(a,b)$.

An argument is made here based upon the prime factorizations of $ab/\gcd(a,b)$
and $\lcm(a,b)$.  The latter may be thought of as the union of two sets $A$ and $B$.
The former, in terms of these sets, may be thought of as the left-hand side of the
following equation.
\begin{equation*}
(A-B)\cup(B-A)\cup (A\cap B) = A\cup B
\end{equation*}
The set $A$ contains primes in the prime factorization of $a$, and $B$ contains
primes in the prime factorization of $b$.  The proof goes through by the
equality of the equation above.

This proof is not precise, but the intuition is correct, I believe.

\section*{Problem 9}

If $a$ and $b$ are integers and $n$ is a positive integer, prove that $a\equiv b\pmod n$
if and only if $n$ divides $a-b$.

Let $a = nq_a + r_a$ and $b=nq_b+r_b$.  If $r_a=r_b$, then
$a-b=n(q_a-q_b)$, showing that $n|(a-b)$.

Now let $a = nq_a + r_a$ and $b=nq_b+r_b$ again and write
\begin{equation*}
a-b = n(q_a-q_b)+r_a-r_b.
\end{equation*}
Now since $n|(a-b)$, we must have $n|(r_a-r_b)$ so that there exists an integer
$k$ such that $r_a=r_b+nk$.  But since $0\leq r_a,r_b<n$, we have $|r_a-r_b|<n$,
and therefore, $k=0$.  It follows that $r_a=r_b$.

\section*{Problem 10}

Let $d=\gcd(a,b)$.  If $a=da'$ and $b=db'$, show that $\gcd(a',b')=1$.

Let $u,v\in\Z$ be integers such that $d=au+bv$.  Then $1=(a/d)u+(b/d)v$.
Then, by Theorem 0.2, we have $\gcd(a',b')=1$, because there is no smaller
positive integer than 1.

\section*{Problem 11}

Let $n$ be a fized positive integer greater than 1.  If $a\equiv a'\pmod n$ and
$b\equiv b'\pmod n$, prove that $a+b\equiv a'+b'\pmod n$ and $ab=a'b'\pmod n$.

Letting $a'=a+k_an$ and $b'=b+k_bn$, we find that
\begin{equation*}
a' + b' = a + b + n(k_a+k_b),
\end{equation*}
and that
\begin{equation*}
a'b' = ab = n(ak_b + bk_a + k_ak_bn),
\end{equation*}
showing that $n$ divides both $a'b'-ab$ and $a'+b'-(a+b)$.
The proof now goes through by Problem 9.

\section*{Problem 12}

Let $a$ and $b$ be positive integers and let $d=\gcd(a,b)$ and $m=\lcm(a,b)$.
If $t$ divides both $a$ and $b$, prove that $t$ divides $d$.  If $s$ is a multiple of
both $a$ and $b$, prove that $s$ is a multiple of $m$.

By Theorem 0.2, $d$ is a linear combination of $a$ and $b$, and therefore,
any common divisor of $a$ and $b$, such as $t$, also divides $d$.

To see that $m$ divides $s$, simply notice that all common multiples
of $a$ and $b$ are generated by all positive multiples of $m$.

\section*{Problem 13}

Let $n$ and $a$ be positive integers and let $d=\gcd(a,n)$.  Show that the
equation $ax\equiv 1\pmod n$ has a solution if and only if $d=1$.

If $d=1$, there exist integers $u,v\in\Z$ such that $1=au+nv$, showing
that $n$ divides $1-au$.  Letting $x=u$, the equation in $x$ above has
a solution by Problem 9.

Now suppose the equation above has a solution in $x$.  Then $n$ divides $1-ax$
and there exists $v\in\Z$ such that $1=nv+ax$.  It now follows that $d=1$
by Theorem 0.2, because there is no positive integer smaller than 1.

\section*{Problem 24}

(Generalized Euclid's Lemma)  If $p$ is a prime and $p$ divides
$a_1a_2\dots a_n$, prove that $p$ divides $a_i$ from some $i$.

The case $n=2$ is covered by Euclid's Lemma.  Now suppose, for a fixed integer
$k>2$, that the generalized lemma holds in the case $n=k-1$.  Now
consider the case $n=k$.  If $p$ does not divide $a_n$, then
clearly $p$ divides $a_1a_2\dots a_{n-1}$ by Euclid's Lemma.
Then, by our inductive hypothesis, $p$ must divide $a_i$ for an
integer $i\in[1,n-1]$.  We have now proven the general lemma
by the principle of mathematical induction.

\section*{Problem 25}

Use the Generalized Euclid's Lemma (see Exercise 24) to establish the
uniqueness portion of the Fundamental Theorem of Arithmetic.

Suppose an integer $n$ has two different prime factorizations $p_1^{a_1}\dots p_r^{a_r}$
and $q_1^{b_1}\dots q_s^{b_s}$.  By the Generalized Euclid's Lemma, if
$p\in\{p_i\}_{i=1}^r$, then $p\in\{q_i\}_{i=1}^s$, because $p$ divides $n$.
Conversely, if $p\in\{q_i\}_{i=1}^s$, then $p\in\{p_i\}_{i=1}^r$ by the same reason.
It follows that $\{p_i\}_{i=1}^r=\{q_i\}_{i=1}^s$, which is a contradiction, and
therefore, no integer $n$ has two different prime factorizations.



\end{document}