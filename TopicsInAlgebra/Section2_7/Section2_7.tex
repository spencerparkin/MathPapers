\documentclass[12pt]{article}

\usepackage{amsmath}
\usepackage{amssymb}
\usepackage{amsthm}

\title{Section 2.7 Exercises\\Herstein's Topics In Algebra}
\author{Spencer T. Parkin}

\newtheorem{theorem}{Theorem}[section]
\newtheorem{definition}{Definition}[section]
\newtheorem{corollary}{Corollary}[section]
\newtheorem{identity}{Identity}[section]
\newtheorem{lemma}{Lemma}[section]
\newtheorem{result}{Result}[section]

%\newcommand{\gcd}{\mbox{gcd}}
\newcommand{\lcm}{\mbox{lcm}}
\newcommand{\Z}{\mathbb{Z}}
\newcommand{\R}{\mathbb{R}}
\newcommand{\C}{\mathbb{C}}

\begin{document}
\maketitle

\section*{Problem 3}

Let $G$ be a finite abelian group of order $|G|$ and suppose the integer
$n$ is relatively prime to $|G|$.  Prove that every $g\in G$ can be written as
$g=x^n$ with $x\in G$.

I'm doubtful I could have done this problem without the hint Herstein provides.

Let $\phi:G\to G$ be defined as $\phi(y)=y^n$.  We then easily see that
\begin{equation*}
\phi(xy)=(xy)^n=x^ny^n=\phi(x)\phi(y),
\end{equation*}
since $G$ is abelian, and so $\phi$ is a homomorphism.
Consider the kernel of $\phi$.  Note that
\begin{equation*}
y^n=e\implies y=e,
\end{equation*}
since the order of $y\neq e$ cannot divide a number $n$ that is coprime with $|G|$,
by Lagrange's theorem.  It follows that $\phi$ is an isomorphism.
Now since $G$ is finite, we can easily claim that every element has
the form $x^n$ for some $x\in G$.  (To find such an $x$ for a given $g$,
just let $x=\phi^{-1}(g)$.)

\section*{Problem 4}

\subsection*{Part A}

Given any group $G$ and a subset $U$, let $\hat{U}$ be the smallest subgroup
of $G$ which contains $U$.  Prove there is such a subgroup $\hat{U}$ in $G$.

I would write
\begin{equation*}
\hat{U}=\{g\in G|g\in\prod_{w\in W}w,W\subseteq V\},
\end{equation*}
where the set $V$ is given by
\begin{equation*}
V=\{u^z|u\in U,z\in\Z\}.
\end{equation*}
I believe this is the smallest subgroup of $G$ containing $U$, because
no element is added unnecessarily.

One drawback of this formulation, however, is that it makes it difficult to write
the form of a general element of the group.  If we restrict ourselves to finite or even
countably infinite subset $U$ of $G$, then we can write a general element $u\in\hat{U}$ as
\begin{equation*}
u = \prod_i u_i^{z_i},
\end{equation*}
where $\{u_i\}\subseteq U$ and $\{z_i\}\subseteq\Z$ are each finite or countably infinite sequences.

\subsection*{Part B}

If $gug^{-1}\in U$ for all $g\in G$, $u\in U$, prove that $\hat{U}$ is a normal
subgroup of $G$.

Let $u\in\hat{U}$.  We then see that
\begin{equation*}
gug^{-1} = \prod_i gu_i^{z_i}g^{-1} = \prod_i (gu_ig^{-1})^{z_i}\in\hat{U}.
\end{equation*}

\section{Problem 5}

Let $U=\{xyx^{-1}y^{-1}|x,y\in G\}$.  In this case $\hat{U}$ is usually written $G'$
and is called the {\it commutator subroup of $G$}.

\subsection*{Part A}

Prove that $G'$ is normal in $G$.

By part B of problem 4, we need only show that for any commuator $u\in U$, and
any $g\in G$, we have $gug^{-1}\in U$.  Let $u=xyx^{-1}y^{-1}$, and see that
\begin{equation*}
gug^{-1}=gxyx^{-1}y^{-1}g^{-1}=(gxg^{-1})(gyg^{-1})(gxg^{-1})^{-1}(gyg^{-1})^{-1}\in U.
\end{equation*}

\subsection*{Part B}

Prove that $G/G'$ is abelian.

For $a,b\in G$, we have
\begin{equation*}
G'aG'b(G'a)^{-1}(G'b)^{-1}=Gaba^{-1}b^{-1}=G\implies G'aG'b=G'bG'a.
\end{equation*}

\subsection*{Part C}

If $G/N$ is abelian, prove that $N\supseteq G'$.

For all $a,b\in G$,
\begin{equation*}
Naba^{-1}b^{-1}=N\implies aba^{-1}b^{-1}\in N\implies G'\subseteq N.
\end{equation*}

I have to try to say something here concerning the significance of commutator groups.
For a non-abelian group, this shows that the largest abelian factor group we can find
is found by mod-ing out by $G'$.  And I remember reading somewhere that the order of
this factor group in comparison to the order of $G$ is somehow a measure of ``how abelian''
the group $G$ is.

\subsection*{Part D}

Prove that if $H$ is a subgroup of $G$ and $H\supseteq G'$, then $H$ is normal in $G$.

If $H=G'$, we're done.  So let $H\supset G'$.  Now if $h\in G'$ and $g\in G$,
clearly $ghg^{-1}\in G'\subset H$ by the normality of $G'$, so let $h\in H-G'$.
Now since $c=ghg^{-1}h^{-1}\in G'$, we have $ghg^{-1}=ch\in H$ by closure in $H$.

\section*{Problem 17}

Let $G$ be the group of real numbers under addition and let $N$ be the
subgroup of $G$ consisting of all the integers.  Prove that $G/N$ is
isomorphic to the group of all complex numbers of absolute vlaue 1
under multiplication.

Let $\phi:\R/\Z\to\C$ be defined by
\begin{equation*}
\phi(\Z+r) = \exp(2\pi r i).
\end{equation*}
Now for any $a,b\in\R$, we have
\begin{equation*}
\Z+a=\Z+b \iff a=b+z,
\end{equation*}
for some integer $z\in\Z$.  We then have
\begin{equation*}
\exp(2\pi a i) = \exp(2\pi(b+z) i) = \exp(2\pi b i)\exp(2\pi z i) = \exp(2\pi b i).
\end{equation*}
Thus far we have shown that $\phi$ is well-defined and onto-to-one.
Clearly $\phi$ is onto $\C$.  Is $\phi$ operation preserving?
\begin{align*}
 \phi(\Z+a+\Z+b) &=\phi(\Z+a+b) = \exp(2\pi(a+b)i) \\
&=\exp(2\pi a i)\exp(2\pi b i) = \phi(\Z+a)\phi(\Z+b)
\end{align*}

\end{document}