\documentclass[12pt]{article}

\usepackage{amsmath}
\usepackage{amssymb}
\usepackage{amsthm}
\usepackage{graphicx}
\usepackage{float}

\title{Master Elements}
\author{Spencer}

\newtheorem{theorem}{Theorem}[section]
\newtheorem{definition}{Definition}[section]
\newtheorem{corollary}{Corollary}[section]
\newtheorem{identity}{Identity}[section]
\newtheorem{lemma}{Lemma}[section]
\newtheorem{result}{Result}[section]

\begin{document}
\maketitle

\begin{definition}
For any group $G$, let an element $m\in G$ be a \textbf{master element} if for
every $g\in G$, there exists a sequence $\{r_i\}_{i=1}^k\subset G$ such that
\begin{equation}\label{equ_master_ele}
g = \prod_{i=1}^k r_imr_i^{-1}.
\end{equation}
\end{definition}

Calling a group \textbf{mastered} if it has a non-empty subset of master elements,
we give the following lemma.

\begin{lemma}[Gallian]
The cyclic groups are the only mastered Abelian groups.
\end{lemma}
\begin{proof}
Notice that equation \eqref{equ_master_ele_factorization} immediately reduces to
\begin{equation*}
g = m^k.
\end{equation*}
Clearly every element of every cyclic group is of this
form, and so every master element is a generator of the group.
This is not the case, however, for any non-cyclic Abelian group.
\end{proof}

We now describe a class of permutation groups that are all mastered.
Our convention for composition is that, for permutations $a$ and $b$, the
composition $ab$ maps domain elements through $a$ first, then $b$.
Similarly, products of cycles are evaluated from left to right.
The notation $x^a=y$ is used instead of $a(x)=y$ to avoid
the idea that $a(x)$ is a composition of the permutation $a$
with the $1$-cycle $(x)$.

\begin{lemma}
Let $p_i=(p_{i,1},\dots,p_{i,n})$ be one of $m$ permutations, each an $n$-cycle
of elements in a domain $\Omega$,
and let $G=\langle\{p_i\}_{i=1}^m\rangle$.
If there exists $1\leq j\leq m$ such that for any $1\leq k\leq m$,
we can find $r\in G$ in the form
\begin{equation*}
r = (p_{k,1},p_{j,1})\dots(p_{k,n},p_{j,n})q,
\end{equation*}
where $q$ is a permutation that, for all $1\leq i\leq m$, has $p_{j,i}^q=p_{j,i}$,
then $G$ is a mastered group.
\end{lemma}
\begin{proof}
Since every element of $G$ is a product of the generators, it suffices to
show that every generator factors as shown in equation \eqref{equ_master_ele_factorization}.
By hypothesis, it is easy to see that
\begin{equation*}
p_k = rp_jr^{-1},
\end{equation*}
showing that $p_j$ is a master element of $G$.
\end{proof}
\begin{corollary}\label{cor_sym_grp_mastered}
The symmetric group on a domain $\Omega$ of size $n$ is a mastered group.
\end{corollary}
\begin{proof}
Note that $S_n=\{(1,2)(2,3)\dots(n-1,n)\}$.  Now choose, arbitrarily,
$m=(1,2)$ to be our master element.  Then, for any generator,
we have $(x,x+1)=rmr^{-1}$, where
\begin{equation*}
r = \prod_{j=1}^{x-2}(x-i,x-i-1)(x-i+1,x-i).
\end{equation*}
\end{proof}
The distinction between an idealy master group and one that is merely mastered comes
into play when we consider exploiting the mastered property of a group for the
purpose of factoring its elements in terms of a set of generators for the group.
\begin{definition}
Call a group $G$ \textbf{ideally} mastered if for every $g\in G$, there
exists $r\in G$ such that
\begin{equation*}
|A(grmr^{-1})|<|A(g)|,
\end{equation*}
where $A:G\to\Omega$ is a function defined as
\begin{equation*}
A(g) = \{i\in\Omega|i^g\neq i\}.
\end{equation*}
\end{definition}
\begin{lemma}\label{lem_sym_grp_ideally_mastered}
The symmetric group on a domain $\Omega$ of size $n$ is an ideally mastered group.
\end{lemma}
\begin{proof}
Note that $S_n=\langle\{(x,y)|\mbox{$x,y\in\Omega$ and $x\neq y$}\}\rangle$.  We now again choose,
arbitrarily, $m=(1,2)$ to be our master element.  Then, for any generator, we have
$(x,y)=rmr^{-1}$, where
\begin{equation*}
r = \prod_{i=0}^{x-2}(x-i,x-i-1)\prod_{j=0}^{y-3}(y-i,y-i-1).
\end{equation*}
\end{proof}

For any $g\in S_n$, finding a factorization of $g$ in terms of the
generators found in lemma \ref{lem_sym_grp_ideally_mastered}, or even those of corollary \ref{cor_sym_grp_mastered},
is trivial.  For other
mastered groups, however, finding a factorization in terms of the
generators may not be so easy.  So we consider the following algorithm
for factoring an element $g$ in an ideally mastered group $G$.

Let $g_1=g$, and then, while $g_k\neq e$, let $g_{k>1}=g_{k-1}r_kmr_k^{-1}$ where
$|A(g_k)|<|A(g_{k-1})|$.  The idea here is that if we can factor each $r_k$ in terms
of the generators, and we know the factorization of $m$ in terms of the generators,
then we've deduced a factorization of $g$.  At each iteration, the crux is finding $r_k$.

Can we find a test for a mastered group being ideal?  Can we find a test for a group being mastered for
that matter?  Can we prove something about finding $r_k$?  Clearly we can go down the generator
tree, but can we show an upper-bound on how far we'd have to go?


\end{document}