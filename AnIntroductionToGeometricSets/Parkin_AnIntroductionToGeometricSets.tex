\documentclass{birkjour}

\usepackage{tikz}
\usepackage{graphicx}
\usepackage{hyperref}

\newtheorem{thm}{Theorem}[section]
\newtheorem{cor}[thm]{Corollary}
\newtheorem{lem}[thm]{Lemma}
\newtheorem{prop}[thm]{Proposition}
\theoremstyle{definition}
\newtheorem{defn}[thm]{Definition}
\theoremstyle{remark}
\newtheorem{rem}[thm]{Remark}
\newtheorem*{ex}{Example}
\numberwithin{equation}{section}

\newcommand{\R}{\mathbb{R}}
\newcommand{\B}{\mathbb{B}}
\newcommand{\G}{\mathbb{G}}
\newcommand{\V}{\mathbb{V}}
\newcommand{\gd}{\dot{g}}
\newcommand{\gh}{\hat{g}}
\newcommand{\Gd}{\dot{G}}
\newcommand{\Gh}{\hat{G}}
\newcommand{\nvai}{\infty}
\newcommand{\nvao}{o}
\newcommand{\grade}{\mbox{grade}}

%\received{}\accepted{}

\begin{document}

\title{An Introduction To Geometric Sets}

\author{Spencer T. Parkin}
\address{102 W. 500 S., \\
Salt Lake City, UT  84101} \email{spencerparkin@outlook.com}

%\subjclass{Primary 14J70; Secondary 14J29}

%\dedicatory{To my dear wife Melinda.}

\begin{abstract}
Abstract goes here...
\end{abstract}

\keywords{Key words go here...}

\maketitle

\section{Introduction}

\subsection{Motivation}

% relate to algebraic geometry here

\subsection{Conventions}

This paper uses capital letters $A,B,C$ to denote blades of various grades,
while using lower case letters $a,b,c$ to denote vectors.  Scalars are written
using greek letters $\alpha,\beta,\gamma$.  Grades of blades $A,B,C$ are
denoted by lower case letters $r,s,t$, respectively.  Lower case letters $i,j,k$ are
used as indices.  Capital letters $R,S,T$
are used to denote subsets of interest of $n$-dimensional euclidean space $\R^n$.
Lower case letters $x,y,z$ are reserved for denoting points taken from $\R^n$.
We will use $\G$ to denote our geometric algebra, and $\V$ to denote an $m$-dimensional vector
space generating it.  $\B$ will denote the set of all blades taken from $\G$.  The
scalars of $\V$, and therefore $\G$, are taken from the field of real numbers $\R$.
The capital letter $I$ will be used to denote the unit psuedo-scalar of $\G$.  We assume
that $I$ is invertible with respect to the geometric product.

We will let the geometric product take precedence over the inner and outer products,
and the outer product take precedence over the inner product.

No particular signature of our geometric algebra $\G$ is assumed in this paper.

\section{Preliminaries}

The results of this paper will depend upon the following preliminary material.
If desired, the reader is welcome to skip this material and refer back to it only as needed.

\subsection{Identities}

Given a vector $a\in\V$ and a blade $A\in\B$, central to all of geometric algebra is the identity
\begin{equation}\label{equ_vec_blade_geo_prod}
aA = a\cdot A + a\wedge A.
\end{equation}
The inner and outer products of \eqref{equ_vec_blade_geo_prod} may be written in terms
of the geometric product as
\begin{align}
a\wedge A &= \frac{1}{2}\left(aA + (-1)^rAa\right),\label{equ_vec_blade_outer_prod} \\
a\cdot A &= \frac{1}{2}\left(aA - (-1)^rAa\right),\label{equ_vec_blade_inner_prod}
\end{align}
where $r=\grade(A)$.  Then, realizing that $m=\grade(I)$, and that by \eqref{equ_vec_blade_geo_prod},
we have $vI=v\cdot I$, we can use equation \eqref{equ_vec_blade_inner_prod} to establish the
commutativity of vectors in $\V$ with the unit psuedo-scalar $I$ as
\begin{equation}\label{equ_blade_psuedo_scalar_commutativity}
aI = -(-1)^mIa.
\end{equation}
Using equation \eqref{equ_blade_psuedo_scalar_commutativity} in conjunction with equation \eqref{equ_vec_blade_inner_prod},
we find that
\begin{equation}\label{equ_vec_blade_inner_prod_dual}
(a\cdot A)I=a\wedge AI.
\end{equation}
(In verifying this identity, it helps to realize that for any integer $i$, we have $(-1)^i=(-1)^{-i}$.)
Replacing $A$ in equation \eqref{equ_vec_blade_inner_prod_dual} with $AI$, we find that
\begin{equation}\label{equ_vec_blade_outer_prod_dual}
(a\wedge A)I=a\cdot AI.
\end{equation}

Refering to \eqref{equ_vec_blade_inner_prod}, another important formulation of the inner product between a vector and a blade
is given by
\begin{equation}\label{equ_vec_blade_inner_prod_series}
a\cdot A=-\sum_{i=1}^r(-1)^i(a\cdot a_i)A_i,
\end{equation}
where $A$ is factored as $\bigwedge_{i=1}^r a_i$, and we define $A_i$ as
\begin{equation}
A_i=\bigwedge_{j=1,j\neq i}^r a_i.
\end{equation}
This leads to the following recursive formulation.
\begin{equation}
a\cdot A = (a\cdot a_1)A_1-a_1\wedge(a\cdot A_1)
\end{equation}
If a blade $B\in\B$ has grade $s$ and factorization $\bigwedge_{i=1}^s b_i$, then we can express
the product $A\cdot B$ recursively as
\begin{equation}
A\cdot B = \left\{\begin{array}{ll}
A_r\cdot(a_r\cdot B) & \mbox{if $r\leq s$,} \\
(A\cdot b_1)\cdot B_1 & \mbox{if $r\geq s$.}
\end{array}\right.
\end{equation}

\subsection{Lemmas}

The following lemmas will help us prove the results of this paper.

\begin{lem}\label{lem_desired_blade_factorization}
If $A\in\B$ is a non-zero blade of grade $r>0$ and $\{b_i\}_{i=1}^r$ is a set
of $r$ linearly independent vectors such that for all $b\in\{b_i\}_{i=1}^r$, we have
$b\wedge A=0$, then there exists a scalar $\beta\in\R$ such that
\begin{equation}\label{equ_desired_blade_factorization}
A=\beta\bigwedge_{i=1}^r b_i.
\end{equation}
\end{lem}
\begin{proof}
Letting $\bigwedge_{i=1}^r a_i$ be a factorization of the $r$-blade $A$, it is clear
that if $b_i\wedge A=0$, then there exists a set of $r$ scalars $\{\alpha_{i,j}\}_{j=1}^r$
such that
\begin{equation}
b_i=\sum_{j=1}^r\alpha_{i,j}a_i.
\end{equation}
We then have
\begin{equation}
\bigwedge_{i=1}^r b_i=(\det M)\bigwedge_{i=1}^r a_i,
\end{equation}
where the $r\times r$ matrix $M$ is given by
\begin{equation}
M = \left[\begin{array}{cccc}
\alpha_{1,1} & \alpha_{1,2} & \dots & \alpha_{1,r} \\
\alpha_{2,1} & \alpha_{2,2} & \dots & \alpha_{2,r} \\
\vdots & \vdots & \ddots & \vdots \\
\alpha_{r,1} & \alpha_{r,2} & \dots & \alpha_{r,r}
\end{array}\right].
\end{equation}
Now since $\{b_i\}_{i=1}^r$ is a linearly independent set of vectors,
$\det M\neq 0$, and we may choose $\beta=(\det M)^{-1}$ and
equation \eqref{equ_desired_blade_factorization} now holds.
\end{proof}

\begin{lem}
Generalize the previous lemma to the case when the number of linearly independent vectors
is less than the grade of the blade in question.
\end{lem}
\begin{proof}
\end{proof}

\begin{lem}\label{lem_lin_indep_blades}
For a non-zero $r$-blade $A$ factored as $\bigwedge_{i=1}^r a_i$, the set of
$(r-1)$-blades $\{A_i\}_{i=1}^r$ is linearly independent.
\end{lem}
\begin{proof}
Suppose there exists a non-trivial set of $r$ scalars $\{\alpha_i\}_{i=1}^r$
such that $0 = \sum_{i=1}^r\alpha_i A_i$.
Then, without loss of generality, suppose that $\alpha_r\neq 0$, and rearrange
our equation as $-\alpha_rA_r=\sum_{i=1}^{r-1}\alpha_iA_i$.  Now notice
that while $a_r\wedge-\alpha_rA_r\neq 0$, we have $a_r\wedge\sum_{i=1}^{r-1}\alpha_iA_i=0$,
which is a contradiction.
\end{proof}

Note that a perhaps more elegant proof of Lemma~\ref{lem_lin_indep_blades} could have been given
under the assumption
of a euclidean signature; in which case, the Gram-Schmidt orthogonalization process would have
allowed us to choose, without loss of generality, an orthogonal factorization of the blade $A$.
Doing so, $A$ becomes the versor $\prod_{i=1}^r a_i$, and we may write
\begin{equation}
0 = \sum_{i=1}^r \alpha_iA_i \iff
0=\left(\sum_{i=1}^r\alpha_iA_i\right)A = -\sum_{i=1}^r(-1)^{r-i}A_i^2\alpha_ia_i.
\end{equation}

\begin{lem}[The Zero Product Property]\label{lem_zero_prod_property}
For any two non-zero blades $A,B\in\B$ of grades $r$ and $s$, respectively, if $AB=0$ and at least one of $A$ and $B$
is invertible, then at least one of $A$ and $B$ is zero.
\end{lem}
\begin{proof}
Without loss of generality, suppose $B^{-1}$ exists.  We then see that
\begin{equation}
A=ABB^{-1}=0B^{-1}=0.
\end{equation}
\end{proof}

Notice the requirment here of Lemma~\ref{lem_zero_prod_property} that at least one of $A$ and $B$ be invertible.
This requirement comes about in consideration of the square of a non-zero null-vector.

\section{Results}

\subsection{Foundation}

Our discussion begins with a non-zero function $p:\R^n\to\V$.  No definition
of this function will be given.

\begin{defn}[Direct And Dual Representation]\label{def_gh_and_gd}
For the two functions $\gh:\B\to\R^n$ and $\gd:\B\to\R^n$, given by
\begin{align}
\gh(A) &= \{x\in\R^n|p(x)\wedge A=0\},\\
\gd(A) &= \{x\in\R^n|p(x)\cdot A=0\},
\end{align}
we say that $A$ directly represents the set of points $\gh(A)$ and
dually represents the set of points $\gd(A)$.
\end{defn}

From Definition~\ref{def_gh_and_gd}, it is important to take away the realization that a given blade $A\in\B$
represents two subsets of $\R^n$ simultaneously; namely, $\gh(A)$ and $gd(A)$.  Which we choose
to think of $A$ as being a representative of at any given time is completely arbitrary.

It should also be clear from Definition~\ref{def_gh_and_gd} that the subset of $\R^n$ represented by a
blade $A$, (directly or dually), remains invariant under any non-zero scaling of the blade $A$.

Finally, now enters this paper's object of study: the {\it geometric set}.

\begin{defn}[Geometric Set]\label{def_geo_set}
A subset $R\subset\R^n$ for which there exists a blade $A\in\B$ such that $\gh(A)=R$ is what
we'll refer to as a {\it geometric set}.
\end{defn}

In the course of our study, we will find the concept of {\it irreducibility} important.

\begin{defn}[Irreducible/Reducible Blade]\label{def_irreducible_blade}
Given an $r$-blade $A\in\B$, if there does not exist an $s$-blade $B\in\B$ with $s<r$
such that $\gh(A)=\gh(B)$, then $A$ is what we'll refer to as an {\it irreducible} blade.
A blade that is not irreducible is referred to as {\it reducible}.
\end{defn}

\subsection{Consequences}

Having given the definitions in the previous section, we may focus now on the results that follow
from these definitions.

\begin{lem}[Dual Relationship Between Representations]\label{lem_dual_relationship_in_rep}
For any geometric set $R\subset\R^n$, if $A\in\B$ is a blade such that $\gh(A)=R$,
then $\gd(AI)=R$, and similarly, if $A\in\B$ is a blade such that $\gd(A)=R$, then $\gh(AI)=R$.
\end{lem}
\begin{proof}
By the identity of equation \eqref{equ_vec_blade_outer_prod_dual}, and Lemma~\ref{lem_zero_prod_property},
the first of these two latter statements is proven by
\begin{equation}
0=p(x)\wedge A=-(p(x)\cdot AI)I\iff p(x)\cdot AI=0,
\end{equation}
while the second, by the identity of equation \eqref{equ_vec_blade_inner_prod_dual},
and again Lemma~\ref{lem_zero_prod_property}, is proven by
\begin{equation}
p(x)\cdot A=0\iff 0=(p(x)\cdot A)I=p(x)\wedge AI.
\end{equation}
\end{proof}

In other words, Lemma~\ref{lem_dual_relationship_in_rep} is telling us that for a single given geometric set, the
algebraic relationship between a blade directly (dually) representative of that set,
and a blade dually (directly) representative of that set, is simply that, up to scale, they
are duals of one another.

Of course, there will also be a geometric relationship between the geometry that is directly represented by a single
given blade $A\in\B$, and the geometry that is dually represented by $A$, but this would depend upon how we choose to define
the function $p:\R^n\to\V$.

\begin{lem}
For any geometric set $R\subset\R^n$, there exists a blade $A\in\B$ such that
$\gd(A)=R$.
\end{lem}

\begin{lem}
For all non-zero blades $A\in\B$, we have $\gh(A)\cap\gd(A)=\emptyset$.
\end{lem}

\begin{lem}
If $A\in\B$ is an irreducible $r$-blade with $\gh(A)\neq\emptyset$, then there exists a set of $r$ points $\{x_i\}_{i=1}^r\subset\R^n$
and a scalar $\alpha\in\R$ such that
\begin{equation*}
A=\alpha\bigwedge_{i=1}^r p(x_i).
\end{equation*}
\end{lem}
\begin{proof}
Let $j$ be the largest integer for which there exists a set of $j$ points $\{x_i\}_{i=1}^j\subset\gh(A)$ such that
$\bigwedge_{i=1}^j p(x_i)=0$.  Clearly $j\geq 1$, because $\gh(A)$ is non-empty; and clearly $j\leq r$ because of
the requirement that for each $x\in\{x_i\}_{i=1}^j$, we have $p(x)\wedge A=0$.  Now if $j=r$, we're
done by Lemma~\ref{lem_desired_blade_factorization}.  Therefore,
suppose $j<r$.  In this case, there must exist, by Lemma~\ref{}, a factorization of $A$ of the form
\begin{equation}
A=B\wedge\bigwedge_{i=1}^j p(x_i),
\end{equation}
where $B$ is a blade of grade $r-j$.  But it must now follow that $\gh(B)=\emptyset$ and
therefore
\begin{equation}
\gh(A)=\gh\left(\bigwedge_{i=1}^j p(x_i)\right),
\end{equation}
which contradicts the irreducibility of the blade $A$.  This proof is wrong.
\end{proof}

% if A = B^C and v^A=0 and v^B != 0, then is v^C=0?  No!!!

% important question comes about while considering converse of A=lB ==> gh(A)=gh(B)
% we often want to algebraic relate independently formulated geometric sets.

\begin{thm}[The Fundamental Theorem Of Geometric Set Representation]
For every geometric set $R\subset\R^n$, there exists, up to scale, a unique,
irreducible blade $A\in\B$ such that $\gh(A)=R$.
\end{thm}

% note that this is similar to the nulzinsnatch of algebraic geometry which states
% that there is a unique radical ideal representative of every algebraic set.

% talk about unions and intersections of geometric sets

% conclude that there may be more to learn about geometric sets.

\begin{thebibliography}{9}

\end{thebibliography}

\end{document}