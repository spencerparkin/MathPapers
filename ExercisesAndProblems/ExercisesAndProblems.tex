\documentclass[12pt]{article}

\usepackage{amsmath}
\usepackage{amssymb}
\usepackage{amsthm}
\usepackage{graphicx}
\usepackage{float}

\title{Exercises and Problems\\of Alan Mcdonald's\\Vector and Geometric Calculus}
\author{Spencer T. Parkin}

\newcommand{\G}{\mathbb{G}}
\newcommand{\V}{\mathbb{V}}
\newcommand{\R}{\mathbb{R}}
\newcommand{\Q}{\mathbb{Q}}
\newcommand{\B}{\mathbb{B}}
\newcommand{\nvao}{o}
\newcommand{\nvai}{\infty}

\begin{document}
\maketitle

\section*{Exercise 1.4 (Genarlized)}

When do two line parameterizations intersect?
\begin{align*}
x_1(t) &= c_1 + tv_1 \\
x_2(t) &= c_2 + tv_2
\end{align*}
We're interested in a scalar $t\in\R$ such that $x_1(t)=x_2(t)$.
It is clear that zero, one or infinitely many such $t$ may exist.
Solving the equation
\begin{equation*}
c_1+tv_1=c_2+tv_2,
\end{equation*}
we get
\begin{equation*}
t = (c_1-c_2)(v_2-v_1)^{-1}.
\end{equation*}
Clearly there is no solution when $v_1=v_2$, unless $c_1=c_2$, in which case
$x_1(t)=x_2(t)$ for all $t$.  On the other hand, $t$ may not be a scalar by
this equation, in which case we may conclude that the lines do not touch.
Letting $\Delta c=c_2-c_1$ and $\Delta v=v_2-v_1$, we have
\begin{equation*}
t = -\Delta c\cdot \Delta v^{-1} - \Delta c\wedge\Delta v^{-1},
\end{equation*}
showing that the lines intersect only when $\Delta c\wedge \Delta v=0$.

\section*{Problem 1.1.1}

Let $x(t)=a\cos(\theta i)+b\sin(\theta j)$, where $0\leq\theta\leq 2\pi$, parameterize a curve.
Show that the points on the curve are on the ellipse with equation $x^2/a^2+y^2/b^2=1$.

We need only show that $x(t)$ satisfies the implicit equation for all $t$.
Doing so, we see that
\begin{align*}
(a\cos\theta)^2/a^2 + (b\sin\theta)^2/b^2 = \cos^2 \theta+\sin^2 \theta = 1
\end{align*}
by the Pythagorean Theorem.

\section*{Problem 1.3.1}

Show that $\rho=2a\sin\phi\cos\theta$ is the equation of the sphere of
radius $|a|$ with center at $(a,0,0)$.

With $x=\rho\sin\phi\cos\theta$, notice that $\rho=2ax/\rho$ so that
we have $x^2+y^2+z^2=\rho^2=2ax$.  All that remains is to
complete the square, giving us $(x-a)^2+y^2+z^2=a^2$.

\section*{Exercise 2.1}

\subsection*{Part a)}

Show that the union of (arbitrariy many) open sets is open.

Let $\{U_\alpha\}$ be a set of arbitrarily many open sets, and
let $U=\cup_\alpha U_\alpha$.  Choosing any $x\in U$, there
must exist $\alpha$ such that $x\in U_\alpha$.  Now let $N$ be
a neighborhood of $x$ contained in $U_\alpha$, and see that
$x\in N\subset U_\alpha\subseteq U$.  It follows that a neighborhood
$N$ of $x$ exists in $U$ for all $x\in U$, and so $U$ is open.

\subsection*{Part b)}

Show that the intersection $O_1\cap O_2$ of two open sets is open.

If the intersection is empty, then we're done, since the empty set is vacuously open.
Supposing the intersection to be non-empty, for any $x\in O_1\cap O_2$, we see
that $x\in O_1$ and $x\in O_2$, and therefore, there exist neighborhoods $N_1$ and
$N_2$ of $x$ such that $x\in N_1\subset O_1$ and $x\in N_2\subset O_2$.  Clearly,
$N_1\cap N_2\neq\emptyset$ and there exists a neighborhood $N$ of $x$ in $N_1\cap N_2$.
It follows that $x\in N\subset N_1\cap N_2\subset O_1\cap O_2$, showing that $O_1\cap O_2$ is open.

\subsection*{Part c)}

The intersection of finitely many open sets is open.  Show that the intersection
of infinitely many open sets need not be open.

Let $\{U_\alpha\}$ be an infinite set of open sets such that $\cap_\alpha U_\alpha$
is a singleton set.  For example,
\begin{equation*}
\{0\} = \bigcap_{r>0}(r,r).
\end{equation*}
Clearly $\{0\}$ is closed, because there exists no neighborhood of 0 in $\{0\}$.

\section*{Exercise 2.5}

Given an example of a set which is neither open nor closed.

Consider the rationals $\Q\subset\R$.  It is clearly not an open set, nor is its complement, the irrationals.

\section*{Problem 2.1.3}

Show that every open set $U$ in $\R^n$ is a union of neighborhoods of points of $U$.

For any $x\in U$, let $N_x$ denote any neighborhood of $x$ in $U$.  We then see that
\begin{equation}
U = \bigcup_{x\in U} x\subseteq \bigcup_{x\in U} N_x \subseteq U.
\end{equation}
It follows that $U=\cup_{x\in U}N_x$.

\pagebreak
\section*{Problem 2.3.4}

Let $f:U\subseteq \R^m\to\R^n$, where $U$ is an open connected
set, be continuous.  Show that the range of $f$ is connected.

For any two vectors $y_1,y_2\in f(U)$, let $x_1,x_2\in U$ be the vectors
such that $y_1=f(x_1)$ and $y_2=f(x_2)$.  Then, by virtue of $U$ being
a connected set, there exists a continuous parameterization $x(t):[a,b]\to U$
such that $x(a)=x_1$ and $x(b)=x_2$.  Letting $y(t):[a,b]\to\R^n$ be
defined as $y(t)=f(x(t))$, we see that $y$ is continuous, because the composition
of continuous functions is continuous.  Furthermore, $y(a)=y_1$ and $y(b)=y_2$.
It follows now that $f(U)$ is connected.

(Didn't use fact that $U$ is open set?)

\end{document}