\documentclass{article}
\usepackage{amsmath}
\usepackage{amssymb}
\title{Chapter 14 Homework}
\author{Spencer}
\addtolength{\oddsidemargin}{-.575in}
\addtolength{\evensidemargin}{-.575in}
\addtolength{\textwidth}{1.0in}
\addtolength{\topmargin}{-.575in}
\addtolength{\textheight}{1.25in}
\begin{document}
\maketitle

\newcommand{\Z}{\mathbb{Z}}
\newcommand{\R}{\mathbb{R}}
\newcommand{\N}{\mathbb{N}}
\newcommand{\lcm}{\mbox{lcm}}
\newcommand{\ann}{\mbox{Ann}}

\section*{Problem 6}

Find all maximal ideals in $\Z_{n>1}$.

Let $\sigma(n)$ denote the number of positive divisors of $n$,
and let $\{d_k\}_{k=1}^{\sigma(n)}$ be the ordered set of all
divisors of $n$ such that $d_i<d_j$ for all $1\leq i<j\leq\sigma(n)$.
(Notice that $d_{\sigma(n)}=n$.)
Choose any integer $k\in[1,\sigma(n)]$ and let $a\in\langle d_k\rangle$ and $z\in\Z_n$.
Then $a=i\cdot d_k$ and $z=j\cdot 1$ for some integers $i$ and $j$.
We then see that $az=(i\cdot d_k)(j\cdot 1)=(i+j)\cdot d_k\in\langle d_k\rangle$.
The same reasoning shows that $za\in\langle d_k\rangle$.
It follows that $\langle d_k\rangle$ is an ideal of $\Z_n$ for all $k$.
Recall that $\{\langle d_k\rangle\}_{k=1}^{\sigma(n)}$ is the set of all
subrings of $\Z_n$, so we now want to know which of these are maximal.
If $n$ is prime, then the only maximal ideal is $\langle 0\rangle$.
Let $n=p_1^{a_1}\dots p_r^{a_r}$ be a composite number having $r$ prime factors.
It should now be easy to see that $\{\langle p_k\rangle\}_{k=1}^r$ is
the set of all maximal ideals of $\Z_n$, since at least
one of these contains any other given
ideal, and each is properly contained only by $\Z_n$.

\section*{Problem 7}

Let $a$ belong to a commutative ring $R$.  Show that $aR=\{ar:r\in R\}$ is
an ideal of $R$.  If $R$ is the ring of even integers, list the elements of $4R$.

First note that $\{0\}\subseteq aR\subseteq R$, showing that $aR$ is non-empty.
Now let $u,v\in aR$.  Then $u=ar_u$ and $v=ar_v$ for some $r_u,r_v\in R$, and
$u-v=ar_u-ar_v=a(r_u-r_v)\in aR$ since $r_u-r_v\in R$.  Then let $r\in R$,
and using the commutativity of $R$, we see that $ur=ru=rar_u=arr_u\in aR$
since $rr_u\in R$.  It follows that $aR$ is an ideal of $R$ by the Ideal Test.

Let $R$ be the ring of even integers.
Then $4R=\{4r:r\in R\}=\{0,\pm 8, \pm 16, \pm 24, \pm 32, \dots\}$.

\section*{Problem 10}

If $A$ and $B$ are ideals of a ring, show that the sum of $A$ and
$B$, $A+B=\{a+b:a\in A,b\in B\}$, is an ideal.

Let $u,v\in A+B$.  Then $u=a_u+b_u$ and $v=a_v+b_v$ for some
$a_u,a_v\in A$ and $b_u,b_v\in B$.  So
$u-v=(a_u+b_u)-(a_v+b_v)=(a_u-a_v)+(b_u-b_v)\in A+B$ since
$a_u-a_v\in A$ and $b_u-b_v\in B$.  This shows that $A+B$ is
a subgroup.

For any $r$ in the ring containing $A$ and $B$,
let $x\in r(A+B)$.  Then $x=r(a+b)=ra+rb$ for some $a\in A$ and $b\in B$.
But $ra\in A$ and $rb\in B$, since $A$ and $B$ are ideals.
It follows that $x\in A+B$.  A similar argument shows that if
$x\in (A+B)r$, then $x\in A+B$.  We can now say that $A+B$ is ideal.

\section*{Problem 12}

If $A$ and $B$ are ideals of a ring, show that the product of $A$ and $B$,
$AB=\{a_1b_1+a_2b_2+\dots+a_nb_n:a_i\in A,b_i\in B,n>0\}$, is an ideal.

Let $u,v\in AB$ so that for some $i,j>0$, we have $u=a_1b_1+\dots+a_ib_i$ and
$v=a_1'b_1'+\dots+a_j'b_j'$.  Write $-v=(-a_1')b_1'+\dots+(-a_j')b_j'$.
It is then clear that $u-v=u+(-v)\in AB$.
Now let $r\in R$.  Then $ru=ra_1b_1+\dots+ra_ib_i\in AB$ since
$ra_k\in A$ for all $1\leq k\leq i$, and $ur=a_1b_1r+\dots+a_ib_ir\in AB$
since $b_kr\in B$ for all $1\leq k\leq i$.  The proof now follows from
the Ideal Test.

\section*{Problem 14}

Let $A$ and $B$ be ideals of a ring.  Prove that $AB\subseteq A\cap B$.

Let $x=a_1b_1+a_2b_2+\dots+a_nb_n$ be any element of $AB$.
Notice that $a_kb_k\in A$ for all $k$ since $A$ is ideal.
This shows that $x\in A$.
Then $a_kb_k\in B$ for all $k$ since $B$ is ideal.
This shows that $x\in B$ also.
So $x\in A\cap B$.

\section*{Problem 26}

Show that $\R[x]/\langle x^2+1\rangle$ is a field.

It was proven in Example 14 of the text that $\langle x^2+1\rangle$
is maximal in $\R[x]$.  So it follows directly from Theorem 14.4
that $\R[x]/\langle x^2+1\rangle$ is a field.

\section*{Problem 28}

Show that $A=\{(3x,y):x,y\in\Z\}$ is a maximal ideal of $\Z\oplus\Z$.

Suppose $B$ is an ideal of $\Z\oplus\Z$ that properly contains $A$.
Let $b\in B-A$.  Then $b=(u,v)$ where $u\not\equiv 0\pmod{3}$, but
one of $u\pm 1$ is congruent to 0 modulo 3.  So there must exist $a\in A$ such
that $a-b=(1,1)$ or $b-a=(1,1)$.  In either case,
$(1,1)\in B\implies B=\Z\oplus\Z$, since $B$ is ideal.
We can now say that $A$ is maximal.

\section*{Problem 32}

In $Z[x]$, the ring of polynomials with integer coeficients, let
$I=\{f(x)\in\Z[x]:f(0)=0\}$.  Prove that $I$ is not a maximal ideal.

It is clear that $I=\langle x\rangle$.
It was shown in Example 17 of the text that
$\langle x\rangle$ is an ideal of $\Z[x]$.
Now notice that $2+\langle x\rangle\in\Z[x]/\langle x\rangle$
has no multiplicative inverse.  So $\Z[x]/\langle x\rangle$ is
not a field.  So by Theorem 14.4, $\langle x\rangle$ is not maximal.

\section*{Problem 36}

Prove that $I=\langle 2+2i\rangle$ is not a prime ideal of $\Z[i]$.
How many elements are in $\Z[i]/I$?  What is the characteristic of $\Z[i]/I$?

Notice that $4\in I$ since $4=(1-i)(2+2i)$.  Then since $4(1+I)=0+I$, we
need only confirm that $2\not\in I$ to verify that $\Z[i]/I\approx\Z_4$.
(This reasoning follows from what we know about cyclic subgroups.)
So assume $2\in I$.  Then there exit integers $a$ and $b$ such that
$2=(a+bi)(2+2i)\implies 1=a-b+(a+b)i\implies a=1/2\not\in\Z$.
By contradiction, $2\not\in I$.

To show that $I$ is not prime, notice that $(2+I)^2=0+I$, showing that $\Z[i]/I$
is not an integral domain.  It follows from Theorem 14.3 that $I$ is not prime.
Alternatively, notice that while
$2^2\in I$, we know that $2\not\in I$, showing also that $I$ is not prime.

Since $\Z[i]/I$ is a cyclic group under its addition operation,
it's characteristic is going to be the order
of $\Z[i]/I$, which is the order of its generator.  Recall that by LaGrange's
Theorem, all elements raised to the additive power of the order of the group
are the additive identity.

\pagebreak
\section*{Problem 40}

Let $a$ and $b$ belong to a commutative ring $R$.  Prove that $\{x\in R:ax\in bR\}$
is an ideal.

Let $u$ and $v$ be in this set so that $au,av\in bR$.  Then there exist
$x_u,x_v\in R$ such that $au=bx_u$ and $av=bx_v$.
We then see that $a(u-v)=au-av=bx_u-bx_v=b(x_u-x_v)\in bR$, showing that
$u-v$ is in the said set.  It is therefore a subgroup of $R$.
We now want to show that the set absorbs elements from $R$.
So let $r\in R$ and notice that $aur=aru=arbx_u=b(arx_u)\in bR$, showing that
$ur$ and $ru$ belong to the said set.  The proof now follows from the Ideal Test.

\section*{Problem 42}

Let $R$ be a commutative ring and let $A$ be any subset of $R$.  Show that the
annihilator of $A$, $\ann(A)=\{r\in R:\mbox{$ra=0$ for all $a\in A$}\}$, is an ideal.

Let $u,v\in\ann(A)$.  Then to show that $u-v$ is an annihilator of $A$,
notice that $(u-v)a=ua-va=0-0=0$.  For any $r\in R$,
to show that $ur$ and $ru$ are annihilators of
$A$, notice that $ura=rua=r0=0$.  The proof now follows from the Ideal Test.

\section*{Problem 52}

Show that $\Z[i]/\langle 1-i\rangle$ is a field.
How many elements does this field have?

Since $1-i+\langle 1-i\rangle=0+\langle 1-i\rangle\implies 2+\langle
1-i\rangle=0+\langle 1-i\rangle$, we need only check that $1\not\in\langle 1-i\rangle$
to show that $\Z[i]/\langle 1-i\rangle\approx\Z_2$.  It would then follow that
$\Z[i]/\langle 1-i\rangle$ is a field since $\Z_2$ is a field.
We therefore want to find integers $a$ and $b$ such that
$(a+bi)(1-i)=1\implies (a+b)+(b-a)i=1$.  This requires $1=a+b$ and $0=-a+b$.
Adding these equations results in the conclusion that $b=1/2\not\in\Z$.
The proof now follows by contradiction.

We might also go about proving the result by showing that $\langle 1-i\rangle$
is maximal in $R[i]$.  It would then follow by Theorem 14.4 that
$R[i]/\langle 1-i\rangle$ is a field.
Suppose $I$ is an ideal of $R[i]$ properly containing $\langle 1-i\rangle$.
To characterize the elements of $I$ not in $\langle 1-i\rangle$, let $u+vi$
be such an element, and notice that
$(u+vi)/(1-i)=((u-v)+(u+v)i)/2\not\in R[i]$ implies that $u$ and $v$ have
different parity.  (Note that complex division is unambiguated by the fact that
$R[i]$ is a commutative ring.)  Letting $x,y\in\Z$, we now consider two cases.
In the first case, we have $(2x+1)+2yi\in I\implies 1+2x+2yi+I=0+I\implies 1+I=0+I\implies 1\in I$ since
$(2x+2yi)/(1-i)=((2x+2y)+(2y-2x)i)/2\in\Z[i]\implies
2x+2yi\in\langle 1-i\rangle\subset I$.  In the second case, we have
$2x+(2y+1)i\in I\implies i+2x+2y+I=0+I\implies i+I=0+I\implies 1+I=(i+I)^4=I^4=I
\implies 1\in I$.  In either case, we have $1\in I$, showing that $I$ absorbs
all elements of $R[i]$, so that $I=R[i]$.

\end{document}
