\documentclass{birkjour}

%\usepackage{tikz}
%\usepackage{graphicx}
\usepackage{hyperref}

\newtheorem{thm}{Theorem}[section]
\newtheorem{cor}[thm]{Corollary}
\newtheorem{lem}[thm]{Lemma}
\newtheorem{prop}[thm]{Proposition}
\theoremstyle{definition}
\newtheorem{defn}[thm]{Definition}
\theoremstyle{remark}
\newtheorem{rem}[thm]{Remark}
\newtheorem*{ex}{Example}
\numberwithin{equation}{section}

\newcommand{\R}{\mathbb{R}}
\newcommand{\B}{\mathbb{B}}
\newcommand{\G}{\mathbb{G}}
\newcommand{\V}{\mathbb{V}}
\newcommand{\gd}{\dot{g}}
\newcommand{\gh}{\hat{g}}
\newcommand{\Gd}{\dot{G}}
\newcommand{\Gh}{\hat{G}}
\newcommand{\nvai}{\infty}
\newcommand{\nvao}{o}
\newcommand{\grade}{\mbox{grade}}

\begin{document}

\title{On The Problem Of Intersecting\\Algebraic Surfaces Using\\Geometric Algebra}

\author{Spencer T. Parkin}
\address{102 W. 500 S., \\
Salt Lake City, UT  84101} \email{spencerparkin@outlook.com}

%\subjclass{Primary 14J70; Secondary 14J29}

%\dedicatory{To my dear wife Melinda.}

\begin{abstract}
Abstract goes here...
\end{abstract}

\keywords{Keywords go here...}

\maketitle

\section{Introduction}

The intersection of two algebraic surfaces of the same dimension may be
realized in different ways.  A natural realization of such an intersection would
simply be that of an algebraic surface, real or imaginary\footnote{The surfaces
in question may have no real intersection.  Cases of tangency should also be considered.},
of dimension one
less than those taken in the intersection.  In \cite{}, the intersection of quadratic surfaces
are found as parametric curves through space.
In this paper, however, we will attempt to realize the intersection of two algebraic surfaces as a
reinterpretation of a given characterization of the intersection with what may be considered
its canonical characterization.

For example, suppose we wish to realize a conic section given as the intersection of a plane
and a conical surface.  If we know enough about this intersection to begin with, then we may
be justified in reinterpreting this intersection as that of a plane and an elliptical cylinder whose
axis meets the plane at a right angle.  The benefit of doing so is in noticing that the latter
characterization of the given intersection lends itself to analysis through decomposition.
That is, we can easily break the canonical form of the intersection down into its characteristic
parts; which parts, collectively, help us realize the intersection as, in this case, the foci and orientation of
the ellipse that is the intersection of the elliptical cylinder and the plane.  By this analysis
we have effectively found the intersection that was originally given to us.

The rest of this paper will now begin exactly where the first section\footnote{The reader
need not read all of \cite{}; just the first section.} of \cite{} ended, and
so it is assumed that, before continuing on, the reader has become familiar with
all terms and definitions given in that section.

\section{Further Developments Of Our Model}

What has been given thus far in \cite{} is a general model of geometry based upon
a geometric algebra $\G$.  Continuing with this model, we wish here to develop it
further for the purpose of finding intersections as described in the introductory section.

To that end, we begin by letting $B\in\G$ be a $k$-blade, and then consider the surface $\gh(B)$.
If now there exist $k$ points $\{x_i\}_{i=1}^k\subseteq\gh(B)$ such that $\bigwedge_{i=1}^k p(x_i)\neq 0$,
then it is clear that there must exist a scalar $\beta\in\R$ such that
\begin{equation}\label{equ_point_factorization}
B = \beta\bigwedge_{i=1}^k p(x_i).
\end{equation}
Being able to find such a factorization of $B$ is perhaps desirable for many reasons, but the
reason it is sought after in this paper is explained as follows.

Given two $k$-blades $A,B\in\G$, if there exists a scalar $\lambda\in\R$
such that $A=\lambda B$, it is clear that $\gh(A)=\gh(B)$.  If, however, we know that
the blades $A$ and $B$ each characterize the same geometry, (even though they may do so in different ways),
we have $\gh(A)=\gh(B)$, but it does not necessarily follow that there exists a
scalar $\lambda\in\R$ such that $A=\lambda B$.  If such a scalar did exist, then
we would clearly have an algebraic means of relating the two characterizations so that
an analysis\footnote{If $A$ was composed as some intersection we had wished to take,
then $B$ may be the canonical form
that we wish to decompose.} by decomposition of either $A$ or $B$ could move forward.

Returning now to equation \eqref{equ_point_factorization}, notice that if $B$ has
such a factorization, then clearly so does $A$, and we may write $A=\alpha\bigwedge_{i=1}^k p(x_i)$.
Letting
\begin{equation*}
\lambda=\frac{\alpha}{\beta},
\end{equation*}
we now have $A=\lambda B$.

Seeing now why such factorizations as that in equation \eqref{equ_point_factorization} are desireable for
our purposes, let us consider when such factorizations exist.

We may first observe that if $p$ is a linear function, then $\{p(x_i)\}_{i=1}^k$ is linearly
independent if and only if $\{x_i\}_{i=1}^k$ is linearly independent.
Requiring $p$ be linear, however, we could never represent non-linear surfaces within
our model.

% I can show other forms of p and what conditions we get linear independence.

% The idea was that even though I can't get a full point factorization,
% I can show that two characterization share a large sub-space in common,
% which means that one blade (char.) may be a versor (even) multiple of the other,
% as apposed to a scalar multiple.  A successful intersection should be taken
% before any more writing is done in this paper.

\end{document}
