\documentclass{birkjour}

\usepackage{tikz}
\usepackage{graphicx}
\usepackage{hyperref}
\usepackage{algpseudocode}
\usepackage{graphicx}

\newtheorem{thm}{Theorem}[section]
\newtheorem{cor}[thm]{Corollary}
\newtheorem{lem}[thm]{Lemma}
\newtheorem{prop}[thm]{Proposition}
\theoremstyle{definition}
\newtheorem{defn}[thm]{Definition}
\theoremstyle{remark}
\newtheorem{rem}[thm]{Remark}
\newtheorem*{ex}{Example}
\numberwithin{equation}{section}

\newcommand{\R}{\mathbb{R}}
\newcommand{\B}{\mathbb{B}}
\newcommand{\G}{\mathbb{G}}
\newcommand{\V}{\mathbb{V}}
\newcommand{\M}{\mathbb{M}}
\newcommand{\J}{\mathbb{J}}
\newcommand{\gd}{\dot{g}}
\newcommand{\gh}{\hat{g}}
\newcommand{\Gd}{\dot{G}}
\newcommand{\Gh}{\hat{G}}
\newcommand{\nvai}{\infty}
\newcommand{\nvao}{o}
\newcommand{\grade}{\mbox{grade}}
\newcommand{\meet}{{\it meet}\;}
\newcommand{\join}{{\it join}\;}
\newcommand{\vd}{\dot{v}}
\newcommand{\vh}{\hat{v}}
\newcommand{\reverse}{\tilde}

%\received{}\accepted{}

\begin{document}

\title{The Meet, Join, And Geometric Sets}

\author{Spencer T. Parkin}
\address{102 W. 500 S., \\
Salt Lake City, UT  84101} \email{spencerparkin@outlook.com}

%\subjclass{Primary 14J70; Secondary 14J29}

%\dedicatory{To my dear wife Melinda.}

\begin{abstract}
A follow-up to \cite{}, this paper shows that the intersection of
any two goemetric sets is geometric, then goes on to consider other
results about geometric sets in terms of the \meet and \join operations.
\end{abstract}

%\keywords{Geometric Algebra, Model Of Geometry, Geometric Set}

\maketitle

\section{Introduction}

In \cite{} the notion of a geometric set was introduced, and it was shown that under
some circumstances, the intersection of two geometric sets is geometric.  In this paper
we show that this is the case under all circumstances.

All conventions and notation used in \cite{} are carried forward here, and
so for brevity, will not be needlessly repeated.

\section{The Meet And Join Of Vector Sub-Spaces}

An excellent treatment of \meet and \join is given in \cite{}.
Unlike \cite{}, however, here we will not, as they said, abuse language by referring to
any blade as a vector sub-space.\footnote{The present author has abused language
in this way in other writings, but now repents of this, at least in the present paper.}  In an attempt to be as
rigorous as possible, we will introduce the following definition.
\begin{defn}
For any blade $B\in\B$, we let
\begin{align}
\vh(B) &= \{v\in\V|v\wedge B=0\}, \\
\vd(B) &= \{v\in\V|v\cdot B=0\}.
\end{align}
\end{defn}

Realize that for any two blades $A,B\in\B$, we have $\vh(A)=\vh(B)$ if and only if
there exists a scalar $\lambda\in\R$ such that $A=\lambda B$.

As operations (as well defined functions), \meet and \join operate on vector spaces, not blades.
We therefore do not speak of taking the \meet or \join of two blades.  Rather, the blades of a geometric
algebra will help us calculate the \meet and \join of vector sub-spaces represented by those blades.

\begin{defn}[The \meet of two vector sub-spaces]\label{def_meet}
For any two blades $A,B\in\B$, the \meet $\M$ of $\vh(A)$ and $\vh(B)$ is the
vector space given by
\begin{equation}
\M = \vh(A)\cap \vh(B) = \{x\in\V|\mbox{$x\in\vh(A)$ and $x\in\vh(B)$}\}.
\end{equation}
\end{defn}
From Defintion~\ref{def_meet} we see that the \meet of two vector subspaces
is the largest common sub-space.  Notice that \meet is clearly communtative.

\begin{defn}[The \join of two vector sub-spaces]\label{def_join}
For any two blades $A,B\in\B$, the \join $\J$ of $\vh(A)$ and $\vh(B)$ is the
vector space given by
\begin{equation}
\J = \vh(A)+\vh(B) = \{x_1+x_2\in\V|\mbox{$x_1\in\vh(A)$ and $x_2\in\vh(B)$}\}.
\end{equation}
\end{defn}
From Definition~\ref{def_join} we see that the \join of two vector spaces
is the smallest common super-space.  Notice here too the commutativity of \join.

Some care should be taken with Definition~\ref{def_meet} and
Definition~\ref{def_join} by easily verifying that $\M$ and $\J$ each satisfy the
necessary properties of a vector space.

At this point a natural question arrises.  Given two blades $A,B\in\B$, how do we find blades
$M,J\in\B$ such that $\vh(M)=\vh(A)\cap\vh(B)$ and $\vh(J)=\vh(A)+\vh(B)$?
While it is not immediately clear if a closed-form formula exists for an $M$ or a $J$ in terms
of $A$ and $B$, \cite{} shows that there are many identities that relate all four blades,
and \cite{} gives an algorithm for calculating the \join of any two given blades.  Once the
\join is known, it is then easy to calculate the \meet, and vice-versa.
In any case, this paper is merely concerned with the realization that
there must exist blades $A',B'\in\B$ such that
\begin{equation}
\vh(J) = \vh(A'\wedge M\wedge B'),
\end{equation}
where $A'\wedge M=A$ and $M\wedge B'=B$.

% citation: http://www.geometricalgebra.net/downloads/meet_and_join_by_grunberg_20080328.pdf

\section{The Intersection Of Geometric Sets}

We now come to the main result of this paper, stated with the following theorem.

\begin{thm}
The intersection of any two geometric sets is geometric.
\end{thm}
\begin{proof}
For any pair of geometric sets $R$ and $S$, let $A,B\in\B$ be a pair of blades
such that $\gd(A)=R$ and $\gd(B)=S$.  In the case that $A\wedge B\neq 0$,
we have
\begin{equation}
\gd(A\wedge B)=R\cap S.
\end{equation}
On the other hand, if $A\wedge B=0$, consider a blade $J$ given by
\begin{equation}
J = A'\wedge M\wedge B',
\end{equation}
where $A'\wedge M=A$ and $M\wedge B'=B$.  In this case, we have
\begin{align}
\gd(J) &= \gd(A')\cap\gd(M)\cap\gd(B') \\
 &=(\gd(A')\cap\gd(M))\cap(\gd(M)\cap\gd(B')). \\
 &= R\cap S.
\end{align}
In any case, notice that a blade directly representative of the \join is also a blade dually
representative of the intersection.
\end{proof}

Theorem~\ref{} then begs the question, if the \join gives us intersections, what
does the \meet give us?  It is not hard to see that if blades $A$ and $B$ are directly
representative of $R$ and $S$, respectively, then $\gh(M)=R\cap S$.  But what
does the \meet give us if $A$ and $B$ are dually representative of $R$ and $S$?
This may be the same as asking: what does the \join give us if $A$ and $B$ are directly
representative of $R$ and $S$?  In that case, one way to proceed is to consider
irreducible representatives.

\end{document}