\documentclass{birkjour}

\usepackage{amsmath}
\usepackage{amssymb}
\usepackage{amsthm}
\usepackage{graphicx}
\usepackage{float}
\usepackage{url}

\newtheorem{thm}{Theorem}[section]
\newtheorem{cor}[thm]{Corollary}
\newtheorem{lem}[thm]{Lemma}
\newtheorem{prop}[thm]{Proposition}
\theoremstyle{definition}
\newtheorem{defn}[thm]{Definition}
\theoremstyle{remark}
\newtheorem{rem}[thm]{Remark}
\newtheorem*{ex}{Example}
\numberwithin{equation}{section}

\newcommand{\G}{\mathbb{G}}
\newcommand{\V}{\mathbb{V}}
\newcommand{\R}{\mathbb{R}}
\newcommand{\Poly}{\mathbb{P}}

\begin{document}

\title{A Model Of Algebraic Sets\\Using\\Geometric Algebra}

\author{Spencer T. Parkin}
\email{spencer.parkin@gmail.com}

\numberwithin{equation}{section}

\subjclass{Primary 14J70; Secondary 14J29}

\keywords{Quadric Surface, Quartic Surface, Geometric Algebra, Quadric Model, Conformal Model}

%\dedicatory{To Melinda and Naomi}

\begin{abstract}
Blah.
\end{abstract}

\maketitle

\section{Introduction And Motivation}

Letting $\R^n$ denote an $n$-dimensional Euclidean space, we are
going to let $\Poly^n$ denote the set of all polynomials $p:\R^n\to\R$
of any degree.  Then, letting $P\subseteq\Poly^n$ be any set of such
polynomials, we then define the zero set $Z$ of $P$, denoted $Z(P)$,
as the set given by
\begin{equation}\label{equ_zero_set_of_polys}
Z(P) = \{x\in\R^n|\mbox{$p(x)=0$ for all $p\in P$}\}.
\end{equation}
Every subset $S$ of $\R^n$ for which there exists a set $P\subseteq\Poly^n$
such that $S=Z(P)$ is what we refer to as an algebraic set.  It is well known
that for any algebraic set $S\subseteq\R^n$, there is always such a subset $P$
of $\Poly^n$ of finite cardinality.

Given the definition in equation \eqref{equ_zero_set_of_polys}, it is easy to show that
the subsets $S$ of $\R^n$ that are the geometries of
CGA, and other similar models of geometry based upon geometric algebra, are simply algebraic sets.
The goal of this paper is to show that there exists a model of geometry, based upon geometric algebra,
where every possible algebraic set has a representative in the form of an element of that geometric algebra.
A desire to come up with such a model of geometric algebra is motivated by the admittedly fanciful
dream of the German mathematician Leibniz, referred to in \cite{} and claimed to
have already been realized in \cite{}.  In any case, it would seem that a generalization
of CGA or any CGA-like model to one that hosts the set of all algebraic sets in $\R^n$ would bring
us closer to such a goal.  Work to this end has already been done in \cite{Parkin13a,Parkin13b} which
has brought us, the reader and writer, to the present paper.

\section{Blah}

We begin with an examination of $\Poly^n$ as a linear space, observing that
it is of countably infinite dimension.  A set of basis vectors for this space may be taken
as the set of all unit monomials in the $n$ variables of an arbitrary point in $\R^n$.
The number $c_k$ of such monomials homogeneous of a degree $k$ is given by
\begin{equation}
c_k=\sum_{i=1}^n\binom{n}{i}p(k,i),
\end{equation}
where $p(k,i)$ is the number of partitions of $k$ of size $i$.
% the combinatorics of the matter isn't important

% Possible citations:
% http://mathdl.maa.org/images/upload_library/22/Ford/DesmondFearnleySander.pdf
% Invariant algebras and geometric reasoning, Hongbo Li
% http://www.xtec.cat/~rgonzal1/proceedings.pdf

\bibliographystyle{amsplain}
\bibliography{Parkin_AModelOfAlgebraicSetsUsingGA}

\end{document}