\documentclass[12pt]{article}

\usepackage{amsmath}
\usepackage{amssymb}
\usepackage{amsthm}

\title{Chapter 12 Exercises\\Gallian's Book on Abstract Algebra}
\author{Spencer T. Parkin}

\newtheorem{theorem}{Theorem}[section]
\newtheorem{definition}{Definition}[section]
\newtheorem{corollary}{Corollary}[section]
\newtheorem{identity}{Identity}[section]
\newtheorem{lemma}{Lemma}[section]
\newtheorem{result}{Result}[section]

%\newcommand{\gcd}{\mbox{gcd}}
\newcommand{\lcm}{\mbox{lcm}}
\newcommand{\abs}{\mbox{abs}}
\newcommand{\Z}{\mathbb{Z}}
\newcommand{\R}{\mathbb{R}}
\newcommand{\G}{\mathbb{G}}
\newcommand{\stab}{\mbox{stab}}
\newcommand{\aut}{\mbox{Aut}}
\newcommand{\inn}{\mbox{Inn}}
\newcommand{\orb}{\mbox{orb}}

\begin{document}
\maketitle

\section*{Exercise 14}

Let $a$ and $b$ belong to a ring $R$ and let $m$ be an integer.  Prove
that $m\cdot (ab)=(m\cdot a)b=a(m\cdot b)$.

Notice that
\begin{equation*}
m\cdot (ab)=\underbrace{ab+\dots+ab}_m=a(\underbrace{b+\dots+b}_m)=a(m\cdot b).
\end{equation*}
A proof that $m\cdot (ab)=(m\cdot a)b$ is similar.  This is just repeated use of the distributive
properties.  We might have used induction, I suppose; but that seems like over-kill.

\section*{Exercise 15}

Show that if $m$ and $n$ are integers and $a$ and $b$ are elements from a ring,
then $(m\cdot a)(n\cdot b)=(mn)\cdot(ab)$.

Notice that
\begin{align*}
(m\cdot a)(n\cdot b) &= (m\cdot a)(\underbrace{b+\dots+b}_n) \\
 &= \underbrace{(m\cdot a)b+\dots+(m\cdot a)b}_n \\
 &= \underbrace{m\cdot (ab)+\dots+ m\cdot(ab)}_n = (nm)\cdot (ab).
\end{align*}

\section*{Exercise 17}

Show that a ring that is cyclic under addition is commutative.

Let $R=\langle g\rangle$ and let $a,b\in R$.
Then there exist integers $m$ and $n$ such that
$a=m\cdot g$ and $b=n\cdot g$.  Now notice that
\begin{equation*}
ab=(m\cdot g)(n\cdot g)=(mn)\cdot g^2=(nm)\cdot g^2=(n\cdot g)(m\cdot g)=ba
\end{equation*}
by Exercise 15.

\section*{Exercise 18}

Let $a$ belong to a ring $R$.  Let $S=\{x\in R|ax=0\}$.  Show that $S$ is a subgring of $R$.

Notice that $0\in S$, which we must have if $S$ is to be an Abelian group.
Now let $x\in S$, and see that $a(-x)=-(ax)=-0=0$, showing that $-x\in S$.
Now let $x,y\in S$, and see that $a(x+y)=ax+ay=0+0=0$, showing that
$x+y\in S$.  Thus far we have shown that $S$ is a group under
the additive operation of $R$.  Letting $x,y\in S$ once again,
see that $axy=(ax)y=0\cdot y=0$, showing that $xy\in S$.
We can now claim by Theorem 12.3 that $S$ is a subring of $R$.

\section*{Exercise 22}

Let $R$ be a commutative ring with unity and let $U(R)$ denote the set of
units of $R$.  Prove that $U(R)$ is a group under the multiplication of $R$.
(This group is called the {\it group of units of $R$}.)

Seeing that the unity 1 of $R$ is a unit, we have $1\in U(R)$, so $U(R)$ is not empty.
Now notice that $x\in U(R)$ if and only if $x^{-1}\in R$ exists, and clearly $(x^{-1})^{-1}=x$,
so $x^{-1}\in U(R)$ too.  Now let $x,y\in U(R)$.  Then, seeing that $(xy)^{-1}=y^{-1}x^{-1}\in R$,
we must have $xy\in U(R)$.  Why did $R$ need to be a commutative ring?  Did I miss something?

\section*{Exercise 30}

Suppose that there is an integer $n>1$ such that $x^n=x$ for all elements
$x$ of some ring.  If $m$ is a positive integer and $a^m=0$ for some $a$,
show that $a=0$.

If $m=1$, we're done.  So let $m>1$.

If $n>m$, then let $n=m+k$ and we have $a=a^n=a^ma^k=0\cdot a^k = 0$.
If $n=m$, then $0=a^m=a^n=a$.
If $n<m$, then let $m_1=m$ and we have $0=a^{m_1}=a^na^{m_1-n}=a^{m_1-n+1}$.
If we then let $m_2=m_1-n+1$, we're done if $m_2\leq n$; otherwise, we have
$m_2<m_1$, and $0=a^{m_2}=a^na^{m_2-n}=a^{m_2-n+1}$.
Now let $m_3=m_2-n+1$, and continue this process, which must terminate with
the conclusion that $a=0$.

\section*{Exercise 36}

Let $m$ and $n$ be positive integers and let $k$ be the least common multiple of $m$ and $n$.
Show that $mZ\cap nZ=kZ$.

If $x\in mZ\cap nZ$, then $x$ is a multiple of $m$ and $n$.  But $\lcm(m,n)$ divides
all such multiples, so there exists $z\in Z$ such that $x=zk\implies x\in kZ$.
Now if $x\in kZ$, there exists $z\in Z$ such that $x=zk=z\lcm(m,n)$.
So $x$ is a multiple of $n$ and $m$.  But $mZ\cap nZ$, by construction,
contains all such multiples.  So $x\in mZ\cap nZ$.

\end{document}