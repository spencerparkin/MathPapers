\documentclass{article}
\usepackage{amsmath}
\usepackage{amssymb}
\title{Chapter 9 Homework}
\author{Spencer}
\addtolength{\oddsidemargin}{-.575in}
\addtolength{\evensidemargin}{-.575in}
\addtolength{\textwidth}{1.0in}
\addtolength{\topmargin}{-.575in}
\addtolength{\textheight}{1.25in}
\begin{document}
\maketitle

%\section*{Problem 3}
%
%Show that if $G$ is the internal direct product of $H_1, H_2, \dots, H_n$ and
%$i\neq j$ with $1\leq i\leq n$, $1\leq j\leq n$, then $H_i\cap H_j=\{e\}$.
%
%Without loss of generality, let $1\leq i<j\leq n$.
%Suppose $h\in H_i\cap H_j$ where $h\neq e$.
%Then $e^{i-1}he^{j-i-1}=h\in H_1H_2\dots H_{j-1}$ and $h\in H_j$.
%But the definition of the internal direct product $H_1\times H_2\times\dots\times H_n$
%requires $(H_1H_2\dots H_{j-1})\cap H_j=\{e\}$.  The result now follows
%by contradiction.
%
%It should be noted that the converse is not nessisarily true.
%Counter example?

\section*{Problem 36}

Determine all subgroups of $\mathbb{R}^*$ (nonzero reals under multiplication) of index 2.

$\mathbb{R}^*$ is Abelian, and therefore so is any one of its subgroups.
But Abelian groups are also normal, so all subgroups of $\mathbb{R}^*$ are normal.
Suppose $S$ is a subgroup of $\mathbb{R}^*$ where $|\mathbb{R}^*:S|=2$.
Then $|\mathbb{R}^*/S|=2$, and so the only non-identity coset has order 2.
Surely the identity coset raised to the power of 2 is the identity coset
as well.  So let $aS$, where $a\in\mathbb{R}^*$, be any coset in
$\mathbb{R}^*/S$, and we see that $(aS)^2=a^2S=S\implies a^2\in S$, showing that
the square of every non-zero real number is in $S$.  Therefore, $\mathbb{R}^+\subseteq S$.
Suppose $\mathbb{R}^+\subset S$.  It follows that $S=\mathbb{R}^*$ by closure since in that case
$S$ must contain an negative real number.  So by contradiction, $S=\mathbb{R}^+$ is
the only subgroup of $\mathbb{R}^*$ having an index of 2.

\section*{Problem 38}

Let $H$ be a normal subgroup of $G$ and let $a$ belong to $G$.
If the element $aH$ has order 3 in the group $G/H$ and $|H|=10$,
what are the possibilities for the order of $a$?

We see that $a^3\in H$ since $(aH)^3=a^3H=H$.  Then by LaGrange's Theorem,
$|a^3|$ divides $|H|=10$.  So the possibilities for $|a^3|$ are 1, 2, 5,
and 10.  Using Theorem 4.1, $|a|/|a^3|=\gcd(|a|,3)$, showing that $|a|$
must be a multiple of $|a^3|$.  Letting $|a|=k|a^3|$, we see that
$k=\gcd(k|a^3|,3)$.  If $|a^3|=1$, $k=3$ works.  If $|a^3|=2$, then
$k=1$ works.  If $|a^3|=5$, then $k=1,3$ work again.  If $|a^3|=10$,
then $k=1,3$ work yet again.  So the choices for $|a|$ are 2, 3, 5,
10, 15, and 30.  I'm not sure how to narrow it down further.

\section*{Problem 40}

An element is called a square if it can be expressed in the form $b^2$
for some $b$.  Suppose that $G$ is an Abelian group and $H$ is a subgroup
of $G$.  If every element of $H$ is a square and every element of $G/H$
is a square, prove that every element of $G$ is a square.

First note that $H$, being a subgroup of an Abelian group, must itself
be Abelian, and therefore a normal subgroup of $G$ so that $G/H$ makes sense.

What we want to show is that for any element $g\in G$, there exists an
element $g'\in G$ such that $g=(g')^2$.  Consider the coset $gH$.  We know
that there exists $a\in G$ such that $(aH)^2=a^2H=gH$.  So there is a pair
$h_1,h_2\in H$ such that $a^2h_1=gh_2$.  But each of these are squares,
so let $h_1=(h_1')^2$ and $h_2=(h_2')^2$ and we have $a^2(h_1')^2=g(h_2')^2$.
Rearranging this, and using the Abelian property of $G$, we get
$g = (ah_1'h_2'^{-1})^2$ showing that $g$ is the square of some element in $G$.

\section*{Problem 46}

Let $G$ be an Abelian group and let $H$ be the subgroup consisting of all
elements of $G$ that have finite order.  Prove that every nonidentity element
in $G/H$ has infinite order.

First let us show that $H$ is a subgroup of $G$.  Notice that $e\in H$ since $|e|=1$
so that $H$ is non-empty.  Now choose $a,b\in H$.
Then the order of $|ab^{-1}|\leq|a||b^{-1}|=|a||b|$, since
$(ab^{-1})^{|a||b^{-1}|}=(a^{|a|})^{|b^{-1}|}((b^{-1})^{|b^{-1}|})^{|a|}=e$, by
the Abelian property of $G$.  This shows that $ab^{-1}\in H$.
So $H$ is a subgroup of $G$ by the one-step subgroup test.

If $G$ is a finite group, then $H=G$, and $G/H=\{eH\}$ so that there are
no nonidentity elements.  We therefore only care about infinite groups $G$.

Recall that the representatives of a coset are all members of the coset.

If $G$ is an infinite group, then all representatives of the identity
coset of $G/H$, (all elements of $H$), are the elements of $G$ having finite order.
For every other coset in $G/H$, their representatives have infinite order.
If $G=H$, then $G/H=\{eH\}$ as before, and we're done.  Suppose $H\subset G$.
Let $a\in G-H$ so that $aH$ is a nonidentity element of $G/H$.
Then $a$ has infinite order, and we need to show that no power
of $a$ has finite order, or else we would have $(aH)^n=a^nH=H$ for some
integer $n>1$.  Suppose this is so.  Then $a^n\in H$ so that $|a^n|<\infty$.
But then $(a^n)^{|a^n|}=a^{n|a^n|}=e$, contradicting the fact that $|a|=\infty$.

\section*{Problem 50}

Show that the intersection of two normal subgroups of $G$ is a normal subgroup of $G$.

First let us show that the intersection of two subgroups $A$ and $B$ of $G$ is a subgroup of $G$.
Notice that $e\in A\cap B$.  Let $a,b\in A\cap B$.
This implies that $a,b\in A\implies ab^{-1}\in A$, and that $a,b\in B\implies ab^{-1}\in B$.
Therefore, $ab^{-1}\in A\cap B$, and the intersection of $A$ and $B$ is a subgroup of $G$
by the one-step subgroup test.

Now notice that for all $g\in G$, we have $gA=Ag$ and $gB=Bg$.
Therefore...
\begin{equation*}
g(A\cap B) = gA\cap gB = Ag\cap Bg = (A\cap B)g
\end{equation*}
...showing that all right cosets of the intersection are equal to all left
cosets of the intersection.  $A\cap B$ is therefore a normal subgroup of $G$ also.

\section*{Problem 56}

Let $G$ be a group and let $G'$ be the subgroup of $G$ generated by the
set $S=\{x^{-1}y^{-1}xy:x,y\in G\}$.  Do some proofs about it.

First, for any $g\in G'$, define the notation $S_g$ to be the ordered set of elements
in $S$ whose product is $g$, so we can write $g=\prod_{s\in S_g}s$.  This is an ordered product
using the operation of $G$ and the order of $S_g$.
Let $-S_g$ denote $S_g$ in the reverse order.

Let's show that the commutator $G'$ generated by $S$ is
a subgroup of $G$.
Notice that $e\in S$ and $S\subseteq G'\subseteq G$ by definition and
the closure of $G$.
The closure of $G'$ is easy to see, since the product of two products
from $S$ is yet another product from $S$.  What remains to be shown is
that $G'$ is closed under the operation of taking inverses.  Notice
that $S$ is closed under this operation since if $x,y\in G$, then
$(x^{-1}y^{-1}xy)^{-1}=y^{-1}x^{-1}yx\in S$.  Now let $g\in G'$.
Then $g^{-1}=\prod_{a\in-S_g}a^{-1}\in G'$ since $g^{-1}$ is
the product of elements in $S$.  So $G'$ is a subgroup of $G$ by
the two-step subgroup test.

Now let's show that $G'$ is a characteristic subgroup of $G$.  We need to
show that if $\phi$ is an automorphism of $G$, then
$\phi(G')=G'$.  Notice that it suffices to show that $\phi(G')\subseteq G'$
since the proof of this also shows that
$\phi^{-1}(G')\subseteq G'\implies \phi(G')\supseteq G'$.
Let $s\in S$.  Then there exists $x,y\in G$ such that $s=x^{-1}y^{-1}xy$
and we see that $\phi(s)=\phi(x)^{-1}\phi(y)^{-1}\phi(x)\phi(y)\in S$ by
the definition of $S$ and the properties of $\phi$.
Now let $g\in G'$ so that $g=\prod_{s\in S_g}s$.
Then $\phi(g) = \prod_{s\in S_g}\phi(s)\in G'$.

\subsection*{Part a.}

Prove that $G'$ is normal in $G$.

Let $g\in G'$ so that $g=\prod_{a\in S_g}a$.
We want to show that for all $b\in G$, that $bgb^{-1}\in G'$.
The proof would then follow by the normal subgroup test.
Since $bgb^{-1}=\prod_{a\in S_g}bab^{-1}$, it suffices to show
that $bab^{-1}\in S$ for any $a\in S_g$.  Let $a=x^{-1}y^{-1}xy$ for some $x,y\in G$.
Then $bab^{-1} = u^{-1}v^{-1}uv$ where $u^{-1}=bx^{-1}b^{-1}$ and
$v^{-1}=by^{-1}b^{-1}$.  We see now that $bab^{-1}\in S$ since $u,v\in G$
by the closure of $G$.  This completes the proof.

\subsection*{Part b.}

Prove that $G/G'$ is Abelian.

What we want to show is that for any $a,b\in G$, that
$(aG')(bG') = (bG')(aG')$.  This is true if and only
if $(ab)G'=(ba)G'$.  Using Property 4 of the Lemma for cosets,
this is true if and only if $(ab)^{-1}ba=b^{-1}a^{-1}ba\in G'$.
But this must be true since $b^{-1}a^{-1}ba\in S$ and $S\subseteq G'$.
This completes the proof.

\subsection*{Part c.}

If $G/N$ is Abelian, prove that $G'\leq N$.

If it can be shown that $S\subseteq N$, then the proof follows
from being able to find any element of $G'$ in $N$ by using the
closure property of $N$.  In other words, if $S\subseteq N$, by the closure
of $N$, the group generated by $S$, namely $G'$, must be in $N$.

To show that $S\subseteq N$, let $a,b\in G$, and note that...
\begin{equation*}
(aN)(bN)=(bN)(aN)\implies (ab)N=(ba)N\implies b^{-1}a^{-1}ba\in N
\end{equation*}
...by the Abelian property of $G/N$.  This shows that any element of
$S$ is in $N$.  This completes the proof.

\subsection*{Part d.}

Prove that if $H$ is a subgroup of $G$ and $G'\leq H$, then $H$ is normal in $G$.

Let $g\in G$ and let $a\in H\subseteq G$.
Then $(ga)G' = (ag)G'$ since $(ga)^{-1}ag = a^{-1}g^{-1}ag\in S\subseteq G'$.
Then $(ag)G'=G'(ag)$ since $G'$ is normal in $G$ by part a. above.
We now see that...
\begin{equation*}
gH = \bigcup_{a\in H}(ga)G'=\bigcup_{a\in H}G'(ag)=Hg
\end{equation*}
...showing that $H$ is normal in $G$.  This is what we wanted.

\section*{Problem 62}

Show that if $N$ is a characteristic subgroup of $G$, then $N$ is a normal subgroup of $G$.

Let $a\in G$ and define $\phi(x)=axa^{-1}$ to be the inner automorphism of $G$ induced by $a$.
Since $N$ is a characteristic subgroup of $G$, we have $aNa^{-1}=\phi(N)=N\implies aN=Na$,
showing that $N$ is a normal subgroup of $G$.

\section*{Problem 66}

If $H$ is a normal subgroup of $G$ and $|H|=2$, prove that $H$ is
contained in the center of $G$.

By LaGrange's Theorem, $H=\langle h\rangle$ for some $h\in G$ of order 2,
and so $H=\{e,h\}$.  Now since $H$ is normal in $G$, for all $g\in G$,
we have $gH=Hg$.  But this must mean that for all $g\in G$, $ge=eg$ and
$gh=hg$, because we can't ever have $ge=hg$, since this implies that $h=e$.
We can now conclude that $h$ commutes with all other elements in $G$.
This is also true of $e$, so all elements of $H$ commute with all elements
of $G$, and so, by definition, $H$ must be contained in the center of $G$.

\end{document}
