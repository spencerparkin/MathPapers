\documentclass[12pt]{article}

\usepackage{amsmath}
\usepackage{amssymb}
\usepackage{amsthm}

\title{Chapter 11 Exercises\\Gallian's Book on Abstract Algebra}
\author{Spencer T. Parkin}

\newtheorem{theorem}{Theorem}[section]
\newtheorem{definition}{Definition}[section]
\newtheorem{corollary}{Corollary}[section]
\newtheorem{identity}{Identity}[section]
\newtheorem{lemma}{Lemma}[section]
\newtheorem{result}{Result}[section]

%\newcommand{\gcd}{\mbox{gcd}}
\newcommand{\lcm}{\mbox{lcm}}
\newcommand{\abs}{\mbox{abs}}
\newcommand{\Z}{\mathbb{Z}}
\newcommand{\R}{\mathbb{R}}
\newcommand{\G}{\mathbb{G}}
\newcommand{\stab}{\mbox{stab}}
\newcommand{\aut}{\mbox{Aut}}
\newcommand{\inn}{\mbox{Inn}}
\newcommand{\orb}{\mbox{orb}}

\begin{document}
\maketitle

\section*{Understanding the last part of Lemma 1}

We have $p^nm=|HK|=|H||K|$ with $\gcd(p^n,m)=1$.
If $p$ divides $|K|$, then $K$ has an element of order $p$
by Theorem 9.5; call it $k$.  But $k\in K\implies k^m=e\implies |k|=p|m$
by Corollary 2 of Theorem 4.1.  But this is a contradiction, since $\gcd(p,m)=1$,
and therefore $p$ does not divide $|K|$.  Now suppose that $m=qr$ with $q$ prime
and that $q$ divides $|H|$.
Then $H$ has an element of order $q$; call it $h$.  But $h\in H\implies h^{p^n}=e\implies |h|=q|p^n$,
but $\gcd(q,p)=1$, so by contradiction, $q$ does not divide $|H|$.

We can now argue that no prime in the factorization $p^n$ appears in $|K|$ and
that no prime in the factorization of $m$ appears in $|H|$.  So $|H|=p^n$ and $|K|=m$.

\section*{Exercise 11}

Prove that every finite Abelian group can be expressed as the
(external) direct product of cyclic groups of orders $n_1,n_2,\dots,n_t$, where $n_{i+1}$
divides $n_i$ for $i=1,2,\dots,t-1$.

It is clear by the Fundamental Theorem of Finite Abelian Groups that such
a group $G$ can, for some $k$ integers $m_1$ through $m_k$, always be written as
\begin{equation*}
G=Z_{m_1}\oplus\dots\oplus Z_{m_k}.
\end{equation*}
We now describe an algorithm for achieving the desired arrangement.
First, sort the product so that for any two distinct integers $i,j\in[1,k]$
with $i<j$, we have $m_i\geq m_j$.  If we then have the property
that for each such pair of integers, $m_j|m_i$, we're done.
Otherwise, find any two distinct integers $i,j\in[1,k]$ where $|Z_{m_i}|$
and $|Z_{m_j}|$ are coprime and collapse them into a single
group in the product as $Z_{m_im_j}$.  This does not change the
group represented, up to isomorphism, by Corollary 2 of Theorem 8.2.
Now reduce the integer $k$ by one and go back to the first step with
a new set of $k$ integers $m_1$ through $m_k$.

What would remain to be shown here is that this algorithm is
correct, which is to say that it will always terminate.  So suppose
we have a group $G$ where this algorithm doesn't terminate.
Then we can always find a pair of integers $i,j\in[1,k]$ such
that $m_i$ and $m_j$ are coprime, and therefore, we can
continue to collapse the product indefinitely.  But this is only
possible if the $|G|=\infty$, which is a contradiction, because
$|G|$ is finite.  So the algorithm will always terminate.
(This proof is not quite right to me, but it's good enough for now.)

\section*{Exercise 20}

Suppose that $G$ is a finite Abelian group that has exactly one subgroup
for each divisor of $|G|$.  Show that $G$ is cyclic.

By the Fundamental Theorem of Finite Abelian Groups, we may write $G$ as
\begin{equation*}
G=Z_{n_1}\oplus\dots\oplus Z_{n_k},
\end{equation*}
for a set of $k$ integers $n_1$ through $n_k$.
Suppose there exist distinct integers $i,j\in [1,k]$ such that
$d=\gcd(n_i,n_j)\neq 1$.  It follows that $Z_{n_i}$ and $Z_{n_j}$ each
have an element of order $d$ by the Fundamental Theorem of Cyclic Groups;
call them $z_i$ and $z_j$, respectively.
We then see that $G$ has two distinct elements of order $d$, (a divisor of $|G|$ by
Lagrange's Theorem), namely, $(e_1,\dots,a_i,\dots,e_k)$
and $(e_1,\dots,a_j,\dots,e_k)$, that each generate their own distinct subgroups of $G$
of order $d$.  But this violates the premise of the group $G$, so we can conclude
that no such integers $i$ and $j$ exist.  It now follows by Corollary 1 of Theorem 8.2
that $G$ is cyclic.

\section*{Exercise 21}

Characterize those integers $n$ such that the only Abelian groups of order $n$ are cyclic.

Let $n=p_1^{n_1}\dots p_k^{n_k}$ be the prime factorization of $n$ where
the primes $p_i$ are distinct.  If for each $i$, we have $n_i=1$, then we can
be assured that Abelian groups of order $n$ are cyclic.  If there is any $i$, such that $n_i>1$,
then there is an Abelian group of order $n$ that is non-cyclic, because there is
an Abelian group of order $p_i^{n_i}$ that is non-cyclic.

\section*{Exercise 31}

Without using Legrange's Theorem, show that an Abelian group of odd order
cannot have an element of even order.

By the Fundamental Theorem of Finite Abelian Groups,
$G$ has the form $Z_{n_1}\oplus\dots\oplus Z_{n_k}$
for $k$ integers $n_1$ through $n_k$.  It follows that
$|G|=|n_1|\dots |n_k|$.
Suppose now that there exists an integer $i\in[1,k]$ such that $n_i$ is even.
It would then follow that $|G|$ is even, which is a contradiction.
Therefore, each $n_i$ is odd.
It now follows by the Corllary of the Fundamental Theorem of Cyclic Groups (Theorem 4.3),
that for no integer $i$ does $Z_i$ have an element of even order.
Considering now an element $(a_1,\dots,a_k)\in G$, it is clear
by Theorem 8.1, that it does not have even order, because
$\lcm(|a_1|,\dots,|a_k|)$ cannot be even.

\end{document}