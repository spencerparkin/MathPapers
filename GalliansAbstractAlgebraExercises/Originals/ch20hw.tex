\documentclass{article}
\usepackage{amsmath}
\usepackage{amssymb}
\title{Chapter 20 Homework}
\author{Spencer}
\addtolength{\oddsidemargin}{-.575in}
\addtolength{\evensidemargin}{-.575in}
\addtolength{\textwidth}{1.0in}
\addtolength{\topmargin}{-.575in}
\addtolength{\textheight}{1.25in}
\begin{document}
\maketitle

\newcommand{\Z}{\mathbb{Z}}
\newcommand{\R}{\mathbb{R}}
\newcommand{\N}{\mathbb{N}}
\newcommand{\Q}{\mathbb{Q}}

\section*{Problem 8}

Let $F=\Z_2$ and let $f(x)=x^3+x+1\in F[x]$.  Suppose that $a$ is a zero of
$f(x)$ in some extension of $F$.  How many elements does $F(a)$ have?
Express each member of $F(a)$ in terms of $a$.  Write out a complete
multiplication table for $F(a)$.

We see that $f(x)$ is irreducible over $F$ since $\deg f(x)=3$ and it
has no zeros in $F$.  Using Theorem 20.3, we can find an extension
field of $F$ where $f(x)$ factors into a product of linear factors,
namely $F[x]/\langle f(x)\rangle\approx F(a)$.  This will also be
the smallest extension of $F$ containing all zeros of $f(x)$.
By the same theorem, $\{1,a,a^2\}$ is a basis for this extension
field over $F$.  That is, $F[x]/\langle f(x)\rangle\approx\{c_0+c_1a+c_2a^2:c_{0,1,2}\in F\}$.

The first and second rows and columns of the $8\times 8$ multiplication table
are boring, so here is the $6\times 6$ lower-right hand corner of that table.
\begin{equation*}
\begin{array}{cccccccc}
 & \vline & a & 1+a & a^2 & 1+a^2 & a+a^2 & 1+a+a^2 \\
\hline
a & \vline & a^2 & \cdot & \cdot & \cdot & \cdot & \cdot \\
1+a & \vline & a+a^2 & 1+a^2 & \cdot & \cdot & \cdot & \cdot \\
a^2 & \vline & a+1 & 1+a+a^2 & a+a^2 & \cdot & \cdot & \cdot \\
1+a^2 & \vline & 1 & a^2 & a & 1+a+a^2 & \cdot & \cdot \\
a+a^2 & \vline & 1+a+a^2 & 1 & 1+a^2 & 1+a & a & \cdot \\
1+a+a^2 & \vline & 1+a^2 & a & 1 & a+a^2 & a^2 & 1+a
\end{array}
\end{equation*}
It is lower triangular because it is equal to its transpose.

\section*{Problem 12}

Let $F=\Q(\pi^3)$.  Find a basis for $F(\pi)$ over $F$.

Without reading Chapter 21, I have no idea what $\Q(\pi^3)$
looks like, but all we need to know is that it contains $\pi^3$,
does not contain $\pi$, and contains the rationals.
Now let $p(x)=x^3-\pi^3\in F[x]$.  By Theorem 17.1, this is
irreducible over $F$.  We can now appeal to Theorem 20.3 to
find that $F[x]/\langle p(x)\rangle\approx F(\pi)$, and
a basis for this field is given by $\{1,\pi,\pi^2\}$ over $F$.

\section*{Problem 16}

Suppose that $\beta$ is a zero of $f(x)=x^4+x+1$ in some field extension $E$
of $\Z_2$.  Write $f(x)$ as a product of linear factors in $E[x]$.

Some work shows that $f(x)$ is irreducible over $\Z_2$, and so it is clear that
$f(x)$ has no zeros in $\Z_2$.  Using Theorem 20.3, we see that the smallest
extension field of $\Z_2$ containing all zeros of $f(x)$ is given by
$\Z_2[x]/\langle f(x)\rangle\approx \Z_2(\beta)$,
and is spanned by the basis $\{1,\beta,\beta^2,\beta^3\}$
over $\Z_2$.  It follows that $\Z_2[x]/\langle f(x)\rangle$ is a splitting field
for $f(x)$ over $\Z_2$.

Let us now write $f(x)$ as a product of linear factors in $\Z_2(\beta)[x]$.
Clearly, $x+\beta$ is a factor of $f(x)$.  A lot of work on scratch paper shows that
$f(x)$ factors as...
\begin{equation*}
(x+\beta)(x+\beta+1)(x+\beta^2)(x+\beta^2+1)
\end{equation*}

\section*{Problem 20}

Let $F$ be a field, and let $a$ and $b$ belong to $F$ with $a\neq 0$.
If $c$ belongs to some extension of $F$, prove that $F(c)=F(ac+b)$.

To show that $F(c)\subseteq F(ac+b)$, we must show that $c\in F(ac+b)$.
This is clearly true since $-b\in F(ac+b)$ and $a^{-1}\in F(ac+b)$ so
that $c=a^{-1}((ac+b)-b)\in F(ac+b)$.

To show that $F(ac+b)\subseteq F(c)$, we must show that $ac+b\in F(c)$.
This is clearly true since $a,b,c\in F(c)$ so that $ac+b\in F(c)$.

\section*{Problem 22}

Show that if $f(x)$ and $g(x)$ are relatively prime in $F[x]$,
they are relatively prime in $K[x]$, where $K$ is any extension field of $F$.

Because $f(x)$ and $g(x)$ are relatively prime, there exist polynomials
$r(x),s(x)\in F[x]$ such that $f(x)r(x)+g(x)s(x)=1$.
Now suppose that there exists $h(x)\in K[x]$ with $h(x)\not\in F[x]$
such that $f(x)=f'(x)h(x)$
and $g(x)=g'(x)h(x)$ for polynomials $f'(x),g'(x)\in K[x]$.
It follows that...
\begin{equation*}
f'(x)h(x)r(x)+g'(x)h(x)s(x)=h(x)(f'(x)r(x)+g'(x)s(x))=1
\end{equation*}
But then this implies that $\deg h(x)=0$ and therefore $h(x)\in F[x]$.
The result now follows by contradiction.

\section*{Problem 26}

Express $x^8-x$ as a product irreducibles over $\Z_2$.

Notice that $x^8-x=x^8+x$.  We factored this in the last exam.
The result was...
\begin{equation*}
x(x+1)(x^3+x+1)(x^3+x^2+1)
\end{equation*}

\section*{Problem 30}

Show that $x^4+x+1$ over $\Z_2$ does not have any multiple zeros in any
field extension of $\Z_2$.

This result follows directly from the results of Problem 16 above.
In Problem 16, we were general enough in determining the complete
factorization of this polynomial over its splitting field over $\Z_2$.
By the Corollary of Theorem 20.4, splitting fields are unique.
So any field containing all of the zeros of this polynomial will
contain a subfield isomorphic to its splitting field over $\Z_2$.

If there is a hole in my logic here, we can appeal to the Corollary
of Theorem 20.9.  In Problem 16, we found this polynomial's complete
factorization over its splitting field and saw no multiple zeros.

\section*{Problem 32}

Show that $x^{21}+2x^9+1$ has multiple zeros in some extension of $\Z_3$.

Because $\Z_3$ has characteristic 3, we can apply Theorem 20.6.
Let $f(x)=x^{21}+2x^9+1$ and let $g(x)=x^7+2x^3+1$.
Then $f(x)=g(x^3)$, and so $f(x)$ has a multiple zero.

\pagebreak
\section*{Problem 34}

Find the splitting field for $f(x)=(x^2+x+2)(x^2+2x+2)$ over $\Z_3[x]$.
Write $f(x)$ as a product of linear factors.

Clearly both quadratic factors are irreducible over $\Z_3$.
Using the result of Problem 24, a splitting field of $f(x)$ over $\Z_3$
is $\Z_3(a)(b)$, where $a$ is a zero of $x^2+x+2$ in some extension of
$\Z_3$, and $b$ is a zero of $x^2+2x+2$ in some extension of $\Z_3$.

By Theorem 20.3, we have $\Z_3(a)\approx \Z_3[x]/\langle x^2+x+2\rangle$,
where $\{1,a\}$ is a basis for it over $\Z_3$.  Using Theorem 20.3 again,
we have $\Z_3(a)(b)\approx\Z_3(a)[x]/\langle x^2+2x+2\rangle$ where
$\{1,b\}$ is a basis for it over $\Z_3(a)$.  Using the proof of Theorem 21.5,
a basis for $\Z_3(a)(b)$ is $\{1,a,b,ab\}$ over $\Z_3$.

At this point, let us reason that because $x^2+x+2$ splits in $\Z_3(a)$
and $x^2+2x+2$ splits in $\Z_3(b)$, we can factor them in their respective splitting fields,
and trust that the product of these factorizations is the factorization of $f(x)$
over $\Z_3(a,b)[x]$.  After working it out on scratch paper, I get...
\begin{equation*}
f(x)=(x+2a)(x+a+1)(x+2b)(x+b+2)
\end{equation*}
I think this works because $\Z_3(a)$ and $\Z_3(b)$ are subfields of $\Z_3(a,b)$.

\end{document}