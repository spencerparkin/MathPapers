\documentclass{birkjour}

\usepackage{float}
\usepackage{hyperref}

\newtheorem{thm}{Theorem}[section]
\newtheorem{cor}[thm]{Corollary}
\newtheorem{lem}[thm]{Lemma}
\newtheorem{prop}[thm]{Proposition}
\theoremstyle{definition}
\newtheorem{defn}[thm]{Definition}
\theoremstyle{remark}
\newtheorem{rem}[thm]{Remark}
\newtheorem*{ex}{Example}
\numberwithin{equation}{section}

\newcommand{\R}{\mathbb{R}}
\newcommand{\B}{\mathbb{B}}
\newcommand{\G}{\mathbb{G}}
\newcommand{\V}{\mathbb{V}}
\newcommand{\gd}{\dot{g}}
\newcommand{\gh}{\hat{g}}
\newcommand{\Gd}{\dot{G}}
\newcommand{\Gh}{\hat{G}}
\newcommand{\nvai}{\infty}
\newcommand{\nvao}{o}
\newcommand{\grade}{\mbox{grade}}

%\received{}\accepted{}

\begin{document}

\title{Another Way To Represent Geometric Sets}

\author{Spencer T. Parkin}
\email{spencerparkin@outlook.com}

%\subjclass{Primary 14J70; Secondary 14J29}

%\dedicatory{To my dear wife Melinda.}

\begin{abstract}
Abstract...
\end{abstract}

\keywords{Key words...}

\maketitle

\section{Blades And Spades}

As stated in the abstract of \cite{Fontijne10}, blades are a powerful tool for representing geometry.
They may be thought of as doing so under the definition of a \emph{geometric set} as given in \cite{Parkin15}, or as
doing so under an isomorphism as given in \cite{Hestenes01}.  In \cite{Parkin13}, another way
to represent geometry was presented that uses vectors homogeneous of a grade equal to the degree
of the polynomial whose zero set is the geometry in question.  In this paper, yet another way to represent
geometry in a geometric algebra is presented that is very much similar to that given in \cite{Parkin15,Hestenes01,Dorst07},
and certainly related, as we'll see; but different enough, and exhibiting enough desirable characteristics, that it may warrent attention.  Its conception begins with
a new term as found in Table~\ref{tbl_terms} among its traditional counter-parts.

\begin{table}[H]\label{tbl_terms}\caption{A few terms used in GA}
\begin{tabular}{p{1cm}p{9cm}}
Term & Definition \\
\hline
Blade & An outer product of zero or more linearly-independent vectors. \\
Versor & A geometric product of zero or more invertible vectors, not necessarily forming a linearly-independent set. \\
Spade & A geometric product of zero or more vectors, not necessarily forming a linearly-independent set.
\end{tabular}
\end{table}

The difference here between ``versor'' and ``spade'' may be subtle, but justifies the new term as it must be emphasized that each
vector in the product being invertible will not be a requirement for us, nor would we want it to be.  No where in
the derivision of any identity of this paper involving spades, nor in many proofs involving them, will we rely upon invertibility
with respect to the geometric product.  It is this kind of invertibility, however, that may make spades, as apposed to blades, in the
cases where invertible vector factors do appear, a desirable
alternative to representing what we'll see are the exact same classes of geometry; namely, geometric sets.

Similar to the concept of grade, that of rank will be introduced in this paper with respect to spades.  As an $r$-blade
refers to a blade of grade $r$, we will let an $r$-spade refer to a spade of rank $r$.  If an element of a geomtric
algebra can be written as any geometric product of vectors, then it is a spade.  The rank of that spade is then the smallest
possible number of vectors for which it can be written as such a product.\footnote{While the correctness of many identities of this paper
do not require a spade to be written in the most compact form, the concept of rank would be ill-defined without its consideration.}   Note that blades of grade zero
are indistinguishable from spades of the same rank as each denotes the set of all scalars.

% Check derivisions where I have assumed a grade-k part of a spade expansion is a blade.  This may not be the case,
% but due to linearity, the derivision may still go through.

\section{Representation By Blades}

The representation of geometric sets by spades is best appreciated when viewed in contrast to what may be its complementary representation by blades.
In any case, it may go beyond instructive to give the latter representation before the former as each are so closely related.

\section{Representation By Spades}

\begin{thebibliography}{9}

\bibitem{Dorst07}
L. Dorst, et. al.,
\emph{Geometric Algebra for Computer Science}.
Morgan Kaufmann, 1st edition, 2007.

\bibitem{Fontijne10}
D. Fontijne, et. al.,
\emph{Efficient algorithms for factorization and join of blades}.
Geometric Algebra Computing, Springer London, pp. 457-476, 2010.

\bibitem{Hestenes01}
D. Hestenes,
\emph{Old Wine in New Bottles: A new algebraic framework for computational geometry}.
Advances in Geometric Algebra with Applications in Science and Engineering,
Birkhauser Boston, pp. 3-17, 2001.

\bibitem{Parkin15}
S. Parkin,
\emph{An Introduction To Geometric Sets}.
Advances in Applied Clifford Algebras, Volume 25, Issue Unknown, pp. 639-655, 2015.

\bibitem{Parkin13}
S. Parkin,
\emph{The Mother Minkowski Algebra Of Order $M$}.
Advances in Applied Clifford Algebras, Volume 24, Issue 1, pp. 193-203, 2013.

\end{thebibliography}

\end{document}