\documentclass{birkjour}

\usepackage{amsmath}
\usepackage{amssymb}
\usepackage{amsthm}
\usepackage{graphicx}
\usepackage{float}
\usepackage{url}

\newtheorem{thm}{Theorem}[section]
\newtheorem{cor}[thm]{Corollary}
\newtheorem{lem}[thm]{Lemma}
\newtheorem{prop}[thm]{Proposition}
\theoremstyle{definition}
\newtheorem{defn}[thm]{Definition}
\theoremstyle{remark}
\newtheorem{rem}[thm]{Remark}
\newtheorem*{ex}{Example}
\numberwithin{equation}{section}

\newcommand{\G}{\mathbb{G}}
\newcommand{\V}{\mathbb{V}}
\newcommand{\R}{\mathbb{R}}
\newcommand{\Z}{\mathbb{Z}}
\newcommand{\Poly}{\mathbb{P}}

\begin{document}

\title{A Model Of Algebraic Sets\\Using\\Geometric Algebra}

\author{Spencer T. Parkin}
\email{spencer.parkin@gmail.com}

\numberwithin{equation}{section}

\subjclass{Primary 14J70; Secondary 14J29}

\keywords{Quadric Surface, Quartic Surface, Geometric Algebra, Quadric Model, Conformal Model}

%\dedicatory{To Melinda and Naomi}

\begin{abstract}
Blah.
\end{abstract}

\maketitle

\section{Introduction And Motivation}

Letting $\R^n$ denote an $n$-dimensional Euclidean space, we are
going to let $\Poly^n$ denote the set of all polynomials $p:\R^n\to\R$
of any degree.  Then, letting $P\subseteq\Poly^n$ be any set of such
polynomials, we then define the zero set $Z$ of $P$, denoted $Z(P)$,
as the set given by
\begin{equation}\label{equ_zero_set_of_polys}
Z(P) = \{x\in\R^n|\mbox{$p(x)=0$ for all $p\in P$}\}.
\end{equation}
Every subset $S$ of $\R^n$ for which there exists a set $P\subseteq\Poly^n$
such that $S=Z(P)$ is what we refer to as an algebraic set.  It is well known
that for any algebraic set $S\subseteq\R^n$, there is always such a subset $P$
of $\Poly^n$ of finite cardinality.

Given the definition in equation \eqref{equ_zero_set_of_polys}, it is easy to show that
the subsets $S$ of $\R^n$ that are the geometries of
CGA, and other similar models of geometry based upon geometric algebra, are simply algebraic sets.
The goal of this paper is to show that there exists a model of geometry, based upon geometric algebra,
where every possible algebraic set has a representative in the form of an element of that geometric algebra.
A desire to come up with such a model of geometric algebra is motivated by the admittedly fanciful
dream of the German mathematician Leibniz, referred to in \cite{} and claimed to
have already been realized in \cite{}.  In any case, it would seem that a generalization
of CGA or any CGA-like model to one that hosts the set of all algebraic sets in $\R^n$ would bring
us closer to such a goal.  Work to this end has already been done in \cite{Parkin13a,Parkin13b} which
has brought us, the reader and writer, to the present paper.

\section{Blades As Algebraic Sets}

We begin with an examination of $\Poly^n$ as a linear space, observing that
it is of countably infinite dimension.  A set of basis vectors for this space may be taken
as the set of all unit-monomials in anywhere from 1 to $n$ of the $n$ variable components
of an arbitrary point in $\R^n$.\footnote{The
number $c_k$ of such monomials homogeneous of a degree $k$ is given by
\begin{equation}
c_k=\sum_{i=1}^n\binom{n}{i}p(k,i),
\end{equation}
where $p(k,i)$ is the number of partitions of size $i$ of the integer $k$.  The
combinatorics of the matter are not important, but they are pointed out here as an added
measure of the reality of the set of monomials we are talking about.}
Considering now any mapping from the set $\Z^+$ of
positive integers to this said set $\{g_i\}_{i=1}^\infty$ of unit-monomials,
and letting $\V^\infty$ denote a simple Euclidean vector space of countably infinite
dimension, if we define the function $p:\R^n\to\V^\infty$ as
\begin{equation}\label{equ_form_vector}
p(x) = \sum_{i=1}^\infty g_i(x)e_i,
\end{equation}
where $\{e_i\}_{i=1}^\infty$ is any orthonormal basis for $\V^\infty$, then for
any polynomial $f\in\Poly^n$, there must exist a unique vector $v\in\V^\infty$ such that
\begin{equation}\label{equ_form_coef_split}
f(x) = p(x)\cdot v.
\end{equation}
It now follows that for any subset $P$ of $\Poly^n$, there exists a blade $B\in\G(\V^\infty)$,
such that
\begin{equation}
Z(P) = \{x\in\R^n|p(x)\cdot B=0\}.
\end{equation}
Clearly, we need only consider such finite sets $P=\{f_i\}_{i=1}^k$ that are linearly independent
sets of $k$ polynomials.  Then, for each polynomial $f_i$, there is an associated vector $v_i$ by
equation \eqref{equ_form_coef_split}, and the set $\{v_i\}_{i=1}^k$ must clearly be a linearly independent
set of $k$ vectors.
We may then take $B$ to be the $k$-blade
\begin{equation}
B=\bigwedge_{i=1}^k v_i,
\end{equation}
seeing that the equation $p(x)\cdot B=0$ becomes
\begin{equation}
0 = -\sum_{i=1}^k (-1)^i(p(x)\cdot v_i)B_i,
\end{equation}
where $B_i$ denotes the product $B$ with $v_i$ removed.  Realize
that $\{B_i\}_{i=1}^k$ is a linearly independent set of $(k-1)$-blades,
and therefore $p(x)\cdot B=0$ if and only if $p(x)\cdot v_i=0$ for all $i\in\Z_k$,
where we'll let $\Z_k$ denote $[1,k]\cap\Z^+$ for convenience.

Already we have fulfilled the promise of the introductory section of this paper,
but we will continue now with one further development that is motivated by a desire
to pair our current ability to intersect geometries with an ability to take their union.

We start by letting $\{\V_i^\infty\}_{i=1}^\infty$ be a countably infinite set of
vector spaces isomorphic to $\V^\infty$.  With the exception of the zero vector,
we consider these vector spaces as pair-wise disjoint, so that for any pair of non-zero
vectors $v_i\in\V_i^\infty$ and $v_j\in\V_j^\infty$ with $i\neq j$, we have $v_i\cdot v_j=0$.
Let $\V$ simply denote the vector space spanned by the set of vectors $\cup_{i\in\Z^+}\{e_{ij}\}_{j=1}^\infty$,
where for each $i\in\Z^+$, the set $\{e_{ij}\}_{j=1}^\infty$ is an orthonormal basis for
the vector space $\V_i^\infty$.
It is clear that the dimension of $\V$,
like that of $\V^\infty$ above, is countably infinite as the union of countably many countable
sets is countable.\footnote{This may come as some comfort, since if you already thought that
the dimension of $\V^\infty$ was ridiculously huge, the dimension of the vector space $\V$ is no bigger.
In practice, of course, a computer
program, for example, would work in a finite-dimensional vector space, thereby restricting
the number of possible geometries to those that are zero sets of polynomials of a specific form.
This is exactly what's going on in CGA, with the additional modification of altering the signature
of the geometric algebra to accommodate the conformal transformations.  Non-Euclidean signatures
are not considered in this paper.}

Given a blade $B\in\G(\V)$, we now define a function $q$ as a mapping from
the set of all blades in $\G(\V)$ to $\Z^+$, so that $i\in q(B)$ if and only if
there exists a vector $v\in\V_i^\infty$ such that $v\wedge B=0$.  With this in place,
we define the set $G$ of $B$, denoted $G(B)$, as the set
\begin{equation}
G(B) = \left\{x\in\R^n\left|\bigwedge_{i\in q(B)} p_i(x)\cdot B=0\right.\right\},
\end{equation}
where for each $i\in\Z^+$, we define $p_i$ similar to equation \eqref{equ_form_vector} as
\begin{equation}
p_i(x) = \sum_{j=1}^\infty g_j(x)e_{ij}.
\end{equation}

Given any two blades $A,B\in\G(\V)$, we can now show that
there exists a rotor $R\in\G(\V)$, such that
\begin{equation}
G(A)\cup G(B)=G(A\wedge B'),
\end{equation}
where $B'=RBR^{-1}$.  Letting $R$ be the rotor given by
\begin{equation}\label{equ_rotor_b_to_b_prime}
R = \prod_{i\in q(A)}\prod_{j=1}^\infty(1-e_{ij}e_{(i+k)j}),
\end{equation}
where $k=\max q(A)$, observe that $q(A)\cap q(B')$ is empty.
If it was already the case that $q(A)\cap q(B)$ is empty, then we may have taken
$R$ as simply the identity rotor.  In any case, a rotor $R\in\G(\V)$ always exists
such that for any pair of blades $A$ and $B$, the sets $q(A)$ and $q(B')$ are disjoint.
Furthermore, it is not hard to show that for the rotor $R$ given
in equation \eqref{equ_rotor_b_to_b_prime}, we have $G(B)=G(B')$.  Then, without loss of generality,
we may rewrite $C=A\wedge B'$ as $C=\bigwedge_{i=1}^k C_k$, where for
each $i\in\Z_k$, we have $C_i\in\G(V_i^\infty)$.
(Notice that $q(C)=\Z_k$.)
It is now clear that
\begin{align}
\bigwedge_{i\in\Z_k} p_i(x)\cdot C &= \pm\bigwedge_{i\in\Z_{k-1}} p_i(x)\cdot
(p_k(x)\cdot C_k)\wedge\bigwedge_{i\in\Z_{k-1}} C_i \\
 &= \pm\bigwedge_{i\in\Z_k}p_i(x)\cdot C_i.\label{equ_union_product}
\end{align}
(Notice that the outer product in \eqref{equ_union_product} becomes
a scalar product in the case that every $C_i$ is a vector.)
It is now clear that $G(C)$ is indeed the union of $G(A)$ and $G(B)$.

Having established the union operation, it is not clear at this point how
the intersection operation is preserved.  We begin with the observation
that for all blades $A,B\in\V$ such that $q(A)=q(B)$ with $|q(A)|=|q(B)|=1$, we have
$G(A)\cap G(B)=G(A\wedge B)$.  On the other hand, if $q(A)\neq q(B)$ with
the persisting requirement that $|q(A)|=|q(B)|=1$,
then there clearly exists a rotor $R$ such that $G(A)\cap G(B)=G(A\wedge B')$,
where $B'=RBR^{-1}$.  Performing an intersection in the case that one of $|q(A)|$
and $|q(B)|$ is not one requires a little more effort.  What we'll show is that,
in general, for any two blades $A,B\in\G(\V)$, there exist a blade $C\in\G(\V)$
such that
\begin{equation}
G(A)\cap G(B)=G(C).
\end{equation}
We begin by making the observation that
\begin{equation}
\bigcup_{i\in q(A)}G(A_i)\cap\bigcup_{i\in q(B)}G(B_i) =
\bigcup_{\substack{i\in q(A)\\j\in q(B)}} G(A_i)\cap G(B_j),
\end{equation}
where for each $i\in q(A)$, we have $A=\bigwedge_{i\in q(A)}A_i$ with
each $A_i\in\G(\V_i^\infty)$.  The same can be
said for each $B_i$ in terms of $B$.  We now carefully select
a set of rotors $\{R_i\}$ and $\{S_j\}$ such that
\begin{equation}
C=\bigwedge_{\substack{i\in q(A)\\j\in q(B)}}A'_i\wedge B'_j,
\end{equation}
where for each $i\in q(A)$, we have $A_i'=R_iA_iR_i^{-1}$ and $G(A_i)=G(A_i')$,
where for each $j\in q(B)$, we have $B_j'=S_jB_jS_j^{-1}$ and $G(B_j)=G(B_j')$,
where for each $(i,j)\in q(A)\times q(B)$, $q(A_i')=q(B_j')$, and lastly, where for each $(i,j)\neq(k,l)$,
we have $q(A_i'\wedge B_j')\cap q(A_k'\wedge B_l')$ empty.  Admittedly, this is
quite terrible.

Notice that, assuming $A\wedge B\neq 0$, the grade of $A\wedge B$ is
not necessarily the grade of $A'\wedge B'$.  This means that we cannot
always apply a rotor to one or both of $A$ and $B$ before taking them
in an outer product to perform an intersection operation.

% zarisky topology...arbitrary intersections of algebraic sets are algebraic, while finite unions are algebraic.
% union of countably many countable sets is countable

% for every blade B with q(B)>1, does there exist a blade A with q(A)=1
% where G(B)=G(A)?

\section{Comments And Criticisms}

A valid complaint, which the reader may have had by this point,
is that in most circumstances, a rotor needs to be applied to one or
both of the two blades $A$ and $B$ before they may be taken in the
outer product in the performance of a union or intersection operation.
Understandably, something like this has to happen if, borrowing a computer
science term, we are able to mathematically overload the outer product.
Owing to the definition of the $q$ function, what's further difficult about this is
that, prior to a union or intersection
operation, one would, in practice, need to
perform some sort of analysis on $A$ and $B$.

% program keeps blades in memory in factored form all the time?

% not ideal

% Possible citations:
% http://mathdl.maa.org/images/upload_library/22/Ford/DesmondFearnleySander.pdf
% Invariant algebras and geometric reasoning, Hongbo Li
% http://www.xtec.cat/~rgonzal1/proceedings.pdf

\bibliographystyle{amsplain}
\bibliography{Parkin_AModelOfAlgebraicSetsUsingGA}

\end{document}