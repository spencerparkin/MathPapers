\documentclass{article}
\usepackage{amsmath}
\usepackage{amssymb}
\title{Chapter 3 Homework}
\author{Spencer}
\addtolength{\oddsidemargin}{-.575in}
\addtolength{\evensidemargin}{-.575in}
\addtolength{\textwidth}{1.0in}
\addtolength{\topmargin}{-.575in}
\addtolength{\textheight}{1.25in}
\begin{document}
\maketitle

\section*{Problem 1}

$Z_{12} = \{0,1,2,\dots,10,11\}$.
And we see that $|0| = 1$, $|1| = 12$, $|2| = 6$,
$|3| = 4$, $|4| = 3$, $|5| = 12$, $|6| = 2$,
$|7| = 12$, $|8| = 6$, $|9| = 3$, $|10| = 6$,
and $|11| = 12$.

The orders of all elements divide the order of the group.  Elements that are coprime with
the order of the group have the same order as the group.  This is the result of a theorem
that states that if $\gcd(a,n)=1$, then the set $\{a,2a,3a,\dots,na\}$ is a complete system
of residues modulo $n$.

$U(10)=\{1, 3, 7, 9\}$.  And we see that $|1|=1$, $|3|=4$, $|7|=4$, and $|9|=2$.
Here, $|U(10)|=4$ and the order of an element of $U(10)$ is going to be 4 when
it's a primitive root of 10.  The order of $U(n)$ in general is the totative of $n$.

$U(12)=\{1, 5, 7, 11\}$.  And we see that $|1|=1$, $|5|=2$, $|7|=2$, and $|11|=2$.
We have $|U(12)|=4$.

$D_4=\{R_0, R_{90}, R_{180}, R_{270}, H, V, D, D'\}$.  And we see that
$|R_0|=1$, $|R_{90}|=4$, $|R_{180}|=2$, $|R_{270}|=4$, $|H|=2$, $|V|=2$,
$|D|=2$, and $|D'|=2$.  Here, $|D_4|=8$.

\section*{Problem 2}

Let $Q$ be the group of rational numbers under addition and let $Q^{*}$ be the
group of nonzero rational numbers under multiplication.  In $Q$, list the
elements in $\left\langle\frac{1}{2}\right\rangle$.  In $Q^{*}$, list the elements
in $\left\langle\frac{1}{2}\right\rangle$.

In $Q$ we have...
\begin{equation*}
\left\langle\frac{1}{2}\right\rangle = \left\{\dots, -2, -\frac{3}{2}, -\frac{1}{2}, 0, \frac{1}{2}, 1, \frac{3}{2}, 2, \dots\right\}
\end{equation*}

In $Q^{*}$ we have...
\begin{equation*}
\left\langle\frac{1}{2}\right\rangle = \left\{ \dots, -\frac{1}{8}, -\frac{1}{4}, -\frac{1}{2}, 1, \frac{1}{2}, \frac{1}{4}, \frac{1}{8}, \dots \right\}
\end{equation*}

\section*{Problem 3}

Let $Q$ and $Q^{*}$ be as in Problem 2.  Find the order of each element in $Q$ and in $Q^{*}$.

In $(Q,+)$, $|0|=1$.  Let $q\in (Q,+)$ and $q\neq 0$.  Since $nq\neq 0$ for all $n\in\mathbb{N}$, we
see that $|q|=\infty$.

In $Q^{*}$ under multiplication, $|1|=1$, as is the case with every group's identity.
Let $q\in Q^{*}$ where $q\neq 1$.  Since $q^n\neq 1$ for all $n\in\mathbb{N}$, we see that
$|q|=\infty$.

\section*{Problem 4}

Prove that in any group, an element and its inverse have the same order.

Let $a$ be a member of any group.  By definition we have $a^{|a|}=e$.
Using this we have...
\begin{equation*}
(a^{-1})^{|a|} = (a^{|a|})^{-1} = e^{-1} = e
\end{equation*}
This shows that $|a^{-1}|\leq|a|$.
By definition, $(a^{-1})^{|a^{-1}|}=e$.  Using this we have...
\begin{equation*}
e = (a^{-1})^{|a^{-1}|} = (a^{|a^{-1}|})^{-1}\implies a^{|a^{-1}|} = e
\end{equation*}
This shows that that $|a|\leq|a^{-1}|$.
It now follows that $|a|=|a^{-1}|$.

\section*{Problem 6}

Let $x$ belong to a group.  If $x^2\neq e$, and $x^6=e$, prove that $x^4\neq e$ and
$x^5\neq e$.  What can we say about the order of $x$?

Suppose $x^4=e$.  Then $x^6=x^4\implies x^2=e$.  But $x^2\neq 2$.  By contradiction, $x^4\neq e$.
Suppose $x^5=e$.  Then $x^6=x^5\implies x=e$.  But if $x^2\neq e$, then $x$ is not its
own inverse.  But $e$ is its own inverse.  By contradiction, $x^5\neq e$.

If the order of $x$ is not 3, then it is 6, but I do not see how we can prove which one it is.
What I think we can prove is that if $|x|=n$, then $x^{nk+r}=e$ for all $k\in\mathbb{Z}$ and $r=0$,
and $x^{nk+r}\neq e$ for all $k\in\mathbb{Z}$ and $0<r<n$.  The case when $r=0$ is obvious.
When $0<r<n$ we have $x^{nk+r}=(x^n)^kx^r = e^kx^r = x^r\neq e$ since if $x^r=e$, then $r$ would
be the order of $x$, not $n$.

\section*{Problem 8}

Show that $U(14)=\langle 3\rangle = \langle 5\rangle$.  Is $U(14)=\langle 11\rangle$?

\begin{align*}
U(14) &= \{ 1, 3, 5, 9, 11, 13 \} \\
\langle 3\rangle &= \{ 3, 9, 13, 11, 5, 1 \} \\
\langle 5\rangle &= \{ 5, 11, 13, 9, 3, 1 \} \\
\langle 11\rangle &= \{ 11, 9, 1 \}
\end{align*}
After some review of my number theory text, I've learned that the cyclic subgroup of $U(n)$
generated by $x\in U(n)$ is equal to $U(n)$ when $x$ is a primitive root of $n$.  That is, $x$ must
not only be coprime with $n$, but we also need the order of $x$ to be the order of $U(n)$.
It is easy, however, to show that the cyclic subgroup of $U(n)$ generated by any $x\in U(n)$
is a subset of $U(n)$ by using induction.

\section*{Problem 12}

Suppose $H$ is a proper subgroup of $\mathbb{Z}$ under addition and $H$ contains
18, 30, and 40.  Determine $H$.

Right away we can see that...
\begin{equation*}
\{18i+30j+40k:i,j,k\in\mathbb{Z}\}\subseteq H
\end{equation*}
...and it can be shown that this subset is a group, but I do not see how we can
conclude that this subset is $H$.  Couldn't there exist some integer that is
not a linear combination of 18, 30, and 40 that is in $H$?  For example, who is
to say that $31\not\in H$?

\section*{Problem 13}

For each divisor $k>1$ of $n$, let $U_k(n)=\{x\in U(n):x\mod{k}=1\}$.
Prove that $U_k(n)$ is a subgroup of $U(n)$.

By the definition of $U_k(n)$ it is clear that $U_k(n)\subseteq U(n)$.  It is also
clear that $U_k(n)$ is not empty because $1\mod{k}=1$.  Let $a,b\in U_k(n)$.
Then since $a\mod{k}=1$ and $b\mod{k}=1$, we see that $ab\mod{k}=1$ and
therefore $ab\in U_k(n)$.  Let $a\in U_k(n)$.  Then $a^{-1}=a$ since
$a^2\mod k=1$.  So $a^{-1}\in U_k(n)$.  By the two-step subgroup test, $U_k(n)$
must be a subgroup of $U(n)$.  It is unsettling that the fact that $k|n$ was never
used in my proof.  Did I miss something?

\section*{Problem 14}

If $H$ and $K$ are subgroups of $G$, show that $H\cap K$ is a subgroup of $G$.

Clearly $H\cap K\subseteq G$ and $e\in H\cap K$ where $e$ is the identity of $G$.
Let $a\in H\cap K$.  Then $a^{-1}\in H$ since $H$ is a subgroup of $G$ and $a^{-1}\in K$
since $K$ is a subgroup of $G$.  Since $a^{-1}\in H$ and $a^{-1}\in K$, we see that
$a^{-1}\in H\cap K$.  Let $a,b\in H\cap K$.  Then $ab\in H$ since $H$ is a subgroup of $G$
and $ab\in K$ since $K$ is a subgroup of $G$.  Since $ab\in H$ and $ab\in K$, we see that
$ab\in H\cap K$.  By the two-step subgroup test, $H\cap K$ is a subgroup of $G$.
This proof could be generalized to any set of subgroups of $G$.

\section*{Problem 15}

Let $G$ be a group.  Show that $Z(G)=\bigcap_{a\in G}C(a)$.

Define $Z_S(G)=\{x\in G:ax=xa,\forall a\in S\}$.
Now let's show that $Z_S(G)=Z_A(G)\cap Z_B(G)$ where $S=A\cup B$
and $A\cap B=\emptyset$.

If $a\in Z_A(G)\cap Z_B(G)$, then $a$ is an element of $G$ that commutes with all
elements in $A$ and $B$.
But this implies that $a$ commutes with all elements in $S$.  Therefore $a\in Z_S(G)$.
If $a\in Z_S(G)$, then $a$ is an element in $G$ that commutes with all elements in $S$.
But this implies that $a$ commutes with all elements in $A$ and $B$.  Therefore
$a\in Z_A(G)\cap Z_B(G)$.  We have now shown that $Z_S(G)=Z_A(G)\cap Z_B(G)$.

Noting that $Z(G)=Z_G(G)$, we can recursively apply our new formula to get...
\begin{equation*}
Z(G) = \bigcap_{s\in S} Z_{s}(G)
\end{equation*}
...where $S$ is the set of all singleton sets taken from $G$.  We can now
note that $Z_{s}(G)=C(a)$ for the only element $a\in s$.
The result now follows.

\section*{Problem 16}

Let $G$ be a group, and let $a\in G$.  Prove that $C(a)=C(a^{-1})$.

\begin{equation*}
x\in C(a)\Leftrightarrow ax=xa\Leftrightarrow axa^{-1}=x\Leftrightarrow xa^{-1}=a^{-1}x
\Leftrightarrow x\in C(a^{-1})
\end{equation*}

\section*{Problem 20}

If $H$ is a subgroup of $G$, then by the centralizer $C(H)$ of $H$ we mean the
set $\{x\in G:xh=hx,\forall h\in H\}$.  Prove that $C(H)$ is a subgroup of $G$.

Notice that $C(H)\subseteq G$ and that $e\in C(H)$ where $e$ is the identity of $G$.
Let $a,b\in C(H)$.  Then $ah=ha$ and $bh=hb$ for all $h\in H$.  Therefore,
$abh=ahb=hab$ for all $h\in H$.  So $ab\in C(H)$.  Let $a\in C(H)$.  Then
$ah=ha$ for all $h\in H$.  Multiplying on the left and right by $a^{-1}$
shows that $ha^{-1}=a^{-1}h$ for all $h\in H$ so that $a^{-1}\in C(H)$ also.
By the two-step subgroup test, $C(H)$ is a subgroup of $G$.

What is troubling to me about this proof is that it does not use the fact that
$H$ is a subgroup of $G$.  $H$ might as well be any subset of $G$.

\section*{Problem 21}

Must the centralizer of an element of a group be Abelian?

Let $G$ be a group and $a\in G$.  Consider the subgroup $C(a)$.
Let $x,y\in C(a)$ with $x\neq y$.  Then $x$ commutes with $a$ and
$a$ commutes with $y$.  Is this transitive?  That is, does this imply
that $x$ commutes with $y$?  The answer is no.  For example, let
$a=e$, then let $x$ and $y$ be any two non-commuting elements of $G$
where $G$ is a non-Abelian group.

\section*{Problem 22}

Must the center of a group be Abelian?

Let $G$ be a group and consider $Z(G)$.  Let $a,b\in Z(G)$.
Then $a$ commutes with all other elements of $G$ and so does $b$.
But $a$ and $b$ are elements of $G$, so they commute with one another.
Therefore, $Z(G)$ must be Abelian.

\section*{Problem 23}

Let $G$ be an Abelian group with identity $e$ and let $n$ be come
fixed integer.  Prove that the set of all elements of $G$ that
satisfy the equation $x^n=e$ is a subgroup of $G$.

Let $G_n$ denote the set $\{x\in G:x^n=e\}$.  By its definition
we see that $G_n\subseteq G$ and that $e\in G_n$.
Let $a,b\in G_n$.  Then $a^n=e$ and $b^n=e$.  Since $a,b\in G$
we may use the Abelian property of $G$ to show that
$e = a^nb^n = (ab)^n$.  Then since $ab\in G$ and $(ab)^n=e$,
we see that $ab\in G_n$.  Now let $a\in G_n$ so that $a^n=e$.
This rearranges to $e=(a^{-1})^n$.  Since $(a^{-1})^n=e$ and
$a^{-1}\in G$, we see that $a^{-1}\in G_n$.  So by the two-step
subgroup test, $G_n$ must be a subgroup of $G$.

\section*{Problem 26}

Suppose $n$ is an even positive integer and $H$ is a subgroup of $\mathbb{Z}_n$.
Prove that either every member of $H$ is even or exactly half of the members of $H$
are even.

If every member of $H$ is even, then we don't run into any problems with
closure.  That is, the sum of two even numbers will always be even.  So it
is possible for all members of $H$ to be even.

Suppose not every member of $H$ is even.  Let $\{a_1,a_2,\dots,a_m\}$
be the set of all odd numbers in $H$.  Notice that $a_i\neq a_j$ for
all $i\neq j$.  Choose any integer $j\in[1,m]$ and let $b_i=a_j+a_i\mod{n}$
for every integer $i\in[1,m]$.  Then $\{b_i\}_{i=1}^m$ is a set of
$m$ distinct even numbers modulo $n$.  So we know that $H$ contains at least $m$
even numbers.  Are there any more than this?  Suppose $b$ is an even number
in $H$ that we have failed to account for.  Notice that we can generate the
set of all odd numbers in $H$ as $\{b_i+a_j\mod{n}\}_{i=1}^m=\{a_1,a_2,\dots,a_m\}$
since $H$ is a group and must therefore be closed.  But
$b+a_j\not\equiv b_i+a_j\pmod{n}$ for all integers $i\in[1,m]$.  So
$b+a_j\mod{n}\not\in\{a_1,a_2,\dots,a_m\}$, contradicting the fact that
we have accounted for all odd numbers in $H$.  So there must be $m$ odd
numbers and $m$ even numbers for a total of $n=2m$ numbers in $H$.

\section*{Problem 27}

Suppose a group contains elements $a$ and $b$ such that $|a|=4$, $|b|=2$,
and $a^3b=ba$.  Find $|ab|$.

$a^4=e\implies a^3=a^{-1}$.  Then $a^3b=ba\implies a^{-1}b=ba$.
Then $(ab)^2=abab=aa^{-1}b^2=e$.  So the order of $ab$ is no greater than 2.
To show that the order of $ab$ is not less than 2, we must show that
$ab\neq e$.  Note that $b^2=e\implies b=b^{-1}$.  So if $ab=e$, then $a=b$.
But then we would have $|a|=2$ instead of $|a|=4$.  So the the order of $ab$
is not 1.  It is therefore 2.

\section*{Problem 28}

Consider the elements $A=\left[\begin{array}{cc}0&-1\\1&0\end{array}\right]$ and
$B=\left[\begin{array}{cc}0&1\\-1&-1\end{array}\right]$ from $SL(2,\mathbb{R})$.
Find $|A|$, $|B|$, and $|AB|$.

The matrix $A$ represents a CCW rotation of $\pi/2$ radians in the plane, so
$A^4=I$ is a rotation of 0 radians and the order of $A$ is 4.
Some work shows that $|B|=3$.
Notice that $B$ can be factored into a CW rotation of $\pi/2$ radians following
a shear in $x$.
\begin{equation*}
B = 
\left[\begin{array}{cc}0&1\\-1&0\end{array}\right]
\left[\begin{array}{cc}1&1\\0&1\end{array}\right]
\end{equation*}
Therefore, when we compute $AB$, the rotations cancel and we're left
with the shear...
\begin{equation*}
AB = \left[\begin{array}{cc}1&1\\0&1\end{array}\right]
\end{equation*}
It can now be shown that...
\begin{equation*}
(AB)^n = \left[\begin{array}{cc}1&n\\0&1\end{array}\right]
\end{equation*}
...for all $n>0$ so that $|AB|=\infty$.  It is not surprising that the order of a shear
is infinity since the continual concatination of this transform with itself
will always yield a bigger shear.  It might be interesting to note that a rotation
can be factored into three shears.

\section*{Problem 35}

Prove that a group of even order must have an element of order 2.

The only element having order 1 is the identity.  There are then
an odd number of elements left to consider.  Suppose there is no
element having order 2.  This means that no element is its own inverse.
Therfore, since an element and its inverse are inverses of one another,
and since every element must have an inverse that is not itself, we
must be able to count the rest of the elements in pairs of inverses.
But this can't be done if there are an odd number of elements left,
so one of them must be its own inverse, and therefore have order 2.

\section*{Problem 41}

Find a cyclic subgroup of order 4 in $U(40)$.
\begin{equation*}
\langle 3\rangle = \{ 3, 9, 27, 1 \}
\end{equation*}

\section*{Problem 45}

Let $H$ be a subgroup of $\mathbb{R}$ under addition.
Let $K=\{2^a:a\in H\}$.
Prove that $K$ is a subgroup of $\mathbb{R}^{*}$ under multiplication.

Since $2^a\neq 0$ for all $a\in H$, we see that $K\subset\mathbb{R}^{*}$.
We also see that $K\neq\emptyset$ since $1\in K$ since $0\in H$.
Let $x,y\in K$.  Then $x=2^a$ and $y=2^b$ where $a,b\in H$.  Since $H$
is a group we see that $y^{-1}=2^{-b}\in K$ and then that
$xy^{-1}=2^a2^{-b}=2^{a-b}\in K$ since $a-b\in H$.  The set $K$ is
therefore a subgroup of $\mathbb{R}^{*}$ under multiplication
by the one-step subgroup test.

\section*{Problem 48}

Let $H=\{a+bi:a,b\in\mathbb{R},ab\geq 0\}$.  Prove or disprove that $H$
is a subgroup of $\mathbb{C}$ under addition.

$H$ is not a group under addition since $1\in H$, $-i\in H$, but $1-i\not\in H$.

\section*{Problem 52}

Let $G$ be a finite group with more than one element.
Show that $G$ has an element of prime order.

By Problem 35, if the group has an even number of elements, then there
is an element having order 2, which is a prime.  So what remains to be shown
is that every group having an odd number of elements has an element with
prime order.

So suppose we have a group with odd order.  Err...

\end{document}