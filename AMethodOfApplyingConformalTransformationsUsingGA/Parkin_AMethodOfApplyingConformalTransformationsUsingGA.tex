\documentclass{ecgd-l}

%     If you need symbols beyond the basic set, uncomment this command.
%\usepackage{amssymb}

%     If your article includes graphics, uncomment this command.
%\usepackage{graphicx}

%     If the article includes commutative diagrams, ...
%\usepackage[cmtip,all]{xy}


%     Update the information and uncomment if AMS is not the copyright
%     holder.
%\copyrightinfo{2009}{American Mathematical Society}

\newtheorem{theorem}{Theorem}[section]
\newtheorem{lemma}[theorem]{Lemma}

\theoremstyle{definition}
\newtheorem{definition}[theorem]{Definition}
\newtheorem{example}[theorem]{Example}
\newtheorem{xca}[theorem]{Exercise}

\theoremstyle{remark}
\newtheorem{remark}[theorem]{Remark}

\numberwithin{equation}{section}

\newcommand{\G}{\mathbb{G}}
\newcommand{\V}{\mathbb{V}}
\newcommand{\R}{\mathbb{R}}
\newcommand{\nvai}{\infty}
\newcommand{\nvao}{o}

\begin{document}

% \title[short text for running head]{full title}
\title{A Method Of Applying Conformal Transformations Using Geometric Algebra}

%    Only \author and \address are required; other information is
%    optional.  Remove any unused author tags.

%    author one information
% \author[short version for running head]{name for top of paper}
\author{Spencer T. Parkin}
\address{Weber State University\\3848 Harrison Blvd\\Ogden, UT  84408}
\curraddr{}
\email{}
\thanks{}

%    \subjclass is required.
\subjclass[2010]{Primary }

\date{}

\dedicatory{}

%    Abstract is required.
\begin{abstract}
It is shown that any conformal transformation may be applied to a polynomial
of degree $m$
by first converting the polynomial to an $m$-vector taken from a geometric algebra,
applying a versor to that $m$-vector, and then converting the resulting $m$-vector
back into a polynomial.
\end{abstract}

\maketitle

\section{Preliminaries}

For this paper, we assume the reader is already familiar with geometric algebra and the
conformal model of geometric algebra.
(See \cite{Dorst07} for introductory material on geometric algebra.  See \cite{Dorst07,Hestenes01} for material
on conformal geometric algebra.)  Despite what conventions
may be used in other papers on geometric algebra, here we will let the outer product
take precedence over the inner product, and the geometric product take precedence
over the inner and outer products.

As there are different ways of defining the inner product for different purposes, we must
take a moment here to define the inner product used in this paper.  It is as follows.
Among vectors, the inner product is a bilinear form defining the signature of our geometric algebra
which will be given in the next section.  For any vector $v$ and $k$-blade $A$, we define
\begin{equation}\label{equ_def_inner_prod}
v\cdot A = -\sum_{i=1}^k(-1)^i(v\cdot a_i)A_i,
\end{equation}
where here, the $k$-blade $A$ may be factored as $A=\bigwedge_{i=1}^k a_i$,
and for each integer $i\in[1,k]$, we let
\begin{equation*}
A_i=\bigwedge_{\substack{j=1\\j\neq i}}^k a_j.
\end{equation*}
We let $v$ commute with $A$ as
\begin{equation*}
v\cdot A=-(-1)^kA\cdot v.
\end{equation*}
It is sometimes convenient to rewrite equation \eqref{equ_def_inner_prod} as
\begin{equation*}
v\cdot A = (v\cdot a_1)A_1 - (v\cdot A_1)\wedge a_1,
\end{equation*}
which gives a recursive version of the definition.  For the $k$-blade $A$ and
$l$-blade $B$, we define
\begin{equation}\label{equ_A_dot_B}
A\cdot B = \left\{\begin{array}{ll}
A_k\cdot (a_k\cdot B) & \mbox{if $k\leq l$,} \\
(A\cdot b_1)\cdot B_1 & \mbox{if $k\geq l$,}
\end{array}\right.
\end{equation}
where here, the $l$-blade $B$ may be factored as $B=\bigwedge_{i=1}^l b_i$, and for each
integer $i\in[1,k]$, we let
\begin{equation*}
B_i=\bigwedge_{\substack{j=1\\j\neq i}}^l b_j.
\end{equation*}
In the case that $k=l$, either part of the piece-wise function of equation \eqref{equ_A_dot_B} may be used to
evaluate $A\cdot B$.

\section{The Geometric Algebra}

We begin by letting $\{\G_i\}_{i=1}^m$ be a sequence of $m$ Minkowski geometric algebras upon
which the conformal model of $n$-dimensional euclidean space may be imposed.
Each $\G_i$ is the very geometric algebra found in \cite{Hestenes01}.  If $\V_i$ is a vector space
generating $\G_i$, then $\{e_{i,-},e_{i,+}\}\cup\{e_{i,j}\}_{j=1}^n$ is a set of basis
vectors generating $\V_i$, where $\{e_{i,j}\}_{j=1}^n$ is an orthonormal basis
for an $n$-dimensional euclidean space, and each of $e_{i,-}$ and $e_{i,+}$, taken from \cite{LiRockwood01}, is given by
\begin{align*}
e_{i,-} &= \frac{1}{2}\nvai_i + \nvao_i, \\
e_{i,+} &= \frac{1}{2}\nvai_i - \nvao_i,
\end{align*}
where here, $\nvao_i$ and $\nvai_i$ are, for each $\G_i$, the familiar null-vectors
representative of the points at origin and infinity, respectively.
With the exception of zero, we consider the sequence of geometric algebras $\{\G_i\}_{i=1}^m$
to be pair-wise disjoint.

Taking our cue from the ``mother algebra'' in \cite{DoranHestenes93},
we are now interested in forming the smallest geometric algebra $\G$
containing each $\G_i$ as a geometric sub-algebra.  This geometric
algebra $\G$ is therefore generated by the vector space $\V$, given by
\begin{equation*}
\V=\bigoplus_{i=1}^m \V_i.
\end{equation*}

We now introduce a function $\Psi_{i,j}:\G\to\G$ defined as
\begin{equation*}
\Psi_{i,j}(E) = \left\{\begin{array}{ll}
S_{i,j,1}E(S_{i,j,1})^{-1} & \mbox{if $i\neq j$,} \\
E & \mbox{if $i=j$}\end{array}\right.,
\end{equation*}
where $S_{i,j,k}$ is the constant given by
\begin{equation}\label{equ_S}
S_{i,j,k} = \left(1-(-1)^ke_{i,-}e_{j,-}\right)\left(1+(-1)^ke_{i,+}e_{j,+}\right)\prod_{r=1}^n\left(1-(-1)^ke_{i,r}e_{j,r}\right).
\end{equation}
Take notice that
\begin{equation*}
(S_{i,j,1})^{-1} = 2^{-(n+2)}S_{i,j,0}.
\end{equation*}
Our definition of $\Psi_{i,j}$ is motivated by the fact that for any vector $v_i\in\V_i$ and its
corresponding vector $v_j\in\V_j$, we have
\begin{equation*}
v_j = \Psi_{i,j}(v_i).
\end{equation*}
Being in correspondence, this means that for all integers $k\in[1,n]$,
we have
\begin{equation*}
v_i\cdot e_{i,k}=v_j\cdot e_{j,k},
\end{equation*}
as well as
\begin{align*}
v_i\cdot\nvao_i &= v_j\cdot\nvao_j, \\
v_i\cdot\nvai_i &= v_j\cdot\nvai_j.
\end{align*}
Notice that $\Psi_{i,j}(v_j)=-v_i$.
For any vector $v\not\in\V_i$ and $v\not\in\V_j$, the function $\Psi_{i,j}$ leaves $v$ invariant.

Though $S_{i,j,1}$ in equation \eqref{equ_S} is not a versor of $\G$, we will now show that $\Psi_{i,j}$ is an outermorphism.
(See \cite{Hestenes91} for a definition of outermorphism.)  It suffices to show that
for any two vectors $a,b\in\V$, we have
\begin{equation*}
a\cdot b = \Psi_{i,j}(a)\cdot \Psi_{i,j}(b).
\end{equation*}
To see this, begin by rewriting $a$ and $b$ as
\begin{align*}
a &= \sum_{k=1}^m a_k, \\
b &= \sum_{k=1}^m b_k,
\end{align*}
where for each pair $(a_k,b_k)$, we have $a_k,b_k\in\V_k$.  We then have
\begin{equation*}
a\cdot b=\sum_{k=1}^m a_k\cdot b_k,
\end{equation*}
where we can make the observation that for any $k\in[1,m]$, we have
\begin{equation*}
a_k\cdot b_k = \Psi_{i,j}(a_k)\cdot\Psi_{i,j}(b_k).
\end{equation*}
We now simply see that
\begin{equation*}
\Psi_{i,j}(a)\cdot\Psi_{i,j}(b)=\sum_{k=1}^m\Psi_{i,j}(a_k)\cdot\Psi_{i,j}(b_k)
\end{equation*}
to complete the proof.

\section{Geometric Representation}

Having now set forth our geometric algebra $\G$, we're ready to discuss
geometric representation.
Letting $\R^n$ be the $n$-dimensional euclidean vector sub-space
of $\V_1$, our geometric representation scheme is to let a geometry be the
set of all points $x\in\R^n$, such that
\begin{equation*}
\bigwedge_{j=1}^m p_j(x)\cdot A=0,
\end{equation*}
where $A$ is an $m$-vector (not necessarily an $m$-blade) of $\G$, and
where the function $p_j:\R^n\to\V_1$, reminding us of the conformal mapping found in \cite{Hestenes01}, is defined as
\begin{equation*}
p_j(x)=\Psi_{1,j}\left(\nvao_1+x+\frac{1}{2}x^2\nvai_1\right).
\end{equation*}
Here, it is the $m$-vector $A$ that serves as the representative of our geometry.
It is not hard to see that the point-set generated by $A$ through equation \eqref{}
is simply the zero set of a polynomial of degree at most $2m$.\footnote{An algebraic
set is the zero set of one or more polynomial functions.  It can be shown that such sets may be represented by
the blades of a geometric algebra, but the geometric representation method of this paper is restricted to those sets
that are the zero set of one and only one polynomial function.}

For any polynomial of degree $r$ with $m<r\leq 2m$, there does not necessarily exist an $m$-vector
$A$ representative of its zero set.  For all polynomials of degree $r$ with $0\leq r\leq m$, however,
there does exist such an $m$-vector $A$.  Taking advantage of the linearity of the inner product,
it suffices to show that this is the case for every monomial of such a degree.  Indeed, the $m$-vector
$A$, (in this case an $m$-blade), is given by
\begin{equation*}
A = \lambda\bigwedge_{j=1}^r a_j\wedge\bigwedge_{j=r+1}^m\nvai_j,
\end{equation*}
where $\lambda\in\R$ is a scalar, $\{a_j\}_{j=1}^r$ is a set of $r$ vectors with
each $a_j\in\V_j$,
and the first product is one in the case that $r=0$.
To see this, let $x_j=\Psi_{1,j}(x)$ and write
\begin{align*}
\bigwedge_{j=1}^m p_j(x)\cdot A &= \bigwedge_{j=1}^r x_j\wedge\bigwedge_{j=r+1}^m\nvao_j\cdot A \\
 &= \lambda(-1)^{m-r}\bigwedge_{i=1}^r x_j \cdot\bigwedge_{i=1}^r a_j \\
 &= \lambda(-1)^{m-r}\prod_{i=1}^r x_j\cdot a_j,
\end{align*}
which shows that for an appropriate choice of the vectors in $\{a_j\}_{j=1}^r$,
and that of $\lambda$, we can formulate $A$ as being representative of any monomial in $n$
independent variables $\{x\cdot e_{1,j}\}_{j=1}^n$.

It is not difficult to convert between polynomial functions and $m$-vectors $A$
of our geometric algebra $\G$.  That is, not difficult if we are using a computer
algebra system.  Never-the-less, once implemented, we have a nice subroutine
allowing us to convert between the two forms.

\section{Applying Conformal Transformations To Geometries}

We now come to the main result of this papers, which is to show
that the conformal transformations are easily applicable to any geometry
that may be represented as an $m$-vector by equation \eqref{}.\footnote{Again,
by this we mean easy for a computer algebra system.}
We simply observe that if $V\in\G_1$ is a versor representative
of a conformal transformation, such as those that can be found in \cite{Dorst07}, then the desired
geometry is given by the set of all points $x\in\R^n$, such that
\begin{equation*}
\bigwedge_{j=1}^m V^{-1}p_jV\cdot A=0.
\end{equation*}
It will not be hard to show that the element $A'$ representative
of this very set of points by equation \eqref{} is given by
\begin{equation*}
A' = WAW^{-1},
\end{equation*}
where the versor $W$ is given by
\begin{equation*}
W = \prod_{j=1}^m V_j,
\end{equation*}
where $V_j=\Psi_{1,j}(V)$.
Taking advantage of the linearity of the inner product once again,
we need only prove our main result in the case that $A$ is an $m$-blade.
Let $\{a_j\}_{j=1}^m$ be a set $m$ vectors, such that
\begin{equation*}
A = \bigwedge_{j=1}^m a_j.
\end{equation*}
We then have
\begin{align*}
\bigwedge_{j=1}^m V^{-1}p_jV\cdot\bigwedge_{j=1}^m a_j &= \prod_{j=1}^m V^{-1}p_jV\cdot a_j \\
 &= \prod_{j=1}^m p_j\cdot Va_jV^{-1} \\
&= \bigwedge_{j=1}^m p_j\cdot\bigwedge_{j=1}^m Va_jV^{-1} \\
&= \bigwedge_{j=1}^m p_j\cdot\bigwedge_{j=1}^m Wa_jW^{-1} \\
&= \bigwedge_{j=1}^m p_j\cdot WAW^{-1}.
\end{align*}
This completes the proof.

\section{A Trial Implementation}

Putting theory into practice, a computer algebra system with
visualization capabilities was implemented and used to generate Figure~\ref{},
Figure~\ref{} and Figure~\ref{}.

\bibliographystyle{amsplain}
\bibliography{Parkin_AMethodOfApplyingConformalTransformationsUsingGA}

\end{document}