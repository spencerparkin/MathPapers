\documentclass{birkjour}

%\usepackage{tikz}
%\usepackage{graphicx}
\usepackage{hyperref}

\newtheorem{thm}{Theorem}[section]
\newtheorem{cor}[thm]{Corollary}
\newtheorem{lem}[thm]{Lemma}
\newtheorem{prop}[thm]{Proposition}
\theoremstyle{definition}
\newtheorem{defn}[thm]{Definition}
\theoremstyle{remark}
\newtheorem{rem}[thm]{Remark}
\newtheorem*{ex}{Example}
\numberwithin{equation}{section}

\newcommand{\R}{\mathbb{R}}
\newcommand{\B}{\mathbb{B}}
\newcommand{\G}{\mathbb{G}}
\newcommand{\V}{\mathbb{V}}
\newcommand{\gd}{\dot{g}}
\newcommand{\gh}{\hat{g}}
\newcommand{\Gd}{\dot{G}}
\newcommand{\Gh}{\hat{G}}
\newcommand{\nvai}{\infty}
\newcommand{\nvao}{o}
\newcommand{\grade}{\mbox{grade}}

\begin{document}

\title{On The Problem Of Intersecting\\Quadric Surfaces Using\\Geometric Algebra}

\author{Spencer T. Parkin}
\address{102 W. 500 S., \\
Salt Lake City, UT  84101} \email{spencerparkin@outlook.com}

%\subjclass{Primary 14J70; Secondary 14J29}

%\dedicatory{To my dear wife Melinda.}

\begin{abstract}
Abstract goes here...
\end{abstract}

\keywords{Keywords go here...}

\maketitle

\section{Introduction}

In light of the paper \cite{}, some encouragement has been given to the present
author to develop a model of geometry, similar to the conformal model of geometric
algebra, but not limited in representation to any proper subset of the set of all
quadric surfaces.  But for such a model to achieve adaquite similarity to the conformal model,
it must preserve one of more of its most desirable features; preferably, all of them.
For example, in \cite{}, the set of all conformal transformations were preserved, but
intersections were not.

The goal of this paper, therefore, is to
preserve the intersection property.  In other words, we want to find a model of geometry based upon
geometric algebra giving us all quadric surfaces and the ability to intersect them
as effortlessly as can be done in the conformal model.  If nothing else, the attempt
to do so in this paper will shed light on the feasability of such an endeaver, and thereby
bring us closer to anwering the question of whether it can even be done.

\section{The Intersection Property}

Let us begin by taking a closer look at exactly what the intersection property is.
Upon initial inspection, one might suppose that this property is nothing more than
the ability to formulate the intersection of two given geometries in a way consistent
with the representation of any geometry of the model, but this is not
enough.  Such a formulation has no usefulness if it does not submit to an
analysis yielding the geometric characteristics of the intersection.

This having been said, the outer product's ability to intersect geometries represented
by blades in the conformal model is really not at all interesting.  What is interesting
is the realization that we can equate one characterization of an intersection
with another, and this is the key to finding intersections in the conformal model.
The reason for this is that while one such characterization is composed as the
intersection we wish to take, the other characterization lends itself to
analysis through decomposition.

For example, suppose we wish to take the planar intersection of a conical surface.
If we know that the resulting conic section is an ellipse, then we can choose to
interpret this intersection as that of a plane and an elliptical cylinder meeting
the plane at right angles.  This latter characterization will have an easily found decomposition
yielding all features of the ellipse.  Having found all such features, we can then say that we've
fully realized the given section, whereas before this we were only able to represent it.

The ability to find our model of geometry, (the one promised in the introductory section of
this paper), being quite difficult, the bringing of the example just given in the preceding paragraph
to fruition will become the impetus for all choices we make in finding the model.
Even if our model can do nothing more than this one example, we will consider or goal achieved.

So that no further delay be made, we will now let the remainder of this paper begin
exactly where the first section of \cite{} ended, assuming all results and definitions up to that point.
That said, we now introduction the function $p:\R^n\to\V$ as
\begin{equation}
p(x) = e_0 + x + (x\cdot e_2)(x\cdot e_3)e_4 + \dots + (x\cdot e_1)^2e_7,
\end{equation}
the vectors in $\{e_i\}_{i=0}^9$ forming an orthonormal basis for a 10-dimensional vector space.

Some results of this paper will depend upon this definition of $p$ while
others will not.

\end{document}
