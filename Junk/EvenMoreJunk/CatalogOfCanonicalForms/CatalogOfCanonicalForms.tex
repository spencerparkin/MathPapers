

\section{Catalog of Canonical Forms}

For reference, this section catalogs canonical dual representations of the geometries in the
conformal model of 3-dimensional space, as well as common transformations represented
by versors.  In each sub-section about dual geometric representations, the blade $B\in\G$ is
assumed to represent the geometry in question, while in sub-sections about
versor transformations, the versor $V\in\G$ represents the transformation in question.
In addition to the composition
of each geometry's dual representation, a sequence of steps
are also provided that show how one can decompose this representation
into the variables that characterize the geometry.  A similar set of composition
and decomposition steps are provided for the versor transformations.

\subsection{Points}

Points (round points) are characterized by a Euclidean point $x\in\V^n$ and
a non-zero scalar (weight) $\lambda\in\R$.
\begin{equation*}
B = \lambda\left(\nvao + x + \frac{1}{2}x^2\nvai\right)
\end{equation*}
We may decompose this as follows.
\begin{align*}
\lambda &= -\nvai\cdot B \\
v &= \nvao\wedge\nvai\cdot\frac{B}{\lambda}\wedge\nvao\wedge\nvai
\end{align*}

\subsection{Spheres}

Spheres are characterized by a Euclidean point (center) $x\in\V^n$, a
non-zero radius $r\in\R$ and a non-zero scalar (weight) $\lambda\in\R$.
\begin{equation*}
B = \lambda\left(\nvao + x + \frac{1}{2}(x^2\pm r^2)\nvai\right)
\end{equation*}
(Say something about imaginary spheres.)
We may decompose this as follows.
\begin{align*}
\lambda &= -\nvai\cdot B \\
x &= \nvao\wedge\nvai\cdot\frac{B}{\lambda}\wedge\nvao\wedge\nvai \\
r^2 &= x^2 + 2\nvao\cdot\frac{B}{\lambda}
\end{align*}
Alternatively, we can find $r^2$ as simply the square of $B^2/\lambda$.
Using what we know about spherical reflections, we can find $x$ by
reflecting $\nvai$ into the sphere.

\subsection{Planes}

Planes are characterized by a Euclidean point (center, if you will) $x\in\V^n$,
a unit-normal $v\in\V^n$ and a non-zero scalar (weight) $\lambda\in\R$.
\begin{equation*}
B = \lambda(v + (x\cdot v)\nvai)
\end{equation*}
If $T=1-\frac{1}{2}x\nvai$, we may also formulate $B$ as $Tv\tilde{T}$.
We may decompose $B$ as follows.
\begin{align*}
v &= \nvao\cdot\frac{B}{\lambda}\wedge\nvai \\
x &= -v\left(\nvao\cdot\frac{B}{\lambda}\right)
\end{align*}
Notice here that any original weight, normal and position used in
the composition of $B$ are not recoverable in the decomposition of $B$.
Here, $x$ will be the point on the plane closest to the origin.

\subsection{Circles}

Circles are characterized by a Euclidean point (center) $x\in\V^n$,
a unit-normal $v\in\V^n$, a non-zero radius $r\in\R$ and a non-zero scalar (weight) $\lambda\in\R$.
\begin{equation*}
B = \lambda(v+(x\cdot v)\nvai)\wedge\left(\nvao+x+\frac{1}{2}(x^2\pm r^2)\nvai\right)
\end{equation*}
(Say something about imaginary circles.)  We may decompose this as follows.
\begin{align*}
v &= \nvao\wedge\nvai\cdot\frac{B}{\lambda}\wedge\nvai \\
x &= v\left(\nvao\wedge\nvai\cdot\frac{B}{\lambda}\wedge\nvao\nvai\right) \\
r^2 &= x^2 - 2v\left((x\cdot v)x-\nvao\wedge\nvai\cdot\nvao\wedge \frac{B}{\lambda}\right)
\end{align*}

\subsection{Lines}

Lines are characterized by a Euclidean point (center, if you will) $x\in\V^n$,
a unit-normal $v\in\V^n$ and a non-zero scalar (weight) $\lambda\in\R$.
\begin{equation*}
B = \lambda\left(vi - \left(x\cdot vi\right)\wedge\nvai\right)
\end{equation*}
We may decompose this as follows.
\begin{align*}
v &= \left(\nvao\cdot\frac{B}{\lambda}\wedge\nvai\right)i \\
x &= -v\left(\nvao\cdot\frac{B}{\lambda}\right)i
\end{align*}

\subsection{Point-Pairs}

Point-pairs are characterized by a Euclidean point (center) $x\in\V^n$,
a unit-normal $v\in\V^n$, a non-zero radius $r\in\R$ and a non-zero scalar (weight) $\lambda\in\R$.
\begin{equation*}
B = \lambda(vi-(x\cdot vi)\wedge\nvai)\wedge\left(\nvao+x+\frac{1}{2}(x^2\pm r^2)\nvai\right)
\end{equation*}
(Say something about imaginary point-pairs.)  We may decompose this as follows.
\begin{align*}
v &= -\left(\nvao\wedge\nvai\cdot\frac{B}{\lambda}\wedge\nvai\right)i \\
x &= -v\left(\nvao\wedge\nvai\cdot\frac{B}{\lambda}\wedge\nvao\nvai\right)i \\
r^2 &= -x^2+2v\left((x\cdot v)v+\left(\nvao\wedge\nvai\cdot\nvao\wedge\frac{B}{\lambda}\right)i\right)
\end{align*}

\subsection{Flat Points}

Flat-points are characterized by a Euclidean point $x\in\V^n$ and a non-zero
scalar (weight) $\lambda\in\R$.
\begin{equation*}
B = \lambda(i+xi\wedge\nvai)
\end{equation*}
We may decompose this as follows.
\begin{align*}
\lambda &= -(B\wedge\nvai)i \\
x &= \left(\nvao\cdot\frac{B}{\lambda}\right)i
\end{align*}

\subsection{Tangent Points}

A tangent-point is characterized by a Euclidean point $x\in\V^n$, a unit-normal $v\in\V^n$
and a non-zero scalar (weight) $\lambda\in\R$.  Dual canonical forms of tangent points for
grades 2 and 3 are given by the dual canonical forms of circles and point-pairs, respectively,
with a radius $r$ of zero.  For example, given any $r>0$, simplyfing the following equation recovers
the dual form of a tangent point for grade 2.
\begin{equation*}
B = \lambda(v+(x\cdot v)\nvai)\wedge\left(\nvao+x-rv+\frac{1}{2}((x-rv)^2-r^2)\nvai\right),
\end{equation*}
The reader will notice that $r$ cancels itself out.  The decomposition steps for tangent
points are the same as those given for circles and point-pairs.  The recovered radius
will be zero in the case of tangent points.

\subsection{Free Blades}

Address free-blades here.

\subsection{Rotate-Translate Transformations}

Such a transformation is characterized by a Euclidean translation vector $t\in\V^n$, a unit-axis $a\in\V^n$,
an angle $\theta\in\R$ and a scalar (weight) $\lambda\in\R$.
\begin{equation*}
V = \lambda\left(1-\frac{1}{2}t\nvai\right)\left(\cos\frac{\theta}{2}-ai\sin\frac{\theta}{2}\right),
\end{equation*}
Notice that $V$ here is not a blade.  It is an even versor.  If the blade $B\in\G$ represents
a geometry, (directyl or dually), the transformation of $B$ by $V$ is given by $VBV^{-1}$,
in the case that we wish to the apply the rotation first, then the translation.
We may decompose this type of transformation as follows.
\begin{align*}
\lambda^2 &= V\tilde{V} \\
R &= -\nvao\cdot\frac{V}{\lambda}\wedge\nvai \\
T &= \frac{V}{\lambda}\tilde{R} \\
\theta &= 2\cos^{-1}\left\langle R\right\rangle_0 \\
a &= \frac{1}{\sin(\theta/2)}\left\langle R\right\rangle_2 i \\
t &= 2\nvao\cdot(1-T)
\end{align*}
(Say something about the polar decomposition of $V$.)
