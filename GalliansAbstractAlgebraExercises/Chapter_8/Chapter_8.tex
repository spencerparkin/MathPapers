\documentclass[12pt]{article}

\usepackage{amsmath}
\usepackage{amssymb}
\usepackage{amsthm}

\title{Chapter 8 Exercises\\Gallian's Book on Abstract Algebra}
\author{Spencer T. Parkin}

\newtheorem{theorem}{Theorem}[section]
\newtheorem{definition}{Definition}[section]
\newtheorem{corollary}{Corollary}[section]
\newtheorem{identity}{Identity}[section]
\newtheorem{lemma}{Lemma}[section]
\newtheorem{result}{Result}[section]

%\newcommand{\gcd}{\mbox{gcd}}
\newcommand{\lcm}{\mbox{lcm}}
\newcommand{\abs}{\mbox{abs}}
\newcommand{\Z}{\mathbb{Z}}
\newcommand{\R}{\mathbb{R}}
\newcommand{\G}{\mathbb{G}}
\newcommand{\stab}{\mbox{stab}}
\newcommand{\aut}{\mbox{Aut}}
\newcommand{\inn}{\mbox{Inn}}
\newcommand{\orb}{\mbox{orb}}

\begin{document}
\maketitle

\section*{Understanding Theorem 8.3}

We want to show that $U(st)\approx U(s)\oplus U(t)$.
Let $\phi(x)=(x\bmod s,x\bmod t)$.
For $x,y\in U(st)$, if $\phi(x)=\phi(y)$,
then $x\equiv y\pmod s$ and $x\equiv y\pmod t$.
Then, since $\gcd(s,t)=1$, it is clear that $x\equiv y\pmod{st}$.
(See Problem 15 of Chapter 0.)  So $\phi$ is one-to-one.
It is also onto since $\phi$ is onto-to-one and maps a finite
set to another of the same cardinality.  That $\phi$ is
operation preserving is a matter of showing that for
any $x,y\in U(st)$, we have
\begin{align*}
(xy\bmod st)\bmod m &= (x\bmod m)(y\bmod m)\bmod st \\
 &= (xy\bmod m)\bmod st,
\end{align*}
where $m$ is $s$ or $t$.  With enough thought, this is
intuitive enough to warrent justification by virtue of being clear.

\section*{Problem 1}

Prove that the external direct product of any finite number of groups is a group.

There is clearly an identity element.  Closure is clear.  Inverses are clear.
Associativity is clear.  I think that's it.

\section*{Problem 2}

Show that $Z_2\oplus Z_2\oplus Z_2$ has seven subgroups of order 2.

Considering $\langle (a,b,c)\rangle$ for any $(a,b,c)\in Z_2\oplus Z_2\oplus Z_2$
for which not all of $a$, $b$ and $c$ are zero, it is clear that there are $2^3-1=7$
such cases.  Are there any more subgroups of order 2?  Hmmm...

\section*{Problem 4}

Show that $G\oplus H$ is Abelian if and only if $G$ and $H$ are Abelian.

If $G$ and $H$ are Abelian, then for any $(g_0,h_0),(g_1,h_1)\in G\oplus H$,
we have
\begin{equation*}
(g_0,h_0)(g_1,h_1)=(g_0g_1,h_0h_1)=(g_1g_0,h_1h_0)=(g_1,h_1)(g_0,h_0),
\end{equation*}
showing that $G\oplus H$ is Abelian too.  If $G\oplus H$ is Abelian, then
for any $a,b\in G$, and $e_H\in H$, we have
\begin{equation*}
(ab,e_H)=(a,e_H)(b,e_H)=(b,e_H)(a,e_H)=(ba,e_H),
\end{equation*}
which implies that $ab=ba$, showing that $G$ is Abelian.
A similar argument shows that $H$ is Abelian.

\section*{Problem 5}

Prove or disprove that $Z\oplus Z$ is a cyclic group.

Supping $Z\oplus Z$ to be cyclic, there exists $(a,b)\in Z\oplus Z$
such that $Z\oplus Z=\langle(a,b)\rangle$.  Clearly $a\neq 0$ and $b\neq 0$.
Now if $(x,y)\in Z\oplus Z$, then so is $(x+1,y)$, and there exist
integers $z_0,z_1\in Z$ such that
\begin{align*}
(x,y) &= (z_0a,z_0b), \\
(x+1,y) &= (z_1a,z_1b).
\end{align*}
It follows that $z_0b=z_1b\implies z_0=z_1$, and then that
$z_0a+1=z_1a\implies a(z_0-z_1)=1\implies 0=1$,
which is a contradiction.  The group $Z\oplus Z$ is therefore
not cyclic.

\section*{Problem 6}

Prove, by comparing orders of elements, that $Z_8\oplus Z_2$ is
not isomorphic to $Z_4\oplus Z_4$.

Notice that $Z_8\oplus Z_2$ has an element of order 8,
but $Z_4\oplus Z_4$ does not.

\section*{Problem 18}

Let $m>2$ be an even integer and let $n>2$ be an odd integers.
Find a formula for the number of elements of order 2 in $D_m\oplus D_n$.

Here, $D_m$ has $m+1$ elements of order 2, because $R_{180}\in D_m$.
Then, $D_n$ has $n$ elements of order 2, because $R_{180}\not\in D_n$.
If an element $(a,b)\in D_m\oplus D_n$ has order two, then we must
have one or both of $a$ and $b$ of order two, and any of $a$ and $b$
not of order two, being that of one, and there is only one element
of order one; namely, the identity of either $D_m$ or $D_n$, whichever applicable.  So there are $m+n+mn$ elements
of order two in $D_m\oplus D_n$, I think.

\section*{Problem 32}

Let $(a_1,a_2,\dots,a_n)\in G_1\oplus G_2\oplus\dots\oplus G_n$.  Give a necessary and
sufficient condition for $|(a_1,a_2,\dots,a_n)|=\infty$.

It may be reasonable to say that $\lcm(|a_1|,\dots,|a_n|)=\infty$
if and only if there exists an integer $i\in [1,n]$ such that $|a_i|=\infty$.

\section*{Problem 39}

Suppose that $n_1,n_2,\dots,n_k$ are positive even integers.  How many elements
of order 2 does $Z_{n_1}\oplus Z_{n_2}\oplus\dots\oplus Z_{n_k}$ have?  How many
are there if we drop the requirement that $n_1,n_2,\dots,n_k$ must be even?

If $n>0$ is even, then $Z_n$ has exactly one element of order 2; namely, n/2.
So if $n_1,\dots,n_k>2$ are even, then there are
\begin{equation*}
\binom{k}{1}+\binom{k}{2}+\dots+\binom{k}{k}=2^k - \binom{k}{0} = 2^k - 1.
\end{equation*}
elements of order $2$ in $Z_{n_1}\oplus\dots\oplus Z_{n_k}$.
Dropping the even requirement, if $0\leq j\leq k$ is the number
of integers in $\{n_i\}_{i=1}^k$ that are even, then there
would be $2^j-1$ elements of order 2, because there is no
element of order 2 in $Z_n$ when $n$ is odd.

\section*{Problem 49}

let $p$ be a prime.  Prove that $Z_p\oplus Z_p$ has exactly $p+1$ subgroups
of order $p$.

Letting $(i,j)\in Z_p\oplus Z_p$, we have $\langle (i,j)\rangle$ as a cyclic
subgroup of $Z_p\oplus Z_p$ of order $p$
whenever $i$ and $j$ are not both zero.  So to count the number of such
subgroups, we need to avoid redundant cases.
Let us start by counting $\langle (i,0)\rangle$ and $\langle (0,i)\rangle$ with
$i\neq 0$.  This gives us two subgroups of order $p$.
We then get the rest of the subgroups by counting $\langle (i,j)\rangle$
for all cases such that neither $i$ nor $j$ is zero, and $|i-j|=0,1,\dots,p-2$.
In total, we have $p-1+1+1=p+1$ subgroups of order $p$.

The back of the book gives a much more elegant answer.

\section*{Problem 56}

Let $p$ and $q$ be odd primes and let $m$ and $n$ be positive integers.
Explain why $U(p^m)\oplus U(q^n)$ is not cyclic.

By Carl Gauss and Problem 14, we have
\begin{equation*}
U(p^m)\oplus U(q^n)\approx Z_{p^m-p^{m-1}}\oplus Z_{q^n-q^{n-1}}.
\end{equation*}
Now notice that $p^m-p^{m-1}\geq 2$.  The same can be said of
$q^n-q^{n-1}$.  Now simply realize that $Z_i\oplus Z_j$ is
not cyclic for all $i,j\geq 2$ by a proof similar to that
given in problem 5.

\end{document}