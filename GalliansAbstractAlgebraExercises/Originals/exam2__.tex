\documentclass{article}
\usepackage{amsmath}
\usepackage{amssymb}
\addtolength{\oddsidemargin}{-.575in}
\addtolength{\evensidemargin}{-.575in}
\addtolength{\textwidth}{1.0in}
\addtolength{\topmargin}{-.575in}
\addtolength{\textheight}{1.25in}
\begin{document}

\newcommand{\Z}{\mathbb{Z}}
\newcommand{\Q}{\mathbb{Q}}
\newcommand{\GF}{\mbox{GF}}
\newcommand{\al}{\alpha}

\section*{Problem 1}

Show that any automorphism on $\GF(p^n)$ acts as the
identity on $\GF(p)$.

Let $\phi$ be an automorphism of $\GF(p^n)$ and let $z\in\GF(p)\subseteq\GF(p^n)$.
Since $\GF(p)\approx\Z_p$, we can think of $z$ as a member of $\Z_p$.
Now notice that we must have $\phi(0)=0$ and $\phi(1)=1$, because
0 is the additive identity of $\Z_p$, and 1 is the multiplicative
identity of $\Z_p^*$.  We then see that for any $z\in\Z_p$, we have
\begin{equation*}
\phi(z)=\phi(\underbrace{1+\dots+1}_z)=\underbrace{\phi(1)+\dots+\phi(1)}_z=\underbrace{1+\dots+1}_z=z.
\end{equation*}
Notice that not all members of $\GF(p^n)$ can be written as the sum of unities.
Only the members of $\GF(p)$ can.

%Question: Is an automorphism of a group restricted to a subgroup an automorphism of the subgroup?

%Any automorphism of $\GF(p^n)$ restricted to $\GF(p)$ is
%an automorphism of $\GF(p)$.  Therefore, it suffices to show
%that the identity on $\GF(p)$ is the only automorphism
%of $\GF(p)\approx\Z_p$.  So let us look at any field automorphism
%$\phi$ of $\Z_p$.  We must have $\phi(0)=0$ and $\phi(1)=1$, because
%0 is the additive identity of $\Z_p$ and 1 is the multiplicative
%identity of $\Z_p^*$.
%We then see that for any $z\in\Z_p$, we have
%\begin{equation*}
%\phi(z)=\phi(\underbrace{1+\dots+1}_z)=\underbrace{\phi(1)+\dots+\phi(1)}_z=\underbrace{1+\dots+1}_z=z.
%\end{equation*}
%It is important to note that it is not true in generaly that any
%automorphism of a field extension fixes the extended field.  Where the
%above proof may fail in that case is in the ability to write $z$ as the sum of ones.

\section*{Problem 2}

Let $F$ be a finite field.  Show that the product of all nonzero elements of $F$ is $-1$.

Wilson's Theorem proves this result for all finite fields of prime order.
In fact, generalizing Wilson's Theorem to all finite fields can be done
by generalizing one of its proofs.  If it can be shown that $1$ and $-1$ are
the only elements of any finite field that are their own multiplicative
inverse, then the result goes through.  Consider all solutions to
the equation $x^2-1=0$ in $F$.  Notice that $x^2-1$ splits as $(x-1)(x+1)$ in $F[x]$.
Now since $F$ is a field, it is also an integral domain.
So the only solutions to the equation $(x-1)(x+1)=0$ are 1 and $-1$.
When we now consider the product of all nonzero elements of $F$, we see that
we can pair every element with its inverse, with the exception
of 1 and $-1$, reducing the product to $(1)(-1)=-1$ as desired.

\section*{Problem 3}

Consider the polynomial $p(x)=x^3+x+1$ in $\Z_2[x]$.  Show this polynomial is irreducible.
Let $\alpha$ be a zero of $p(x)$ in some extension field of $\Z_2$.  Write out the
multiplication table for $\Z_2(\alpha)$.

Since $\deg p(x)<4$ and $p(0)\neq 0$ and $p(1)\neq 0$, we have $p(x)$ irreducible over $\Z_2$.
Using Theorem 20.3, we have $\Z_2(\alpha)\approx\Z_2[x]/\langle p(x)\rangle$ and
$\{1,\al,\al^2\}$ as a basis for it over $\Z_2$.
\begin{equation*}
\begin{array}{cccccccc}
 & \vline & \al & \al+1 & \al^2 & \al^2+1 & \al^2+\al & \al^2+\al+1 \\
\hline
\al & \vline & \al^2 & \cdot & \cdot & \cdot & \cdot & \cdot \\
\al+1 & \vline & \al^2+\al & \al^2+1 & \cdot & \cdot & \cdot & \cdot \\
\al^2 & \vline & \al+1 & \al^2+\al+1 & \al^2+\al & \cdot & \cdot & \cdot \\
\al^2+1 & \vline & 1 & \al^2 & \al & \al^2+\al+1 & \cdot & \cdot \\
\al^2+\al & \vline & \al^2+\al+1 & 1 & \al^2+1 & \al+1 & \al & \cdot \\
\al^2+\al+1 & \vline & \al^2+1 & \al & 1 & \al^2+\al & \al^2 & \al+1
\end{array}
\end{equation*}

\section*{Problem 4}

Let $R$ be a commutative ring with unity that has more than one element.
Prove that if for every nonzero element $a\in R$ we have $aR=R$, then $R$ is a field.

What we need to show is that every nonzero element is a unit.
By hypothesis, for any $r\in R$, there exists $b\in R$
such that $ab=r$.  Choosing $r=1$, we see that there
exists $b\in R$ such that $ab=1$.  But this is true for all $a\in R$.
So every $a\in R$ is a unit, which is what we wanted to show.

\end{document}