\documentclass[12pt]{article}

\usepackage{amsmath}
\usepackage{amssymb}
\usepackage{amsthm}
\usepackage{graphicx}
\usepackage{float}

\newcommand{\lcm}{\mbox{lcm}}

\title{Abstract Algebra Exercises}
\author{Spencer T. Parkin}

\begin{document}
\maketitle

These problems are taken from Gallian's, "Contemporary Abstract Algebra."

\section*{Chapter 0}

\subsection*{Exercise 7}

Show that if $a$ and $b$ are positive integers, then $ab=\lcm(a,b)\gcd(a,b)$.

We first make the observation that if $x$ and $y$ are common
multiples of $a$ and $b$ with $x<y$, then $x|y$.  Therefore,
if a given integer $q$ is a common multiple $a$ and $b$, we can
show that it is the least such multiple if there is no integer $k$
such that $q/k$ is also a common multiple.

Letting $d=\gcd(a,b)$,
we now make an argument that $\lcm(a,b)=ab/d$ by an examination
of the prime factorization of $ab/d$.  First, notice that $ab/d$ is a
common multiple of $a$ and $b$, since we may write $a=da'$ and
$b=db'$ and see that $ab/d = a'b'd$.  To see that $ab/d$ is the least common multiple
of $a$ and $b$, we notice that for an integer $k$ such that $k|(ab/d)$, we must
have $a\nmid(ab/(kd))$ or $b\nmid(ab/(kd))$, because division
of $ab/d$ by $k$ must remove a non-redundant divisor of $a$ or $b$ appearing in
the prime factorization of $ab/d$.
Division of $ab$ by $d$ removes all redundant divisors of $a$ and $b$
in the prime factorization of $ab$.

This is not a very good proof, but it makes intuitive sense.

\subsection*{Exercise 10}

Let $d=\gcd(a,b)$.  If $a=da'$ and $b=db'$, show that $\gcd(a',b')=1$.

Notice that if $x$ is any common divisor of $a$ and $b$, then $x|d$.
Therefore, there are no non-trivial divisors of $a/d$ and $b/d$.  That is,
division by $d$ removes all non-trivial common divisors.

\subsection*{Exercise 13}

Let $n$ and $a$ be positive integers and let $d=\gcd(a,n)$.  Show that the
equation $ax\mod n=1$ has a solution if and only if $d=1$.

Suppose that $\gcd(a,n)=1$.  It then follows by Theorem 0.2 that
there exists an integral linear combination of $a$ and $n$ that is
equal to one.  But this is just what it means for $ax\mod n=1$ when
we write it as $ax+ny=1$ for some integer $x$ and some integer $y$.
Now suppose that $ax\mod n=1$ has a solution.  Then there is an
integral linear combination of $a$ and $n$ such that $ax+ny=1$.  Now
suppose $d>1$.  It would then follow that $d|1$, which is a contradiction,
so we must have $d=1$.

\section*{Chapter 1}

\subsection*{Exercise 5}

For $n\geq 3$, describe the elements of $D_n$.  How many elements does $D_n$ have?

The group $D_n$, when $n\geq 3$, will have $n$ rotation operations and $n$ reflections operations.
So the group will have order $2n$.  The group $D_2$ has a 2 rotation and 2 reflection operations that
are the same, so it must have order 2.  The group $D_1$ has order 1.

\subsection*{Exercise 6}

In $D_n$, explain geometrically why a reflection followed by a reflection must be a rotation.

Rotations preserve the winding order of the $n$-gon, but reflections do not.  An even
number of reflection will leave the winding order of the $n$-gon invariant.  Then since the
rotations are the set of all winding preserving operations, two successive reflections
must be a rotation.

\subsection*{Exercise 7}

In $D_n$, explain geometrically why a rotation followed by a rotation must be a rotation.

Because the set of all rotations in $D_n$ forms its own sub-group.

\subsection*{Exercise 8}

In $D_n$, explain geometrically why a rotation and a reflection taken together in either order must be a reflection.

An odd number of reflections combined with any number of rotations does not preserve winding order.
The only non-winding-order-preserving operations are the reflections.  So any rotation and
reflection combination must be a reflection.

\subsection*{Exercise 12}

For any integer $n>2$, show that there are at least two elements in $U(n)$ that satisfy $x^2=1$.

The trivial case is $1^2=1$.  Now notice that for all $n>2$, we have $\gcd(n,n-1)=1$.
Notice that $(n-1)^2\equiv 1\pmod n$.

\subsection*{Exercise 23}

Prove that every group table is a Latin Square; that is, each element of the group appears
exactly once in each row and each column.

If the group table was not a latin square, then there must exist three distinct elements $a,b,c$
such that $ab=ac$.  Multiplying this equation on the left by $a^{-1}$, we find that $b=c$,
which is a contradiction.  So three such elements cannot exist in any group table.

\subsection*{Exercise 29}

Let $G$ be a finite group.  Show that the number of elements $x$ of $G$ such
that $x^3=e$ is odd.  Show that the number of elements $x$ of $G$ such that
$x^2=e$ is even.

\end{document}