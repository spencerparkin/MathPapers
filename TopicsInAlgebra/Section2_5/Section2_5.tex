\documentclass[12pt]{article}

\usepackage{amsmath}
\usepackage{amssymb}
\usepackage{amsthm}

\title{Section 2.5 Exercises\\Hertein's Topics In Algebra}
\author{Spencer T. Parkin}

\newtheorem{theorem}{Theorem}[section]
\newtheorem{definition}{Definition}[section]
\newtheorem{corollary}{Corollary}[section]
\newtheorem{identity}{Identity}[section]
\newtheorem{lemma}{Lemma}[section]
\newtheorem{result}{Result}[section]

%\newcommand{\gcd}{\mbox{gcd}}
\newcommand{\lcm}{\mbox{lcm}}
\newcommand{\Z}{\mathbb{Z}}

\begin{document}
\maketitle

\section*{Understanding Theorem 2.5.1}

Here, a leap was made for me in realizing that for every $h_1,h_2\in H\cap K$,
we have $hh_1\neq hh_2$ and $h_1^{-1}k\neq h_2^{-1}k$.  This is obvious by
the cancelation proprety.  So what he does is show that $o(H\cap K)$ duplicates
of the element $hk$ exists in the $o(HK)$ possible multiplications of an element
taken from $H$ with that of $K$.  He then shows that this lower bound is also
an upper bound, and the rest goes through.

\end{document}