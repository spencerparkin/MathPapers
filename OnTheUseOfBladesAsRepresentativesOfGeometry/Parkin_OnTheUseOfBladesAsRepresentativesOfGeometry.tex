\documentclass{birkjour}

\usepackage{tikz}
\usepackage{graphicx}
\usepackage{hyperref}

\newtheorem{thm}{Theorem}[section]
\newtheorem{cor}[thm]{Corollary}
\newtheorem{lem}[thm]{Lemma}
\newtheorem{prop}[thm]{Proposition}
\theoremstyle{definition}
\newtheorem{defn}[thm]{Definition}
\theoremstyle{remark}
\newtheorem{rem}[thm]{Remark}
\newtheorem*{ex}{Example}
\numberwithin{equation}{section}

\newcommand{\R}{\mathbb{R}}
\newcommand{\B}{\mathbb{B}}
\newcommand{\G}{\mathbb{G}}
\newcommand{\V}{\mathbb{V}}
\newcommand{\gd}{\dot{g}}
\newcommand{\gh}{\hat{g}}
\newcommand{\Gd}{\dot{G}}
\newcommand{\Gh}{\hat{G}}
\newcommand{\nvai}{\infty}
\newcommand{\nvao}{o}
\newcommand{\grade}{\mbox{grade}}

%\received{}\accepted{}

\begin{document}

\title{On The Use Of Blades As\\Representatives Of Geometry}

\author{Spencer T. Parkin}
\address{102 W. 500 S., \\
Salt Lake City, UT  84101} \email{spencerparkin@outlook.com}

%\subjclass{Primary 14J70; Secondary 14J29}

%\dedicatory{To my dear wife Melinda.}

\begin{abstract}
Abstract goes here...
\end{abstract}

\keywords{Key words go here...}

\maketitle

\section{Introduction And Motivation}

In many models of geometry based upon geometric algebra (see \cite{}), blades are used
to represent geometries.  Seeing a great deal of commonality between these models, a formal
treatment of this idea deserves to be given in an abstract setting.\footnote{For example,
the use of ideals of polynomial rings as representatives of algebraic sets is studied in
the setting of abstract algebra.}  To the author's knowledge, this is the first treatment of its kind.

\section{Foundation}

To lay the foundation of our work, we introduce $\V$ as denoting an $m$-dimensional vector
space generating a geometric algebra denoted by $\G$.  We leave the signature of this geometric algebra
unspecified, but in cases where a proof depends upon signature, one is given as either
euclidean or non-euclidean.\footnote{The Gram-Schmidt orthogonalization process is applicable
to all blades taken from and only from geometric algebras having euclidean signatures.}
The scalars of $\V$, (and therefore $\G$), are taken from the field of real numbers denoted by $\R$.\footnote{To be more abstract,
we could have used any field with characteristic 1, but there will be no foreseable advantage to doing so in this paper.}
We will let $\R^n$ denote $n$-dimensional euclidean space,\footnote{Some models of geometry find affine space to be the natural
space within which to work, but this will not be the case in this paper.} and let $\B$ denote the set of all blades taken from $\G$.
Lastly, we will let $p:\R^n\to\V$ be an unspecified function we'll use in the following definition which launches us forth into a general
theory of models of geometry based upon blades in a geometric algebra.  In our abstract setting, the definition of
this function does not matter.  All that matters is that it is a function.

\begin{defn}[Direct And Dual Representation]\label{def_blade_rep_geo}
For any blade $B\in\B$, we say that $B$ directly represents the set of all points $x\in\R^n$ such that
$p(x)\wedge B=0$, and say that $B$ dually represents the set of all points $x\in\R^n$ such that
$p(x)\cdot B=0$.  For convenience, we introduction the following functions using set-builder notation.
\begin{align*}
\gh(B) &= \{x\in\R^n|p(x)\wedge B=0\} \\
\gd(B) &= \{x\in\R^n|p(x)\cdot B=0\}
\end{align*}
\end{defn}
From Definition~\ref{def_blade_rep_geo}, it's important to take away the realization that a given blade $B\in\B$ represents two geometries
simultaneously; namely, $\gh(B)$ and $\gd(B)$.  Which geometry we choose to think of $B$ as being a representative of at any given time is completely
arbitrary.\footnote{In some literature on geometric algebra, a blade $B$ intended to represent some peice of geometry directly or dually
is referred to as a ``geometry'' or a ``dual geometry,'' respectively.  This is confusing and not practiced in this paper.  A blade is a blade; and when
we refer to geometry, we will use proper language in identifying what represents it and how it does so.  In this paper, a geometry is a subset of
$\R^n$ that can be represented dually or directly by some blade $B\in\B$ under Definition~\ref{def_blade_rep_geo}.}

It should also be clear from Definition~\ref{def_blade_rep_geo} that the geometry represented by a blade $B$, (directly or dually), remains invariant
under any non-zero scaling of the blade $B$.  Something interesting happens, however, when we take the dual of $B$, as our first lemma shows.

\begin{lem}[Dual Something]\label{lem_dual_rep}
For any subset $S$ of $\R^n$, if there exists $B\in\B$ such that $\gh(B)=S$, then $\gd(BI)=S$, where
$I$ is the unit psuedo-scalar of $\G$.  Similarly, if there exists $B\in\B$ such that $\gd(B)=S$, then $\gh(BI)=S$.
\end{lem}
\begin{proof}
The first of these two statements is proven by
\begin{equation*}
0=p(x)\wedge B=-(p(x)\cdot BI)I\iff p(x)\cdot BI=0,
\end{equation*}
while the second is proven by
\begin{equation*}
p(x)\cdot B=0\iff 0=(p(x)\cdot B)I=p(x)\wedge BI.
\end{equation*}
See identities \eqref{equ_dual_of_v_dot_B} and \eqref{equ_dual_of_v_dot_dual_B} of Section~\ref{sec_useful_identities}.
\end{proof}

In words, Lemma~\ref{lem_dual_rep} is telling us that for a single given geometry, the algebraic relationship between a
blade directly (dually) representative of that geometry, and a blade dually (directly) representative of that geometry, is simply
that, up to scale, they are duals of one another.

Of course, there will also be a geometric relationship between the geometry that is directly represented by a single given
blade $B\in\B$, and the geometry that is dually represented by $B$, but this depends upon the definition of our function
$p$, which we choose, in this paper, to leave open for speculation.

% what's next?

% need to cover intersections and union-like operations.

% need to cover existence and uniqueness of irreducible blades...introduce this interms of converse of obvious statement.
% given algorithm for finding irreducible blade.

% cover transformations by versors and how that relates to understanding the action of a versor on p?

\section{Useful Identities}\label{sec_useful_identities}

In this section we give a number of useful algebraic identities that would otherwise distract us from the flow of the
paper if given in the main body.  This section is not intended as a complete review of geometric algebra.
See \cite{} for such a review.

Letting $v\in\V$ and $B\in\B$, recall that
\begin{equation}\label{equ_v_geoprod_B_identity}
vB = v\cdot B+v\wedge B.
\end{equation}
Also recall that
\begin{align}
v\wedge B &= \frac{1}{2}(vB+(-1)^{\grade(B)}Bv),\label{equ_v_wedge_B_identity} \\
v\cdot B &= \frac{1}{2}(vB-(-1)^{\grade(B)}Bv).\label{equ_v_dot_B_identity}
\end{align}
Realizing that $\grade(I)=m$, and that by \eqref{equ_v_geoprod_B_identity}, we have $vI=v\cdot I$, we can use
equation \eqref{equ_v_dot_B_identity} to establish the commutativity of vectors in $\V$ with the unit psuedo-scalar $I$ as
\begin{equation}\label{equ_v_commute_I}
vI = -(-1)^mIv.
\end{equation}
Using equation \eqref{equ_v_commute_I} in conjunection with equation \eqref{equ_v_dot_B_identity}, we find that
\begin{equation}\label{equ_dual_of_v_dot_B}
(v\cdot B)I = v\wedge BI.
\end{equation}
In verifying this identity, it helps to realize that for any integer $k$, $(-1)^k=(-1)^{-k}$.
Replacing $B$ in equation \eqref{equ_dual_of_v_dot_B} with $BI$, we find that
\begin{equation}\label{equ_dual_of_v_dot_dual_B}
v\wedge B = -(v\cdot BI)I.
\end{equation}


\begin{thebibliography}{9}

\end{thebibliography}

\end{document}