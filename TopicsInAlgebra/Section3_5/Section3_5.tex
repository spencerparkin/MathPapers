\documentclass[12pt]{article}

\usepackage{amsmath}
\usepackage{amssymb}
\usepackage{amsthm}

\title{Section 3.5 Exercises\\Herstein's Topics In Algebra}
\author{Spencer T. Parkin}

\newtheorem{theorem}{Theorem}[section]
\newtheorem{definition}{Definition}[section]
\newtheorem{corollary}{Corollary}[section]
\newtheorem{identity}{Identity}[section]
\newtheorem{lemma}{Lemma}[section]
\newtheorem{result}{Result}[section]

%\newcommand{\gcd}{\mbox{gcd}}
\newcommand{\lcm}{\mbox{lcm}}
\newcommand{\Z}{\mathbb{Z}}
\newcommand{\R}{\mathbb{R}}
\newcommand{\cl}{\mbox{Cl}}
\newcommand{\aut}{\mbox{Aut}}

\begin{document}
\maketitle

\section*{Thoughts}

If $R$ is a commutative ring and $a\in R$, then I think it's fair to define the ideal of $R$ generated by $a$ as
\begin{equation*}
I = \{ra|r\in R\},
\end{equation*}
and write $I=\langle a\rangle$.  Clearly $I$ is non-empty.  Let $x,y\in I$.  Then $x=r_xa$ and $y=r_ya$,
and we have $x+y=(r_x+r_y)a\in I$.  Also, $-x=(-r_x)a\in I$.  So $I$ is a subgroup of $R$.
We also have $xy=(r_xr_ya)a\in I$, so it's a subring of $R$.  And lastly, for any $r\in R$, we have
$xr=rx=(rr_x)a\in I$, so it's an ideal of $R$.

It should also be remarked that $\langle a\rangle$ is the smallest possible ideal of $R$ containing $a$.
If we knew $I$ was an ideal of $R$ containing $a$, then it must also contain all elements of the form $ra$.
After throwing those into $I$, this is the soonest we form a set that is an ideal of $R$.

What now may be of interest is to consider any ideal $I$ of $R$, choose $a\in I$, and consider
the relationship
\begin{equation*}
\langle a\rangle\subseteq I\subseteq R.
\end{equation*}
Notice that $\langle a\rangle$ is not only an ideal of $R$, but also of $I$.
It may also be of interest to consider
the case that $\langle a\rangle\neq I$.  In that case, choose $b\in I-\langle a\rangle$,
and see that
\begin{equation*}
\langle a\rangle\cup\langle b\rangle\subseteq I.
\end{equation*}
Indeed, if we write
\begin{equation*}
I = \bigcup_{a\in S}\langle a\rangle,
\end{equation*}
then this clearly holds when $S=I$.  But there are certainly cases where $S$ is a proper subset of $I$.
I suppose if $S$ is finite, we can say that the ideal $I$ is finitely generated.

\section*{Problem 4}

Let $R$ be the ring of all real-valued continuous functions on the closed unit interval.
If $M$ is a maximal ideal of $R$, prove that there exists a real number $\gamma$,
$0\leq\gamma\leq 1$, such that $M=M_{\gamma}=\{f(x)\in R|f(\gamma)=0\}$.

Let $f\in R$ be a non-zero-valued continuous function on all of $[0,1]$, and suppose $I$ is an ideal of $R$
containing it.  Now letting $g\in R$ be any member of $R$, does there exist a function $h\in R$ such that
$fh=g$?  Clearly there must, since $f$, being non-zero on $[0,1]$, allows us to write $h=g/f$.
We can now conclude that $I=R$, and that for every properly contained ideal $I$ of $R$, if $f\in I$,
then there exists $\gamma\in[0,1]$ such that $f(\gamma)=0$.

Now let $f\in R$ be a function with {\it exactly one} zero $\gamma\in[0,1]$, and consider the ideal $\langle f\rangle$.
Notice that all $g\in\langle f\rangle$ have this same zero, even if possibly others.
It is not clear, however, whether $\langle f\rangle$ contains all functions of $R$ having this zero.
Persuing this, we let $g\in R$ be any such function, and ask: can we find $h\in R$ such that $fh=g$?  Consider
\begin{equation*}
h(x)=\left\{\begin{array}{ll} g(x)/h(x) & x\neq\gamma, \\ 0 & x=\gamma. \end{array}\right.
\end{equation*}
The problem here is that $h$ need not be continuous at $\gamma$.  That is, we need not have
$\lim_{x\to\gamma} h(x)=0$.  The limit may, in fact, not even exist!

Leaving this line of thinking for a moment, can there exist a proper ideal $I$ of $R$ with the
property that there does not exist $x\in[0,1]$ such that for all $f\in I$, we have $f(x)=0$?\footnote{I suspect that
in such an ideal we would be able to construct a function that is non-zero on all of $[0,1]$, which would lead us to
contradict the fact that it's a proper ideal.}
Let's suppose for the moment that no such ideal can exist.  In that case, we can claim that
for every proper ideal $I$ of $R$, we must have $V(I)$ non-empty, where this is defined as
\begin{equation*}
V(I)=\{x\in[0,1]|\mbox{$f(x)=0$ for all $f\in I$}\}.
\end{equation*}
But then we can also establish the relationship that for any two ideals $I,J\subset R$,
if $V(I)\subset V(J)$, then $I\supset J$.  If $|V(I)|=1$, then must we have $I=M_{\gamma}$
with $\gamma\in V(I)$?  Can it be shown that if $|V(I)|>1$, then $I$ is not maximal?
Note that if $V(I)>1$, then $I$ cannot contain every function of $R$ having one of the zeros in $V(I)$,
because then it must contain a function that is non-zero on all of $[0,1]$.  (In such a case, $I=R$, and we have $|V(I)|=0$,
a contradiction.)  For example, if $V(I)=\{\alpha,\beta\}$, we can construct $f:[0,1]\to\R$ that is non-zero on all of $[0,1]$
as the sum of two continuous functions $h$ and $g$, each having exactly one zero: $\alpha$ and $\beta$, respectively.
Therefore, $h$ and $g$ cannot co-exist in $I$.
So if $|V(I)|>1$, we know that $I$ is properly contained in $M_{\gamma}$, where $\gamma\in V(I)$;
so $I$ is not maximal in $R$.

\end{document}