\documentclass{article}
\usepackage{amsmath}
\usepackage{amssymb}
\title{Chapter 24 Homework}
\author{Spencer}
\addtolength{\oddsidemargin}{-.575in}
\addtolength{\evensidemargin}{-.575in}
\addtolength{\textwidth}{1.0in}
\addtolength{\topmargin}{-.575in}
\addtolength{\textheight}{1.25in}
\begin{document}
\maketitle

\newcommand{\Z}{\mathbb{Z}}
\newcommand{\R}{\mathbb{R}}
\newcommand{\N}{\mathbb{N}}
\newcommand{\Q}{\mathbb{Q}}
\newcommand{\cl}{\mbox{cl}}

\section*{Problem 2}

Calculate all conjugacy classes for the quaternions.

Some work shows that the conjugacy classes are
\begin{equation*}
\{e\},\{a,a^3\},\{a^2\},\{b,ba^2\},\{ba,ba^3\}.
\end{equation*}

\section*{Problem 8}

Find all Sylow 3-subgroups of $A_4$.

The group $A_4$ has order $4!/2=24=2^3\cdot 3$, so we're looking for
all subgroups of order 3.  Since 3 is prime, all such subgroups are cyclic.
So the set of subgroups is given by
$\{\langle a\rangle|\mbox{$a\in A_4$ and $|a|=3$}\}$.
Using Table 5.1, these groups are given by
\begin{align*}
\{\alpha_1,\alpha_5,\alpha_9\},\\
\{\alpha_1,\alpha_6,\alpha_{11}\},\\
\{\alpha_1,\alpha_7,\alpha_{12}\},\\
\{\alpha_1,\alpha_8,\alpha_{12}\}.
\end{align*}
This was given away in Example 2 of the chapter.

\section*{Problem 12}

Show that every group of order 56 has a proper nontrivial normal subgroup.

By Sylow's First Theorem, there is at least one subgroup of order 7.
By Sylow's Third Theorem, the number of such subgroups is congruent
to 1 modulo 7 and divides 56.  The only possible number of such
subgroups is therefore 1 or 8.  If there is only 1 subgroup of order 7,
then it is a nontrivial normal subgroup by the corollary to Sylow's
Third Theorem.  If there are 8 subgroups of order 7, then because
they are cyclic, they account for $8\cdot(7-1)=48$ distinct elements,
leaving $56-48=8$ remaining elements, including the identity element.
It follows that these 8 elements must form a subgroup of order 8,
By Sylow's First Theorem.
Furthermore, this is the only subgroup of order 8, because every
one of the 48 elements previously mentioned have order 7, and 7 does
not divide 8.

\pagebreak
\section*{Problem 24}

Prove that a group of order 105 contains a subgroup of order 35.

By Sylow's First Theorem, there is a subgroup of order 5, and a subgroup
of order 7.  By Sylow's Third Theorem, there
may be at most 21 groups of order 5, and 15 groups of order 7.  Suppose there
are more than one of both types of subgroups.  Since these are cyclic
of prime order, this would generate
$21\cdot(5-1)=84$ elements of order 5,
and $15\cdot(7-1)=90$ elements of order 7.  But $84+90>105$, so at least one
of these two types of subgroup is unique, and therefore normal by the corollary to
Sylow's Third Theorem.
It then follows by the result of Exercise 51 on
Page 194, that if $H$ is a subgroup of order 5, and $K$ is a subgroup of order 7,
then $HK$ is a subgroup of $G$, since at least one of $H$ and $K$ is normal.
Now notice that $|HK|=|H||K|/|H\cap K|$, but $H\cap K=\{e\}$ since $H$ and $K$
are cyclic groups of distinct prime orders.  It follows that $|HK|=|H||K|=5\cdot 7=35$,
which is what we wanted.

\section*{Problem 28}

Suppose that $G$ is a group of order 60 and $G$ has a normal
subgroup $N$ of order 2.  Show that $G$ has normal subgroups
of orders 6, 10, and 30.  Show that $G$ has subgroups of
orders 12 and 20.  Show that $G$ has a cyclic subgroup of order 30.

Cauchy tells us that $G$ has an element $\alpha$ of order 3 and so a
cyclic subgroup of order 3.  Since $N$ is normal, $N\langle\alpha\rangle$
is a subgroup of $G$.  Since $N$ is cyclic too, $N\cap\langle\alpha\rangle=\{e\}$,
and so $|N\langle\alpha\rangle|=|N||\langle\alpha\rangle|=2\cdot 3=6$.
A similar argument shows that if $\beta$ is an element of order 5, then
$N\langle\beta\rangle$ is a subgroup of order 10.

If I could make an argument for why there is a normal subgroup of order 5,
(perhaps it's the only Sylow 5-subgroup), this would imply the existence
of subgroups of order 30 and 20.

If there was only one Sylow 2-subgroup, it would be normal, and this would
imply the existence of a subgroup of order 12.

If we could find a homomorphism from $G$ to $G$ whose kernel was $N$,
then its image would be a subgroup of order 30, but I can't find such
a homomorphism.  I don't know how to show that there is an element of order 30.

\section*{Problem 40}

Let $p$ be prime.  If the order of every element of a finite group $G$ is a
power of $p$, prove that $|G|$ is a power of $p$.

Suppose $|G|$ has a prime $q$ other than $p$ in its factorization.
Then by Sylow's First Theorem, there must exist a cyclic subgroup of
order $q$, and its generator is an element of order $q$, contradicting
the fact that all elements of $G$ have orders that are powers of $p$.
So $p$ is the only prime in the factorization of $|G|$, and so $|G|$
is a power of $p$.

\section*{Problem 44}

Determine the groups of order 45.

Sylow's Third Theorem tells us that groups of order 45 have 1 subgroup
of order 5 and 1 subgroup of order 9.  The corollary then tells us that
these are normal subgroups.  Let $G$ be a group of order 45, and let
$H$ and $K$ be its normal subgroups of orders 5 and 9, respectively.
Then these are both normal, and we can describe $G$ as $H\times K$.

\end{document}