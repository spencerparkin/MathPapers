\documentclass[12pt]{article}

\usepackage{amsmath}
\usepackage{amssymb}
\usepackage{amsthm}

\title{Chapter 7 Exercises\\Gallian's Book on Abstract Algebra}
\author{Spencer T. Parkin}

\newtheorem{theorem}{Theorem}[section]
\newtheorem{definition}{Definition}[section]
\newtheorem{corollary}{Corollary}[section]
\newtheorem{identity}{Identity}[section]
\newtheorem{lemma}{Lemma}[section]
\newtheorem{result}{Result}[section]

%\newcommand{\gcd}{\mbox{gcd}}
\newcommand{\lcm}{\mbox{lcm}}
\newcommand{\abs}{\mbox{abs}}
\newcommand{\Z}{\mathbb{Z}}
\newcommand{\R}{\mathbb{R}}
\newcommand{\G}{\mathbb{G}}
\newcommand{\stab}{\mbox{stab}}
\newcommand{\aut}{\mbox{Aut}}
\newcommand{\inn}{\mbox{Inn}}
\newcommand{\orb}{\mbox{orb}}

\begin{document}
\maketitle

\section*{Exercise 29}

Let $G$ be a group of permutations of a set $S$.  Prove that the orbits of the
members of $S$ consitute a partition of $S$.

For any $a,b\in S$, let $a\sim b$ if and only if $a\in\orb_G(b)$.
To see that this forms an equivilance relation on the set $S$,
we begin by noting that for all $a\in S$, we have $a\in\orb_G(a)$,
giving us the reflexive property.  For all $a,b\in S$, if $a\sim b$,
then $a\in\orb_G(b)$ and therefore there exists $\phi\in G$ such
that $\phi(b)=a$.  Seeing that $\phi^{-1}(a)=b$ and $\phi^{-1}\in G$, it is clear
that $b\in\orb_g(a)$ and therefore $b\sim a$, giving us the symmetric property.
Lastly, for all $a,b,c\in S$, if $a\sim b$ and $b\sim c$, there exist $\phi_0,\phi_1\in G$
such that $\phi_0(b)=a$ and $\phi_1(c)=b$.  Then, letting $\phi=\phi_0\phi_1\in G$,
it is clear that $\phi(c)=a$ and therefore $a\in\orb_G(c)$.  It follows that $a\sim c$,
and we have the transitive property.

Now notice that the equivilance class containing $a$ is given by
\begin{equation*}
[a]=\{x\in S|x\sim a\}=\{x\in S|x\in\orb_G(a)\}=\orb_G(a),
\end{equation*}
showing that the orbits of elements in $S$ are the equivilance classes
that partition $S$ by Theorem 0.6.

\end{document}