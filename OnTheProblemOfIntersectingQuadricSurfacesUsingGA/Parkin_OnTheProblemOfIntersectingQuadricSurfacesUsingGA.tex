\documentclass{birkjour}

%\usepackage{tikz}
%\usepackage{graphicx}
\usepackage{hyperref}

\newtheorem{thm}{Theorem}[section]
\newtheorem{cor}[thm]{Corollary}
\newtheorem{lem}[thm]{Lemma}
\newtheorem{prop}[thm]{Proposition}
\theoremstyle{definition}
\newtheorem{defn}[thm]{Definition}
\theoremstyle{remark}
\newtheorem{rem}[thm]{Remark}
\newtheorem*{ex}{Example}
\numberwithin{equation}{section}

\newcommand{\R}{\mathbb{R}}
\newcommand{\B}{\mathbb{B}}
\newcommand{\G}{\mathbb{G}}
\newcommand{\V}{\mathbb{V}}
\newcommand{\Z}{\mathbb{Z}}
\newcommand{\gd}{\dot{g}}
\newcommand{\gh}{\hat{g}}
\newcommand{\Gd}{\dot{G}}
\newcommand{\Gh}{\hat{G}}
\newcommand{\nvai}{\infty}
\newcommand{\nvao}{o}
\newcommand{\grade}{\mbox{grade}}

\begin{document}

\title{On The Problem Of Intersecting\\Quadric Surfaces Using\\Geometric Algebra}

\author{Spencer T. Parkin}
\address{102 W. 500 S., \\
Salt Lake City, UT  84101} \email{spencerparkin@outlook.com}

\subjclass{Primary 14J70; Secondary 14J29}

\dedicatory{To my dear wife Melinda.}

\begin{abstract}
Progress is made on the problem of finding a conformal-like model of geometry
based upon geometric algebra in which intersections of quadric surfaces
may be taken.
\end{abstract}

\keywords{Quadric, Geometric Algebra, Conformal Model, Intersection}

\maketitle

\section{Introduction}

In light of the paper \cite{Parkin13}, some encouragement has been given to the present
author to develop a model of geometry, similar to the conformal model of geometric
algebra, (see\cite{Hestenes01,Dorst07,Lasenby04}), but not limited in representation to any proper subset of the set of all
quadric surfaces.  But for such a model to achieve adequate similarity to the conformal model,
it must preserve one or more of its most desirable features; preferably, all of them.
For example, in \cite{Parkin13}, the set of all conformal transformations were preserved, but
intersections were not.\footnote{A way of representing intersections using the outer
product in the model of \cite{Parkin13} can be found, but its usefulness, if any, is highly questionable.}

The goal of this paper, therefore, is to
preserve the intersection property.  In other words, we want to find a model of geometry based upon
geometric algebra giving us all quadric surfaces and the ability to intersect them
as effortlessly as can be done in the conformal model.  If nothing else, the attempt
to do so in this paper will shed light on the feasibility and practicality of such an endeavor.

\section{The Intersection Property}

Let us begin by taking a closer look at exactly what the intersection property is.
Upon initial inspection, one might suppose that this property is nothing more than
the ability to represent the intersection of any two given geometries in a way consistent
with the representation of any geometry of the model, but this is not
enough.  Such a representation has no usefulness if it does not submit to an
analysis yielding the geometric characteristics of the intersection.

That having been said, we can say now that the outer product's ability to intersect geometries represented
by blades in the conformal model is really not at all interesting.  What is interesting
is the realization that we can equate one characterization of an intersection
with another, and this is the key to finding intersections in the conformal model.
The reason for this is that while one such characterization is composed as the
intersection we wish to take, the other characterization lends itself to
analysis through decomposition.

For example, suppose we wish to take the planar intersection of a conical surface.
If we know or suspect that the resulting conic section is an ellipse, then we can choose to
interpret this intersection as that of a plane and an elliptical cylinder meeting
the plane at right angles.  This latter characterization will have an easily found decomposition
yielding all features of the ellipse.\footnote{Geometries having easily decomposable
representations in our model will be referred to as canonical forms.}
Having found all such features, we can then say that we've
fully realized the given section, whereas before this we were only able to
represent it.\footnote{Note that although non-planar intersections will be as easily
represented in our model of geometry as any other type of intersection,
the technique of finding intersections in this paper
might not be helpful in finding non-planar intersections for the simple reason that such intersections
have no obvious canonical forms.  For more on non-planar intersections, see \cite{Miller87}.}

The quest to find our model of geometry, (the one promised in the introductory section of
this paper), being quite difficult, the bringing of the example just given in the preceding paragraph
to fruition will become the impetus for all choices henceforth made in finding the model.
Even if our model can do nothing more than this one example, we will consider or goal achieved.
Rest assured, however, that along the way, we will find results generally applicable to the problem at hand.

\section{Enter The Model}

So that no further delay be made, we will now let the remainder of this paper begin
exactly where the first section of \cite{Parkin13} ended, assuming all results and definitions up to that point.\footnote{The
reader need read no further than the first section of \cite{Parkin13} before preceding.  In this paper, $\R^n$ is used to
denote $n$-dimensional euclidean space; but we are, for the most part, going to restrict ourselves to the case $n=3$.
The astute reader will recognize when results in this paper generalize to any positive integer $n$.}
That said, we now introduce the function $p:\R^n\to\V$ as
\begin{align*}
p(x) &= e_0 + x \\
 &+ (x\cdot e_2)(x\cdot e_3)e_4 + (x\cdot e_1)(x\cdot e_3)e_5 + (x\cdot e_1)(x\cdot e_2)e_6 \\
 &+ (x\cdot e_1)^2e_7 + (x\cdot e_2)^2e_8 + (x\cdot e_3)^2 e_9,
\end{align*}
the vectors in $\{e_i\}_{i=0}^9$ forming an orthonormal basis for a 10-dimensional
euclidean vector space $\V$ generating our geometric algebra $\G$.\footnote{The reader
should take care to note which results of this paper
depend upon this definition of the function $p$ and which do not.  Also note that signatures
other than the euclidean may be worth considering, but there will be no foreseeable need to do so in this paper.}
This, along with the function $\gh$ and $\gd$, is
sufficient to define the entire model as it is clear that for any quadric
surface, or any intersection of two or more quadric surfaces, there exists a blade
$B\in\G$ representative of this surface as $\gd(B)$.

Continuing with our example, let us now find the canonical form for an ellipse
in a plane.  Using the notation\footnote{This notation is overloaded.  When writing $x_i$, this may mean
the $i^{th}$ component of $x$, or it may mean the $i^{th}$ point in a sequence of points.  The intended
meaning will always be clear from context.} $x_i = x\cdot e_i$, an equation for such an ellipse
may be given by
\begin{equation}\label{equ_ellipse}
\frac{(x_1-h)^2}{a^2} + \frac{(x_2-k)^2}{b^2} - 1 = 0,
\end{equation}
provided $x_3=0$, which is simply an equation for the plane.
Factoring $p(x)$ out of the equation $x_3=0$, we get $p(x)\cdot e_3=0$,
and out of equation \eqref{equ_ellipse}, we get $p(x)\cdot E=0$, where the
vector $E$ is given by
\begin{equation*}
E = \left(\frac{h^2}{a^2}+\frac{k^2}{b^2} - 1\right)e_0 -
  2\frac{h}{a^2}e_1 - 2\frac{k}{b^2}e_2 + \frac{1}{a^2}e_7 + \frac{1}{b^2}e_8.
\end{equation*}
By itself, $E$ here represents an axis-aligned elliptical cylinder.  The ellipse is given by
the set of points $\gd(E\wedge e_3)$.  The 2-blade $E\wedge e_3$ will be the
canonical form of the ellipse that we will use to find a conic intersection that
is an ellipse.

Letting $\lambda\in\R$ be any non-zero scalar, to see that
$B=\lambda E\wedge e_3$ is easily decomposable, we give the following set of
equations.
\begin{align}
h &= (-e_{31}\cdot B)(2e_{37}\cdot B)^{-1}\label{equ_decompose_first} \\
k &= (-e_{32}\cdot B)(2e_{38}\cdot B)^{-1} \\
\lambda &= (h^2e_{37}+k^2e_{38}-e_{30})\cdot B \\
a &= \sqrt{\lambda(e_{37}\cdot B)^{-1}} \\
b &= \sqrt{\lambda(e_{38}\cdot B)^{-1}}\label{equ_decompose_last}
\end{align}
Using these equations, we can recover the ellipse.  Note here that we
are using the convenient notation $e_{ijk\dots} = e_i\wedge e_j\wedge e_k\wedge\dots$.

Now let us formulate the intersection we wish to take.  We will
intersect the $x_3=0$ plane with the conical surface having points satisfying the equation
\begin{equation}\label{equ_conical_surface}
x_1^2+x_2^2 - (x_3+1)^2\tan^2\frac{\pi}{4} = 0,
\end{equation}
where $\frac{\pi}{2}$ is the angle of aperture.\footnote{Yes, this will give
us a circle in the $x_3=0$ plane and not a general ellipse, but to keep things
simpler, our primary example will consider this special case of the ellipse.}  We have submerged it
below the $x_3=0$ plane to get a non-trivial intersection $\gd(C\wedge e_3)$,
the vector $C$ being given by
\begin{align*}
C &= -\left(\tan^2\frac{\pi}{4}\right)e_0 + 2\left(\tan^2\frac{\pi}{4}\right)e_3 + e_7 + e_8 - \left(\tan^2\frac{\pi}{4}\right)e_9 \\
 &= -e_0 + 2e_3 + e_7 + e_8 - e_9.
\end{align*}

\section{Making Use Of The Model}

Being now able to represent all quadric intersections, the task of
algebraically relating them remains.  For example, knowing that
$C\wedge e_3$ is an ellipse, how might we decompose it
as we would $E\wedge e_3$?  The following lemma may be able to help.
\begin{lem}\label{lma_point_factored_blade}
Letting $B\in\G$ be a blade of grade $k$, if there exist $k$ points $\{x_i\}_{i=1}^k\subset\R^n$
such that $\bigwedge_{i=1}^k p(x_i)\neq 0$, and that for all points
$x\in\{x_i\}_{i=1}^k$, we have $x\in\gh(B)$, then there exists a scalar $\lambda\in\R$
such that
\begin{equation}\label{equ_point_factored_blade}
B = \lambda\bigwedge_{i=1}^k p(x_i).
\end{equation}
\end{lem}
\begin{proof}
If $x_i\in\gh(B)$, then $p(x_i)\wedge B=0$, showing that $p(x_i)$ is in the vector
space spanned by any factorization of $B$ in terms of a set of basis vectors.
Then, since $\bigwedge_{i=1}^k p(x_i)\neq 0$,
the set of vectors $\{p(x_i)\}_{i=1}^k$ is linearly independent and therefore
a basis for this vector space.  The blades $B$ and $\bigwedge_{i=1}^k p(x_i)$
must, therefore, be equal, up to scale.
\end{proof}
For a blade $B\in\G$ having a factorization of the form in equation \eqref{equ_point_factored_blade},
we will refer to $B$ as an {\it irreducible blade} for reasons that will become clear shortly.

We now make use of Lemma~\ref{lma_point_factored_blade} in our next lemma.
\begin{lem}\label{lma_equating_blades}
If $A,B\in\G$ are blades of grade $k$ with $\gh(A)=\gh(B)$, and one of these is irreducible,
then there exists a scalar $\lambda\in\R$ such that $A=\lambda B$.
\end{lem}
\begin{proof}
We first establish that if one of the blades $A$ and $B$ is irreducible, then so is the other.
Assuming, without loss of generality, that $A$ is irreducible, let $\{x_i\}_{i=1}^k\subset\R^n$
be a set of $k$ points, and $\alpha\in\R$ be a scalar, such that $A=\alpha\bigwedge_{i=1}^k p(x_i)$.
Now since $\gh(A)=\gh(B)$, it is clear that for all $x\in\{x_i\}_{i=1}^k$, we have $x\in\gh(B)$,
and so it follows by Lemma~\ref{lma_point_factored_blade} that there exists a scalar $\beta\in\R$
such that $B=\beta\bigwedge_{i=1}^k p(x_i)$.

Lastly, we simply realize that if we let $\lambda=\frac{\alpha}{\beta}$, we have
$A=\lambda B$.
\end{proof}

In light of Lemma~\ref{lma_equating_blades}, the following question naturally arises.
Knowing that $\gh((E\wedge e_3)I)=\gh((C\wedge e_3)I)$, is any one of
$(E\wedge e_3)I$ and $(C\wedge e_3)I$ irreducible?\footnote{Here we're letting $I=e_{0123456789}$
be the unit psuedo-scalar of our 10-dimensional geometric algebra $\G$.}  If so, we have found
a way, by Lemma~\ref{lma_equating_blades}, to algebraically relate them so that an analysis
by decomposition of $C\wedge e_3$
as $E\wedge e_3$ can move forward.\footnote{Think of this as replacing
$B$ with $\lambda C\wedge e_3$ in equations \eqref{equ_decompose_first} through
\eqref{equ_decompose_last}.}
Unfortunately, it is not hard to show
that neither of these is irreducible.  To see why, consider the following equation which
expresses the form of $p(x)$ for all points in the $x_3=0$ plane.
\begin{equation}\label{equ_form_of_points_in_plane}
p(x_1e_1 + x_2e_2) = e_0 + x_1e_1 + x_2e_2 + x_1x_2e_6 + x_1^2e_7 + x_2^2e_8
\end{equation}
Now notice that an upper-bound on the dimension of a vector space that can
be spanned by vectors of this form is clearly 6 as there are only 6 components
on the right-hand side of equation \eqref{equ_form_of_points_in_plane}.
But each of $E\wedge e_3$ and $C\wedge e_3$ are 2-blades, making their
duals blades of grade $10-2=8>6$.  It follows that there is no set of 8 points $\{x_i\}_{i=1}^8$
on the ellipse such that $\bigwedge_{i=1}^8 p(x_i)\neq 0$.

Not willing to give up just yet, we arrive at the following lemma.

\begin{lem}\label{lma_irreducible_form_exists}
For every $k$-blade $B\in\G$ with $\gh(B)$ non-empty, there exists a blade $B'\in\G$ of grade $k'\leq k$
such that $\gh(B)=\gh(B')$ and $B'$ is irreducible.
\end{lem}
\begin{proof}
If $B$ is irreducible, then let $k'=k$ and $B'=B$ and we're done.
If $B$ is not irreducible, then let $j$ be the largest possible integer
for which there exists a set of $j$ points $\{x_i\}_{i=1}^j\subseteq\gh(B)$
with $\bigwedge_{i=1}^j p(x_i)\neq 0$, (clearly $0<j<k$), and write
\begin{equation*}
B = B_0\wedge\bigwedge_{i=1}^j p(x_i)
\end{equation*}
for some blade $B_0$ of grade $k-j$.  Now realize that for any $x\in\gh(B)$,
if $x\in\gh(B_0)$, then $p(x)\wedge\bigwedge_{i=1}^j p(x_i)\neq 0$ and $x\not\in\{x_i\}_{i=1}^j$,
which is a contradiction.  Therefore, if $x\in\gh(B)$, then, letting $k'=j$ and $B'=\bigwedge_{i=1}^jp(x_i)$,
we have $x\in\gh(B')$.  Conversely, if $x\in\gh(B')$, then clearly $x\in\gh(B)$.  It follows that
$\gh(B)=\gh(B')$ and $B'$ is irreducible.
\end{proof}

Seeing that $B'$ is potentially a reduction in grade of the blade $B$, but one
in which the geometry represented by $B$ is certainly not sacrificed,
we will say that $B$ is reducible
in the case that $k'<k$.  In the case that $k'=k$, it is clear that $B$ is irreducible.

What we see now is that $(C\wedge e_3)I$ and $(E\wedge e_3)I$ are reducible blades,
and that finding an algebraic relation between them may be possible if either
one or both of them can be reduced.  Since finding an
irreducible canonical form of the ellipse $(E\wedge e_3)I$
does not seem in the least bit trivial, let us take a moment now to consider reducing the
intersection $(C\wedge e_3)I$ we wish to find.

To do this, we set out to find the largest irreducible factor of $(C\wedge e_3)I$.
Clearly, if $x\in\gd(C\wedge e_3)$, then $p(x)$ is a factor of $(C\wedge e_3)I$; so
the question of when a set of points produces a linearly independent set of
vectors naturally arises.  It is immediately clear that if the set of $j$ points $\{x_i\}_{i=1}^j$
is linearly independent, then so is the set of $j$ vectors $\{p(x_i)\}_{i=1}^j$, but the following
lemma helps us do a little better than this.
\begin{lem}\label{lma_non_co_planar}
If a given set of $j>2$ points $\{x_i\}_{i=1}^j$ are non-co-planar for a plane of
dimension $j-2$, then the set of vectors $\{p(x_i)\}_{i=1}^j$ is linearly independent.
\end{lem}
\begin{proof}
Proving the contrapositive of the lemma, let $\{\lambda_i\}_{i=1}^j$ be
a set of scalars in $\R$, not all zero, such that $0=\sum_{i=1}^j\lambda_i p(x_i)$.
It follows that $0=\sum_{i=1}^j\lambda_i(e_0+x_i)$ and therefore
$0=\sum_{i=1}^j\lambda_i$ and $0=\sum_{i=1}^j\lambda_i x_i$.
Now realize that if there exists an integer $a\in[1,j]$ such that $\lambda_a\neq 0$,
then there must exist an integer $b\in[1,j]-\{a\}$ such that $\lambda_b\neq 0$.
Without loss of generality, let $a=j$ so that $1\leq b\leq j-1$, and write
\begin{equation*}
0 = \sum_{i=1}^j\lambda_ix_i = \sum_{i=1}^{j-1}\lambda_ix_i - \left(\sum_{i=1}^{j-1}\lambda_i\right)x_j = -\sum_{i=1}^{j-1}\lambda_i(x_j-x_i),
\end{equation*}
which shows that the set of vectors $\{x_j-x_i\}_{i=1}^{j-1}$ is linearly dependent.
It now follows that the $(j-1)$-dimensional simplex determined by the points in $\{x_i\}_{i=1}^j$
has no $(j-1)$-dimensional hyper-volume.  That is,
\begin{equation*}
0 = \frac{1}{(j-1)!}\bigwedge_{i=1}^{j-1}(x_j-x_i).
\end{equation*}
But this can only be if the $j$ points are co-planar for a hyper-plane of dimension $j-2$.
\end{proof}
Lemma~\ref{lma_non_co_planar} is a good start, but there are certainly more conditions on $\{x_i\}_{i=1}^j$ to be found upon which
$\bigwedge_{i=1}^j p(x_i)\neq 0$.  The non-linearity of our function $p$ makes these conditions difficult to find, to say the least.
Nevertheless, Lemma~\ref{lma_non_co_planar} can help guide our initial choice of probing vectors in the na\"{i}ve blade factorization
algorithm;\footnote{A better blade factorization algorithm can been found in \cite{Dorst10}; but remember, we're not looking
for just any factorization; we're looking for one having a factor of the form \eqref{equ_point_factored_blade} of largest possible grade.
What further complicates the matter is that, in finding
our desired factorization, we need points on
the intersection we are in the process of trying to find.  Calculus methods may be in order.} and we, as
Lemma~\ref{lma_unique_grade_of_irreducible_form} below will show,
do not have to complete this factorization.  We need go only so far as to know
that we have found the irreducible factor we're trying to find.

Lemma~\ref{lma_unique_grade_of_irreducible_form} will depend upon
Lemma~\ref{lma_irreducibles_factor_as_irreducibles}, which we now give.

\begin{lem}\label{lma_irreducibles_factor_as_irreducibles}
If $B\in\G$ is an irreducible $k$-blade with $k>2$ and $B_0,B_1\in\G$ is any factorization
of $B$ in terms of two blades as $B=B_0\wedge B_1$, each having non-zero grade,
then both $B_0$ and $B_1$ are irreducible blades.
\end{lem}
\begin{proof}
Let $\{x_i\}_{i=1}^k\subseteq\gh(B)$ be a set of $k$ points on the geometry represented
by $B$ such that $\bigwedge_{i=1}^k p(x_i)\neq 0$.
Then, seeing that $B=B_0\wedge B_1$, it is clear that there must exist a partition of
$\{x_i\}_{i=1}^k$ into two non-empty sets, one of cardinality $j=\grade(B_0)$ with $0<j<k$, and
the other of cardinality $k-j=\grade(B_1)$, such that for all $x$ in the first partition, we
have $x\in\gh(B_0)$; and for all $x$ in the second partition, we have $x\in\gh(B_1)$.
Without loss of generality, let the first partition be $\{x_i\}_{i=1}^j$,
and the second $\{x_i\}_{i=j+1}^k$.  Then, since $\bigwedge_{i=1}^j p(x_i)$
and $\bigwedge_{i=j+1}^k p(x_i)$ are each non-zero, it follows, by
Lemma~\ref{lma_point_factored_blade},
that there exist scalars $\lambda_0,\lambda_1\in\R$ such that $B_0=\lambda_0\bigwedge_{i=1}^j p(x_i)$,
and $B_1=\lambda_1\bigwedge_{i=j+1}^k p(x_i)$.
\end{proof}

\begin{lem}\label{lma_unique_grade_of_irreducible_form}
Given any blade $B\in\G$, if $k$ points $\{x_i\}_{i=1}^k\subset\R^n$ can be found
such that $\gh(B)=\gh(R)$ with $R=\bigwedge_{i=1}^k p(x_i)$, then there is no
irreducible blade $B'$ of a grade greater than $k$, or less than $k$, such that $\gh(B)=\gh(B')$.
\end{lem}
\begin{proof}
Suppose, contrary to the lemma, that $B'\in\G$ is an irreducible blade of grade greater than $k$ such that $\gh(B)=\gh(B')$.
Letting $B_0\in\G$ be a blade such that $B'=R\wedge B_0$, we see that $B_0$ is irreducible
by Lemma~\ref{lma_irreducibles_factor_as_irreducibles}.  But if $\gh(R)=\gh(B)$, then
we must have $R\wedge B_0=0$, which is a contradiction, because $B'\neq 0$.

Suppose now, also contrary to the lemma, that $B'\in\G$ is an irreducible blade of grade less than $k$ such that $\gh(B)=\gh(B')$.
Letting $B_0\in\G$ be a blade such that $R=B'\wedge B_0$, we see that $B_0$ is irreducible
by Lemma~\ref{lma_irreducibles_factor_as_irreducibles}.  But if $\gh(B')=\gh(B)$, then
we must have $B'\wedge B_0=0$, which is a contradiction, because $R\neq 0$.
\end{proof}

Concerning irreducible blades, we see that while Lemma~\ref{lma_irreducible_form_exists} has dealt with
the question of existence, Lemma~\ref{lma_unique_grade_of_irreducible_form} has dealt with the question of uniqueness.  Lemmas \ref{lma_irreducible_form_exists}, \ref{lma_unique_grade_of_irreducible_form}
and \ref{lma_equating_blades}, all taken together, we can
say now that every blade representative of a non-empty geometry has an irreducible form that
is unique, up to scale.
Lemma~\ref{lma_irreducibles_factor_as_irreducibles} shows that irreducible blades always factor
as irreducible blades.

By Lemma~\ref{lma_unique_grade_of_irreducible_form}, we are justified in
claiming that, up to scale, the irreducible form of
our desired intersection $\gd(C\wedge e_3)$ is given by $\gh(R)$, where
\begin{equation}\label{equ_irreducible_form_of_our_intersection}
R = e_{01267} - e_{01268} + e_{12678}
\end{equation}
is a blade of grade 5 in our algebra.\footnote{Choosing any 5 points $\{x_i\}_{i=1}^5$ on
the unit circle in the plane for which $\bigwedge_{i=1}^5 p(x_i)\neq 0$, you'll find that $\bigwedge_{i=1}^5 p(x_i)$
is a scalar multiple of $R$ in equation \eqref{equ_irreducible_form_of_our_intersection}.  It is also not
hard to show that $\gh(R)=\gd(C\wedge e_3)$.}
Admittedly, knowledge of the intersection was used
to obtain this form; but, assuming that there is an algorithm for finding it
that does not depend upon such knowledge, let's move forward unashamed
and undiscouraged.

Having now fully reduced the 8-blade $(C\wedge e_3)I$ down to the 5-blade $R$, we know
that there must exist a 3-blade $R_0$ such that the equation
\begin{equation*}
(R_0\wedge R)I = E\wedge e_3
\end{equation*}
has real solutions in $a$, $b$, $h$ and $k$.
Indeed, by examination, it is not hard to see that if we let $R_0=e_{459}$, then this is the case.
The 2-blade $(R_0\wedge R)I$ may now be decomposed using equations \eqref{equ_decompose_first} through \eqref{equ_decompose_last}.
As the reader can check, we get the unit circle at origin.

\section{Closing Remarks}

To see any efficacy in this, admittedly, anticlimactic paper, we have to look at what
we've learned from our attempt at finding the model of geometry we had hoped
for in the introductory section, and then try to ask the right questions moving forward.
To the thoughtful reader, a few of these are given as follows.

\begin{enumerate}
\item Is there a condition on a vector $v\in\V$ such that if this condition
is satisfied, we have, for any blade
$B\in\G$, $\gh(B)=\gh(v\cdot B)$?  If $\gd(v)=\R^n$, it is
not hard to show that $\gh(B)\subseteq\gh(v\cdot B)$.

\item Is there an algorithm for fully reducing a given blade that does not
require knowledge of the intersection it represents?  Such an algorithm, if it
exists, must surely take into account how we have defined the function $p$
and any properties we can derive from it.  {\it Note that it may
be possible to factor an irreducible blade in terms of vectors where not all
of them are of the form $p(x)$ for some point $x\in\R^n$.}

\item Removing the need for such an algorithm altogether, is there a model
for quadric surfaces in which all blades are irreducible?  With the exception
of flat points, all blades of the conformal model are irreducible.

\item Even if we can always find the irreducible form $R$ of a desired intersection,
will the blade $R_0$, making $(R_0\wedge R)I$ a scalar multiple of our canonical
form, always be obvious by inspection?

\item Removing the need to find $R_0$, is it better to find and consider
only irreducible canonical forms?

\item If we had chosen an arbitrary plane,
instead of the $x_3=0$ plane, how might this have complicated our example?

\item Is there any way to deal with non-planar intersection in this model?
\end{enumerate}

Lastly, it must be said here that geometric algebra just may not be the right tool
for studying the intersections of algebraic surfaces.  For those readers possessing a savy
for algebraic geometry, the article \cite{Fulton83}, and similar publications by the same author,
give the modern approach to intersection theory.

\begin{thebibliography}{9}

\bibitem{Lasenby04}
A. Lasenby, J. Lasenby, R. Wareham, {\it A covariant approach
to geometry using geometric algebra}.  Technical report, Cambridge University
Engineering Department, 2004.

\bibitem{Dorst07}
L. Dorst, D. Fontijne and S. Mann, {\it Geometric algebra for computer
science}. Morgan Kaufmann, 2007.

\bibitem{Dorst10}
L. Dorst, D. Fontijne, {\it Efficient Algorithms for Factorization and Join
of Blades}.  Geometric Algebra Computing, E. Bayro-Corrochano, G Scheuermann, Eds.,
Springer, London, 2010, pp. 457-476.

\bibitem{Hestenes01}
D. Hestenes, {\it Old wine in new bottles: A new algebraic
framework for computational geometry}. Advances in Geometric
Algebra with Applications in Science and Engineering (2001), 1-14.

\bibitem{Miller87}
J. Miller, {\it Geometric approaches to nonplanar quadric surface
intersection curves}. ACM Transactions on Graphics Vol. 6, No. 4, pp.
274-307.

\bibitem{Parkin13}
S. Parkin, {\it Mother Minkowski Algebra Of Order $M$}.
Advances in Applied Clifford Algebras (2013).

\bibitem{Fulton83}
W. Fulton, {\it Introduction To Intersection Theory In Algebraic Geometry}.
Expository Lecture, CBMS Regional Conference, 1983.

\end{thebibliography}

\end{document}
