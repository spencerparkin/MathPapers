\documentclass[12pt]{article}

\usepackage{amsmath}
\usepackage{amssymb}
\usepackage{amsthm}
\usepackage{graphicx}
\usepackage{float}

\title{On The Multiplicative Inverse Of Multivectors\\With Respect To The Geometric Product}
\author{Spencer T. Parkin}

\newcommand{\G}{\mathbb{G}}
\newcommand{\V}{\mathbb{V}}
\newcommand{\R}{\mathbb{R}}
\newcommand{\B}{\mathbb{B}}
\newcommand{\nvao}{o}
\newcommand{\nvai}{\infty}

\newtheorem{theorem}{Theorem}[section]
\newtheorem{definition}{Definition}[section]
\newtheorem{corollary}{Corollary}[section]
\newtheorem{identity}{Identity}[section]
\newtheorem{lemma}{Lemma}[section]
\newtheorem{result}{Result}[section]

\begin{document}
\maketitle

We begin by extending the idea of linear independent amoung vectors to that of blades.
\begin{definition}\label{def_lin_indep_set}
For a given set $\{A_i\}_{i=1}^n$ of $n$ non-zero blades of various grades, we say that it is
a linearly independent set if there does not exist a linear combination $L$, given by
\begin{equation}
L = \sum_{i=1}^k \alpha_iA_i,
\end{equation}
of these blades, where $L=0$ and not all scalars in $\{\alpha_i\}_{i=1}^n$ are zero.
\end{definition}
It is clear that if no two blades in $\{A_i\}_{i=1}^n$ are of the same grade, then
$\{A_i\}_{i=1}^n$ must be a linearly independent set.  If this is not the case, however,
the set may not be linearly independent.  If a set of blades is not linearly independent,
we will call it linearly dependent.
\begin{definition}\label{def_irreducible_set}
For any given set of $n$ non-zero blades $\{A_i\}_{i=1}^n$, we say that it
is an irreducible set if there does not exist a set of $m$ non-zero blades $\{B_i\}_{i=1}^m$,
with $m<n$, such that
\begin{equation}
\sum_{i=1}^n A_i=\sum_{i=1}^m B_i.
\end{equation}
\end{definition}
A non-irreducible set will be referred to as reducible.  In the context
of Definition~\ref{def_irreducible_set}, we will refer to $\{B_i\}_{i=1}^m$
as a reduction of the set $\{A_i\}_{i=1}^n$.

We now find that the irreducible sets are the linearly independent sets.
\begin{lemma}
A given set of $n$ non-zero blades $\{A_i\}_{i=1}^n$ is irreducible if and only if
it is linearly independent.
\end{lemma}
\begin{proof}
Clearly the lemma goes through in the case $n=1$.  Therefore, we will consider now only the
cases $n>1$.  We prove the contrapositive in each direction.

Let $\{A_i\}_{i=1}^n$ be a linearly dependent set.  Then, without loss of generality,
we may write
\begin{equation}
A_n = \sum_{i=1}^{n-1}\alpha_i A_i.
\end{equation}
Then, for each blade in $\{B_i\}_{i=1}^{n-1}$, letting $B_i=(\alpha_i+1)A_i$, we have
\begin{equation}
\sum_{i=1}^n A_i = \sum_{i=1}^{n-1} B_i,
\end{equation}
showing that $\{A_i\}_{i=1}^n$ is a reducible set.

Now assume that $\{A_i\}_{i=1}^n$ is a reducible set.  Let $\{B_i\}_{i=1}^m$
be a linearly independent reduction of this set.  (To find such a set, take
any reduction of $\{A_i\}_{i=1}^n$ and combine any pair of linearly
dependent blades until the set is linearly independent.)  Considering now
$\{B_i\}_{i=1}^m$ to be a basis for a linear space, we must show
that $\{A_i\}_{i=1}^m$ also spans this space to establish its linear dependence.
Hmmm...
\end{proof}

For the discussion to follow, we will find it convenient to think of a
multivector $M$ as a sum
\begin{equation}
M = \sum_{i=1}^n A_i,
\end{equation}
where the set of blades $\{A_i\}_{i=1}^n$ is an irreducible set.
(Clearly a multivector can always be written in terms of such a set.)
Furthermore, we will overload the notation $M$ as denoting
both the multivector $M$ and the linear space spanned by
the set of blades $\{A_i\}_{i=1}^n$.  It will therefore
make sense to think of multivectors in $M$.

Consider now the multiplicative inverse of a multivector $M$ with
respect to the geometric product.  The geometric product not being
generally commutative, we will restrict our attention in this paper to the inverse $M^{-1}$
of $M$, such that
\begin{equation}\label{equ_m_m_inv}
1=MM^{-1}.
\end{equation}
Such an inverse may or may not exist.
If it does exists, it is clearly unique by the zero-product property of the geometric product.

\begin{lemma}\label{lem_inverse_form}
For any invertible multivector $M$, if it can be written in
terms of the blades in the irreducible set $\{A_i\}_{i=1}^n$, then its inverse $M^{-1}$ is a linear
combination of the blades in this set.
\end{lemma}
\begin{proof}
Given such a multivector $M$, it is natural to write $M^{-1}$ as
\begin{equation}
M^{-1} = M_0 + M_1,
\end{equation}
where $M_0\in M$ and $\mbox{span}(M_1)\cap\mbox{span}(M)=\emptyset$.
Put another way, for all multivectors $A\in M_0$, we have $A\in M$; and for
all multivectors $A\in M_1$, we have $A\not\in M_1$.
Clearly $M^{-1}$ takes on such
a form, because any multivector may be written in such a form with respect to $M$.
In the case of $M^{-1}$, however, we must show that $M_1=0$ and
that $\mbox{span}(M_0)=\mbox{span}(M)$.

To that end, notice that for all $B\in M_1$, it is clear that there
does not exist a blade $A\in M$ such that $\langle AB\rangle_0$ is a non-zero scalar.
It follows that $\langle MB\rangle_0=0$, and then that $\langle MM_1\rangle_0=0$.
It then follows that
\begin{equation}
1 = MM^{-1} = \langle MM^{-1}\rangle_0 = \langle MM_0\rangle_0 + \langle MM_1\rangle_0
\end{equation}
if and only if $\langle MM_0\rangle_0=1$ and $\langle MM_1\rangle_0=0$.
Now notice that there does not exist a multivector $A\in M_0$ and a multivector
$B\in M_1$ such that
\begin{equation}
MA+MB=0.
\end{equation}
It follows that for all integers $i>0$, we must have $\langle MM_0\rangle_i=0$
and $\langle MM_1\rangle_i=0$.  We can now say that
\begin{equation}
1 = MM^{-1} = MM_0 + MM_1
\end{equation}
if and only if $MM_0=1$ and $MM_1=0$, and therefore
$M_1=0$ by the zero product property.

What we want to show now is that $\mbox{span}(M_0)=\mbox{span}(M)$.
It is clear by definition that $\mbox{span}(M_0)\subseteq\mbox{span}(M)$.
\end{proof}

Knowing the form of which $M^{-1}$ takes on
in terms of $M$ brings us a long way towards formulating $M^{-1}$
in terms of $M$.  Writing $M=\sum_{i=1}^n A_i$ and then
$M^{-1}=\sum_{i=1}^n \alpha_i A_i^{-1}$, we may now expand
equation \eqref{equ_m_m_inv} to arrive at the following system of equations.
\begin{align}
1 &= \sum_{i=1}^n \alpha_i \\
0 &= \sum_{i=1}^n\sum_{j=i+1}^n\left(\alpha_iA_jA_i^{-1}+\alpha_jA_iA_j^{-1}\right).
\end{align}
Recall that for any non-zero blade $A_i$ that $A_i^{-1}$ and $A_i$ are scalar multiples
of one another.  We may therefore apply Lemma~\ref{lem_inverse_form} in writing $M^{-1}$
as $\sum_{i=1}^n \alpha_i A_i^{-1}$.

% does it somehow follow that left and right inverses are the same?
% i think that everything results in the same system of equations, so yeah, i guess so.

\end{document}