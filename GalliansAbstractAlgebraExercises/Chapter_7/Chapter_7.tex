\documentclass[12pt]{article}

\usepackage{amsmath}
\usepackage{amssymb}
\usepackage{amsthm}

\title{Chapter 7 Exercises\\Gallian's Book on Abstract Algebra}
\author{Spencer T. Parkin}

\newtheorem{theorem}{Theorem}[section]
\newtheorem{definition}{Definition}[section]
\newtheorem{corollary}{Corollary}[section]
\newtheorem{identity}{Identity}[section]
\newtheorem{lemma}{Lemma}[section]
\newtheorem{result}{Result}[section]

%\newcommand{\gcd}{\mbox{gcd}}
\newcommand{\lcm}{\mbox{lcm}}
\newcommand{\abs}{\mbox{abs}}
\newcommand{\Z}{\mathbb{Z}}
\newcommand{\R}{\mathbb{R}}
\newcommand{\G}{\mathbb{G}}
\newcommand{\stab}{\mbox{stab}}
\newcommand{\aut}{\mbox{Aut}}
\newcommand{\inn}{\mbox{Inn}}
\newcommand{\orb}{\mbox{orb}}

\begin{document}
\maketitle

\section*{Exercise 6}

Let $n$ be a positive integer.  Let $H=\{0,\pm n, \pm 2n, \pm 3n, \dots\}$.
Find all left cosets of $H$ in $Z$.  How many are there?

There are $n$ of them.  They are in $\{k+H\}_{k=0}^{n-1}$.

\section*{Exercise 8}

Suppose that $a$ has order 15.  Find all of the left cosets of $\langle a^5\rangle$ in $\langle a\rangle$.

Since $|\langle a^5\rangle|=3$, there are $15/3=5$ such cosets.  They are in $\{a^kH\}_{k=0}^4$.

\section*{Exercise 9}

Let $|a|=30$.  How many left cosets of $\langle a^4\rangle$ in $\langle a\rangle$ are there.  List them.
By Theorem 4.2, we have $\langle a^4\rangle=\langle a^2\rangle$, which has order $30/2=15$.
There are therefore $30/15=2$ left cosets, and they are $\langle a^4\rangle$ and $a\langle a^4\rangle$.

\section*{Exercise 10}

Let $a$ and $b$ be nonidentity elements of different orders in a group $G$ of
order 155.  Prove that the only subgroup of $G$ that contains $a$ and $b$ is $G$ itself.

Since $155=5\cdot 31$, it is clear that the only possible orders of non-identity elements
are 5, 31 and 155.
 If one of $a$ and $b$ has order 155, say $|a|=155$, then
clearly $a$ is not in any one of the proper and non-trivial sub-groups,
because 155 does not divide 5 or 31,
so the result follows in this case.
If $a$ and $b$ have orders less than 155, say $|a|=5$ and $|b|=31$,
then $a$ can be a subgroup of order 5, but not of 31, and $b$ can
be in a subgroup of order 31, but not of 5, because 5 does not divide 31, and
31 does not divide 5.  The result goes through in this case too.
We have now exhausted all cases.

\section*{Problem 12}

Let $C^*$ be the group of nonzero complex numbers under multiplication
and let $H=\{a+bi\in C^*|a^2+b^2=1\}$.  Give a geometric description
of the cosets of $H$.

Considering the polar form of complex numbers, it is easy to see that these
are all possible concentric circles in the plane, centered at origin.

\section*{Problem 16}

Recall that, for any integer $n$ greater than 1, $\phi(n)$ denotes the number of
integers less than $n$ and relatively prime to $n$.  Prove that if $a$ is any
integer relatively prime to $n$, then $a^{\phi(n)}\equiv 1\pmod n$.

Notice that $a\in U(n)$ and that $|U(n)|=\phi(n)$.  This result then
follows directly from Corollary 4 of Legrange's Theorem.

\section*{Problem 18}

Use Corollary 2 of Lagrange's Theorem to prove that the order of $U(n)$ is even
when $n>2$.

We must show that when $n>2$, that $U(n)$ always has an element of order 2.
Notice that $|n-1|=2$ for all $n>2$.

\section*{Problem 20}

Suppose $H$ and $K$ are subgroups of a group $G$.  If $|H|=12$ and $|K|=35$,
find $|H\cap K|$.

The subgroup $H\cap K$ being that of $H$ and $K$, the order of any element of
$H\cap K$ must divide 12 and 35; but these are
are relatively prime, so the order of all elements in $H\cap K$ is 1.
The subgroup $H\cap K$ must therefore be the trivial subgroup having
order 1.

\section*{problem 21}

Suppose that $G$ is an Abelian group with an odd number of elements.
Show that the product of all of the elements of $G$ is the identity.

Notice that no element is its own inverse by Lagrange's Theorem,
because 2 does not divide any odd number.  Then, since making
an element equivilant to its inverse produces an equivilance
relation on a group, it partitions the group, and
we can say that
\begin{equation*}
e = \prod_{i=1}^{|G|} a_ia_i^{-1}
\end{equation*}
is a product of all the elemnts of $G$.  Now notice that by the
Abelian property of $G$, any rearrangement of this product is the identity.

\section*{Problem 22}

Suppose that $G$ is a group with more than one element and $G$ has no
proper, nontrivial subgroups.  Prove that $|G|$ is prime.  (Do not assume
at the outset that $G$ is finite.)

This group is not the trivial group, so it has nonidentity elements.
Consider the cyclic subgroup generated by any non-identity element.
It is now clear that the group $G$ is cyclic, because the cyclic subgroup
just found cannot be proper.  Furthermore, this cyclic subgroup must
be of prime order, (and hence be finite), or else it would have nontrivial, proper subgroups
by Theorem 4.3.  (Notice that we did not need Lagrange's Theorem.)

\section*{Problem 23}

Let $|G|=15$.  If $G$ has only one subgroup of order 3 and only one of order 5,
prove that $G$ is cyclic.  Generalize to $|G|=pq$, where $p$ and $q$ are prime.

By Corollary 3 of Legrange's Theorem, groups of prime order are cyclic.
Therefore, $G$ must have an element $a\in G$ of order $p$,
and an element $b\in G$ of order $q$.  Now choose $c\in G$ such that $c\not\in\langle a\rangle$
and $c\not\in\langle b\rangle$.  (Such an element exists, because $G\neq\langle a\rangle\cup\langle b\rangle$ since
$|G|>|a|+|b|-1$.)
Clearly $c$ is not the identity element.
Further, $|c|$ can't be $p$, otherwise there would exist more than one cyclic
subgroup of order $p$.  The order of $c$ can't be $q$ for a similar reason.
Therefore, since $|c|$ must divide $|G|=pq$, it is clear that $|c|=|G|$,
and we have $G=\langle c\rangle$.

\section*{Problem 24}

Let $G$ be a group of order 25.  Prove that $G$ is cyclic or $g^5=e$ for all $g\in G$.

If $G$ is cyclic, we're done.  Suppose $G$ is not cyclic.  Then there does not
exist $g\in G$ such that $|g|=25$.  The only possible orders for $g$ are
then, by Legrange's Theorem, one and five.  The only element of order one
is the identity, and clearly $e^5=e$.  It follows that $g^5=e$ for all $g\in G$.

\section*{Exercise 29}

Let $G$ be a group of permutations of a set $S$.  Prove that the orbits of the
members of $S$ consitute a partition of $S$.

For any $a,b\in S$, let $a\sim b$ if and only if $a\in\orb_G(b)$.
To see that this forms an equivilance relation on the set $S$,
we begin by noting that for all $a\in S$, we have $a\in\orb_G(a)$,
giving us the reflexive property.  For all $a,b\in S$, if $a\sim b$,
then $a\in\orb_G(b)$ and therefore there exists $\phi\in G$ such
that $\phi(b)=a$.  Seeing that $\phi^{-1}(a)=b$ and $\phi^{-1}\in G$, it is clear
that $b\in\orb_g(a)$ and therefore $b\sim a$, giving us the symmetric property.
Lastly, for all $a,b,c\in S$, if $a\sim b$ and $b\sim c$, there exist $\phi_0,\phi_1\in G$
such that $\phi_0(b)=a$ and $\phi_1(c)=b$.  Then, letting $\phi=\phi_0\phi_1\in G$,
it is clear that $\phi(c)=a$ and therefore $a\in\orb_G(c)$.  It follows that $a\sim c$,
and we have the transitive property.

Now notice that the equivilance class containing $a$ is given by
\begin{equation*}
[a]=\{x\in S|x\sim a\}=\{x\in S|x\in\orb_G(a)\}=\orb_G(a),
\end{equation*}
showing that the orbits of elements in $S$ are the equivilance classes
that partition $S$ by Theorem 0.6.

\section*{Problem 32}

Prove that 3, 5, and 7 are the only three consecutive odd integers that are prime.

I'm not sure how to solve this problem in group-theory terms.  Nevertheless...

Notice that $\{2k+1\}_{k=0}^\infty$ is the set of odd numbers.
Now consider the subset of odd numbers $\{2(3k+1)+1\}_{k=0}^\infty$
and notice that every member is divisible by 3.  But now also notice that
one of every three consecutive odd numbers is in this subset.

\section*{Problem 33}

Let $G$ be a group of order $p^n$ where $p$ is prime.  Prove that the center of
$G$ cannot have order $p^{n-1}$.

I have to use theorems in chapters yet to come to get this done.  I'm not sure
how else to do it.

Suppose $|Z(G)|=p^{n-1}$.  Then, seeing that $Z(G)$ is always
a normal subgroup of $G$, the factor group $G/Z(G)$ has
order $p^n/p^{n-1}=p$, and is therefore cyclic.
Therefore, there exists an element $g\in G$ such that $G/Z(G)=\{g^kZ(G)|k\in\Z\}$.
Now since the cosets of $Z(G)$ cover $G$, if $a,b\in G$, there must exist integers $i,j\in\Z$
and elements $x,y\in Z(G)$ such that $a=g^ix$ and $b=g^jy$.
It is then easy to see that $ab=ba$, showing that $G$ is
Abelian, and therefore $Z(G)=G$, but this contradicts the supposition that $Z(G)$ is a proper
subgroup of $G$.  It follows that $|Z(G)|\neq p^{n-1}$.

Now peaking at the back of the book, I see that a much easier
proof exists.

\section*{Problem 36}

Let $G$ be a finite Abelian group and let $n$ be a positive integer that is relatively prime to $|G|$.
Show taht the mapping $a\to a^n$ is an automorphism of $G$.

Letting $\phi(x)=x^n$ be defined on $G$, it is clear that for all $x,y\in G$,
we have $\phi(xy)=(xy)^n=x^ny^n=\phi(x)\phi(y)$ by the Abelian property
of $G$, showing that $\phi$ is operation preserving.

Let $x,y\in G$ again and suppose that $\phi(x)=\phi(y)$.
Then $x^n=y^n\implies e=x^n(y^n)^{-1}=(xy^{-1})^n$
by the Abelian property of $G$.  Now since $(xy^{-1})^n=e$,
we see that $|xy^{-1}|$ divides $n$, (by Corollary 2 of Theorem 4.1), but it also divides $|G|$, (by
Corollary 2 of Theorem 7.1.)
Then since $n$ and $|G|$ are coprime, it follows that $|xy^{-1}|=1$,
and therefore, $xy^{-1}=e\implies x=y$, and we see that
$\phi$ is one-to-one.  The function $\phi$ must then also be onto,
because it is one-to-one and maps from one finite set to another
of the same cardinality.

\section*{Problem 38}

Let $G$ be a group of order $pqr$, where $p$, $q$ and $r$ are distinct primes.
If $H$ and $K$ are subgroups of $G$ with $|H|=pq$ and $|K|=qr$, prove
that $|H\cap K|=q$.

We know that $H\cap K$ is a subgroup of $G$.  Let us first show
that $H\cap K$ is not the trivial subgroup of $G$.  Supposing this
to be the case, we have...I don't know.

Let $g\in H\cap K$ be a non-identity element.
Also being an element of $H$, we must have $|g|$ dividing $pq$.
Being an element of $K$, we must have $|g|$ dividing $qr$.
It follows that since $|g|>1$, we must have $|g|=q$.
Then, since all non-identity elements of $H\cap K$ have order $q$,
it follows that $H\cap K$ is cyclic of order $q$.

\section*{Problem 45}

If $G$ is a finite group with fewer than 100 elements and $G$ has subgroups
of orders 10 and 25, what is the order of $G$?

The smallest possible order of $G$ is $\lcm(10,25)=50$.  Any larger
common multiple of 10 and 25 is at least 100.  So $|G|=50$.

\end{document}