\documentclass[12pt]{article}

\usepackage{amsmath}
\usepackage{amssymb}
\usepackage{amsthm}

\title{Section 2.8 Exercises\\Hertein's Topics In Algebra}
\author{Spencer T. Parkin}

\newtheorem{theorem}{Theorem}[section]
\newtheorem{definition}{Definition}[section]
\newtheorem{corollary}{Corollary}[section]
\newtheorem{identity}{Identity}[section]
\newtheorem{lemma}{Lemma}[section]
\newtheorem{result}{Result}[section]

%\newcommand{\gcd}{\mbox{gcd}}
\newcommand{\lcm}{\mbox{lcm}}
\newcommand{\Z}{\mathbb{Z}}

\begin{document}
\maketitle

\section*{Problem 16}

Let $\phi(n)$ be the Euler $\phi$-function.  If $a>1$ is an integer, prove that
$n|\phi(a^n-1)$.

Consider the group $U_m$ with $m=2^n-1$.
Since $|U_m|=\phi(m)$, if we can exhibit an element of $U_m$
with order $n$, then the result goes through by Lagrange's Theorem.
Notice that
\begin{equation*}
(a^{n-1})a + (-1)(a^n-1) = 1.
\end{equation*}
This shows that $\gcd(a,a^n-1)=1$; and therefore, $a\in U_m$.
Then clearly, we have
\begin{equation*}
a^n\equiv 1\pmod{m},
\end{equation*}
so $|a|$ divides $n$.  But since $a^k-1<a^n-1=m$ for all $0\leq k<n$,
we must have $|a|=n$.  Now by Lagrange's Theorem, the order of
the cyclic subgroup generated by $a$, which is $n$, must divide $\phi(m)$.

\end{document}