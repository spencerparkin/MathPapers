\documentclass{birkjour}

\usepackage{amsmath}
\usepackage{amssymb}
\usepackage{amsthm}
\usepackage{graphicx}
\usepackage{float}
\usepackage{url}

\newtheorem{thm}{Theorem}[section]
\newtheorem{cor}[thm]{Corollary}
\newtheorem{lem}[thm]{Lemma}
\newtheorem{prop}[thm]{Proposition}
\theoremstyle{definition}
\newtheorem{defn}[thm]{Definition}
\theoremstyle{remark}
\newtheorem{rem}[thm]{Remark}
\newtheorem*{ex}{Example}
\numberwithin{equation}{section}

\newcommand{\G}{\mathbb{G}}
\newcommand{\V}{\mathbb{V}}
\newcommand{\R}{\mathbb{R}}
\newcommand{\Z}{\mathbb{Z}}
\newcommand{\Poly}{\mathbb{P}}
\newcommand{\Gi}{\dot{G}}
\newcommand{\Go}{\hat{G}}

\begin{document}

\title{A Model Of Algebraic Sets\\Using\\Geometric Algebra}

\author{Spencer T. Parkin}
\email{spencer.parkin@gmail.com}

\numberwithin{equation}{section}

\subjclass{Primary 14J70; Secondary 14J29}

\keywords{Quadric Surface, Quartic Surface, Geometric Algebra, Quadric Model, Conformal Model}

%\dedicatory{To Melinda and Naomi}

\begin{abstract}
Blah.
\end{abstract}

\maketitle

\section{Introduction And Motivation}

Letting $\R^n$ denote an $n$-dimensional Euclidean space, we are
going to let $\Poly^n$ denote the set of all polynomials $f:\R^n\to\R$
of any degree.  Then, letting $P\subseteq\Poly^n$ be any set of such
polynomials, we define the zero set $Z$ of $P$, denoted $Z(P)$,
as the set given by
\begin{equation}\label{equ_zero_set_of_polys}
Z(P) = \{x\in\R^n|\mbox{$f(x)=0$ for all $f\in P$}\}.
\end{equation}
Every subset $S$ of $\R^n$ for which there exists a set $P\subseteq\Poly^n$
such that $S=Z(P)$ is what we refer to as an algebraic set.  It is well known
that for any algebraic set $S\subseteq\R^n$, there is always such a subset $P$
of $\Poly^n$ of finite cardinality.

Given the definition in equation \eqref{equ_zero_set_of_polys}, it is easy to show that
the subsets $S$ of $\R^n$ that are the geometries of
CGA, and other similar models of geometry based upon geometric algebra, are simply algebraic sets.
The goal of this paper is to show that there exists a model of geometry, based upon geometric algebra,
where every possible algebraic set has a representative in the form of an element of that geometric algebra.
A desire to come up with such a model of geometric algebra is motivated by the admittedly fanciful
dream of the German mathematician Leibniz, referred to in \cite{} and claimed to
have already been realized in \cite{}.  In any case, it would seem that a generalization
of CGA or any CGA-like model to one that hosts the set of all algebraic sets in $\R^n$ would bring
us closer to such a goal.  Work to this end has already been done in \cite{Parkin13a,Parkin13b} which
has brought us, the reader and writer, to the present paper.

\section{Blades As Algebraic Sets}

We begin with an examination of $\Poly^n$ as a linear space, observing that
it is of countably infinite dimension.  A set of basis vectors for this space may be taken
as the set of all unit-monomials in anywhere from 1 to $n$ of the $n$ variable components
of an arbitrary point in $\R^n$.\footnote{The
number $c_k$ of such monomials homogeneous of a degree $k$ is given by
\begin{equation*}
c_k=\sum_{i=1}^n\binom{n}{i}p(k,i),
\end{equation*}
where $p(k,i)$ is the number of partitions of size $i$ of the integer $k$.  The
combinatorics of the matter are not important, but they are pointed out here as an added
measure of the reality of the set of monomials we are talking about.}
Considering now any mapping from the set $\Z^+$ of
positive integers to this said set $\{g_i\}_{i=1}^\infty$ of unit-monomials,\footnote{For example,
if $n=2$, then we might take $g_1(x,y)=1$, $g_2(x,y)=x$, $g_3(x,y)=y$, $g_3(x,y)=x^2$,
$g_4(x,y)=xy$, $g_5(x,y)=y^2$, and so on.}
and letting $\V^\infty$ denote a simple Euclidean vector space of countably infinite
dimension, if we define the function $p:\R^n\to\V^\infty$ as
\begin{equation}\label{equ_form_vector}
p(x) = \sum_{i=1}^\infty g_i(x)e_i,
\end{equation}
where $\{e_i\}_{i=1}^\infty$ is any orthonormal basis for $\V^\infty$, then for
any polynomial $f\in\Poly^n$, there must exist a unique\footnote{Let $v_1,v_2\in\V^\infty$
be two such vectors.  Then $0=f(x)-f(x)=p(x)\cdot (v_1-v_2)$.  Now notice that there
does not exist $x\in\R^n$ such that $p(x)=0$.  It follows that $v_1=v_2$.}
vector $v\in\V^\infty$ such that
\begin{equation}\label{equ_form_coef_split}
f(x) = p(x)\cdot v.
\end{equation}
It now follows that for any subset $P$ of $\Poly^n$, there exists a blade $B\in\G(\V^\infty)$,
such that
\begin{equation}
Z(P) = \{x\in\R^n|p(x)\cdot B=0\}.
\end{equation}
Clearly, we need only consider such finite sets $P=\{f_i\}_{i=1}^k$ that are linearly independent
sets of $k$ polynomials.  Then, for each polynomial $f_i$, there is an associated vector $v_i$ by
equation \eqref{equ_form_coef_split}, and the set $\{v_i\}_{i=1}^k$ must clearly be a linearly independent
set of $k$ vectors.
We may then take $B$ to be the $k$-blade
\begin{equation}
B=\bigwedge_{i=1}^k v_i,
\end{equation}
seeing that the equation $p(x)\cdot B=0$ becomes
\begin{equation}\label{equ_intersection_of_geometries}
0 = -\sum_{i=1}^k (-1)^i(p(x)\cdot v_i)B_i,
\end{equation}
where $B_i$ denotes the product $B$ with $v_i$ removed.  Realize
that $\{B_i\}_{i=1}^k$ is a linearly independent set of $(k-1)$-blades,
and therefore $p(x)\cdot B=0$ if and only if $p(x)\cdot v_i=0$ for all $i\in\Z_k$.
For convenience, we will let $\Z_k$ denote the set of $k$ integers in $[1,k]\cap\Z^+$.

Already we have fulfilled the promise of the introductory section of this paper,
but we will continue now with one further development that is motivated by a desire
to pair our current ability to intersect geometries with an ability to take their union.

We start by letting $\{\V_i^\infty\}_{i=1}^\infty$ be a countably infinite set of
vector spaces isomorphic to $\V^\infty$.  With the exception of the zero vector,
we consider these vector spaces as pair-wise disjoint, so that for any pair of non-zero
vectors $v_i\in\V_i^\infty$ and $v_j\in\V_j^\infty$ with $i\neq j$, we have $v_i\cdot v_j=0$.
Let $\V$ simply denote the vector space spanned by the set of vectors $\cup_{i\in\Z^+}\{e_{ij}\}_{j=1}^\infty$,
where for each $i\in\Z^+$, the set $\{e_{ij}\}_{j=1}^\infty$ is an orthonormal basis for
the vector space $\V_i^\infty$.
It is clear that the dimension of $\V$,
like that of $\V^\infty$ above, is countably infinite as the union of countably many countable
sets is countable.\footnote{This may come as some comfort to the reader. If you already thought that
the dimension of $\V^\infty$ was ridiculously huge, the dimension of the vector space $\V$ is no bigger.
In practice, of course, a computer
program, for example, would work in a finite-dimensional vector space, thereby restricting
the number of possible geometries to those that are zero sets of polynomials of a specific form.
This is exactly what's going on in CGA, with the additional modification of altering the signature
of the geometric algebra to accommodate the conformal transformations.  Non-Euclidean signatures
are not considered in this paper.}

\begin{defn}[The $q$ Function]\label{def_q_func}
Given a blade $B\in\G(\V)$, the function $q$ is a mapping from
the set of all blades in $\G(\V)$ to $\Z^+$ so that $i\in q(B)$ if and only if
there exists a vector $v\in\V_i^\infty$ such that $v\wedge B=0$.
\end{defn}

With Definition~\ref{def_q_func} in place, we may now proceed with the following two definitions.

\begin{defn}[The Algebraic Set $\Gi$]\label{def_gi}
Given a blade $B\in\G(\V)$, the set $\Gi$ of $B$, denoted $\Gi(B)$, is the set
\begin{equation}
\Gi(B) = \left\{x\in\R^n\left|\bigwedge_{i\in q(B)} p_i(x)\cdot B=0\right.\right\},
\end{equation}
where for each $i\in\Z^+$, we define $p_i$ similar to equation \eqref{equ_form_vector} as
\begin{equation}
p_i(x) = \sum_{j=1}^\infty g_j(x)e_{ij}.
\end{equation}
\end{defn}
For any product of the form $\prod_{i\in S}$ or $\bigwedge_{i\in S}$, where $S$ is
a set of positive integers, we take the terms in the product to be in ascending order
with respect to the index $i$.

\begin{defn}[The Algebraic Set $\Go$]\label{def_go}
Given a blade $B\in\G(\V)$, the set $\Go$ of $B$, denoted $\Go(B)$, is the set
\begin{equation}
\Go(B) = \left\{x\in\R^n\left|\bigwedge_{i\in q(B)}p_i(x)\wedge B=0\right.\right\}.
\end{equation}
\end{defn}

We now proceed to investigate the consequences of Definition~\ref{def_gi}
and that of Definition~\ref{def_go}.

\section{The Algebraic Set $\Gi$}

As promised, there now exist conditions under which we are able to take the
union of any two geometries represented by blades in $\G(\V)$.  The result is
as follows.

\begin{lem}[The Union Of Geometries]
For any two blades $A,B\in\G(\V)$ with the property that $q(A)\cap q(B)$ is empty,
we have
\begin{equation}
\Gi(A)\cup\Gi(B)=\Gi(A\wedge B).
\end{equation}
\end{lem}
\begin{proof}
Without loss of generality,
we may write $C=A\wedge B$ as $C=\bigwedge_{i=1}^k C_k$, where for
each $i\in\Z_k$, we have $C_i\in\G(V_i^\infty)$.
(Notice that $q(C)=\Z_k$.)  It is now clear that
\begin{align}
\bigwedge_{i\in\Z_k} p_i(x)\cdot C &= \pm\bigwedge_{i\in\Z_{k-1}} p_i(x)\cdot
(p_k(x)\cdot C_k)\wedge\bigwedge_{i\in\Z_{k-1}} C_i \\
 &= \pm\bigwedge_{i\in\Z_k}p_i(x)\cdot C_i.\label{equ_union_product}
\end{align}
(Notice that the outer product in \eqref{equ_union_product} becomes
a scalar product in the case that every $C_i$ is a vector.)
It is now clear that $\Gi(C)$ is indeed the union of $\Gi(A)$ and $\Gi(B)$.
\end{proof}
It is not hard to show that for any blade $B\in\G(\V)$, there exists
a rotor $R$, such that $B'$, given by $B'=RBR^{-1}$, has the property
$\Gi(B')=\Gi(B)$ while $q(B')$ is mapped to any other set of integers
of size $|q(B)|$.  This fact can be used to adjust any given pair
of blades $A,B\in\G(\V)$, if needed, so that $q(A)\cap q(B)$ is empty.
Admittedly, the need for any such adjustment prior to a union operation
seems to detract from our dream of an algebra where the geometric elements
would combine effortlessly in products that perform desired geometric operations.
Unfortunately, things don't get any better as the next lemma shows.

\begin{lem}[The Intersection Of Geometries]
For any two blades $A,B\in\G(\V)$ such that $q(A)=q(B)$
with $|q(A)|=|q(B)|=1$, we have
\begin{equation}
\Gi(A)\cap\Gi(B)=\Gi(A\wedge B).
\end{equation}
\end{lem}
\begin{proof}
Revisit the conversation of this paper surrounding equation \eqref{equ_intersection_of_geometries}.
\end{proof}
In this case, a simple rotor adjustment to one of $A$ and $B$ will only help
in the case that $|q(A)|=|q(B)|=1$.  If either one of $|q(A)|$ or $|q(B)|$
is greater than one, however, more adjustments are needed before an intersection can
be taken.  The following equation illustrates why.
\begin{equation}\label{equ_intersection_expansion}
\bigcup_{i\in q(A)}\Gi(A_i)\cap\bigcup_{i\in q(B)}\Gi(B_i) =
\bigcup_{\substack{i\in q(A)\\j\in q(B)}} \Gi(A_i)\cap \Gi(B_j),
\end{equation}
Here we have considered $A$ as the blade $\bigwedge_{i\in q(A)}A_i$, where
for each $i\in q(A)$, we have $A_i\in\G(\V_i^\infty)$.  We have similiarly
considered $B$ in terms of the blades $\{B_i\}_{i\in q(B)}$.  It is now
easy to see by the right-hand side of equation \eqref{equ_intersection_expansion}
that there exists a blade $C$
with the property that $\Gi(C)=\Gi(A)\cap\Gi(B)$, but we do not necessarily
have $\mbox{grade}(C)=\mbox{grade}(A\wedge B)$, showing that a rotor
adjustment, which is grade preserving, to one or both of $A$ and $B$ cannot
be of general help.  We need the factorizations of $A$ and $B$ to formulate $C$.



% What may be helpful is a lemma showing that for any blade B,
% there exists a blade B' such that $Gi(B)=G(B')$ with q(B')=1.

% examing Gi(A ^ B) for general A,B and given boolean exprsesion (using set unions/intersections)
% do same for Go(A ^ B).

\section{The Algebraic Set $\Go$}

\section{Relating $\Gi$ with $\Go$}

% for every blade B with q(B)>1, does there exist a blade A with q(A)=1
% where G(B)=G(A)?



% Possible citations:
% http://mathdl.maa.org/images/upload_library/22/Ford/DesmondFearnleySander.pdf
% Invariant algebras and geometric reasoning, Hongbo Li
% http://www.xtec.cat/~rgonzal1/proceedings.pdf

\bibliographystyle{amsplain}
\bibliography{Parkin_AModelOfAlgebraicSetsUsingGA}

\end{document}