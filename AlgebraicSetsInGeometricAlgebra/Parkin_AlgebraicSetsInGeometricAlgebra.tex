\documentclass{birkjour}

%\usepackage{tikz}
%\usepackage{graphicx}
%\usepackage{hyperref}
%\usepackage{algpseudocode}
%\usepackage{graphicx}

\newtheorem{thm}{Theorem}[section]
\newtheorem{cor}[thm]{Corollary}
\newtheorem{lem}[thm]{Lemma}
\newtheorem{prop}[thm]{Proposition}
\theoremstyle{definition}
\newtheorem{defn}[thm]{Definition}
\theoremstyle{remark}
\newtheorem{rem}[thm]{Remark}
\newtheorem*{ex}{Example}
\numberwithin{equation}{section}

\newcommand{\C}{\mathbb{C}}
\newcommand{\F}{\mathbb{F}}
\newcommand{\R}{\mathbb{R}}
\newcommand{\B}{\mathbb{B}}
\newcommand{\G}{\mathbb{G}}
\newcommand{\V}{\mathbb{V}}
\newcommand{\gd}{\dot{g}}
\newcommand{\gh}{\hat{g}}
\newcommand{\Gd}{\dot{G}}
\newcommand{\Gh}{\hat{G}}
\newcommand{\nvai}{\infty}
\newcommand{\nvao}{o}
\newcommand{\grade}{\mbox{grade}}

%\received{}\accepted{}

\begin{document}

\title{Algebraic Sets In Geometric Algebra}

\author{Spencer T. Parkin}
\email{spencerparkin@outlook.com}

%\subjclass{Primary 14J70; Secondary 14J29}

\dedicatory{To my dear wife Melinda.}

\begin{abstract}
Abstract...
\end{abstract}

\keywords{Key words...}

\maketitle

\section{Introduction}

% Note anywhere that geometric sets can be represented using spades/versors too?

\section{The Intersection Of Two Circles In The Plane}

In this section we solve the same problem twice; once with the conformal model, and another using
a more general model.

\subsection{Using The Conformal Model}

\subsection{Using A More General Model}

Let $\F$ denote an algebraicly closed field, and $\V^2(\F)$ a 2-dimensional, euclidean vector-space with
scalars taken from $\F$.  Letting $s_x,s_y$ be a pair of orthonormal vectors
taken from $\F^{2}$, and therefore a basis generating $\V^2(\F)$, we will define, for all $v\in\V^2(\F)$, the notation
$v_x=v\cdot s_x$ and $v_y=v\cdot s_y$.  We now let $\V^6(\F)$ denote a 6-dimensional, euclidean vector space
generated by the set of orthonormal basis vectors
\begin{equation*}
\{e,e_x,e_y,e_{xy},e_{xx},e_{yy}\},
\end{equation*}
and we define the mapping $p:\V^2(\F)\to\V^6(\F)$ as
\begin{equation*}
p(v) = e + v_xe_x + v_ye_y + v_xv_ye_{xy} + v_x^2e_{xx} + v_y^2e_{yy}.
\end{equation*}
For convenience, we'll set $s_x=e_x$ and $s_y=e_y$ so that $\V^2(\F)$ is a sub-space of $\V^6(\F)$.
Doing so, equation \eqref{} becomes
\begin{equation*}
p(v) = e + v + v_xv_ye_{xy} + v_x^2e_{xx} + v_y^2e_{yy}.
\end{equation*}

\begin{thebibliography}{9}

\end{thebibliography}

\end{document}