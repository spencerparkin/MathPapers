\documentclass[12pt]{article}

\usepackage{amsmath}
\usepackage{amssymb}
\usepackage{amsthm}

\title{Section 2.11 Exercises\\Herstein's Topics In Algebra}
\author{Spencer T. Parkin}

\newtheorem{theorem}{Theorem}[section]
\newtheorem{definition}{Definition}[section]
\newtheorem{corollary}{Corollary}[section]
\newtheorem{identity}{Identity}[section]
\newtheorem{lemma}{Lemma}[section]
\newtheorem{result}{Result}[section]

%\newcommand{\gcd}{\mbox{gcd}}
\newcommand{\lcm}{\mbox{lcm}}
\newcommand{\Z}{\mathbb{Z}}
\newcommand{\cl}{\mbox{Cl}}

\begin{document}
\maketitle

\section*{Some Thoughts}

It is interesting to observe that for any subgroup $H$ of a group $G$, and any element $g\in G$,
that $g^{-1}Hg$ is also a subgroup of $G$.  If $H$ is not normal in $G$, then there must exist
$g\in G$ such that $g^{-1}Hg$ is some subgroup of $G$ other than $H$.

For subgroups $A$ and $B$ of a group $G$, say that $A$ is conjugate to $B$, and write this as $A\sim B$,
if and only if there exists an element $g\in G$ such that $g^{-1}Ag=B$.  Does this define an equivalence
relation on the set $S$ of all subgroups of $G$?  Clearly, $A\sim A$ as $e^{-1}Ae=A$.
And if $g^{-1}Ag=B$, we must have $(g')^{-1}Bg'=A$, where $g'=g^{-1}$; proving $B\sim A$.
Lastly, $a^{-1}Aa=B$ and $b^{-1}Bb=C$ implies that
\begin{equation*}
C=b^{-1}a^{-1}Aab=(ab)^{-1}Aab,
\end{equation*}
showing that $A\sim C$.
It follows now from what we know about equivalence relations that
\begin{equation*}
|S|=\sum|\cl(A)|,
\end{equation*}
where here, the sum is taken over all equivalence class of $S$, and therefore,
each $A$ is just one of the possible representatives of each such class.

Let's consider for a moment a subgroup $A$ of $G$ for which $|\cl(A)|=1$.
It is clear that if $A$ is normal in $G$, then $|\cl(A)|=1$.  What about the converse?
If $|\cl(A)|=1$, then there does not exist an element $g\in G$ such that $g^{-1}Ag$
is some subgroup of $G$ other than $A$.  It follows, then, that $g^{-1}Ag=A$ for all
$g\in G$, and therefore, $A$ is normal in $G$.  We can now say that if $N$ is the
number of subgroups normal in $G$, then
\begin{equation*}
|S|=N+\sum|\cl(A)|,
\end{equation*}
where here, each $A$ is not normal in $G$.  We return to this equation later.

Let's now consider, for any subgroup $A$ of $G$, the normalizer of $A$; namely,
\begin{equation*}
N(A) = \{g\in G|gAg^{-1}=A\}.
\end{equation*}
(This is the largest subgroup of $G$ in which $A$ is normal.  See section 2.6, problem 10.)
Notice that its right cosets take the form
\begin{equation*}
N(A)a = \{g\in G|(ga^{-1})A(ga^{-1})^{-1}=A\}=\{g\in G|g(a^{-1}Aa)g^{-1}=A\}.
\end{equation*}
This makes it clear that the number of such cosets is precisely the number of conjugates of $A$.
We can now say that
\begin{equation*}
|\cl(A)|=\frac{|G|}{|N(A)|}.
\end{equation*}

Now suppose $G$ is a group of prime power order.  Specifically, $|G|=p^n$.
We then have
\begin{equation*}
|S| = \sum\frac{|G|}{|N(A)|}=\sum\frac{p^n}{p^{n_A}}=  N+\sum_{n_A<n}\frac{p^n}{p^{n_A}},
\end{equation*}
where for each arbitrarily chosen representative $A$ of each conjugacy class, $p^{n_A}=|N(A)|$.
(Notice that if $A$ is not normal in $G$, then $|N(A)|<|G|$.)  Interestingly, this shows that
\begin{equation*}
|S|\equiv N\pmod{p}.
\end{equation*}

Looking ahead to Lemma 2.12.6, let's consider the number of $p$-Sylow subgroups of $G$.
If $P$ is a $p$-Sylow subgroup of $G$, then by Sylow's Second Theorem, $|Cl(P)|$ accounts
for all $p$-Sylow subgroups in $G$.  It follows immediately that the number of such
groups in $G$ is $|G|/|N(P)|$, where $P$ is any such group.

\section*{Problem 11}

Using Theorem 2.11.2 as a tool, prove that if $|G|=p^n$, $p$ a prime number,
then $G$ has a subgroup of order $p^{\alpha}$ for all $0\leq\alpha\leq n$.

We proceed by strong induction on $n$.  The cases $n=0$ and $n=1$ are trivial.  Assuming all
cases $n-1, n-2, \dots, 1, 0$, we must prove case $n\geq 2$.
For our proof to work, we need only find any non-trivial, normal subgroup $N$ of $G$.
If $G$ is non-abelian, we may let $N=Z(G)$ by Theorem 2.11.2.  If $G$ is abelian,
then...ugh, think about it.  In any case, let $N$ be a non-trivial, normal subgroup of $G$
of order $p^m$ with $0<m<n$.  It then follows by our inductive hypothesis,
that $G$ has subgroups of orders $p^i$ for $0\leq i\leq m$.
Now consider the factor group $G/H$.
By a problem in Gallian's book (cite it here), for every subgroup $K$ of $G/N$,
there exists a subgroup $H$ of $G$ containing $N$ such that $K=H/N$.
And now since $N$ is non-trivial, we know, again by our inductive hypothesis,
that there exist subgroups of $K$ of every possible order.  These
are the orders $p^i$ with $0\leq i\leq n-m$.  It now follows that there
exist subgroups $H$ of $G$ of orders $p^i$ with $m\leq i\leq n$.
And this completes the proof!

\section*{Problem 12}

If $|G|=p^n$, $p$ a prime number, prove that there exist subgroups
$N_i$, $i=0,1,\dots,r$ (for some $r$) such that $G=N_0\supset N_1\supset N_2\supset\dots\supset N_r=\{e\}$
where $N_i$ is a normal subgroup of $N_{i-1}$ and where $N_{i-1}/N_i$ is abelian.

By problems 11 and 14, $G$ has a normal subgroup $H$ of order $p^{n-1}$.  Now since $|G/H|=|G|/|H|=p^n/p^{n-1}=p$,
we see that $G/H$ must be a cyclic group, and therefore abelian.
Now, of course, we can apply this same reasoning to $H$ in finding a normal subgroup $K$ of $H$ of order $p^{n-2}$,
and so on.  Since $G$ is of finite order, this nesting of subgroups must terminate at $\{e\}$.

I believe a group $G$, not necessarily of prime order, but having the above properties otherwise,
is considered to be a solvable group.  It's interesting to consider the solvability of any group.
By Caylay's theorem, any group $G$ is isomorphic to a subgroup of $A(S)$ for an appropriate set $S$.
Now if $|S|=n$, let $\emptyset=S_0\subset S_1\subset\dots\subset S_n=S$ be a sequence of $n$ properly nested
subsets of $S$, and define $H_i=\{\phi\in G|\mbox{for all $x\in S_i$, $\phi(x)=x$}\}$.  It is not hard to see
that each $H_i$ is a normal subgroup of $G$.
Further, for any $0\leq i<j\leq n$, notice that $H_j\leq H_i$, and $H_j$ is normal in $H_i$.  So what's to keep
any group from being solvable?  The only remaining criteria would appear to be the requirement that each $H_j/H_i$ be abelian.
In this general situation, I'm not sure what we can say, if anything, about how abelian the fractor group $H_j/H_i$ is.
A measure of that has something to do with its commutator subgroup, I think.


\section*{Problem 13}

Erf!

\section*{Problem 14}

Prove that any subgroup of order $p^{n-1}$ in a group $G$ of order $p^n$, $p$ a prime number, is normal in $G$.

Let $H$ be a subgroup of $G$ of order $p^{n-1}$.  Clearly the only choice of orders
for the subgroup $N(H)$ is $p^{n-1}$ or $p^n$, but because $N(H)\cap(G-H)$ is non-empty
by Problem 13, we must have $N(H)=G$.  It follows that $H$ is normal in $G$.

\end{document}