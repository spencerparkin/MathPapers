\documentclass[12pt]{article}

\usepackage{amsmath}
\usepackage{amssymb}
\usepackage{amsthm}

\title{Section 2.7 Exercises\\Hertein's Topics In Algebra}
\author{Spencer T. Parkin}

\newtheorem{theorem}{Theorem}[section]
\newtheorem{definition}{Definition}[section]
\newtheorem{corollary}{Corollary}[section]
\newtheorem{identity}{Identity}[section]
\newtheorem{lemma}{Lemma}[section]
\newtheorem{result}{Result}[section]

%\newcommand{\gcd}{\mbox{gcd}}
\newcommand{\lcm}{\mbox{lcm}}
\newcommand{\Z}{\mathbb{Z}}
\newcommand{\R}{\mathbb{R}}
\newcommand{\C}{\mathbb{C}}

\begin{document}
\maketitle

\section*{Problem 17}

Let $G$ be the group of real numbers under addition and let $N$ be the
subgroup of $G$ consisting of all the integers.  Prove that $G/N$ is
isomorphic to the group of all complex numbers of absolute vlaue 1
under multiplication.

Let $\phi:\R/\Z\to\C$ be defined by
\begin{equation*}
\phi(\Z+r) = \exp(2\pi r i).
\end{equation*}
Now for any $a,b\in\R$, we have
\begin{equation*}
\Z+a=\Z+b \iff a=b+z,
\end{equation*}
for some integer $z\in\Z$.  We then have
\begin{equation*}
\exp(2\pi a i) = \exp(2\pi(b+z) i) = \exp(2\pi b i)\exp(2\pi z i) = \exp(2\pi b i).
\end{equation*}
Thus far we have shown that $\phi$ is well-defined and onto-to-one.
Clearly $\phi$ is onto $\C$.  Is $\phi$ operation preserving?
\begin{align*}
 \phi(\Z+a+\Z+b) &=\phi(\Z+a+b) = \exp(2\pi(a+b)i) \\
&=\exp(2\pi a i)\exp(2\pi b i) = \phi(\Z+a)\phi(\Z+b)
\end{align*}

\end{document}