\documentclass[12pt]{article}

\usepackage{amsmath}
\usepackage{amssymb}
\usepackage{amsthm}

\title{Chapter 0 Exercises\\Gallian's Book on Abstract Algebra}
\author{Spencer T. Parkin}

\newtheorem{theorem}{Theorem}[section]
\newtheorem{definition}{Definition}[section]
\newtheorem{corollary}{Corollary}[section]
\newtheorem{identity}{Identity}[section]
\newtheorem{lemma}{Lemma}[section]
\newtheorem{result}{Result}[section]

%\newcommand{\gcd}{\mbox{gcd}}
\newcommand{\lcm}{\mbox{lcm}}

\begin{document}
\maketitle

\section*{Problem 12}

Let $a$ and $b$ be positive integers and let $d=\gcd(a,b)$ and $m=\lcm(a,b)$.
If $t$ divides both $a$ and $b$, prove that $t$ divides $d$.  If $s$ is a multiple of
both $a$ and $b$, prove that $s$ is a multiple of $m$.

By Theorem 0.2, $d$ is a linear combination of $a$ and $b$, and therefore,
any common divisor of $a$ and $b$, such as $t$, also divides $d$.

To see that $m$ divides $s$, simply notice that all common multiples
of $a$ and $b$ are generated by all positive multiples of $m$.

\section*{Problem 24}

(Generalized Euclid's Lemma)  If $p$ is a prime and $p$ divides
$a_1a_2\dots a_n$, prove that $p$ divides $a_i$ from some $i$.

The case $n=2$ is covered by Euclid's Lemma.  Now suppose, for a fixed integer
$k>2$, that the generalized lemma holds in the case $n=k-1$.  Now
consider the case $n=k$.  If $p$ does not divide $a_n$, then
clearly $p$ divides $a_1a_2\dots a_{n-1}$ by Euclid's Lemma.
Then, by our inductive hypothesis, $p$ must divide $a_i$ for an
integer $i\in[1,n-1]$.  We have now proven the general lemma
by the principle of mathematical induction.

\section*{Problem 25}

Use the Generalized Euclid's Lemma (see Exercise 24) to establish the
uniqueness portion of the Fundamental Theorem of Arithmetic.

Suppose an integer $n$ has two different prime factorizations $p_1^{a_1}\dots p_r^{a_r}$
and $q_1^{b_1}\dots q_s^{b_s}$.  By the Generalized Euclid's Lemma, if
$p\in\{p_i\}_{i=1}^r$, then $p\in\{q_i\}_{i=1}^s$, because $p$ divides $n$.
Conversely, if $p\in\{q_i\}_{i=1}^s$, then $p\in\{p_i\}_{i=1}^r$ by the same reason.
It follows that $\{p_i\}_{i=1}^r=\{q_i\}_{i=1}^s$, which is a contradiction, and
therefore, no integer $n$ has two different prime factorizations.

\end{document}