\documentclass{birkjour}

\usepackage{tikz}
\usepackage{graphicx}
\usepackage{hyperref}
\usepackage{algpseudocode}
\usepackage{graphicx}

\newtheorem{thm}{Theorem}[section]
\newtheorem{cor}[thm]{Corollary}
\newtheorem{lem}[thm]{Lemma}
\newtheorem{prop}[thm]{Proposition}
\theoremstyle{definition}
\newtheorem{defn}[thm]{Definition}
\theoremstyle{remark}
\newtheorem{rem}[thm]{Remark}
\newtheorem*{ex}{Example}
\numberwithin{equation}{section}

\newcommand{\R}{\mathbb{R}}
\newcommand{\B}{\mathbb{B}}
\newcommand{\G}{\mathbb{G}}
\newcommand{\V}{\mathbb{V}}
\newcommand{\gd}{\dot{g}}
\newcommand{\gh}{\hat{g}}
\newcommand{\Gd}{\dot{G}}
\newcommand{\Gh}{\hat{G}}
\newcommand{\nvai}{\infty}
\newcommand{\nvao}{o}
\newcommand{\grade}{\mbox{grade}}

\newcommand{\meet}{{\it meet}\;}
\newcommand{\join}{{\it join}\;}
\newcommand{\vecspace}{\overline}

%\received{}\accepted{}

\begin{document}

\title{Further Treatment Of Geometric Sets}

\author{Spencer T. Parkin}
\address{102 W. 500 S., \\
Salt Lake City, UT  84101} \email{spencerparkin@outlook.com}

%\subjclass{Primary 14J70; Secondary 14J29}

%\dedicatory{To my dear wife Melinda.}

\begin{abstract}
Abstract goes here...
\end{abstract}

%\keywords{Geometric Algebra, Model Of Geometry, Geometric Set}

\maketitle

\section{Introduction}

We let $P(\V)$ denote the set of all possible vector sub-spaces of $\V$,
not the power-set of $\V$.

\section{The Meet And Join Of Vector Sub-Spaces}

An excellent treatment of \meet and \join is given in \cite{}.
Unlike \cite{}, however, here we will not, as they said, abuse language by referring to
any blade as a vector sub-space.\footnote{The present author has abused language
in this way in other writings, but repents of this, at least in the present paper.}  In an attempt to be as rigorous as possible,
we will introduce a function $\vecspace{B}:\B\to P(\V)$, and, for any blade $B$, only refer to
$\vecspace{B}(B)$ as a vector sub-space.  For convenience in notation, we write $\vecspace{B}$ instead
of $\vecspace{B}(B)$.
\begin{defn}
For any blade $B\in\B$, we define the function $\vecspace{B}:\B\to P(\V)$ as
\begin{equation}
\vecspace{B} = \{x\in\V|x\wedge B=0\}.
\end{equation}
\end{defn}

In light of Definition~\ref{}, it is important to realize that for any two blades $A,B\in\B$,
and a vector sub-space $V\subseteq\V$, if we have $\vecspace{A}=V$ and $\vecspace{B}=V$, then there
exists a scalar $\lambda\in\R$ such that $A=\lambda B$.

As operations (or well defined functions), \meet and \join operate on vector spaces, not blades.
We therefore do not speak of taking the \meet or \join of two blades.  Rather, the blades of a geometric
algebra will help us calculate the \meet and \join of vector sub-spaces represented by those blades.

\begin{defn}[The \meet of two vector sub-spaces]\label{def_meet}
For any two blades $A,B\in\B$, the \meet $M$ of $\vecspace{A}$ and $\vecspace{B}$ is the
vector space given by
\begin{equation}
M = \vecspace{A}\cap \vecspace{B} = \{x\in\V|\mbox{$x\in \vecspace{A}$ and $x\in \vecspace{B}$}\}.
\end{equation}
\end{defn}
From Defintion~\ref{def_meet}, we see that the \meet of two vector subspaces
is the largest common sub-space.  Notice that \meet is clearly communtative.

\begin{defn}[The \join of two vector sub-spaces]\label{def_join}
For any two blades $A,B\in\B$, the \join $J$ of $\vecspace{A}$ and $\vecspace{B}$ is the
vector space given by
\begin{equation}
J = \vecspace{A}+\vecspace{B} = \{x_1+x_2\in\V|\mbox{$x_1\in \vecspace{A}$ and $x_2\in \vecspace{B}$}\}.
\end{equation}
\end{defn}
From Definition~\ref{def_join}, we see that the \join of two vector spaces
is the smallest common super-space.  Notice here too the commutativity of \join.

Note that some care should be taken with Definition~\ref{def_meet} and
Definition~\ref{def_join} by easily verifying that $M$ and $J$ each satisfy the
necessary properties of a vector space.
Also note that any pair of blades, each representative of the same \meet or \join, are clearly
scalar multiples of one another.

At this point, a natural question arrises.  Given two blades $A,B\in\B$, how do we find a blade
$C\in\B$ such that $\vecspace{C}=\vecspace{A}\cap\vecspace{B}$ and $\vecspace{C}=\vecspace{A}+\vecspace{B}$?
Do this in terms of orthogonal complement...define that first...

% citation: http://www.geometricalgebra.net/downloads/meet_and_join_by_grunberg_20080328.pdf

\end{document}