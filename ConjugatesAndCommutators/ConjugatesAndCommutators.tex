\documentclass[12pt]{article}

\usepackage{amsmath}
\usepackage{amssymb}
\usepackage{amsthm}
\usepackage{graphicx}
\usepackage{float}

\title{Conjugates and Commutators}
\author{Spencer T. Parkin}

\newtheorem{theorem}{Theorem}[section]
\newtheorem{definition}{Definition}[section]
\newtheorem{corollary}{Corollary}[section]
\newtheorem{identity}{Identity}[section]
\newtheorem{lemma}{Lemma}[section]
\newtheorem{result}{Result}[section]

\begin{document}
\maketitle

This note gives some insight into why useful sequences for twisty puzzles so often take
the form of conjugates and commutators.  We begin with a permutation group $G\leq S(\Omega)$,
and for all $g\in G$, let
\begin{equation*}
\phi(g)=\{\omega\in\Omega|\omega^g\neq\omega\}.
\end{equation*}
We then establish, for all $a,b\in G$, the following relations.
\begin{align*}
|\phi(bab^{-1})|&=|\phi(a)|\\
|\phi(aba^{-1}b^{-1})|&=|\phi(a)\cap\phi(b)|\\
 &\leq\mbox{min}(|\phi(a)|,|\phi(b)|)
\end{align*}
The first of these is a consequence of the fact that for any $\omega\in\Omega$, we
have $\omega^a=\omega$ if and only if $\omega^{bab^{-1}}=\omega$.
For the second of these, the relation clearly holds when $a$ and $b$ commute.
If they do not commute, then for any $\omega\in\phi(a)\cup\phi(b)$,
we consider the following three cases.
\begin{align*}
\mbox{case 1:}\; & \omega\in\phi(a)-\phi(b)\\
\mbox{case 2:}\; & \omega\in\phi(b)-\phi(a)\\
\mbox{case 3:}\; & \omega\in\phi(a)\cap\phi(b)
\end{align*}
Clearly, $\omega^{aba^{-1}b^{-1}}=\omega$ for cases 1 and 2, but not case 3.

What we learn here is that conjugates translate a useful sequence into another useful
sequence acting on a different subset of $\Omega$ of the same size in a similar way.
We also learn that the total action performed by a commutator is less than the minimum
action performed by either permutation taken in the commutator product.  This is
essential to descending a stabilizer chain, where getting from one subgroup to a
smaller subgroup requires stabilizing more points of $\Omega$.

\end{document}