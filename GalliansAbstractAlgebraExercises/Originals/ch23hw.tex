\documentclass{article}
\usepackage{amsmath}
\usepackage{amssymb}
\title{Chapter 23 Homework}
\author{Spencer}
\addtolength{\oddsidemargin}{-.575in}
\addtolength{\evensidemargin}{-.575in}
\addtolength{\textwidth}{1.0in}
\addtolength{\topmargin}{-.575in}
\addtolength{\textheight}{1.25in}
\begin{document}
\maketitle

\newcommand{\Z}{\mathbb{Z}}
\newcommand{\R}{\mathbb{R}}
\newcommand{\N}{\mathbb{N}}
\newcommand{\Q}{\mathbb{Q}}

\section*{Problem 6}

Prove that an angle $\theta$ is constructible if and only if $\sin\theta$ is constructible.

If both $\sin\theta$ and $\cos\theta$ are constructible, then we can use them
to construct the angle $\theta$ from a right triangle.  So we want to show that
if $\sin\theta$ constructible, then $\cos\theta$ is too.

If $\sin\theta$ is construcible, then so is everything in $\Q(\sin\theta)$.
Consider now the polynomial
\begin{equation*}
x^2-(1-\sin^2\theta)\in\Q(\sin\theta)[x].
\end{equation*}
If this splits in $\Q(\sin\theta)[x]$, then we're done.  If not, then it is
the minimal polynomial for $\cos\theta$ over $\Q(\sin\theta)$ and
$[\Q(\sin\theta)(\cos\theta):\Q(\sin\theta)]=2$.  This is a power of two, but
does this mean $\cos\theta$ is constructible?  I don't know.
If $\alpha$ is constructible, then $[\Q(\alpha):\Q]=2^k$ for some $k$, but I do not
know if the converse holds.

If $\theta$ is constructible, it is easy to construct both $\sin\theta$ and $\cos\theta$
from a right triangle with hypotenuse of unit length.

\section*{Problem 7}

Prove that $\cos 2\theta$ is constructible if and only if $\cos\theta$ is constructible.

Using the identity $\cos 2\theta=2\cos^2\theta-1$, it is clear that $\cos 2\theta$ is
constructible if $\cos\theta$ is constructible.  Conversley, we're looking for a
root of the polynomial $2x^2-(1+\cos 2\theta)$ over the constructible field $\Q(\cos 2\theta)$.
Just as with the argument given in Problem 6, the root of this polynomial will have degree
1 or 2 over $\Q(\cos 2\theta)$, and since 1 and 2 are powers of two, this may mean the root
is constructible.

\section*{Problem 8}

Prove that $30^{\circ}$ is a construcible angle.

The angle $\pi/6$ is in the constructible field $\Q(\sqrt{3})$, because we can
construct $\pi/6$ from a right triangle with legs $\sqrt{3}/2$ and $1/2$.
Notice that $[\Q(\sqrt{3}):\Q]=2$.

\section*{Problem 9}

Prove that a $45^{\circ}$ angle can be trisected with an unmarked straightedge and a compass.

I do not know why some angles can be trisected and others can't.  We may have even gone
over this in class.

Instead of showing that $\pi/4$ can be trisected, does it suffice to show that
$\pi/12$ is construcible?  Using the result proven in Problem 6, we need only
show that $\cos(\pi/12)=(\sqrt{3}+1)\sqrt{2}/4$ or $\sin(\pi/12)=(\sqrt{3}-1)\sqrt{2}/4$
is constructible.  The $\cos(\pi/12)$ is a root of the polynomial $x^4-x^2+1/16$ in $\Q[x]$.
My calculator says this is irreducible over $\Q$.  (Yes, I cheated.)
It now follows that since $x^4-x^2+1/16$ is irreducible over $\Q$, its roots have a
degree over $\Q$ of $2^2$, and so may be constructible.

\section*{Problem 10}

Prove that a $40^{\circ}$ angle is not constructible.

If $40^{\circ}$ is constructible, then so is $20^{\circ}$.  If $20^{\circ}$ is
constructible, then so is $\cos 20^{\circ}$.  But it was shown in the textbook that
$[\Q(\cos 20^{\circ}):\Q]=3$, which is not a power of two, so $40^{\circ}$ is not
constructible.

\section*{Problem 11}

Show that the point of intersection of two lines in the plane of a field $F$
lies in the plane of $F$.

For the two lines $ax+by+c=0$ and $a'x+b'y+c'=0$, we want a solution to the
matrix equation
\begin{equation*}
\left[\begin{array}{cc} a & b \\ a' & b' \end{array}\right]
\left[\begin{array}{c} x \\ y \end{array}\right] =
\left[\begin{array}{c} -c \\ -c' \end{array}\right].
\end{equation*}
If there is a solution, it can be found using elementary row operations which use
the operations of $F$.

\section*{Problem 12}

Show that the points of intersection of a circle in the plane of a field $F$
and a line in the plane of $F$ are points in the plane of $F$ or in the plane
of $F(\sqrt{\alpha})$, where $\alpha\in F$ and $\alpha$ is positive.  Give an
example of a circle and a line in the plane of $\Q$ whose points of intersection
are not in the plane of $\Q$.

An example is given by the circle $x^2+y^2=1$ and the line $x+y=0$.

Given an equation for a circle and a line, solving the linear equation for
$y$ and plugging it into the circle equation results in a quadratic equation in $x$.
Let $\alpha$ be the descriminent in the quadratic formula used on this quadratic equation.
If $\alpha$ is negative, then there is no intersection, and so no
new number is constructed.  If $\alpha$ is non-negative and the square
of some rational number, then the point or points constructed are in the plane of $F$.
If $\alpha$ is non-negative and not the square of a rational number,
then the point or points are in the plane of $F(\sqrt{\alpha})$.

\section*{Problem 14}

Use the fact that $8\cos^3(2\pi/7)+4\cos^2(2\pi/7)-4\cos(2\pi/7)-1=0$ to prove
that a regular seven-sided polygon is not constructible with an unmarked
straightedge and a compass.

If the seven-sided polygon is constructible, then so is the angle $2\pi/7$, and then
so is $\cos(2\pi/2)$.  It follows that $[\Q(\cos(2\pi/7)):\Q]$ is a power of two.
But if $8x^3+4x^2-4x-1$ is irreducible over $\Q$, then $\cos(2\pi/7)$ will have
degree 3 over $\Q$, which is not a power of two.
Using the Mod 3 Irreducibility Test, let us consider whether $2x^3+x^2+2x+2$ is
irreducible over $\Z_3$.  It has no zeros in $\Z_3$, and some work shows that no
irreducible quadratic in $\Z_3[x]$ divides it.  So $8x^3+4x^2-4x-1$ is irreducible
over $\Q$, and by contradiction, the seven-sided polygon is not constructible.

\section*{Problem 15}

Show that a regular 9-gon cannot be constructed with an unmarked straightedge and a compass.

This implies that a $40^{\circ}=(180/\pi)(2\pi/9)$ angle is constructable, which is a contradiction
by the result proven in Problem 10.

\section*{Problem 16}

Show that if a regular $n$-gon is constructible, then so is a regular $2n$-gon.

This follows directly from the ability to bisect any angle.  Notice that the $2n$-gon
can be constructed using the angle that is half the angle $2\pi/n$ found in the $n$-gon.

\section*{Problem 17}

Show that it is impossible to construct, with an unmarked straightedge and a compass, a square
whose area equals that of a circle of radius 1.

The side-length of the square must be $\sqrt{\pi}$.
So we must have $[\Q(\sqrt{\pi}):\Q]$ be a power of two, which is a finite field extension,
and therefore an algebraic extension.
But since $\pi$ is transcendental over $\Q$, so is $\sqrt{\pi}$,
which is a contradiction.

I'm having trouble convincing myself that $\sqrt{\pi}$ is transcendental because $\pi$ is.

\section*{Problem 19}

Can the cube be "tripled"?

If a cube has side-length $x$, then the tripled cube has side-length $\sqrt[3]{3}x$,
so we need to see if $\sqrt[3]{3}$ is constructible.  The minimal polynomial for
$\sqrt[3]{3}$ over $\Q$ is $x^3-3$ and so $\sqrt[3]{3}$ has degree 3 over $\Q$, which
is not a power of two.  So the cube cannot be "tripled".

\section*{Problem 20}

Can the cube be "quadrupled"?

If I'm not mistaken, $[\Q(\sqrt[3]{4}):\Q]=3$, so the answer is no.

\end{document}