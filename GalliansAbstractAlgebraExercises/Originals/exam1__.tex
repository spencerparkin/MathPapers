\documentclass{article}
\usepackage{amsmath}
\usepackage{amssymb}
\addtolength{\oddsidemargin}{-.575in}
\addtolength{\evensidemargin}{-.575in}
\addtolength{\textwidth}{1.0in}
\addtolength{\topmargin}{-.575in}
\addtolength{\textheight}{1.25in}
\begin{document}

\newcommand{\Z}{\mathbb{Z}}

\section*{Problem 1}

Let $N$ be an ideal of a non-commutative ring $R$.  Define the set
$M=\{r\in R:\mbox{$xr\in N$ for all $x\in R$}\}$.  Show that
$M$ is an ideal of $R$ that contains $N$.

We see that $N$ is both a right and left sided ideal, because it
is an ideal of $R$.
Let $a\in N$.  Then for all $x\in R$, we have $xa\in N$, because
$N$ is a left sided ideal.  This shows that $N\subseteq M$.
Now let $a,b\in M$.  Then for all $x\in R$, we see that $x(a-b)=xa-xb\in N\subseteq M$,
showing that $M$ is a subring of $R$.
For all $x\in R$, we also have $xa\in N\subseteq M$, showing that $M$
is a left sided ideal.
What remains to be shown is that $M$ is a right sided ideal.
Let $a\in M$ and let $x\in R$.  We want to show that $ax\in M$.
So for all $y\in R$, we see that $y(ax)=(ya)x\in N$ since $ya\in N$
and $N$ is a right sided ideal.
It now follows that $M$ is an ideal of $R$.

\section*{Problem 2}

Let $D$ be a principle ideal domain and let $a\in D$ with $a\neq 0$.
Show that there are only a finite number of different ideals in $D$
containing $a$.

Let us first show that $\langle x\rangle=\langle y\rangle$ if and
only if $x$ and $y$ are associates in $D$.

Let $x=uy$ where $u$ is a unit of $D$.  Suppose $z\in\langle x\rangle$.
Then there exists $d\in D$ such that $z=dx=duy\in\langle y\rangle$.
Supposing now that $z\in\langle y\rangle$, we see that there exists
$d\in D$ such that $z=dy=du^{-1}x\in\langle x\rangle$.  It follows
that $\langle x\rangle=\langle y\rangle$.

Now suppose $\langle x\rangle=\langle y\rangle$.  Then there exist
$d_1,d_2\in D$ such that $x=d_1y$ and $y=d_2x$.  So we have
$x=d_1d_2x\implies 1=d_1d_2$, because $D$ is an integral domain.
This shows that $d_1$ and $d_2$ are units, and
it follows that $x$ and $y$ are associates.

Now since $D$ is a principle ideal domain, the set of all
ideals in $D$ containing $a$ is given by
$\{\langle g\rangle:\mbox{$g$ divides $a$}\}$.
Then since $D$ is also a unique factorization domain, we see
that the number of divisors of $a$ that are unique up to associates
is limited by the number of ways we can choose irreducible factors from
the unique factorization of $a$.
But the number of such choices is finite, since the number of irreducible
factors of $a$ is finite.  It now follows that the set of
all ideals containing $a$ is finite.

\section*{Problem 3}

Let $F$ be a field and let $F(x)$ denote the field of quotients of $F[x]$.
Show that there is no element in $F(x)$ whose square is $x$.

Let $f(x),g(x)\in F[x]$ with $g(x)\neq 0$.  Then $f(x)/g(x)\in F(x)$,
and we consider whether or not it is possible that
$((f(x)/g(x))^2=(f(x))^2/(g(x))^2=x/1\implies x(g(x))^2=(f(x))^2$.
Clearly this is impossible since $\deg x(g(x))^2$ is odd and
$\deg f(x)^2$ is even.

\section*{Problem 4}

Express $x^8+x$ as a product of irreducible polynomials in $\Z_2[x]$.

Clearly $x^8+x=x(x^7+1)$.  The latter term has a zero of 1, so this
factors to $x(x-1)(x^6+\dots+1)$.  At this point, if the latter term
has no linear factors, then it has no quintic factors.  If it has no
quadratic factors, then it has no quartic factors.  
Both $x+1$ and $x$ do not divide $x^6+\dots+1$, and no irreducible
quadratic divides it either.
Finally, some work shows that $x(x-1)(x^6+\dots+1)$ further factors to...
\begin{equation*}
x(x-1)(x^3+x+1)(x^3+x^2+1)
\end{equation*}
Since $\Z_2$ is a field, we may apply Theorem 17.1 to show that the remaining
two factors are irreducible.  Clearly, neither of them have zeros in $\Z_2$,
so this is the desired factorization.

\section*{Problem 5}

Let $a+b\sqrt{-5}$ be an element of $\Z[\sqrt{-5}]$ with $b\neq 0$.
Show that $2\not\in\langle a+b\sqrt{-5}\rangle$.

If $2\in\langle a+b\sqrt{-5}\rangle$, then there exists $z\in\Z[\sqrt{-5}]$
such that $2=z(a+b\sqrt{-5})$.  Then we see that
$4=N(2)=N(z(a+b\sqrt{-5}))=N(z)N(a+b\sqrt{-5})=N(z)(a^2+5b^2)$, but
$a^2+5b^2\geq 5$, since $b\neq 0$.  There is no positive integer $N(z)$
such that $4=N(z)(a^2+5b^2)$, so $2\not\in\langle a+b\sqrt{-5}\rangle$.

\end{document}