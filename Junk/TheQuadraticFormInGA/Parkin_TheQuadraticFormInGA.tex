\documentclass{birkjour}

\usepackage{amsmath}
\usepackage{amssymb}
\usepackage{amsthm}
\usepackage{graphicx}
\usepackage{float}

\newtheorem{thm}{Theorem}[section]
\newtheorem{cor}[thm]{Corollary}
\newtheorem{lem}[thm]{Lemma}
\newtheorem{prop}[thm]{Proposition}
\theoremstyle{definition}
\newtheorem{defn}[thm]{Definition}
\theoremstyle{remark}
\newtheorem{rem}[thm]{Remark}
\newtheorem*{ex}{Example}
\numberwithin{equation}{section}

\newcommand{\G}{\mathbb{G}}
\newcommand{\V}{\mathbb{V}}
\newcommand{\Vb}{\mathbb{\overline{V}}}
\newcommand{\W}{\mathbb{W}}
\newcommand{\R}{\mathbb{R}}

\newcommand{\nvao}{o}
\newcommand{\nvai}{\infty}
\newcommand{\nvaob}{\overline{o}}
\newcommand{\nvaib}{\overline{\infty}}

\newcommand{\eminus}{e_{-}}
\newcommand{\eplus}{e_{+}}
\newcommand{\eminusb}{\overline{e}_{-}}
\newcommand{\eplusb}{\overline{e}_{+}}

\begin{document}

\title{The Quadratic Form In Geometric Algebra}

\author{Spencer T. Parkin}
\email{spencer.parkin@gmail.com}

\numberwithin{equation}{section}

\subjclass{Primary 14J70; Secondary 14J29}

\keywords{Quadric Surface, Quadratic Form, Geometric Algebra}

%\dedicatory{To Melinda and Naomi}

\begin{abstract}
Blah.
\end{abstract}

\maketitle

\section{The Quadratic Form In Vector Algebra}

Let $\V^n$ be an $n$-dimensional Euclidean vector space, and identify vectors
in this space with points in $n$-dimensional Euclidean space.  That is,
for any vector $v\in\V^n$, identify this vector with the point at its tip when
its tail is placed at origin.  Letting any subset of $\V^n$ be what we refer to
as a geometry, the goal of this paper is to use geometric algebra in the study of all
such geometries that occur as the zero set of one or more quadratic forms.\footnote{In
algebraic geometry, the zero set of one or more polynomials is called an affine variety.}
A quadratic form $q:\V^n\to\R$ is a quadratic polynomial in the vector components of any
vector $v\in\V^n$.  Specifically, we have
\begin{equation}\label{equ_quadratic_form}
q(v) = C + \sum_{i=1}^n C_i(v\cdot e_i) + \sum_{i=1}^n\sum_{j=1}^n C_{ij}(v\cdot e_i)(v\cdot e_j),
\end{equation}
where $C$, each of $C_i$ and each of $C_{ij}$ are scalars in $\R$.  The coeficients $C$, $C_i$ and $C_{ij}$ collectively determine the geometry that is the zero set of $q$.  Adding a Euclidean vector $e_0$
representative of the origin to $\V^n$ to obtain the $(n+1)$-dimensional Euclidean
vector space $\V^{n+1}$, we see that the quadratic form $q$ is determined by a
symmetric billinear form $B:\V^{n+1}\times\V^{n+1}\to\R$ as
\begin{align}
q(v) &= B(e_0+v,e_0+v) \\
 &= B(e_0,e_0)+2\sum_{i=1}^n B(e_0,e_i)(v\cdot e_i)
 + \sum_{i=1}^n\sum_{j=1}^n B(e_i,e_j)(v\cdot e_i)(v\cdot e_j),
\end{align}
if we let $B(e_0,e_0)=C$, each of $B(e_0,e_i)=B(e_i,e_0)=\frac{1}{2}C_i$ and each of
$B(e_i,e_j)=C_{ij}$.
In turn, we see that the symmetric billinear form $B$ is determined entirely by how it
maps a basis of $\V^{n+1}$.

While the quadratic function $q$ in equation \eqref{equ_quadratic_form} is useful as is, it is
worth abandoning the direct use of this function in favor of finding a way to implement it in
a geometric algebra, because
such an implementation, depending on how it's done, may allow us to exploit certain
properties and theorems of that algebra.
This has already been accomplished with matrix algebra.  See, for example, \cite{}.  The approach taken
in this paper to finding a useful instance of the quadric form in a geometric algebra will be based on
a search for the symmetric bilinear form $B$.

\section{The Quadratic Form In $\G(\V^{n+1}\oplus\Vb^{n+1})$}

Here we let $\Vb^{n+1}$ be an $(n+1)$-dimensional Euclidean vector space
disjoint from and isomorphic to $\V^{n+1}$, and then define the over-bar notation on elements
of the geometric algebra $\G(\W)$, with $\W=\V^{n+1}\oplus\Vb^{n+1}$, as an
outermorphic isomorphism between $\G(\V^{n+1})$ and $\G(\Vb^{n+1})$.
That is, for any element $Q\in\G(\W)$, we have
\begin{equation}
\overline{R} = RQR^{-1},
\end{equation}
where the rotor $R$ is given by
\begin{equation}
R = \prod_{i=0}^n(1-e_i\overline{e}_i).
\end{equation}
Havine done this, for any pair of vectors $x,y\in\V^{n+1}$, we can now find
the symmetric bilinear form $B$ in $\G(\W)$ as
\begin{equation}
B(x,y) = x\overline{y}\cdot\sum_{i=0}^n\sum_{j=0}^n B(e_i,e_j)e_i\overline{e}_j.
\end{equation}
Then, defining the function $S:\V^{n+1}\to\G(\W)$ as
\begin{equation}
S(p) = p\overline{p},
\end{equation}
we may, for any bivector $Q\in\G(\W)$, write the quadratic form as
\begin{equation}\label{equ_quadric_form_in_ga}
q(v) = S(e_0+v)\cdot Q,
\end{equation}
showing that such bivectors $Q$ are representative of $n$-dimensional
quadric surfaces as the set of all vectors $v\in\V^n$ such that $S(e_0+v)\cdot Q=0$.

This approach is especially advantageous in the realization that for
any versor $V\in\G(\V^{n+1})$, we have
\begin{equation}
S(V^{-1}pV)\cdot Q=S(p)\cdot V\overline{V}Q(V\overline{V})^{-1}
\end{equation}
which shows that if we understand how $V$ transforms homogeneous
points $p\in\V^{n+1}$ as $V^{-1}pV$, then we also understand
how $V$ transforms quadric surfaces $Q\in\G(\W)$ as $(V\overline{V})Q(V\overline{V})^{-1}$.
In a variation of this approach that uses the Minkowski geometric algebra $\G(\V^{n+1,1}\oplus\V^{n+1,1})$,
the versors of the conformal model of geometric algebra may be used to transform quadric surfaces.
See \cite{}.

A down-side to this approach, however, is in the fact that we're not using blades
to represent quadric surfaces in the same way that blades are representive of geometries
in the conformal model of geometric algebra.  Consequently, we cannot similarly benefit
from the meet and join operations.
We will attempt to remedy this problem in the next section.

\section{The Quadratic Form In $\G(\V^{(n+1)^2})$}

Notice that in the previous method, the Euclidean space $\V^n$ was
embedded in the representation space $\G(\W)$.  For the method to
follow, we show that this need not be the case.  Specifically, we
do not let $\V^n$ be a vector sub-space of the $(n+1)^2$-dimensional
anti-Euclidean\footnote{If $\{e_i\}_{i=1}^n$ is an orthonormal set of basis
vectors for an $n$-dimensional Euclidean vector space, then the vector space
becomes anti-Euclidean if for each integer $i\in[1,n]$, we redefine $e_i^2=1$ as $e_i^2=-1$.}
vector space $\V^{(n+1)^2}$.
We will, however, continue to let $\V^n$ be a proper vector sub-space of $\V^{n+1}$.

Letting $\{e_{ij}\}$ be a set
of orthonormal basis vectors spanning $\V^{(n+1)^2}$, we will now reintroduce the
function $S:\V^{n+1}\to\V^{(n+1)^2}$ as
\begin{equation}
S(p) = p\otimes p,
\end{equation}
where $\otimes:\V^{n+1}\times\V^{n+1}\to\V^{(n+1)^2}$ is a non-commutative
bilinear operator, defined as
\begin{equation}
x\otimes y = \sum_{i=0}^n\sum_{j=0}^n(x\cdot e_i)(y\cdot e_j)e_{ij},
\end{equation}
and then find that for all pairs of vectors $x,y\in\V^{n+1}$, the symmetric
bilinear form $B$ in $\G(\V^{(n+1)^2})$ is given by
\begin{equation}
B(x,y) = -(x\otimes y)\cdot\sum_{i=1}^n\sum_{j=1}^n B(e_i,e_j)e_{ij},
\end{equation}
showing that the vectors $Q\in\G(\V^{(n+1)^2})$ are representative
of $n$-dimensional quadric surfaces as the set of all vectors $v\in\V^n$
such that $q(v)=0$, where $q$ is again given by equation \eqref{equ_quadric_form_in_ga}.

Immediately we see that the advantage to this approach is that a
non-zero blade $Q\in\G(\V^{(n+1)^2})$ of grade $k$ is representative
of an $(n+1-k)$-dimensional quadric surface.  To see this,
let $Q=Q_1\wedge\dots\wedge Q_k$, and realize that
\begin{equation}
0 = S(p)\cdot\bigwedge_{i=1}^k Q_i = \sum_{i=1}^k (S(p)\cdot Q_i)\bigwedge_{j=1,j\neq i}^k Q_j
\end{equation}
if and only if for all integers $j\in[1,k]$, we have $S(p)\cdot Q_j=0$.  In other words,
$Q$ represents the affine variety generated by the set of all quadratic polynomials
determined by each $Q_j$.

We will refer to $Q$ as a dual quadric if we are interpreting it as being representative
of a quadric surface in terms of the equation
\begin{equation}\label{equ_dual_quadric}
0 = S(e_0+v)\cdot Q.
\end{equation}
Similarly, we will refer to $Q$ as a direct quadric if we are interpreting it as being
representative of such a surface in terms of the equation
\begin{equation}\label{equ_direct_quadric}
0 = S(e_0+v)\wedge Q.
\end{equation}
To see that this is also the previously mentioned affine variety, simply realize that
\begin{align}
 & 0 = S(e_0+v)\wedge Q \\
\mbox{iff}\;\; & 0 = S(e_0+v)\cdot QI,
\end{align}
where $I$ is the unit-psuedo scalar of $\G(\V^{(n+1)^2})$.

Notice that any single blade $Q\in\G(\V^{(n+1)^2})$ is
simultaneously representative of both a dual and direct quadric, which are distinct
pieces of geometry.\footnote{A dual quadric is directly represented by its dual, and
a direct quadric is dually represented by its dual.  As a given blade simultaneously
represents two geometries, (one dually, the other directly), a single given geometry is
simultaneously represented by two distinct blades, (which are duals of one another).}
It is sometimes useful to reinterpret a dual quadric
as a direct quadric, or vice versa.  For example, if the dual intersection of
two dual quadrics is imaginary, the imaginary intersection may be a real quadric
in direct form.

%The immediate down-side to this approach, however, is that we lose
%the versor transformation properties of the prior method.  To see this,
%realize that for any vector $v\in\V^{n+1}$, we have
%\begin{equation}
%S(v^{-1}pv) = 2(p\cdot v^{-1})v\otimes y - S(p),
%\end{equation}
%and that for any vector $w\in\V^{(n+1)^2}$, we have
%\begin{equation}
%w^{-1}S(p)w = \frac{2}{w^2}\sum_{i=1}^n\sum_{j=1}^n(w\cdot e_{ij})^2(p\cdot e_i)^2e_{ij} - S(p).
%\end{equation}
%Given a vector $v\in\V^{n+1}$, there is not always a choice of $w\in\V^{(n+1)^2}$ such
%that $S(v^{-1}pv)=w^{-1}S(p)$w.  (Needs more work.)

Before moving on, there is some question here about how the versors
of $\G(\V^{(n+1)^2})$ act on the quadrics in this geometric algebra.
The first observation we can make is that for any vector $v\in\V^{(n+1)^2}$,
we have
\begin{align}
 & 0 = S(p)\wedge vQv^{-1} \\
\mbox{iff}\;\; & 0 = S(p)\cdot vQv^{-1}I \\
\mbox{iff}\;\; & 0 = S(p)\cdot vQIv^{-1},
\end{align}
since vectors either commute or anti-commute with the unit psuedo-scalar.  This shows that
versors will act the same way on dual and direct quadrics.  We may therefore restrict
our attention to direct quadrics without loss of generality.  Doing so, let $V\in\G(\V^{(n+1)^2})$
be a versor, and notice that
\begin{align}
 & 0 = V^{-1}S(p)V\wedge Q \\
\mbox{iff}\;\;& 0 = S(p)\wedge VQV^{-1},
\end{align}
which shows that the question of how any versor $V$ acts on any quadric $Q$ reduces
to the question of how $V$ acts on $S(p)$.  This also reflects the importance of our
choice in defining $S(p)$ and in choosing the signature of our geometric algebra.
In any case, subject to the signature we have chosen for $\G(\V^{(n+1)^2})$,
and with $S(p)$ defined the way it is, we see that for a vector $v\in\V^{(n+1)^2}$,
we have
\begin{equation}
v^{-1}S(p)v = \sum_{i=1}^n\sum_{j=1}^n(p\cdot e_i)(p\cdot e_j)v^{-1}e_{ij}v.
\end{equation}
Simplify that, if you can.

\section{Point Fitting Quadrics In $\G(\V^{(n+1)^2})$}

Suppose $Q\in\G(\V^{(n+1)^2})$ is a direct quadric of grade $k$, and
that $\{p_i\}_{i=1}^k$ is a set of $k$ homogeneous points taken from $\V^{n+1}$
such that for all integers $i\in[1,k]$, we have $S(p_i)\wedge Q=0$.  Then,
if $\{S(p_i)\}_{i=1}^k$ is a linearly independent set, it follows that
there exists a scalar $\lambda\in\R$ such that
\begin{equation}
\bigwedge_{i=1}^k S(p_i) = \lambda Q.
\end{equation}
What this shows is that, given a set of $k$ points $\{p_i\}_{i=1}^k$, we can find a quadric $Q$
that fits the $k$ points, provided the set $\{S(p_i)\}_{i=1}^k$ is linearly
independent.  Two questions
arise from this.  First, under what circumstances do the $k$ points generate
a linearly independent set $\{S(p_i)\}_{i=1}^k$; and secondly, under
those circumstances, what quadric surface do we get?  These questions
are easy to answer in the conformal model of geometric algebra.  Here, however,
the author is forced to leave them as open questions.  This, of course, does
not stop us from performing the experiment of trying to fit a quadric surface
to a given set of points.  Figures \ref{} and \ref{} illustrate the results
of such an experiment.

From what we have thus far gathered, an $(n+1-k)$-dimensional quadric surface
would be fit to $(n+1)^2-k$ points if it were at all possible to find such a set
of points generating a linearly independent set.  Possible or not, it is easy to show that this
is certainly not the least upper bound on the minimum number of points needed to determine
such a surface.  To see why, define $S_{\leq}:\V^n\to\V^m$ as
\begin{equation}
S_{\leq}(p) = \sum_{i\leq j}(S(p)\cdot e_{ij})e_{ij},
\end{equation}
where $m$ is given by
\begin{equation}
m = \binom{n}{0}+2\binom{n}{1}+\binom{n}{2} = \frac{(n+1)(n+2)}{2},
\end{equation}
and $\V^m$ is a proper vector sub-space of $\V^{(n+1)^2}$ and spanned
by the vectors in $\{e_{ij}\}_{i\leq j}$.
Using now $I_{\leq}$, what we'll use to denote the unit psuedo-scalar of $\G(\V^m)$,
to transition between dual and direct quadrics,\footnote{Notice that quadrics in $\G(\V^m)$
defined using $S_{\leq}$ are still valid quadrics in $\G(\V^{(n+1)^2})$ under the definition
using $S$.  We may think of the
quadrics in $\G(\V^{(n+1)^2})$ satisfying both definitions as being in a reduced form.}
and using $S_{\leq}$ in place of $S$
to define dual and direct quadrics by equations \eqref{equ_dual_quadric} and
\eqref{equ_direct_quadric}, we see that
an $(n+1-k)$-dimensional quadric surface may be fit to $m-k$ points, from which it
is more likely that we'll generate a linearly independent set.

A perhaps better approach to studying the ability to fit quadrics to
points would be the idea of generating a higher dimensional quadric
from a lower dimensional quadric.  For example, letting $Q$ be a $k$-blade, $Q$
represents an $(n+1-k)$-dimensional surface as a direct quadric.  Then,
if we choose any point $p\in\V^n$ not on $Q$ so that $S(p)\wedge Q\neq 0$,
then the quadric $Q'=S(p)\wedge Q$ must be a direct $(n+2-k)$-dimensional quadric containing
both $p$ and all the points of $Q$.  An example of this idea is given in Figure \ref{}.

% Do a 2D ellips (through intersectoin of two 3D ellipsoids).  then add a point not on the ellipsoid.
% Do we get a 3D ellipsoid?

% Do it on the computer?  can we show a figure where i've
% fit points to a quadric?

\section{Switching Between $\G(\V^{n+1}\oplus\Vb^{n+1})$ And $\G(\V^{(n+1)^2})$}

% Compare canonical forms in this section.
% S(a).S(b) = (a.b)^2
% Show Q then f(Q) in the form

If you found the choice of an anti-Euclidean vector
space in section \ref{} odd, the reason for this will
now come to light.  To gain the advantages of working
in both $\G(\V^{n+1}\oplus\Vb^{n+1})$ and $\G(\V^{(n+1)^2})$,
it may not be unreasonable to switch between the two
algebras when needed.  To do this, we simply use the linear
function $f:\V^{(n+1)^2}\to\G(\V^{n+1}\oplus\Vb^{n+1})$,
defined in terms of how it maps the basis vectors of $\V^{(n+1)^2}$ onto the
basis bivectors of the linear sub-space of bivectors in $\G(\V^{n+1}\oplus\Vb^{n+1})$ as follows.
For all pairs of integers $(i,j)\in[0,n]\times[0,n]$, we define
\begin{equation}
f(e_{ij})=e_i\overline{e}_j.
\end{equation}
We now see that for any vector $Q\in\V^{(n+1)^2}$ representative of
a quadric surface through the use of equation \eqref{equ_quadric_intersectable}, the bivector
$f(Q)\in\G(\V^{n+1}\oplus\Vb^{n+1})$ is representative of the same
quadric through the use of equation \eqref{equ_quadric_transformable}.

This gives us the ability to transform any intersection of one or more quadrics
in $\G(\V^{(n+1)^2})$ as we would a single quadric in $\G(\V^{n+1}\oplus\Vb^{n+1})$.
For a given blade $Q\in\G(\V^{(n+1)^2})$, we need only find a factorization
of the blade $Q$ as $Q_i\wedge\dots\wedge Q_k$, then formulate the transformation $Q'$
of $Q$ by a versor $V\in\G(\V^{n+1}\oplus\Vb^{n+1})$ as
\begin{equation}
Q' = \bigwedge_{i=1}^k f^{-1}\left(V\overline{V}f(Q_i)(V\overline{V})^{-1}\right).
\end{equation}
The problem of blade factorization has been given a great deal of treatment in \cite{}.

\section{Intersecting Lines With Quadrics}

Given a homogeneous point $p\in\V^{n+1}$ and a direction vector $v\in\V^n$,
it can be shown that
\begin{equation}\label{equ_quadric_line_intersect_quadratic}
S(p+\lambda v) = S(p)+\lambda v\cdot\nabla S(p)+\lambda^2 S(v),
\end{equation}
where $\lambda\in\R$ is a scalar.  Taking equation \eqref{equ_quadric_line_intersect_quadratic} in the
inner product with a quadric $Q\in\G(\W)$, and then setting this
product to zero, we get a quadratic equation in $\lambda$ with
coeficients $S(p)\cdot Q$, $(v\cdot\nabla S(p))\cdot Q$ and $S(v)\cdot Q$.
In the case that these coeficients are not all zero, the equation can be
solved to find zero, one or two points of intersection.  Otherwise, the quadric
contains the line through $p$ having direction $v$.

This is an easy way to intersect lines and quadrics, but shouldn't there be
a way to do this in $\G(\V^{(n+1)^2})$ using the outer product?  Presumably,
if $Q_L$ is a dual line quadric and $Q$ is any other dual quadric, the 2-blade
$Q_L\wedge Q$ should be a real or imaginary point-pair, or a tangent point.\footnote{A
tangent point in the conformal model of geometric algebra may be thought of as a
degenerate point-pair or a point-pair with radius zero.}
Unfortunately, attempts to find a canonical form for the quadric point-pair
show the impracticality of working in $\G(\V^{(n+1)^2})$.  In theory, however,
once a canonical form is found, any intersection $Q_L\wedge Q$ could
be interpreted in terms of this canonical form as a means to decomposing $Q_L\wedge Q$
into parameters characteristic of the point-pair.

\section{Closing Remarks}

It is not hard to see how the methods of this paper might be generalized
to surfaces of higher degree.  That is, affine varieties generated from a
set of one or more polynomials of higher degree.  But it doesn't seem worth
exploring such an idea until all of the wrinkles can be worked out of the study
of the quadratic form in geometric algebra.  There are many different ways
in which a quadratic polynomial can be encoded in a vector or bivector that
take advantage of geometric algebras with various signatures.  There is a question
of which arrangement is best, or which constructions lend themselves more to solving
one type of problem over another.  It may well be that there is an entirely
better model in existence for quadric surfaces that doesn't use geometric algebra at all.

% synthetic geometry <--> computational geometry

% Study conjugate points WRT a quadric?
% ruled surfaces, lines as generators of quadrics?
% thm: ruled surface in every pencil?

% completing the square?

\bibliographystyle{amsplain}
\bibliography{Parkin_TheQuadraticFormInGA}

% cite http://en.wikipedia.org/wiki/Zero_set
% cite http://www.science.uva.nl/~geer/Ch2.pdf
% cite a work on blade factorization
% cite your own work on the quadric model
% cite your own work on the extension/variation of the quadric model
% maybe cite http://www.montgomerycollege.edu/Departments/planet/planet/Numerical_Relativity/GA-SIG/Conformal%20Geometry%20Papers/Cambridge/Oriented%20Conformal%20Geometric%20Algebra.pdf
% cite http://en.wikipedia.org/wiki/Quadric_(projective_geometry) for line/quad intersect
% cite? http://www.encyclopediaofmath.org/index.php/Quadric  (too many internet sources)

\end{document}