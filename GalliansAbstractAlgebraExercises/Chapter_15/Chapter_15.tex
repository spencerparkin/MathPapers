\documentclass[12pt]{article}

\usepackage{amsmath}
\usepackage{amssymb}
\usepackage{amsthm}

\title{Chapter 15 Exercises\\Gallian's Book on Abstract Algebra}
\author{Spencer T. Parkin}

\newtheorem{theorem}{Theorem}[section]
\newtheorem{definition}{Definition}[section]
\newtheorem{corollary}{Corollary}[section]
\newtheorem{identity}{Identity}[section]
\newtheorem{lemma}{Lemma}[section]
\newtheorem{result}{Result}[section]

%\newcommand{\gcd}{\mbox{gcd}}
\newcommand{\lcm}{\mbox{lcm}}
\newcommand{\abs}{\mbox{abs}}
\newcommand{\Z}{\mathbb{Z}}
\newcommand{\R}{\mathbb{R}}
\newcommand{\G}{\mathbb{G}}
\newcommand{\stab}{\mbox{stab}}
\newcommand{\aut}{\mbox{Aut}}
\newcommand{\inn}{\mbox{Inn}}
\newcommand{\orb}{\mbox{orb}}
\newcommand{\chr}{\mbox{char}\,}

\begin{document}
\maketitle

\section*{Exercise 1}

Prove Theorem 15.1.

For the theorem, we let $\phi$ be a ring homomorphism from a ring $R$ to a ring $S$,
and we let $A$ be a subring of $R$ and let $B$ be an ideal of $S$.

The first part of the theorem states that for any $r\in R$ and any positive integer $n$,
that $\phi(nr)=n\phi(r)$ and $\phi(r^n)=(\phi(r))^n$.

Proof: That $\phi(nr)=n\phi(r)$ follows from Property 2 of Theorem 10.1 for group homomorphisms.
Similarly,
\begin{equation*}
\phi(r^n)=\phi(\underbrace{r\cdot\dots\cdot r}_n)=\underbrace{\phi(r)\cdot\dots\cdot\phi(r)}_n=(\phi(r))^n.
\end{equation*}

The second part of the theorem states that $\phi(A)=\{\phi(a)|a\in A\}$ is a subgring of $S$.

Proof: That $\phi(A)$ is an Abelian group follows from Properties 1 and 3 of Theorem 10.2
for group homomorphisms.  Then, if $a,b\in\phi(A)$, then there exist $x,y\in A$ such
that $\phi(x)=a$ and $\phi(y)=b$.  Then, since $xy\in A$ and $\phi(xy)=\phi(x)\phi(y)=ab$,
we see that $ab\in\phi(A)$.  Having now shown closure of the ring multiplication of $S$
in $\phi(A)$, we can claim that $\phi(A)$ is a subring of $S$.

The third part of the theorem states that if $A$ is an ideal and $\phi$ is onto $S$,
then $\phi(A)$ is an ideal.

Proof: By the second part of this theorem, $\phi(A)$ is a subring, so we need
only prove now that it is an ideal of $S$.  Let $s\in S-\phi(A)$ and $y\in\phi(A)$.  Then since $\phi$ is onto, there
exists $r\in R$ such that $\phi(r)=s$.  Let $x\in A$ such that $\phi(x)=y$.
Then since $rx\in A$, (because $A$ is an ideal of $R$), and $\phi(rx)=\phi(r)\phi(x)=sy$,
we have $sy\in\phi(A)$.  Similarly, since $xr\in A$, (again, because $A$ is an ideal of $R$),
and $\phi(xr)=\phi(x)\phi(r)=ys$, we have $ys\in\phi(A)$.  We can now claim that $\phi(A)$
is an ideal of $S$.

The fourth part of the theorem states that $\phi^{-1}(B)=\{r\in R|\phi(r)\in B\}$ is an ideal of $R$.

Proof: By Property 7 of Theorem 10.2, $\phi^{-1}(B)$ is a subgroup of $R$.  It must be an
Abelian group since all subgroups of rings are Abelian.  Now let $r\in R$ and $x\in\phi^{-1}(B)$.
Then $\phi(r)\in S$ and $\phi(x)\in B$ and since $B$ is an ideal of $S$, $\phi(rx)=\phi(r)\phi(x)\in B$,
showing that $rx\in\phi^{-1}(B)$.  Similar reasoning shows that $xr\in\phi^{-1}(B)$, so $\phi^{-1}(B)$
is an ideal of $R$.

The fifth part of the theorem states that if $R$ is commutative, then $\phi(R)$ is commutative.

Proof: Letting $a,b\in R$, notice that
\begin{equation*}
\phi(a)\phi(b)=\phi(ab)=\phi(ba)=\phi(b)\phi(a).
\end{equation*}

The sixth part of the theorem states that if $R$ has a unity 1, $S\neq\{0\}$, and $\phi$ is onto,
then $\phi(1)$ is the unity of $S$.

Proof: Notice that for all $r\in R$, we have $\phi(1)\phi(r)=\phi(r)$.  Then since $\phi$ is onto,
it follows that $\phi(1)s=s$ for all $s\in S$.  This shows that $\phi(1)$ is either the unity of $S$,
or that $\phi(r)=0$ for all $r\in R$.  Now if $S\neq\{0\}$ and $\phi$ is onto, then we can't
have $\phi(r)=0$ for all $r\in R$.  So $\phi(1)=1$.

The seventh part of the theorem states that $\phi$ is an isomorphism if and only if $\phi$ is onto and
$\ker\phi=\{r\in R|\phi(r)=0\}=\{0\}$.

Proof: This follows immediately from Property 9 of Theorem 10.2.  We need only look at the statement
from a purely group-theoretic stand-point and also realize that $\phi$ will preserve the multiplication
product of the ring.

The eighth and last part of the theorem states that if $\phi$ is an isomorphism from $R$ onto $S$,
then $\phi^{-1}$ is an isomorphism from $S$ onto $R$.

Proof: Realize that $\ker\phi^{-1}$ is the trivial subring of $R$.  This part of the theorem then
follows from the seventh part of the theorem.

\section*{Exercise 2}

Prove Theorem 15.2.

Let $\phi$ be a homomorphism from a ring $R$ to a ring $S$.  Then
$\ker\phi=\{r\in R|\phi(r)=0\}$ is an ideal of $R$.

Proof: From group theory, we already know that $\ker\phi$ is
a normal subgroup of $R$.  Now let $r\in R$ and $x\in\ker\phi$.
Then $\phi(rx)=\phi(r)\phi(x)=\phi(r)\cdot 0=0\implies rx\in\ker\phi$.
Similarly, we have $xr\in\ker\phi$, so $\ker\phi$ is an ideal of $R$.

\section*{Exercise 3}

Prove Theorem 15.3.

Let $\phi$ be a ring homomorphism from $R$ to $S$.  Then the
mapping from $R/\ker\phi$ to $\phi(R)$, given by $r+\ker\phi\to\phi(r)$,
is an isomorphism.  In symbols, $R/\ker\phi\approx\phi(R)$.

Proof: By Theorem 10.3, $\phi$ is a group isomorphism from $R/\ker\phi$
to $\phi(R)$.  Now since $\ker\phi$ is an ideal, $R/\ker\phi$ is a factor ring
by Theorem 14.2.  What remains to be shown is that the mapping preserves
multiplication in $\phi(R)$.  To that end, see that for any pair of elements $x,y\in R$, we have
\begin{equation*}
\Psi(x+\ker\phi)\Psi(y+\ker\phi)=\phi(x)\phi(y)=\phi(xy)=\Psi(rs+\ker\phi),
\end{equation*}
where $\Psi:R/\ker\phi\to\phi(R)$ is the mapping given in the theorem's statement.

\section*{Exercise 4}

Prove Theorem 15.4.

Every ideal of a ring $R$ is the kernel of a ring homomorphism of $R$.
In particular, an ideal $A$ is the kernel of the mapping $r\to r+A$ from
$R$ to $R/A$.

Proof: Define $\phi(r)=r+A$ as the natural homomorphism from $R$ to $R/A$.
It is not hard to see that $\phi$ preserves both operations of $R$ in $R/A$.
Clearly, $\phi(r)=A$ if and only if $r\in A$, so $\ker\phi=A$.

\end{document}