\documentclass{ecgd-l}

% Who will be the copy right owner?
%\copyrightinfo{2013}{American Mathematical Society}

\newtheorem{theorem}{Theorem}[section]
\newtheorem{lemma}[theorem]{Lemma}

\theoremstyle{definition}
\newtheorem{definition}[theorem]{Definition}
\newtheorem{example}[theorem]{Example}
\newtheorem{xca}[theorem]{Exercise}

\theoremstyle{remark}
\newtheorem{remark}[theorem]{Remark}

\numberwithin{equation}{section}

\newcommand{\R}{\mathbb{R}}
\newcommand{\B}{\mathbb{B}}
\newcommand{\G}{\mathbb{G}}
\newcommand{\V}{\mathbb{V}}
\newcommand{\gd}{\dot{g}}
\newcommand{\gh}{\hat{g}}
\newcommand{\Gd}{\dot{G}}
\newcommand{\Gh}{\hat{G}}
\newcommand{\nvai}{\infty}
\newcommand{\nvao}{o}
\newcommand{\grade}{\mbox{grade}}

\begin{document}

\title{The Mother Minkowski Algebra of Order $m$}

\author{Spencer T. Parkin}
\address{102 W. 500 S., Salt Lake City, UT  84101}
\email{spencerparkin@outlook.com}

\subjclass[2010]{Primary 14J70}

% Date?
%\date{}

\begin{abstract}
Put abstract here.
\end{abstract}

\maketitle

\section{Motivation}

Before presenting the Mother Minkowski algebra of order $m$, we lead up to it here with
some background and motivation.  We begin by recalling that an algebraic set is any 
subset of an $n$-dimensional euclidean space $\R^n$ that is also the zero set of one
or more polynomials.  Given a geometric algebra $\G$, we can represents such sets
using blades $B\in\G$ as the set of all points $x\in\R^n$ such that
\begin{equation*}
p(x)\cdot B=0,
\end{equation*}
where $p:\R^n\to\V$ maps points in $\R^n$ to a vector space $\V$ generating
our geometric algebra $\G$.  Though not necessary, $\R^n$ is often embedded in $\V$;
but regardless of this, the function $p$ is necessarily defined in such a way that the expression
$p(x)\cdot B$ is a polynomial in the vector components of $x$ when $B\in\V$.

Letting $\B$ denote the set of all blades found in $\G$,
and letting $P(\R^n)$ denote the power set of $\R^n$,
we will find it useful to define the mapping $\gd:\B\to P(\R^n)$ as
\begin{equation}\label{equ_gd}
\gd(B) = \{x\in\R^n|p(x)\cdot B=0\}.
\end{equation}
To see that $\gd(B)$ is an algebraic set, we first observe that when $B\in\V$,
$\gd(B)$ is the zero set of a polynomial in the vector components of $x$.
Secondly, we observe that if $\bigwedge_{i=1}^k b_i$ is a factorization
of the $k$-blade $B$, each $b_i$ being in $\V$, then
\begin{equation*}
p(x)\cdot B = -\sum_{i=1}^k (-1)^i (p(x)\cdot b_i)B_i,
\end{equation*}
where $B_i$ is given by
\begin{equation*}
B_i = \bigwedge_{j=1,j\neq i}^k b_j,
\end{equation*}
and therefore, since $\{B_i\}_{i=1}^k$ is a linearly independent set, we have
\begin{equation*}
\gd(B) = \bigcap_{i=1}^k \gd(b_i).
\end{equation*}
This model of representing algebraic sets using blades of a geometric algebra presents
some interesting properties.  To begin, if $A,B\in\B$ are blades with $A\wedge B\neq 0$,
then
\begin{equation*}
\gd(A)\cap\gd(B)=\gd(A\wedge B).
\end{equation*}
In this way, the outer product serves to take the intersection of two surfaces.  But we can also
look at the outer product in a different light as an operator that takes at least the union
of its two given surfaces.  To see this, we must consider an alternative interpretation
of blades $B\in\B$ as algebraic sets.  Defining $\gh:\B\to P(\R^n)$ as
\begin{equation}\label{equ_gh}
\gh(B)=\{x\in\R^n|p(x)\wedge B=0\},
\end{equation}
we see that $\gh(B)=\gd(BI)$, where $I$ is the unit psuedo-scalar of $\G$, showing
that the image of $\gh$, like $\gd$, consists of algebraic sets.
Under this new interpretation, we find that for blades $A,B\in\B$, we have
\begin{equation*}
\gh(A)\cup\gh(B)\subseteq\gh(A\wedge B).
\end{equation*}
Exactly what surface we get from $A\wedge B$ in terms of $\gh$ can
be deduced by considering the surface $(A\wedge B)I$ in terms of $\gd$.

What's further a benefit of using blades to represent surfaces are the transformations
performable on such geometries through the use of outermorphisms; in particular,
outermorphisms $f:\B\to\B$ of the form
\begin{equation*}
f(B) = VBV^{-1},
\end{equation*}
where $V$ is a versor of $\G$.  Given such a function, we wish to
compare $\gd(B)$ with $\gd(f(B))$.  Interestingly, to understand
the latter in terms of the former, we need only understand the mapping from $\R^n\to\R^n$,
if any, induced by $V$ through $p$ as being each point $x\in\R^n$ mapped to a point
$y\in\R^n$ satisfying the condition
\begin{equation}\label{equ_induce_mapping}
V^{-1}p(x)V=\lambda p(y),
\end{equation}
$\lambda$ being some scalar in $\R$.
This is, of course, only a well defined mapping, provided that for every point $x\in\R^n$, there exists
such a point $y\in\R^n$, and that it is unique.
Assuming that $V$ and $p$ meet these requirements, and so do indeed induce
such a mapping $h:\R^n\to\R^n$, we can now show that
\begin{equation*}
\gd(f(B)) = h^{-1}(\gd(B)).
\end{equation*}
Notice that by the symmetry
of equation \eqref{equ_induce_mapping}, the same argument we
used to show that $h$ is a well defined
mapping can also be used to show that $h^{-1}$ exists.
We now need only show that
\begin{equation*}
\gd(VBV^{-1})=\{x\in\R^n|V^{-1}p(x)V\cdot B=0\}.
\end{equation*}
Do that here...

\section{The Mother Minkowski Algebra of Order $m$}

Up to this point, we have kept the definition of the function $p$
open to interpretation, because the set of all possibilities for $p$, in terms of the types of geometry
we can consequently do, remains an open question.  What might be the most interesting and significant
definition of $p$ thus far proposed is found in \cite{} and given by
\begin{equation*}
p(x)=\nvao + x + \frac{1}{2}x^2\nvai.
\end{equation*}
Here, the vector space $\V$ is generated by the set
of basis vectors $\{\nvao,\nvai\}\cup\{e_i\}_{i=1}^n$,
where the set of $n$ euclidean vectors $\{e_i\}_{i=1}^n$ span
$\R^n$ as an orthonormal basis for that space, and the
vectors $\nvao$ and $\nvai$ are the null vectors representing the
points at origin and infinity, respectively.  The geometric
algebra generated by $\V$ is called a Minkowski algebra, and the resulting model of
geometry imposed upon this algebra by $p$ using functions \eqref{equ_gd} and \eqref{equ_gh}
is known as the conformal model of geometric algebra.  It has been shown in \cite{} that
the versors of $\G$ generated by $\V$ induce the set of all conformal transformations through $p$.

Building upon the ideas presented in \cite{}, we will now consider
a new model of geometry based upon a geometric algebra $\G$
generated by a vector space $\V$ that is the external direct product
\begin{equation*}
\V = \bigoplus_{i=1}^m \V_i,
\end{equation*}
where for each $\V_i$, the geometric algebra generated by $\V_i$
is a Minkowski algebra.  For all $i\neq j$, we have $\V_i\cap\V_j=\{\vec{0}\}$,
the singleton set containing the zero vector. Moreover, for all $i\neq j$, if $a\in\V_i$
and $b\in\V_j$, we have $a\cdot b=0$.  We will refer to $\G$ as the mother Minkowski
algebra of order $m$.

Letting $\B$ denote the set
of all blades taken from $\G$, we now define the function $\Gd:\B\to P(\R^n)$ as
\begin{equation}\label{equ_Gd}
\Gd(B) = \left\{x\in\R^n\left|\bigwedge_{i=1}^m p_i(x)\cdot B=0\right\}\right.,
\end{equation}
where we define $p_i:\R^n\to\V_i$ as
\begin{equation}\label{equ_p_i}
p_i(x) = \nvao_i + x_i + \frac{1}{2}x_i^2\nvai_i,
\end{equation}
where $\nvao_i,\nvai_i\in\V_i$ are null vectors, and $x_i$ denotes the embedding
of $x$ in the $n$-dimensional euclidean sub-space of $\V_i$.  If more precision is needed here,
we can let $\B_i$ denote the set of all blades generated by $\V_i$,
let $\R_i^n$ denote the $n$-dimensional euclidean sub-space of $\V_i$, and then
work exclusively in $\R_1^n$ by defining an outermorphism that takes any blade in $\B_1$
to its corresponding blade in $\B_i$.  The function $p_i$ can then be defined in terms
of this outermorphism.  Interestingly, an explicit formula for this outermorphism can be found and carried through
all of the equations we'll present in the remainder of this paper, but there is no need to
formally introduce it, because the equations still go through in its absense.

%We now proceed to study the consequences of defining our new model of geometry
%based upon the function $\Gd$ given in equation \eqref{equ_Gd}, first by showing
%that we recover the conformal model of geometric algebra as a special case.

What is immediately clear from the definition of $\Gd$ in equation \eqref{equ_Gd}
is that unless for all integers $i\in[1,m]$, a vector
$v\in\V_i$ exists such that $v\wedge B=0$, we must have $\Gd(B)=\emptyset$.  We will therefore limit our attemtion
to those blades $B\in\B$ having factorizations involving a representative from each $\B_i$.
Doing so, we write $B$ as
\begin{equation*}
B = \bigwedge_{i=1}^m B_i,
\end{equation*}
where each $B_i$ is in $\B_i$, and then see that
\begin{equation}\label{equ_exp_def}
\bigwedge_{i=1}^m p_i(x)\cdot B = (-1)^k\bigwedge_{i=1}^m p_i(x)\cdot B_i,
\end{equation}
where the integer $k$ is given by
\begin{equation*}
k=\sum_{i=1}^m \grade(B_i)\left(\sum_{j=1,j\neq i}^m \grade(B_j)-m+i\right).
\end{equation*}
Subscripting equation \eqref{equ_gd} appropriately, what we now find is that
\begin{equation*}
\Gd(B) = \bigcup_{i=1}^m\gd_i(B_i).
\end{equation*}
This shows that we can represent any union of up to $m$ surfaces taken from the conformal model,
(let $B_i=\nvai_i$ to fill any remaining and unused blade factors),
but if we extend our function $\Gd$ to the set of all $m$-vectors, we can do even better.
To see why, we need only show that any monomial in up to $n$ variables and at most
degree $m$ can be represented by the expression on the right-hand side of equation \eqref{equ_exp_def}.
The $n$ variables are taken from the components of the point $x\in\R^n$.  Let each $B_i$ be a vector in $\V_i$
with $B_i\cdot\nvai_i=0$.
The expression then becomes
\begin{equation*}
(-1)^k\prod_{i=1}^m p_i(x)\cdot B_i.
\end{equation*}
For an appropriate choice of each vector $B_i$, we can formulate any monomial in the components of $x$.
Letting $B$ be a general $m$-vector, (not necessarily an $m$-blade), we see now that the expression
that is the left-hand side of equation \eqref{equ_exp_def} represents any polynomial of at most degree $m$
in the vector components of $x$.  Of course, if $B_i\cdot\nvai_i=0$ for not all vector factors of $B$, what
we get is a polynomial of at most degree $2m$, but we cannot represent all polynomials of up to this degree.

\section{Conformal Transformations}

While this new model certainly expands upon the set of all possible surfaces that may be represented by
the conformal model, not all of the nice properties discussed in the motivation section carry over very easily, if at all.
We will begin, however, to show that the all of the conformal transformations are available in the new model.
To that end...

\bibliographystyle{amsplain}
% Put the bib here!

\end{document}