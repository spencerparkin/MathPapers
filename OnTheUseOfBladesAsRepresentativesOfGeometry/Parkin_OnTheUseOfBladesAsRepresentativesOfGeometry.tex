\documentclass{birkjour}

\usepackage{tikz}
\usepackage{graphicx}
\usepackage{hyperref}

\newtheorem{thm}{Theorem}[section]
\newtheorem{cor}[thm]{Corollary}
\newtheorem{lem}[thm]{Lemma}
\newtheorem{prop}[thm]{Proposition}
\theoremstyle{definition}
\newtheorem{defn}[thm]{Definition}
\theoremstyle{remark}
\newtheorem{rem}[thm]{Remark}
\newtheorem*{ex}{Example}
\numberwithin{equation}{section}

\newcommand{\R}{\mathbb{R}}
\newcommand{\B}{\mathbb{B}}
\newcommand{\G}{\mathbb{G}}
\newcommand{\V}{\mathbb{V}}
\newcommand{\gd}{\dot{g}}
\newcommand{\gh}{\hat{g}}
\newcommand{\Gd}{\dot{G}}
\newcommand{\Gh}{\hat{G}}
\newcommand{\nvai}{\infty}
\newcommand{\nvao}{o}
\newcommand{\grade}{\mbox{grade}}

%\received{}\accepted{}

\begin{document}

\title{On The Use Of Blades As Representatives Of Geometry}

\author{Spencer T. Parkin}
\address{102 W. 500 S., \\
Salt Lake City, UT  84101} \email{spencerparkin@outlook.com}

%\subjclass{Primary 14J70; Secondary 14J29}

%\dedicatory{To my dear wife Melinda.}

\begin{abstract}
Abstract goes here...
\end{abstract}

\keywords{Key words go here...}

\maketitle

\section{Introduction And Motivation}

In many models of geometry that are based upon geometric algebra (see \cite{}), blades are used
to represent geometries.  Seeing a great deal of commonality between these models, a formal
treatment of this idea deserves to be given in an abstract setting in much the same way that,
for example, abstract algebra provides such a setting in which algebraic sets generated by ideals
of a polynomial ring can be studied.  To the best of this author's knowledge, this is the first treatment of its kind.

To keep our discussion from becoming too pedantic, the finer details upon which the major results of this paper will depend are given
in the last section.  Readers needing more familiarity with geometric algebra may want to read that section first.

\section{Enter The Geometric Set}

To lay the foundation of our work, we introduce $\V$ as an $m$-dimensional vector
space generating a geometric algebra denoted by $\G$.  We leave the signature of this geometric algebra
unspecified, but in cases where a proof depends upon signature, one is given as either
euclidean or non-euclidean.\footnote{The Gram-Schmidt orthogonalization process is applicable
to all blades taken from and only from a geometric algebra having no null-vectors.  While many proofs
are simplified under the assumption that an orthogonal basis can be chosen for any given blade, no such
assumption is made in this paper for the sake of generality.}
The scalars of $\V$, (and therefore of $\G$), are taken from the field $\R$ of real numbers.\footnote{To be more abstract,
we could have used any field with characteristic 1, but there will be no foreseable advantage to doing so in this paper.}
We will let $\R^n$ denote $n$-dimensional euclidean space,\footnote{Some models of geometry find affine space to be the natural
space within which to work, but this will not be the case in this paper.} and let $\B$ denote the set of all blades taken from $\G$.
Lastly, we will let $p:\R^n\to\V$ be an unspecified, yet well-defined function that we'll use in the following definition and
throughout the remainder of this paper.\footnote{By leaving $p$ unspecified, we're abstracting away the definition of the function.
We only care that it is a well-defined function.  In some parts of this paper, we will consider the cases where $p$ takes on some desirable
properties.}

\begin{defn}[Direct And Dual Representation]\label{def_blade_rep_geo}
For any blade $B\in\B$, we say that $B$ directly represents the set of all points $x\in\R^n$ such that
$p(x)\wedge B=0$, and say that $B$ dually represents the set of all points $x\in\R^n$ such that
$p(x)\cdot B=0$.  For convenience, we introduction the following functions using set-builder notation.
\begin{align*}
\gh(B) &= \{x\in\R^n|p(x)\wedge B=0\} \\
\gd(B) &= \{x\in\R^n|p(x)\cdot B=0\}
\end{align*}
\end{defn}
From Definition~\ref{def_blade_rep_geo}, it's important to take away the realization that a given blade $B\in\B$ represents two geometries
simultaneously; namely, $\gh(B)$ and $\gd(B)$.  Which geometry we choose to think of $B$ as being a representative of at any given time is completely
arbitrary.

It should also be clear from Definition~\ref{def_blade_rep_geo} that the geometry represented by a blade $B$, (directly or dually), remains invariant
under any non-zero scaling of the blade $B$.  Something interesting happens, however, when we take the dual of $B$, as Lemma~\ref{lem_dual_rep} will show.

\begin{lem}[Dual Relationship Between Representations]\label{lem_dual_rep}
For any subset $S$ of $\R^n$, if there exists $B\in\B$ such that $\gh(B)=S$, then $\gd(BI)=S$, where
$I$ is the unit psuedo-scalar of $\G$.  Similarly, if there exists $B\in\B$ such that $\gd(B)=S$, then $\gh(BI)=S$.
\end{lem}
\begin{proof}
The first of these two statements is proven by
\begin{equation*}
0=p(x)\wedge B=-(p(x)\cdot BI)I\iff p(x)\cdot BI=0,
\end{equation*}
while the second is proven by
\begin{equation*}
p(x)\cdot B=0\iff 0=(p(x)\cdot B)I=p(x)\wedge BI.
\end{equation*}
(See identities \eqref{equ_dual_of_v_dot_B} and \eqref{equ_dual_of_v_dot_dual_B} of Section~\ref{sec_useful_identities}.  Recall the zero-product property of the geometric product.)
\end{proof}

In other words, Lemma~\ref{lem_dual_rep} is telling us that for a single given geometry, the algebraic relationship between a
blade directly (dually) representative of that geometry, and a blade dually (directly) representative of that geometry, is simply
that, up to scale, they are duals of one another.

Of course, there will also be a geometric relationship between the geometry that is directly represented by a single given
blade $B\in\B$, and the geometry that is dually represented by $B$, but this depends upon the definition of our function
$p$, which we choose, in this paper, to leave open to speculation.

With Lemma~\ref{lem_dual_rep} in hand, geometric algebra's equivilant of an algebraic set may be given as follows.\footnote{If $p$ is
defined appropriately, geometric sets are algebraic sets.}

\begin{defn}[Geometric Set]\label{def_geo_set}
A subset $S\subset\R^n$ for which there exists a blade $B\in\B$ such that $\gh(B)=S$ is what we'll refer to as a ``geometric set.''
\end{defn}

By Lemma~\ref{lem_dual_rep}, it is easy to see that Definition~\ref{def_geo_set} is equivilant to a version of itself that replaces $\gh$ with $\gd$.
Therefore, for any geometric set $S$, we can claim the existence of a blade $B\in\B$ such that $\gh(B)=S$ or $\gd(B)=S$.

\section{Properties Of Geometric Sets}

Having now defined the notion of a geometric set, we now consider the properties of such sets.

\begin{thm}[The Intersection Of Two Geometric Sets Is A Geometric Set]
If $R,S\subset\R^n$ are a pair of geometric sets, then $R\cap S$ is also a geometric set.
\end{thm}
\begin{proof}
The case that $R=S$ is trivial.  So we assume that $R\neq S$.
Let $A,B\in\B$ be blades such that $\gd(A)=R$ and $\gd(B)=S$.
Then, since $R\neq S$, we have $A\wedge B\neq 0$.  (Not obvious!  Prove it!)
Go on...
\end{proof}

% need to cover intersections and union-like operations.

% need to cover existence and uniqueness of irreducible blades...introduce this in terms of converse of obvious statement: A=LB ==> gh(A)=gh(B).  (liken to nullzensnats)
% state as theorem.
% give algorithm for finding irreducible blade.

% cover transformations by versors and how that relates to understanding the action of a versor on p?
% p must have desirable property.  (form preserving versors)

% can take anything from CGAIntro.pdf?

\section{Transforming Geometric Sets}

% are these geo sets forming a group using inner automorphisms?

\section{Concluding Remarks}

% may want to ask what else might be missing; i don't know
% don't babble.

\section{Useful Identities And Lemmas}\label{sec_useful_identities}

In this section we give a number of useful algebraic identities and lemmas that would
otherwise distract us from the flow of the paper if given in the main body.\footnote{This
section is not intended as a complete or comprehensive review of geometric algebra.
See \cite{} for such a review.}

Letting $v\in\V$ and $B\in\B$, recall that
\begin{equation}\label{equ_v_geoprod_B_identity}
vB = v\cdot B+v\wedge B.
\end{equation}
Also recall that
\begin{align}
v\wedge B &= \frac{1}{2}(vB+(-1)^sBv),\label{equ_v_wedge_B_identity} \\
v\cdot B &= \frac{1}{2}(vB-(-1)^sBv),\label{equ_v_dot_B_identity}
\end{align}
where $s=\grade(B)$.
Realizing that $\grade(I)=m$, and that by \eqref{equ_v_geoprod_B_identity}, we have $vI=v\cdot I$, we can use
equation \eqref{equ_v_dot_B_identity} to establish the commutativity of vectors in $\V$ with the unit psuedo-scalar $I$ as
\begin{equation}\label{equ_v_commute_I}
vI = -(-1)^mIv.
\end{equation}
Using equation \eqref{equ_v_commute_I} in conjunction with equation \eqref{equ_v_dot_B_identity}, we find that
\begin{equation}\label{equ_dual_of_v_dot_B}
(v\cdot B)I = v\wedge BI.
\end{equation}
(In verifying this identity, it helps to realize that for any integer $k$, we have $(-1)^k=(-1)^{-k}$.)
Replacing $B$ in equation \eqref{equ_dual_of_v_dot_B} with $BI$, we find that
\begin{equation}\label{equ_dual_of_v_dot_dual_B}
v\wedge B = -(v\cdot BI)I.
\end{equation}

Recall that in the case that $v\wedge B$ vanishes, we can say that $v$ is
in the vector space spanned by any factorization of $B$.  Furthermore,
if $B$ is an $s$-blade, at most $s$ linearly independent vectors can be
found where for each such vector $v$, we have $v\wedge B=0$.
Lastly, if $v\wedge B=0$, then there exists a blade $A\in\B$, possibly
of grade zero, such that $B=v\wedge A$.

% The following notion is FALSE!  Don't make the mistake of believing it.
%For any non-zero blade $C\in\B$, if $A,B\in\B$ are blades such that
%$C=A\wedge B$, then for all $v\in\V$ such that $v\wedge C=0$, we have
%$v\wedge A=0$ or $v\wedge B=0$, exclusively.

\begin{lem}
If $B\in\B$ is a non-zero blade of grade $s>0$ and $\{b_i\}_{i=1}^s$ is a linearly
independent set of $s$ vectors such that for all $v\in\{b_i\}_{i=1}^s$, we have
$v\wedge B=0$, then there exists $\lambda\in\R$ such that
\begin{equation*}
B = \lambda\bigwedge_{i=1}^s b_i.
\end{equation*}
\end{lem}
\begin{proof}
The case $s=1$ is trivial.  We now proceed by induction on $s>1$.
Seeing that $b_s\wedge B=0$, there exists $A\in\B$ of positive grade
such that $B=A\wedge b_s$.  Go on...
% Can we claim that the rest of the vectors are in A?
\end{proof}

Returning to the product $v\cdot B$ given in equation \eqref{equ_v_dot_B_identity},
an alternative expansion is given by
\begin{equation}\label{equ_v_dot_B_sum}
v\cdot B = -\sum_{i=1}^s (-1)^i(v\cdot b_i)B_i,
\end{equation}
where $B$ is factored as $\bigwedge_{i=1}^s b_i$, and we define $B_i$ as
\begin{equation*}
B_i = \bigwedge_{j=1,j\neq i}^s b_i.
\end{equation*}
This leads to the following recursive formulation.
\begin{equation*}
v\cdot B=(v\cdot b_1)B_1-b_1\wedge(v\cdot B_1)
\end{equation*}
If a blade $A\in\B$ has grade $r$ and factorization $\bigwedge_{i=1}^r a_i$, then
we can express the product $A\cdot B$ recursively as
\begin{equation*}
A\cdot B = \left\{\begin{array}{ll}
A_r\cdot (a_r\cdot B) & \mbox{if $r\leq s$,} \\
(A\cdot b_1)\cdot B_1 & \mbox{if $r\geq s$.}
\end{array}\right.
\end{equation*}

Interestingly, though it is not at all obvious from equation \eqref{equ_v_dot_B_sum}, the product
$v\cdot B$ is a blade.  It is clearly homogeneous of grade $s-1$, but it is not
immediately clear that it is a blade.  To see that it is a blade, let $\beta=\prod_{i=1}^s v\cdot b_i$,
and let $\beta_i$ be given by
\begin{equation*}
\beta_i=\prod_{j=1,j\neq i}v\cdot b_j.
\end{equation*}
(If $\beta=0$, the following argument can be reduced to a smaller yet similar case; so
we assume $\beta\neq 0$.)
Then, letting $v_i=\beta_1b_1-(-1)^i\beta_ib_i$, notice that for all integers $1<i\leq s$,
we have
\begin{equation*}
v_i\wedge (v\cdot B)=\beta B-\beta B=0.
\end{equation*}
Seeing now that the linear independence of the set of $s-1$ vectors
$\{v_i\}_{i=2}^s$ follows from that of the set of $s$ vectors $\{b_i\}_{i=1}^s$,
we can invoke Lemma~\ref{} in claiming that for some non-zero scalar $\lambda\in\R$,
we have
\begin{equation*}
v\cdot B = \lambda\bigwedge_{i=2}^s v_i,
\end{equation*}
showing that $v\cdot B$ is indeed a blade of grade $s-1$.

\begin{lem}\label{lem_B_i_lin_indep}
For a non-zero $s$-blade $B$ factored as $\bigwedge_{i=1}^s b_i$, the set of
$(s-1)$-blades $\{B_i\}_{i=1}^s$ is linearly independent.
\end{lem}
\begin{proof}
Suppose there exists a non-trivial set of $s$ scalars $\{\beta_i\}_{i=1}^s$
such that $0 = \sum_{i=1}^s\beta_i B_i$.
Then, without loss of generality, suppose that $\beta_s\neq 0$, and rearrange
our equation as $-\beta_sB_s=\sum_{i=1}^{s-1}\beta_iB_i$.  Now notice
that while $b_s\wedge-\beta_sB_s\neq 0$, we have $b_s\wedge\sum_{i=1}^{s-1}\beta_iB_i=0$,
which is a contradiction.
\end{proof}

Note that a perhaps more elegant proof of Lemma~\ref{lem_B_i_lin_indep} could have been given
under the assumption
of a euclidean signature; in which case, the Gram-Schmidt orthogonalization process would have
allowed us to choose, without loss of generality, an orthogonal factorization of the blade $B$.
Doing so, $B$ becomes the versor $\prod_{i=1}^s b_i$, and we may write
\begin{equation*}
0 = \sum_{i=1}^s \beta_iB_i \iff
0=\left(\sum_{i=1}^s\beta_iB_i\right)B = -\sum_{i=1}^s(-1)^{s-i}B_i^2\beta_ib_i.
\end{equation*}


% may want (a.b)=(VaV^-1).(VbV^-1) lemma

\begin{thebibliography}{9}

\end{thebibliography}

\end{document}