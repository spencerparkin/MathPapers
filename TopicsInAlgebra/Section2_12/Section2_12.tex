\documentclass[12pt]{article}

\usepackage{amsmath}
\usepackage{amssymb}
\usepackage{amsthm}

\title{Section 2.12 Exercises\\Herstein's Topics In Algebra}
\author{Spencer T. Parkin}

\newtheorem{theorem}{Theorem}[section]
\newtheorem{definition}{Definition}[section]
\newtheorem{corollary}{Corollary}[section]
\newtheorem{identity}{Identity}[section]
\newtheorem{lemma}{Lemma}[section]
\newtheorem{result}{Result}[section]

%\newcommand{\gcd}{\mbox{gcd}}
\newcommand{\lcm}{\mbox{lcm}}
\newcommand{\Z}{\mathbb{Z}}
\newcommand{\cl}{\mbox{Cl}}

\begin{document}
\maketitle

\section*{Thoughts On First Proof Of Sylow's First Theorem}

It was left to the reader to show that $|G|=n|H|$.
What we want to show is that $H$ has exactly $n$ right
cosets in $G$.  Since $H=\{g\in G|M_1g=M_1\}$, we see
that the every right-coset of $H$ in $G$ can be written as
\begin{equation*}
Ha=\{g\in G|M_1ga^{-1}=M_1\}=\{g\in G|M_1g=M_1a\}.
\end{equation*}
But now $M_1a=M_i$ for some $1\leq i\leq n$, and as $a$ ranges over $G$,
we get every $M_i$ in $\{M_i\}_{i=1}^n$; so there are exactly $n$ such cosets.

\section*{Other Thoughts}

At the end, he uses Lemma 2.12.5 to say that the number of $p$-Sylow subgroups,
which number has the form $1+kp$, must divide the order of the group.  This is not immediately obvious to me.

That this would be the case seems to follow from the proof of Theorem 2.12.3.

\section*{Problem 11}

Let $|G|=pq$, $p$ and $q$ distinct primes, $p<q$.

\subsection*{Part A}

Show that if $p$ doesn't divide $q-1$, then $G$ is cyclic.

Admittedly, without the sample analyses Herstein gives at the end of the chapter,
I couldn't've figured this one out.

By Theorem 2.12.3, we have $1+kq$, for some integer $k$, as the number of $q$-Sylow subgroups of $G$,
and $1+kq$ divides $pq$.
Now since
\begin{equation*}
(1+kq)(1) + q(-k) = 1,
\end{equation*}
we see that $\gcd(1+kq,q)=1$, and therefore, $1+kq$ divides $p$.  But now $p<q$,
so we must have $k=0$ and the number of $q$-Sylow subgroups of $G$ is one.

Turning our attention now to the number of $p$-Sylow subgroups of $G$, we see, again by
Theorem 2.12.3, that there must be, for some integer $k$, $1+kp$ of them, and this divides $pq$.
Again, it is easy to show that $1+kp$ and $p$ are coprime; therefore, $1+kp$ must divide $q$,
and so we write
\begin{equation*}
r(1+kp)=q.
\end{equation*}
Now, if $k>0$, we must have $r=1$ since $q$ is prime.  We then have
\begin{equation*}
kp = q - 1,
\end{equation*}
but, by hypothesis, $p$ does not divide $q-1$, so $k=0$ and the number of $q$-Sylow
subgroups of $G$ is just one.

What we know now is that there are at most $p-1$ elements of order $p$,
and $q-1$ elements of order $q$.  But this doesn't account for all non-identity elements; specifically,
\begin{equation*}
(pq-1)-(p-1)-(q-1)=pq-p-q+1>0.
\end{equation*}
It follows that there must exist an element of order $pq$; hence, $G$ is cyclic.

\subsection*{Part B}

Show that if $p|(q-1)$, then there exists a unique non-abelian group of order $pq$.

I have no idea.  Herstein also brings this up in section 10, problem 10 where more clues can be found.
Clearly there is an existence and then a uniqueness portion to this.  First show existence, then
uniqueness through isomorphism.

We might consider the external direct product of two cyclic groups: one of order $p$, the
other of $q$.  This group has the right order.  Is it non-abelian?  No, it's abelian.

\end{document}