\documentclass[12pt]{article}

\usepackage{amsmath}
\usepackage{amssymb}
\usepackage{amsthm}
\usepackage{graphicx}
\usepackage{float}
\usepackage{centernot}

\title{The Orbit-Shuffler Theorem}
\author{Spencer T. Parkin}

\newtheorem{theorem}{Theorem}[section]
\newtheorem{definition}{Definition}[section]
\newtheorem{corollary}{Corollary}[section]
\newtheorem{identity}{Identity}[section]
\newtheorem{lemma}{Lemma}[section]
\newtheorem{result}{Result}[section]

\begin{document}
\maketitle

We work here to prove a generalization of the Orbit-Stabilizer Theorem.
Let $G$ be a finite permutation group with each $g\in G$ defined over the set $\Omega$.
We say that $G$ is a subgroup of the symmetric group of $\Omega$.
Given a non-trivial subset $P$ of $\Omega$, we define, for any $g\in G$,
\begin{equation*}
P^g = \{p^g|p\in P\}.
\end{equation*}
We also define
\begin{equation*}
G^P = \{g\in G|P^g=P\},
\end{equation*}
and call this the shuffler of $P$ in $G$.  Of course, when $P$ is a singleton set,
this is the stabilizer of the sole member of $P$ in $G$.

\begin{lemma}
$G^P$ is a subgroup of $G$.
\end{lemma}
\begin{proof}
Closure being trivial, we need only show that for any $a\in G^P$, we also have
$a^{-1}\in G^P$.  To do this, we must convince ourselves that since $a$ is a bijection,
we must have $P^a=P$ if and only if $P=P^{a^{-1}}$.
\end{proof}

\begin{lemma}
For any pair of elements $a,b\in G$, we have
\begin{equation*}
P^a=P^b\implies P^{ab^{-1}}=P.
\end{equation*}
\end{lemma}
\begin{proof}
If $p\in P$, then $p^b\in P^b=P^a$ means there exists $q\in P$
such that $p^b=q^a$.  Then $q^{ab^{-1}}=p\implies p\in P^{ab^{-1}}$.
On the other hand, let $p\in P^{ab^{-1}}$.  So there exists $q\in P$ such
that $q^{ab^{-1}}=p$.  Now $q^a\in P^a=P^b\implies p^b\in P^b$.
So there is $r\in P$ such that $p^b=r^b\implies p=r\in P$.
\end{proof}

We now define
\begin{equation*}
P^G = \{P^g|g\in G\},
\end{equation*}
and say that $P^G$ is the orbit of $P$ in $G$.
We can now prove the following theorem.
\begin{theorem}[Orbit-Shuffler Theorem]
A relationship between $|G|$, $|G^P|$ and $|P^G|$ is given by
\begin{equation*}
|G|=|G^P||P^G|.
\end{equation*}
\end{theorem}
\begin{proof}
Letting $T$ be a right-transversal of $G^P$ in $G$, we simply find a bijection
between $T$ and $P^G$.  Map $t\in T$ to $P^t$.  Let $a,b\in T$ such that
$P^a=P^b$.  It follows that $P^{ab^{-1}}=P$, and therefore, $ab^{-1}\in G^P$.
In turn, we have $G^Pa=G^Pb\implies a=b$, since $a$ and $b$ are taken from $T$.
\end{proof}

Defining
\begin{equation*}
G^{[P]}=\{g\in G|\mbox{$p^g=p$ for all $p\in P$}\},
\end{equation*}
it is worth noting where the proof of our theorem would fail
if we replaced $G^P$ with $G^{[P]}$.  Where we run in to trouble
is the implication
\begin{equation*}
P^{ab^{-1}}=P\centernot\implies ab^{-1}\in G^{[P]}.
\end{equation*}

% Can we say that G^P = G^{Omega-P}?  Is this helpful?
%Thoughts...  Would it be useful to build a shuffler chain--the analog of a stabilizer chain?
%In some cases, G^[P] set is a normal subgroup, and I was hoping to find a transversal
%for it.  The idea is that I already have a stabilizer chain for the associated homomorphic image.
%A stabilizer for the kernel would then complete all the information we need to say we have
%a stabilizer chain for the entire group.

\end{document}