\documentclass[12pt]{article}

\usepackage{amsmath}
\usepackage{amssymb}
\usepackage{amsthm}
\usepackage{graphicx}
\usepackage{float}

\title{Examination of Equ. 14.6}
\author{Spencer T. Parkin}

\newcommand{\G}{\mathbb{G}}
\newcommand{\V}{\mathbb{V}}
\newcommand{\R}{\mathbb{R}}
\newcommand{\B}{\mathbb{B}}
\newcommand{\nvao}{o}
\newcommand{\nvai}{\infty}

\newtheorem{theorem}{Theorem}[section]
\newtheorem{definition}{Definition}[section]
\newtheorem{corollary}{Corollary}[section]
\newtheorem{identity}{Identity}[section]
\newtheorem{lemma}{Lemma}[section]
\newtheorem{result}{Result}[section]

\begin{document}
\maketitle

On page 403, it is claimed that the direct form of a real round is given by
\begin{equation*}
\Sigma = \left(\nvao+\frac{1}{2}\rho^2\nvai\right)A_k,
\end{equation*}
where $A_k$ is a purely Euclidean blade.  I will show that this is \emph{also} the
dual form of an imaginary round.  Consider the simple case of $k=1$,
and let $A_k$ be a unit Euclidean vector.  If $\Sigma$ as a
dual imaginary round, then there cannot exist a Euclidean vector $x$
such that $p(x)\cdot\Sigma=0$.  Letting $\sigma=\nvao+\frac{1}{2}\rho^2\nvai$,
we see that
\begin{align*}
p(x)\cdot\Sigma &= p(x)\cdot(\sigma\wedge A_k) \\
 &= (p(x)\cdot\sigma)A_k - (p(x)\cdot A_k)\sigma.
\end{align*}
Now, since $\Sigma\neq 0$, it is clear that $A_k$ and $\sigma$ are linearly independent,
so that $p(x)\cdot\Sigma=0$ if and only if $p(x)\cdot\sigma=0$ and $p(x)\cdot A_k=0$.
Then since $A_k$ is a real dual plane, clearly there exists $x$ such that $p(x)\cdot A_k=0$.
However, notice that $\sigma$ is an imaginary dual round!  It follows that there does not
exist $x$ such that $p(x)\cdot\sigma=0$ and therefore there is no $x$ such that $p(x)\cdot\Sigma=0$,
which is what we wanted to show.

A similar argument can be given for the cases when $k>1$.

\end{document}