\documentclass{article}
\usepackage{amsmath}
\usepackage{amssymb}
\title{Essential Theorems in Abstract Algebra}
\author{Spencer}
\addtolength{\oddsidemargin}{-.575in}
\addtolength{\evensidemargin}{-.575in}
\addtolength{\textwidth}{1.0in}
\addtolength{\topmargin}{-.575in}
\addtolength{\textheight}{1.25in}
\begin{document}
\maketitle

\newcommand{\Z}{\mathbb{Z}}
\newcommand{\N}{\mathbb{N}}
\newcommand{\Q}{\mathbb{Q}}
\newcommand{\Ker}{\mbox{Ker}\,}
\newcommand{\Char}{\mbox{char}\,}

\section*{The One-Step Subgroup Test}

A nonempty subset $H$ of a group $G$ is a subgroup
of $G$ if $ab^{-1}$ is in $H$ whenever $a$ and $b$
are in $H$.

\section*{The Fundamental Theorem of Cyclic Groups}

Every subgroup of a cyclic group is cyclic.
If $|\langle a\rangle|=n$, then for each
positive divisor $k$ of $n$, $\langle a\rangle$
has exactly one subgroup of order $k$ and no others.

If a group is of prime order, then it is cyclic.
This is easy to see since by Lagrange's Theorem,
the order of every element must divide the order
of the group.  Notice that any nonidentity element
must be a generator of a group of prime order.

All cyclic groups are Abelian.  This is easy to
see since all cyclic groups are isomorphic to
one of the $\Z_n$ groups, which are Abelian.

\section*{Lagrange's Theorem}

In a finite group the order of a subgroup divides the
order of the group.  The converse of this theorem applies
to all Abelian groups.  The Sylow Theorem's provide
partial converses to this theorem for non-Abelian groups.

\section*{Normal Subgroup Test}

A subgroup $H$ of $G$ is normal in $G$ if and only if
$xHx^{-1}\subseteq H$ for all $x\in G$.

A normal subgroup is a subgroup $H$ of $G$ where
$aH=Ha$ for all $a\in G$.  That is, the left cosets
are equal to the right cosets.  This is needed to
prove that the coset multiplication operation is well defined
on factor groups.  If $H$ is normal in $G$, then
$G/H$ is a factor group with elements $aH$ for all $a\in G$.
Two cosets $aH$ and $bH$ are equal if and only if $a^{-1}b\in H$.

Recall that $xHx^{-1}$ denotes the set $\{xhx^{-1}:h\in H\}$.

Notice that all subgroups of an Abelian group are normal.

\section*{The $G/Z$ Theorem}

Let $G$ be a group and let $Z(G)$ be the center of $G$.
If $G/Z(G)$ is cyclic, then $G$ is Abelian.

Recall that $Z(G)$ denotes the subgroup of $G$ consisting
of all elements of $G$ that commute with all other
elements of $G$.  That is, $Z(G)=\{a\in G:\mbox{$ax=xa$ for all $x\in G$}\}$.

This theorem is not hard to prove.  The contrapositive
is used most often in practice.

\section*{The Number of Elements in $HK$}

Let $H$ and $K$ be subgroups of a finite group $G$.
It then follows that
\begin{equation*}
|HK|=\frac{|H||K|}{|H\cap K|}.
\end{equation*}
It would be a good idea to prove this for yourself.
Notice that $HK$ is not always a subgroup of $G$.
If it can be shown that $HK=KH$, then $HK$ is
a subgroup of $G$.

\section*{First Isomorphism Theorem}

If $\phi$ is a group homomorphism from $G$ to a group, then
$G/\Ker\phi\approx\phi(G)$.

A group homomorphism is a function from one group into another
that is operation preserving.  That is $\phi(ab)=\phi(a)\phi(b)$
for all $a,b\in G$, assuming $\phi$ is defined on the group $G$.
Notice that the operation in $\phi(ab)$ is that of $G$ and
the operation in $\phi(a)\phi(b)$ is that of $\bar{G}$, assuming
$\phi$ is a $\bar{G}$ valued function, where $\bar{G}$ is some group.

The kernal of a homomorphism $\phi:G\to\bar{G}$, denoted
$\ker\phi$ is the set $\{x\in G:\phi(x)=\bar{e}\}$, where
$\bar{e}$ is the identity element of $\bar{G}$.  Kernels
of homomorphisms are normal subgroups.  The factor group
made by moding about by the kernel of a homomprhism is
isomorphic to the image of the homomorphism.  Homomorphisms
are tools that can be used to gain insight into the structure
of a group.

Let $a,b\in G$.  Recall that $\phi(a)=\phi(b)$
if and only if $a\Ker\phi=b\Ker\phi$ if and only
if $a^{-1}b\in\Ker\phi$.

Also recall that if $\phi(g)=g'$, then
$\phi^{-1}(g')=\{x\in G:\phi(x)=g'\}=g\Ker\phi$.

Lastly, recall that if $|g|$ is finite, then
$|\phi(g)|$ divides $|g|$.

The order of the image of a homomorphism divides the
order of homomorphisms domain and codomain.

Every normal subgroup is the kernel of some homomorphism.
In fact, if $N$ is a normal subgroup of $G$, then the
kernal of the homomorphism defined by $\phi(g)=gN$
has $N$ as its kernel.  Notice that $\phi$ maps
elements from $G$ to the factor group $G/N$.
So the kernel of $\phi$ is all $g\in G$
such that $gH=H\implies g\in H$.

\section*{The Fundamental Theorem of Finite Abelian Groups}

Every finite Abelian group is a direct product of cyclic groups of
prime power order.  To classify non-Abelian groups, we need the
Sylow Theorems.

\section*{The Subring Test}

A nonempty subset $S$ of a ring $R$ is a subring if
$a-b$ and $ab$ are in $S$ whenever $a$ and $B$ are in $S$.

The first part is the One-Step Subgroup Test, making sure
that $S$ is a subgroup $R$, since we can think of rings,
under just the addition operation, as Abelian groups.
The second part makes sure that $S$ is closed under the
multiplication of $R$.

Recall that rings are not in general commutative and
the multiplication of rings is left and right
distributive over addition.  Rings also need not
have a multiplicative identity and even with such
an elements, not all nonzero ring elements
need have a multiplicative inverse.

\section*{The Ideal Test}

A nonempty subset $A$ of a ring $R$ is an ideal if $a-b\in A$
whenever $a$ and $b$ are in $A$; and $ra$ and $ar$
are in $A$ whenever $a\in A$ and $r\in R$.

Again, the first part here is the One-Step Subgroup Test.
The second part verifies not only closure, but that the
subring absorbs elements of $R$, making it an ideal.
We can take rings and modout by ideals just as we can
take groups and modout by normal subgroups.

\section*{The Characteristic of an Integral Domain}

The characteristic of an integral domain is zero or prime.

An integral domain is a commutative ring with unity and
no zero divisors.  If the product of two nonzero elements
is zero, then each of these is a zero divisor.
It is easy to show that integral domains have the cancelation
property.  That is, if $ab=ac\implies b=c$, assuming $a\neq 0$.
Notice that all fields are integral domains.

The characteristic of an integral domain $R$, denoted
$\Char R$, is the least positive integer $n$ such that
all elements raised to the additive power of $n$ are the
additive identity.  If no such integer exists, then
$\Char R=0$.

We have a theorem that states that finite integral domains
are fields.  Infinite integral domains are not always fields.
Recall that $\Z_p$, where $p$ is prime, is a field under
addition and multiplication modulo $p$.

\section*{$R/A$ is a Field if and only if $A$ is Maximal}

Let $R$ be a commutative ring with unity and let $A$
be an ideal of $R$.  Then $R/A$ is a field if and only if
$A$ is maximal.

A field is a commutative ring with unity where all nonzero
elements are units (have multiplicative inverses).

An ideal $A$ is maximal in a ring $R$ if the only
proper ideal containing $A$ is $R$.  A common way
to show that a given subring $A$ is a maximal ideal is to let
there be a proper ideal $A'$ containing $A$, show
that $A'$ contains 1, and then it follows that $A'=R$.

\section*{First Isomorphism Theorem for Rings}

If $\phi$ is a ring homomorphism from $R$ to a ring,
then $R/\Ker\phi\approx\phi(R)$.

A homomorphism is a function from one ring into
another that preserves both ring operations.

Ring homomorphisms map subrings to subrings.
The pullbacks of ideals under homomorphisms are ideals.

A lot of generalizations from The First Isomorphism
Theorem for Groups carry over to rings.

\section*{$F[x]/\langle p(x)\rangle$ is a Field}

Let $F$ be a field and $p(x)$ an irreducible polynomial over $F$.
Then $F[x]/\langle p(x)\rangle$ is a field.

Recall that $F[x]$ denotes the ring of polynomials over
$F$.  That is, all polynomials having coeficients taken
from $F$.  Notice that $F[x]$ is not a field because
not all nonzero elements have multiplicative inverses.
If $F$ is a field, then $F[x]$ is a principle ideal domain.
A prinpcle ideal domain $R$ is an integral domain, (which
is a commutative ring), such that every ideal
has the form $\langle a\rangle=\{ra:r\in R\}$.

An irreducible polynomial $p(x)$ in a polynomial ring $F[x]$
is a polynomial that cannot be factored into the product
of two or more polynomials having positive degree.
For a polynomial $p(x)\in F[x]$, we have a theorem that
tells us that $\langle p(x)\rangle$ is a maximal ideal
in $F[x]$ if and only if $p(x)$ is irreducible over $F$.
This is how the above theorem is proved.

Notice that, in a sense, $F[x]/\langle p(x)\rangle$
contains $F$ as a subfield.

\section*{The Fundamental Theorem of Field Theory}

Let $F$ be a field and let $f(x)$ be a nonconstant
polynomial in $F[x]$.  Then there is an extension field
$E$ of $F$ in which $f(x)$ has a zero.

What we do is start with a polynomial in $F[x]$ of
degree $n$ that has less than $n$ zeros in $F$.
Choose an irreducible factor $p(x)$ of $f(x)$.
We want to extend $F$ to a field $E$ containing
$F$ such that $p(x)$ as at least one zero.
We claim that $E\approx F[x]/\langle p(x)\rangle$
is such a field.  To show that this is so,
see that $x+\langle p(x)\rangle$ is a
zero of $p(x)$ in $F[x]/\langle p(x)\rangle$.

\section*{$[K:F]=[K:E][E:F]$}

If $K$ is a finite extension field of a field $E$ and $E$ is
a finite extension field of the field $F$, then
$[K:F]=[K:E][E:F]$.

Recall that $[E:F]$ denotes the dimension $E$ has as a vector
space over $F$.  It is said to be the degree $E$ has over $F$
as an extension field of $F$.  Say we modout by an irreducible
polynomial $p(x)$ in $F[x]$ of degree $n$ to get $E$.
Then for some zero $a$ of $p(x)$ in $E$, we can express
$E$ as a vector space over $F$ having basis $\{1,a,a^2,\dots,a^{n-1}\}$.

\section*{The Structure of Finite Fields}

The set of nonzero elements of a finite field is a cyclic
group under multiplication.

If $F$ is a finite field, we usually let $F^*$ denote
the cyclic group $F\backslash\{0\}$ under multiplication.

\section*{Sylow's First Theorem}

Let $G$ be a finite group and $p$ a prime.
If $p^k$ divides $|G|$, then $G$ has a subgroup of order $p^k$.

\section*{Sylow's Second Theorem}

If $H$ is a subgroup of a finite group $G$ and $|H|$ is a power
of a prime $p$, then $H$ is contained in some Sylow $p$-subgroup
of $G$.

\section*{Sylow's Third Theorem}

The number of Sylow $p$-subgroups of $G$ is congruent
to one modulo $p$ and divides $|G|$.  Furthermore,
any two Sylow $p$-subgroups of $G$ are conjugate.

A Sylow $p$-subgroup of $G$ is a subgroup of $G$
having order $p^k$ where $p^{k+1}$ does not divide $|G|$.
This means that $k$ is as large as possible.

Two subgroups $H$ and $K$ of a group $G$ are conjugates
of one another if there exists $g\in G$ such that
$H=gKg^{-1}$.
All conjugate pairs are isomorphic to one another.

If it can be shown that there is only one Sylow $p$-subgroup
for a given $p$, then it is a normal subgroup.

\end{document}