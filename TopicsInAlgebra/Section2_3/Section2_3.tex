\documentclass[12pt]{article}

\usepackage{amsmath}
\usepackage{amssymb}
\usepackage{amsthm}

\title{Section 2.3 Exercises\\Hertein's Topics In Algebra}
\author{Spencer T. Parkin}

\newtheorem{theorem}{Theorem}[section]
\newtheorem{definition}{Definition}[section]
\newtheorem{corollary}{Corollary}[section]
\newtheorem{identity}{Identity}[section]
\newtheorem{lemma}{Lemma}[section]
\newtheorem{result}{Result}[section]

%\newcommand{\gcd}{\mbox{gcd}}
\newcommand{\lcm}{\mbox{lcm}}
\newcommand{\Z}{\mathbb{Z}}

\begin{document}
\maketitle

\section*{Problem 8}

If $G$ is a finite group, show that threre exists a postive integer $N$ such that $a^N=e$
for all $a\in G$.

It is not hard to show that for every $a\in G$, there exists an integer $k(a)$ such that $a^{k(a)}=e$.
Further, $a^{nk(a)}=e$ for every integer $n$.  Now let
\begin{equation*}
N=\prod_{a\in G} k(a).
\end{equation*}

\section*{Problem 11}

If $G$ is a group of even order, prove it has an element $a\neq e$ satisfying $a^2=e$.

Clearly $G$ has an odd number of non-identity elements.  Pluck such an element from $G$.
If $a^2=e$, we're done.  If not, pluck its inverse out of $G$ as well.  This leaves us yet
a smaller pool of odd elements to choose from.  Continue this process until we either find an
element being its own inverse, or we're left with just one non-identity element.  Clearly this
last remaining non-identity element must be its own inverse.

\section*{Exercise 26}

\subsection*{Part (a)}

Let $G$ be the group of all $2\times 2$ matrices $\left(\begin{array}{cc}a&b\\c&d\end{array}\right)$
where $a$, $b$, $c$, $d$ are integers modulo $p$, $p$ a prime number, such that $ad-bc\neq 0$.
$G$ forms a group relative to matrix multiplication.  What is $o(G)$?

We find $|G|$ by counting matrices of the said dimension and then subtracting from that the number of
such being singular.  Clearly there are $p^4$ matrices of the desired dimension.  How many of them are singular?
The singularity of a $2\times 2$ matrix occurs whenever a row (column) is a scalar multiple of the other row (column).
Consider the following expression.
\begin{equation*}
(1)p^2 + (p-1)p + (p-1)p + (p-1)^2p.
\end{equation*}
This expression, having 4 terms, represents 4 cases.  In the first case, one row is zero, leaving $p^2$ choices
for the other row.  In the second case, a row is axis-aligned in $p-1$ ways with $p$ ways the other row
is zero or parallel to it.  The third cases is counted like the second, but using the other axis.  In the fourth case,
there are $(p-1)^2$ ways a row is non-zero and non-axis-aligned with $p$ ways the other row is zero or parallel to it.

Putting it all together, we get
\begin{equation*}
|G| = p^4 - p^3 - p^2 + p.
\end{equation*}

\subsection*{Part (b)}

Let $H$ be the subgroup of the $G$ of part (a) defined by
\begin{equation*}
H = \left\{\left.\left(\begin{array}{cc}a&b\\c&d\end{array}\right)\in G\right|ad-bc=1\right\}.
\end{equation*}
What is $o(H)$?

Figure it out...

\end{document}