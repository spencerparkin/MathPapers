\documentclass[12pt]{article}

\usepackage{amsmath}
\usepackage{amssymb}
\usepackage{amsthm}

\title{Chapters 5-8 Supplementary Exercises\\Gallian's Book on Abstract Algebra}
\author{Spencer T. Parkin}

\newtheorem{theorem}{Theorem}[section]
\newtheorem{definition}{Definition}[section]
\newtheorem{corollary}{Corollary}[section]
\newtheorem{identity}{Identity}[section]
\newtheorem{lemma}{Lemma}[section]
\newtheorem{result}{Result}[section]

%\newcommand{\gcd}{\mbox{gcd}}
\newcommand{\lcm}{\mbox{lcm}}
\newcommand{\abs}{\mbox{abs}}
\newcommand{\Z}{\mathbb{Z}}
\newcommand{\R}{\mathbb{R}}
\newcommand{\cl}{\mbox{cl}}

\begin{document}
\maketitle

\section*{Problem 1}

A subgroup $N$ of a group $G$ is called a {\it chracteristic subgroup} if $\phi(N)=N$
for all automorphisms $\phi$ of $G$.  Prove that every subgroup of a cyclic group is characteristic.

Let $\phi$ be an automorphism of $G$, a cyclic group.
Let $a$ be an element of $G$.  Now see that
\begin{equation*}
\phi(\langle a\rangle)=\{\phi(a^k)|k\in\Z\}=\{\phi^k(a)|k\in\Z\}=\langle\phi(a)\rangle.
\end{equation*}
Clearly, $|a|=|\phi(a)|$, so $|\langle a\rangle|=|\langle\phi(a)\rangle|$.  We can
now claim that $\langle a\rangle=\langle \phi(a)\rangle$ by the fundamental
theorem of cyclic groups, because $G$ has one and only one subgroup of each
possible order.

\section*{Problem 2}

Prove that the center of a group is characteristic.

Let $\phi$ be any automorphism of a group $G$.
Letting $a$ be an element in $\phi(Z(G))$ and $g$ an element in $G$,
there must exist an element $a'\in Z(G)$ and an element $g'\in G$ such
that $\phi(a')=a$ and $\phi(g')=g$.  It then follows that
\begin{equation*}
ag=\phi(a')\phi(g')=\phi(a'g')=\phi(g'a')=\phi(g')\phi(a')=ag,
\end{equation*}
showing that $a\in Z(G)$.
Thus far we have shown that $\phi(Z(G))\subseteq Z(G)$.
But $\phi$ is one-to-one, so $\phi(Z(G))$ cannot be a proper
subset of $Z(G)$, and therefore, we must have $\phi(Z(G))=Z(G)$.
Oops, what if $G$ is infinite?

I'm stumped...

\end{document}