\documentclass{birkjour}

\usepackage{float}
\usepackage{hyperref}

\newtheorem{thm}{Theorem}[section]
\newtheorem{cor}[thm]{Corollary}
\newtheorem{lem}[thm]{Lemma}
\newtheorem{prop}[thm]{Proposition}
\theoremstyle{definition}
\newtheorem{defn}[thm]{Definition}
\theoremstyle{remark}
\newtheorem{rem}[thm]{Remark}
\newtheorem*{ex}{Example}
\numberwithin{equation}{section}

\newcommand{\R}{\mathbb{R}}
\newcommand{\C}{\mathbb{C}}
\newcommand{\B}{\mathbb{B}}
\newcommand{\G}{\mathbb{G}}
\newcommand{\V}{\mathbb{V}}
\newcommand{\gd}{\dot{g}}
\newcommand{\gh}{\hat{g}}
\newcommand{\Gd}{\dot{G}}
\newcommand{\Gh}{\hat{G}}
\newcommand{\nvai}{\infty}
\newcommand{\nvao}{o}
\newcommand{\grade}{\mbox{grade}}

%\received{}\accepted{}

\begin{document}

\title{Spades As Representatives Of Geometric Sets}

\author{Spencer T. Parkin}
\email{spencerparkin@outlook.com}

%\subjclass{Primary 14J70; Secondary 14J29}

%\dedicatory{To my dear wife Melinda.}

\begin{abstract}
Abstract...
\end{abstract}

\keywords{Key words...}

\maketitle

\section{Introduction And Motivation}

In \cite{Parkin15} it was shown how \emph{blades} taken from a geometric algebra may be representative of geometric sets.
In this paper we show how \emph{spades} taken from a geometric algebra may also be representative of such sets.

So what is a spade?  Table~\ref{tbl_terms} to follow gives a definition of this new term among a few of its traditional counter-parts.

\begin{table}[H]\label{tbl_terms}\caption{A few terms used in GA}
\begin{tabular}{p{1cm}p{9cm}}
Term & Definition \\
\hline
Blade & An outer product of zero or more linearly-independent vectors. \\
Versor & A geometric product of zero or more invertible vectors, not necessarily forming a linearly-independent set. \\
Spade & A geometric product of zero or more vectors, not necessarily forming a linearly-independent set.
\end{tabular}
\end{table}

The difference between ``versor'' and ``spade'' is a subtle one, but important.  We will not require that each vector
in the product be invertible.  It is the invertibility of the geometric product, however, that provides us with perhaps the
greatest motivation to use spades, as apposed to blades, as representatives of geometric sets.

Similar to the concept of grade, that of rank will be introduced in this paper with respect to spades.  As an $r$-blade
refers to a blade of grade $r$, we will let an $r$-spade refer to a spade of rank $r$.  If an element of a geomtric
algebra can be written as any geometric product of vectors, then it is a spade.  The rank of that spade is then the smallest
possible number of vectors for which it can be written as such a product.\footnote{While the correctness of many identities of this paper
do not require a spade to be written in the most compact form, the concept of rank would be ill-defined without its consideration.}   Note that blades of grade zero
are indistinguishable from spades of the same rank as each denotes the set of all scalars.

\section{Representation By Blades And Spades}

Letting $a$ denote a vector, and $B_r$ a blade of grade $r$ having factorization $\bigwedge_{i=1}^r b_i$, recall
the following identity, albeit in what may be a slightly unfamiliar form.
\begin{equation}\label{equ_a_dot_Br_sum_of_blades}
a\cdot B_r = \langle B_r\rangle_0a - \sum_{i=1}^r(-1)^i(a\cdot b_i)\bigwedge_{\substack{j=1\\j\neq i}}^rb_i
\end{equation}
Fascinatingly, replacing every instance of the outer product in equation \eqref{equ_a_dot_Br_sum_of_blades} with
a geometric product gives us a new identity, equation \eqref{equ_a_dot_Mr_sum_of_spades}, which does indeed hold.  Letting $M_r$ denote
a spade of at most rank $r$, having a factorization of $\prod_{i=1}^r m_i$, we have
\begin{equation}\label{equ_a_dot_Mr_sum_of_spades}
a\cdot M_r = \langle M_r\rangle_0a - \sum_{i=1}^r(-1)^i(a\cdot m_i)\prod_{\substack{j=1\\j\neq i}}^rm_j.
\end{equation}
At this point it is immediately clear by a comparison of equations \eqref{equ_a_dot_Br_sum_of_blades} and \eqref{equ_a_dot_Mr_sum_of_spades}
that a representation of geometric sets can be accomplished by spades as well as by blades.  We now go on to formalize this notion.

Letting $p:\R^n\to\V$ be a vector-valued function from an $n$-dimensional space\footnote{The field of real numbers $\R$ is most typically used.
However, there are often advantages to using the complex numbers $\C$ instead as they form an algebraicly closed field.} to the vector
space $\V$ generating our geometric algebra $\G$, recall from \cite{Parkin15} the characterization of a geometric set as being the largest sub-set of $\R^n$
over which $p$ vanishes in the inner product with any given blade $B_r\in\G$ of grade $r>0$.  In set builder notation, we would write
\begin{equation*}
\gd(B_r) = \{x\in\R^n|p(x)\cdot B_r=0\}.
\end{equation*}
Now letting each $m_i=b_i$, and defining
\begin{equation*}
\Gd(M_r) = \{x\in\R^n|p(x)\cdot M_r=\langle M_r\rangle_0p(x)\},
\end{equation*}
we see, by equations \eqref{equ_a_dot_Br_sum_of_blades} and \eqref{equ_a_dot_Mr_sum_of_spades}, that
\begin{equation*}
\Gd(M_r) = \bigcap_{i=1}^r\Gd(m_i) = \bigcap_{i=1}^r\gd(b_i) = \gd(B_r).
\end{equation*}
This shows that every geometric set has a spade representative of that set.

\section{Proof Of Identities}

\begin{thebibliography}{9}

\bibitem{Parkin15}
S. Parkin,
\emph{An Introduction To Geometric Sets}.
Advances in Applied Clifford Algebras, Volume 25, Issue Unknown, pp. 639-655, 2015.

\end{thebibliography}

\end{document}