\documentclass{birkjour}

\newtheorem{thm}{Theorem}[section]
\newtheorem{cor}[thm]{Corollary}
\newtheorem{lem}[thm]{Lemma}
\newtheorem{prop}[thm]{Proposition}
\theoremstyle{definition}
\newtheorem{defn}[thm]{Definition}
\theoremstyle{remark}
\newtheorem{rem}[thm]{Remark}
\newtheorem*{ex}{Example}
\numberwithin{equation}{section}

\newcommand{\V}{\mathcal{V}}
\newcommand{\G}{\mathcal{G}}
\newcommand{\R}{\mathcal{R}}
\newcommand{\B}{\mathcal{B}}
\newcommand{\Vb}{\overline{\mathcal{V}}}
\newcommand{\Gb}{\overline{\mathcal{G}}}
\newcommand{\Rb}{\overline{\mathcal{R}}}
\newcommand{\Bb}{\overline{\mathcal{B}}}

\newcommand{\VB}{\mathbb{V}}
\newcommand{\GB}{\mathbb{G}}

\newcommand{\nvao}{o}
\newcommand{\nvai}{\infty}
\newcommand{\nvaob}{\overline{o}}
\newcommand{\nvaib}{\overline{\infty}}
\newcommand{\eb}{\overline{e}}

\begin{document}

%-------------------------------------------------------------------------
% editorial commands: to be inserted by the editorial office
%
%\firstpage{1} \volume{228} \Copyrightyear{2004} \DOI{003-0001}
%
%
%\seriesextra{Just an add-on}
%\seriesextraline{This is the Concrete Title of this Book\br H.E. R and S.T.C. W, Eds.}
%
% for journals:
%
%\firstpage{1}
%\issuenumber{1}
%\Volumeandyear{1 (2004)}
%\Copyrightyear{2004}
%\DOI{003-xxxx-y}
%\Signet
%\commby{inhouse}
%\submitted{March 14, 2003}
%\received{March 16, 2000}
%\revised{June 1, 2000}
%\accepted{July 22, 2000}
%
%
%
%---------------------------------------------------------------------------

\title{A Natural Extension Of\\The Conformal Model Of\\Geometric Algebra}

\author{Spencer T. Parkin}

%\address{%
%\\
%\\
%\\
%}

\email{spencerparkin@outlook.com}

%\thanks{}

%\subjclass{Primary 99Z99; Secondary 00A00}
%\keywords{Class file, journal}

\date{January 1, 2004}

%\dedicatory{To Naomi and Dean}

\begin{abstract}
Abstract goes here...
\end{abstract}

\maketitle

\section{Introduction And Motivation}

The conformal model of geometric algebra, (here-after abbreviated CGA), is
a model of geometry based upon a geometric algebra by an interpretation of its blades as being
representative of certain algebraic surfaces.  For 3-dimensional space, such surfaces include the set of all
points, lines, planes, point-pairs, circles and spheres.  If we include the empty-set geometry, this set
of geometries is closed under the intersection operation as well as the set of all conformal transformations.
Interestingly, the blades representative of the geometries of CGA are also
representative of the conformal transformations.  For example, planes represent planar reflections,
and spheres represent spherical inversions.

Once one has come to terms with CGA, it is natural to ask whether there exists a similar model of
geometry based upon a geometric algebra in which higher order algebraic surfaces may be
represented.  The goal of this paper is to explore a possible answer to this question.  Specifically,
we will introduce a new model of geometry based upon a mother Minkowski algebra of order 2
in which all quadratic surfaces have representation.  In CGA, the quadratic surfaces have 
representation in only the rounds of the model, which certainly do not exhaust the set of
all possible types of quadratic surfaces.

\section{The Structure Of The Algebra}

We are going to let $\V$ denote an $(n+2)$-dimensional vector space generating
a Minkownski geometric algebra $\G$ and having an $n$-dimensional euclidean space
$\R^n$ embedded within it.  A set of basis vectors for $\V$ is given by
\begin{equation*}
\{\nvao,\nvai\}\cup\{e_i\}_{i=1}^n,
\end{equation*}
where $\nvao$ and $\nvai$ are the familiar null-vectors representing the points in CGA
at origin and infinity, respectively.  The set $\{e_i\}_{i=1}^n$ is a right-handed, orthonormal set of
euclidean vectors.

Enter now $\Vb$, which we'll use to denote an identical, yet entirely disjoint $(n+2)$-dimensional
vector space generating a Minkowski geometric algebra $\Gb$ and also having an $n$-dimensional
euclidean space $\Rb^n$ embedded within it.  A set of basis vectors for $\Vb$ will employ
an over-bar notation and be given by
\begin{equation*}
\{\nvaob,\nvaib\}\cup\{\eb_i\}_{i=1}^n.
\end{equation*}

Taken together to form the $2(n+2)$-dimensional vector space $\V\cup\Vb$, this new vector space
generates a mother Minkowski algebra $\G\cup\Gb$ of order 2 as outlined in \cite{}.
Working within this algebra, the over-bar notation may become an outermorphism
that lets us map multivectors in $\G$ to their counter-parts in $\Gb$.  Though an explicit formula
for this outermorphism is given by
\begin{equation*}
\overline{M} = SMS^{-1},
\end{equation*}
where $M$ is a multi-vector in $\G$ and $S$ is given by
\begin{equation*}
(1-)(1+)\prod_{i=1}^n(1-e_i\eb_i),
\end{equation*}
once we have become familiar with the algebraic properties of this outermorphism,
we will have no need to carry $S$ through any of our calculations.  A list of
such properties is given as follows.
\begin{equation*}
\begin{array}{l}
\mbox{For all $a\in\V$, we have $a=-\overline{\overline{a}}$.} \\
\mbox{For all $a,b\in\V$, we have $\overline{a}+\overline{b}=\overline{a+b}.$} \\
\mbox{For all $a,b\in\V$, we have $\overline{a}\cdot\overline{b}=a\cdot b$.} \\
\mbox{For all $a,b\in\V$, we have $\overline{a}\wedge\overline{b}=\overline{a\wedge b}$.} \\
\mbox{For all $a,b\in\V$, we have $a\cdot\overline{b}=0$.}
\end{array}
\end{equation*}
The verification of each of these properties is left as an exercise for the reader.

It is at this point in the paper that we introduce an additional layer of
algebraic structure that will facilitate the realization of properties that
we would otherwise be enable to achieve.  Letting $\B$ and $\Bb$ denote the
set of all bivectors in $\G$ and $\Gb$, respectively, we will let $\VB$ denote the linear space
of bivectors $\B\cup\Bb$.  Being a vector space, we are free to let $\GB$ denote the
geometric algebra generated by $\VB$.

So that we do not confuse the inner
and outer products of $\G\cup\Gb$ with those of $\GB$, we will use $\odot$ for the
inner product of $\GB$ and $\vee$ for the outer product of $\GB$.  These
will carry a lower level of precedence than the inner and outer products $\cdot$
and $\wedge$ of $\G\cup\Gb$.

To complete our definition of $\GB$, we need
now only define the signature of the inner product.  We do so as follows.
\begin{equation*}
\mbox{For any two vectors $a,b\in\VB$, we let $a\odot b=a\cdot b$.}
\end{equation*}
In other words, the inner product among the vectors of $\GB$ is simply the
inner product among the bivectors of $\G\cup\Gb$.

\section{The New Model}

Our new model of geometry will be based upon the geometric algebra $\GB$.
A blade $B\in\GB$ will be representative of a geometry as the set of
all points $x\in\R^n$ such that
\begin{equation*}
P(x)\odot B = 0,
\end{equation*}
where the function $P:\R^n\to\VB$ is defined as
\begin{equation*}
P(x) = p(x)\wedge\overline{p}(x),
\end{equation*}
with the function $p:\R^n\to\V$ being defined as it is in \cite{} as
\begin{equation*}
p(x) = \nvao + x + \frac{1}{2}x^2\nvai.
\end{equation*}
In keeping with the methods of CGA, an alternative geometry that is also
represented by the blade $B\in\GB$ is given by the set of all points $x\in\R^n$
such that
\begin{equation*}
P(x)\vee B = 0.
\end{equation*}

As shown in \cite{}, this new model preserves the set of all conformal
transformations.  That is, any one geometry in the set of all geometries that may be represented
in this model may undergo any conformal transformation by the application of a well-formulated
versor, the result of which
being yet another geometry having representation in the model.\footnote{It should be noted
that more than the set of conformal transformations is supported by this new model, but such
additional transformations are not studied in this paper.}
In what remains of this paper, we will therefore concern
ourselves with the goal of preserving the intersection and union properties of CGA.

%\section{Canonical Forms}

% We lose uniqueness of representation up to scale in the new model.
% This means we can't equate one characterization as easily with another as we could in CGA.
% Such equalities are used for composition and decomposition.

%\begin{thebibliography}{1}
%\bibitem{test} A. B. C. Test, \textit{On a Test.} J. of Testing
%\textbf{88} (2000), 100--120.
%\bibitem{latex} G. Gr\"atzer, \textit{Math into \LaTeX.} 3rd Edition,
%Birkh\"auser, 2000.
%\end{thebibliography}

\end{document}