\documentclass[12pt]{article}

\usepackage{amsmath}
\usepackage{amssymb}
\usepackage{amsthm}
\usepackage{graphicx}
\usepackage{float}

\title{Geometric Calculus}
\author{Spencer T. Parkin}

\newcommand{\G}{\mathbb{G}}
\newcommand{\V}{\mathbb{V}}
\newcommand{\R}{\mathbb{R}}
\newcommand{\B}{\mathbb{B}}
\newcommand{\nvao}{o}
\newcommand{\nvai}{\infty}

\newtheorem{theorem}{Theorem}[section]
\newtheorem{definition}{Definition}[section]
\newtheorem{corollary}{Corollary}[section]
\newtheorem{identity}{Identity}[section]
\newtheorem{lemma}{Lemma}[section]
\newtheorem{result}{Result}[section]

\begin{document}
\maketitle

\section{Introduction and Motivation}

This paper draws upon a number of sources, most notably
\cite{}, to give an exposition of basic geometric calculus
in my own words.  Ancient and modern calculus is laden with
confusing notation that is designed to appeal to most human's
intuition, but this leaves simple-minded people, such as myself, feeling
that the calculus conveyed suffers from a great deal lack of rigor.  Put another way,
if the language of calculus as it is inconsistently given in many
different calculus texts were fed to a computer program designed
to compile it into machine language, that program would either fail
every time, or be extremely complicated.
This paper, therefore, is my attempt to put basic geometric calculus in layman's terms
and in a way that does not assume that the reader is on board with every bit
of notation that can only be understood by glimpsing the unspoken mind of the author.

\section{What Is Geometric Calculus?}

Already having an understanding of geometric algebra (GA), we may define
geometric calculus (GC) as the study of limiting processes involving multivector-valued functions
defined on sub-sets of the vector-space generating the geometric algebra containing the
image of such functions.  Before moving on, it is therefore
requisite that we review here the definition of the limit.
Throughout this paper, we will let $\G$ denote a geometric algebra, and $\V$
the vector space generating $\G$.  The scalars of $\V$ will be the real numbers $\R$.
\begin{definition}\label{def_scalar_limit}
Given a multivector-valued function $f:\R\to\G$, we define the limit of $f$ at
$x_0\in\R$, if it exists, as the multivector $L\in\G$, and write
\begin{equation}
\lim_{x\to x_0} f(x) = L,
\end{equation}
if and only if for all real numbers $\epsilon>0$, there exists a real number $\delta>0$ such that
\begin{equation}
|x-x_0|<\delta \implies |f(x)-f(x_0)|<\epsilon.
\end{equation}
\end{definition}
Here, the notation $|\cdot |$ takes the norm of its argument, which norm
is defined and well-known in studies of GA.

As the reader will notice, this definition applies to multivector-valued functions
defined on $\R$.  But what about multivector-valued functions defined on $\V$?
In an effort to answer this question, it is tempting to give what may be a natural extension of
Definition~\ref{def_scalar_limit} by simply replacing $\R$ with $\V$ in the wording
of Definition~\ref{def_scalar_limit}.  While such a definition may prove useful,
what we'll find in our study of GC is that functions defined on $\V$ can be analyzed
in great and perhaps sufficient detail by restricting our use of the limit process to related functions
defined on $\R$.  The first such function we will study is the difference quotient used
to define what we'll call the directional derivative.

\section{The Directional Derivative Of A Function}

Given a function $f:\R\to\G$, we define its derivative at a point $x_0\in\R$ as
$f'(x_0)$, where the function $f':\R\to\G$ is given by
\begin{equation}\label{equ_scalar_deriv}
f'(x) = \lim_{\delta\to 0}\frac{f(x+\delta)-f(x)}{\delta}.
\end{equation}
Extending this idea to a function $f:\V\to\G$, we get the following definition.
\begin{definition}\label{def_dir_deriv}
Given a function $f:\V\to\G$, we define its derivative at a point $x_0\in\V$
and in a direction $a\in\V$ as $f_a(x_0)$, where the function $f_a:\V\to\R$ is given by
\begin{equation}\label{equ_dir_deriv}
f_a(x) = \lim_{\delta\to 0}\frac{f(x+\delta a)-f(x)}{\delta}.
\end{equation}
\end{definition}
Intuitively, we may think of $f_a(x)$ as the instantaneous rate of change
of the function $f$ at $x$ in the direction $a$.

Just as $\frac{d}{dx}$ is often notation used as an operator taking the
derivative of function $f:\R\to\G$ in equation \eqref{equ_scalar_deriv}, we will let
$\nabla_a$ denote an operator used to take the directional derivative of
a function $f:\V\to\G$ as we have in equation \eqref{equ_dir_deriv}.
Specifically, we have $\frac{d}{dx}f(x)=f'(x)$ and $\nabla_a f(x)=f_a(x)$.

\section{The Derivative (Gradient) Of A Function}

When working with functions defined on $\R$, equation \eqref{equ_scalar_deriv} gives
our definition of the derivative.  Equation \eqref{equ_dir_deriv}, however, is not
completely analogous, because it is additionally characterized by a
direction vector $a$.  What, then, might we consider to be the derivative of a function $f:\V\to\G$
at a point $x_0\in\V$, and what interpretation might we attach to it?
In terms of Definition~\ref{def_dir_deriv}, we will  now proceed to give a definition
of the derivative of a function $f:\V\to\G$, denoted $\nabla f$, and sometimes referred
to as the gradient of $f$, and then go on to contemplate its geometric significance.
For the definition to follow, we let $\{e_i\}_{i=1}^n$ be any set of vectors forming
a basis for the $n$-dimensional vector space $\V$.
\begin{definition}\label{def_vec_diriv}
Given a function $f:\V\to\G$, we define its derivative at a point $x_0\in\V$
as $\nabla f(x_0)$, where the function $\nabla f:\V\to\G$ is given by
\begin{equation}\label{equ_vec_deriv}
\nabla f(x) = \sum_{i=1}^n e_if_{e_i}(x).
\end{equation}
\end{definition}
Notice that here, since the image of $f$ is not necessarily confined to $\R$,
the order of multiplication between $e_i$ and $f_{e_i}$ is significant.
The product here is the geometric product.

The reader should be aware that most authors like to introduce the gradient
of $f$ as an operator that acts like a vector.  Intuitively, they write
\begin{equation}\label{equ_nabla_op}
\nabla = \sum_{i=1}^n e_i\nabla_{e_i}.
\end{equation}
When applied to a function $f$, you must realize that $f$ distributes over addition to
each operator $\nabla_{e_i}$.
To make matters even more complicated, when the application of each operator $\nabla_{e_i}$
results in a scalar-valued function, they then go on to let $a\cdot\nabla$, where $a\in\V$
is any given non-zero vector, denote an operator in its own right as
\begin{equation}\label{equ_dir_deriv_op}
a\cdot\nabla = \sum_{i=1}^n (a\cdot e_i)\nabla_{e_i}
\end{equation}
under the same idea of distribution upon application.  Notice that the $\nabla$ operator
is being treated like a vector here so that the inner product $a\cdot\nabla$ may be carried out.
In my opinion, all of this is horribly confusing, unjustified and completely unnecessary.
A claim is then made that $\nabla_a=a\cdot\nabla$ with little or no explanation.  Here we are
going to justify this reasoning; but for now, put equations \eqref{equ_nabla_op} and \eqref{equ_dir_deriv_op}
out of your mind.

Regaining our grip on mathematical rigor, take a second look at Definition~\ref{def_vec_diriv}, and
consider the case that each function $f_{e_i}$ is scalar-valued.  This makes the function $\nabla f$
vector-valued, and for a vector $a\in\V$, we may then write
\begin{equation}
a\cdot \nabla f(x) = \sum_{i=1}^n (a\cdot e_i)f_{e_i}(x).
\end{equation}
At first glance, it may not be entirely clear what geometric significance the
function $a\cdot \nabla f(x)$ has, but the following theorem will relate
it to the direction derivative which we have already come to terms with geometrically.
\begin{theorem}
Given a differentiable and scalar-valued function $f:\V\to\R$ and a non-zero vector $a\in\V$, we have
\begin{equation}
\nabla_a f(x) = a\cdot\nabla f(x).
\end{equation}
\end{theorem}
\begin{proof}
We begin by rewriting equation \eqref{equ_dir_deriv} as
\begin{equation}
\nabla_a f(x) = \lim_{\delta\to 0}\frac{1}{\delta}\left(f\left(x+\delta\sum_{i=1}^n (a\cdot e_i)e_i\right)-
\sum_{i=1}^n f(x)+\sum_{i=1}^{n-1} f(x)\right).
\end{equation}
We now make the observation that for small values of $\delta$, the function $f$ begins to
approximate a linear function.  This allows us to write
\begin{equation}
\nabla_a f(x)=\lim_{\delta\to 0}\frac{1}{\delta}\left(\sum_{i=1}^n (f(x+\delta (a\cdot e_i)e_i)-f(x))\right).
\end{equation}
Now make the substitution $\delta=\frac{\delta'}{a\cdot e_i}$.
\end{proof}
Lots of rigor lacking here.

\end{document}