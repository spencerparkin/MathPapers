\documentclass[12pt]{article}

\usepackage{amsmath}
\usepackage{amssymb}
\usepackage{amsthm}

\title{Section 2.5 Exercises\\Hertein's Topics In Algebra}
\author{Spencer T. Parkin}

\newtheorem{theorem}{Theorem}[section]
\newtheorem{definition}{Definition}[section]
\newtheorem{corollary}{Corollary}[section]
\newtheorem{identity}{Identity}[section]
\newtheorem{lemma}{Lemma}[section]
\newtheorem{result}{Result}[section]

%\newcommand{\gcd}{\mbox{gcd}}
\newcommand{\lcm}{\mbox{lcm}}
\newcommand{\Z}{\mathbb{Z}}

\begin{document}
\maketitle

\section*{Understanding Theorem 2.5.1}

Here, a leap was made for me in realizing that for every $h_1,h_2\in H\cap K$,
we have $hh_1\neq hh_2$ and $h_1^{-1}k\neq h_2^{-1}k$.  This is obvious by
the cancelation proprety.  So what he does is show that $o(H\cap K)$ duplicates
of the element $hk$ exists in the $o(HK)$ possible multiplications of an element
taken from $H$ with that of $K$.  He then shows that this lower bound is also
an upper bound, and the rest goes through.

\section*{Problem 1}

If $H$ and $K$ are subgroups of $G$, show that $H\cap K$ is a subgroup of $G$.
(Can you see that the same proof shows that the intersection of any
number of subgroups of $G$, finite or infinite, is again a subgroup of $G$?)

Let's just go ahead and show the result for an arbitrary intersection.
Let $I$ be an index set for a family of subgroups $H_\alpha$ of $G$.
Is $H=\bigcap_{\alpha\in I} H_\alpha$ a subgroup of $G$?

Clearly $e\in H$, since all subgroups contain the identity element; so $H$ is non-empty.
Now for all $a,b\in H$, and for any $\alpha\in I$, notice that $a\in H\subseteq H_\alpha$ and $b\in H\subseteq H_\alpha$,
so $ab\in H_\alpha$.  It follows that $ab\in H$.  Now since for all $a\in H$, and all $\alpha\in I$, we
have $a\in H\subseteq H_\alpha$, we have $a^{-1}\in H_\alpha$; and therefore, $a^{-1}\in H$.
That $H$ is a subgroup of $G$ now follows by Lemma 2.4.1.

\section*{Problem 2}

Let $G$ be a group such that the intersection of all its subgroups which
are different from $\langle e\rangle$ is a subgroup different form $\langle e\rangle$.
Prove that every element in $G$ has finite order.

We show the contrapositive.  Let $a\in G$ be an element of infinite order.  We must
now show that the intersection of all subgroups, save $\{e\}$, is the trivial subgroup $\{e\}$.
But this is easy.  We need only show that this is the case for the subgroup generated by $a$;
namely, $\langle a\rangle$.  Being isomorphic to the integers, let us just consider $\Z$.
Notice that for any integer $n>0$, $\Z$ has a subgroup with smallest positive non-identity
element equal to $n$.  It follows that the intersection of all subgroups, save $\{0\}$, is $\{0\}$.

\section*{Problem 3}

If $G$ has no nontrivial subgroups, show that $G$ must be finite of prime order.

We first show that $G$ is finite by showing that every infinite group has
at least one nontrivial subgroup.  If an infinite group has no nontrivial subgroups, then
every non-identity element would generate the entire group.  But this is impossible,
because every non-identity element of infinite order generates $\Z$, which has non-trivial subgroups.

Now consider $|G|$.  If $G$ has a non-trivial subgroup $H$, then $|H|$ divides $|G|$
and $1<|H|<|G|$ which implies that $|G|$ is composite.  This is no help.

Let $a\in G$ be a non-identity element.  Clearly we must have $\langle a\rangle=G$.
We now show that the converse of Lagrange's theorem holds for cyclic groups.

Consider the group $\Z_n=\{z\in\Z|0\leq z<n\}$ endowed with addition mod $n$.
Let $d$ be any divisor of $n$.  We must find a subgroup of order $d$ of $\Z_n$.
This is easy when $d=1$ or $d=n$.  Considering $d$ to be a non-trivial divisor,
let's look at $\langle n/d\rangle$.  The order of this subgroup is the order $n/d$ in $Z_n$,
which is clearly $n/(n/d)=d$.  Thus, for every divisor $d$ of $n$, $\Z_n$ has a subgroup of order $d$.

Returning to $\langle a\rangle=G$, we can now say that if $|G|$ was composite,
then it would have a non-trivial subgroup.  We now have our proof by the contrapositive
of this statement.

(Note also that all subgroups of a cyclic group are cyclic, and that there is {\it exactly} one
subgroup of order $d$ for every divisor $d$ of $\Z_n$.  Proof is needed, though.)

\section*{Problem 4}

\subsection*{Part (a)}

If $H$ is a subgroup of $G$, and for $a\in G$, $aHa^{-1}=\{aha^{-1}|h\in H\}$, show that $aHa^{-1}$ is a subgroup of $G$.

Note that $x,y\in aHa^{-1}$ implies that $x=ah_xa^{-1}$ and $y=ah_ya^{-1}$ with $h_x,h_y\in H$.
It follows that $xy=ah_xh_ya^{-1}\in aHa^{-1}$ since $h_xh_y\in H$.
Then clearly $x^{-1}\in aHa^{-1}$ since $x^{-1}=ah_x^{-1}a^{-1}$.
Seeing that $aHa^{-1}\subseteq G$, our proof goes through by Lemma 2.4.1.

\subsection*{Part (b)}

If $H$ is finite, what is $|aHa^{-1}|$?

Let $\phi:H\to aHa^{-1}$ be defined as $\phi(x)=axa^{-1}$.  Then if $\phi(x)=\phi(y)$, then $axa^{-1}=aya^{-1}\implies x=y$,
showing that $\phi$ is one-to-one.  Then since $H$ is finite, $\phi$ is also onto.  It follows that $|H|=|aHa^{-1}|$.

\section*{Problem 5}

For a subgroup $H$ of $G$ define the left coset $aH$ of $H$ in $G$ as the set of all elements of the form
$ah$, $h\in H$.  Show that there is a one-to-one correspondence (bijection) between the set of left cosets
of $H$ in $G$ and the set of right cosets of $H$ in $G$.

The natural mapping to investigate is $\phi(Ha)=aH$ with $a\in G$.  Clearly it is onto (surjective).  Is it well defined?  Is it one-to-one (injective)?

Note that $Hx=Hy$ if and only if $xy^{-1}\in H$.  If $xy^{-1}\in H$, then
$x\in Hy$.  Then since clearly $x\in Hx$, we have $Hx\cap Hy$ non-empty.
But now since the right cosets of $H$ in $G$ partition $G$, we must have $Hx=Hy$.
On the other hand, if $Hx=Hy$, then $x\in Hy\implies x\equiv y\pmod{H}$ by Lemma 2.4.4.

It is also possible to show that $x^{-1}y\in H$ if and only if $xH=yH$.

Hmmm...think about it.

\end{document}