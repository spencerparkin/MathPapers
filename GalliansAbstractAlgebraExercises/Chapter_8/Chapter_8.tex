\documentclass[12pt]{article}

\usepackage{amsmath}
\usepackage{amssymb}
\usepackage{amsthm}

\title{Chapter 8 Exercises\\Gallian's Book on Abstract Algebra}
\author{Spencer T. Parkin}

\newtheorem{theorem}{Theorem}[section]
\newtheorem{definition}{Definition}[section]
\newtheorem{corollary}{Corollary}[section]
\newtheorem{identity}{Identity}[section]
\newtheorem{lemma}{Lemma}[section]
\newtheorem{result}{Result}[section]

%\newcommand{\gcd}{\mbox{gcd}}
\newcommand{\lcm}{\mbox{lcm}}
\newcommand{\abs}{\mbox{abs}}
\newcommand{\Z}{\mathbb{Z}}
\newcommand{\R}{\mathbb{R}}
\newcommand{\G}{\mathbb{G}}
\newcommand{\stab}{\mbox{stab}}
\newcommand{\aut}{\mbox{Aut}}
\newcommand{\inn}{\mbox{Inn}}
\newcommand{\orb}{\mbox{orb}}

\begin{document}
\maketitle

\section*{Understanding Theorem 8.3}

We want to show that $U(st)\approx U(s)\oplus U(t)$.
Let $\phi(x)=(x\bmod s,x\bmod t)$.
For $x,y\in U(st)$, if $\phi(x)=\phi(y)$,
then $x\equiv y\pmod s$ and $x\equiv y\pmod t$.
Then, since $\gcd(s,t)=1$, it is clear that $x\equiv y\pmod{st}$.
(See Problem 15 of Chapter 0.)  So $\phi$ is one-to-one.
It is also onto since $\phi$ is onto-to-one and maps a finite
set to another of the same cardinality.  That $\phi$ is
operation preserving is a matter of showing that for
any $x,y\in U(st)$, we have
\begin{align*}
(xy\bmod st)\bmod m &= (x\bmod m)(y\bmod m)\bmod st \\
 &= (xy\bmod m)\bmod st,
\end{align*}
where $m$ is $s$ or $t$.  With enough thought, this is
intuitive enough to warrent justification by virtue of being clear.

\section*{Problem 1}

Prove that the external direct product of any finite number of groups is a group.

There is clearly an identity element.  Closure is clear.  Inverses are clear.
Associativity is clear.  I think that's it.

\end{document}