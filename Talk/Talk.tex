\documentclass[12pt]{article}

\usepackage{amsmath}
\usepackage{amssymb}
\usepackage{amsthm}
\usepackage{graphicx}
\usepackage{float}

\begin{document}

Good morning, brothers and sisters.  I have been asked to speak on the same topics
covered in Elder M. Russel Ballard's recent general conference address entitled,
``This Is My Work and Glory''.  Brother Jarman advised that this should not be a summary
of Elder Ballard's talk, but that while guided by the subject matter, I should draw
upon my own experiences to express the same faith-promoting message and admonitions.
Being largely about priesthood power and authority, I'm afraid to say that, despite
Brother Jarmin's kind insistence to the contrary, I do not have any great experiences with
the priesthood in my attempts to use it in administering to other people.  Being a fairly
new father with one daughter and a son on the way, perhaps this will change.

Most of my experiences with the priesthood are on the receiving end.  Of the many examples of
this, the one that takes the forefront of my mind at this time would be my being set apart as a
mission a very long time ago.  I could tell that the stake president wasn't speaking his own
mind.  He wasn't saying what he wanted to say or what he wanted for me as a new missionary.
His words came from the Holy Ghost.  They were given to him.  There were blessings, promises, warnings
and counsel given.  I recall being advised to leave everything behind -- counsel that I
wish I had followed.  I was also counseled to use my testimony as a shield against arguments
that would try my faith.  In a related, but seperate priesthood blessing, I recall a promise given that
although I may become home-sick or lonely,
that when I prayed in the mission field, the Lord would give me the comfort and assurance that I was doing His work.

More than once as a missionary on the streets of Los Angeles, I was put into a
very familiar position in which a missionary and his companion find themselves.
A member of the church, less active or bed-ridden, sick or otherwise afflicted,
has reached out to the missionaries and desired a priesthood blessing.  I must admit
to feeling a great deal of relief on such occasions when I was chosen to anoint while
my companion that of sealing the anointing.  All I had to do was get the individual's name right.
To give the body of the priesthood blessing, on the other hand, required a great
deal more.  I knew many missionaries that were very good at it.  They always
had so much to say.  Always being at a loss for words, as our poor unfortunate
Elders Quorum can attest, when it was placed upon me to exercise my priesthood in
giving the blessing, it is needless to say that it didn't go very well.  In any case,
despite my fears, I always wanted to seek Heavenly Father's will in giving priesthood
blessings, that He would inspire and direct me.

As I've read over Elder Ballard's talk again and again, I want to be careful here to
avoid, as Brother Jarmin requested, giving a summary of the address.  But if the
congregation will indulge me for a moment, I, having a
poor knowledge of many gospel topics, not the least of which is the priesthood
power and authority of God, would like to share with you what I hope is not
a misinterpretation of a few teachings and princples conveyed in Elder Ballard's talk,
which things I have, at length, given a great deal of effort toward understanding.
Perhaps such things can only come through prayerful and worthy understanding,
and experience.

One of Elder Ballard's teachings is a distinction he makes between the power and
authority of the priesthood.  While one may be authorized to act in God's name,
the power of the priesthood comes only when that worthy individual is also acting in
accordance with God's will.  I believe that Heavenly Father's will can be made known
to us in the very moment it is saught, if we are worthy of and honor the priesthood we hold.
Elder Ballard quotes President Kimball as declaring, ``The Lord has given to all of us, as holders of
the priesthood, certain of his authority, but we can only tap the powers of heaven on the basis
of our personal righteousness.''  By certain of his authority, I wonder if here President Kimball
is refering to the concept of priesthood keys given to those holding certain priesthood offices.
It would be interesting to learn more and understand the full structure of the priesthood.

Another teaching conveyed by Elder Ballard relates the priesthood and procreative powers
and their respective roles in building an eternal family.  Elder Ballard says, ``In the
eternal perspective, both the procreative power and the priesthood power are shared by husband
and wife.''  
For the procreative power, this sharing is obvious.  For the priesthood power, I believe
we must realize that a priesthood holder never exercises his priesthood for his own benefit,
but only for the benefit of others; in this case, his family.  Furthermore, while the procreative
power simply grows a family by numbers, there must be some quality of the family that is grown
by the application of priesthood power.  I suppose that this quality uniquely qualifies a family
for inclusion in God's kingdom.

% on understanding the role that the priesthood plays in building eternal families / our exhaultation, etc.

\end{document}