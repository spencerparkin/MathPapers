\documentclass[12pt]{article}

\usepackage{amsmath}
\usepackage{amssymb}
\usepackage{amsthm}
\usepackage{graphicx}
\usepackage{float}

\title{Imaginary Rounds}
\author{Spencer T. Parkin}

\newcommand{\G}{\mathbb{G}}
\newcommand{\V}{\mathbb{V}}
\newcommand{\R}{\mathbb{R}}
\newcommand{\B}{\mathbb{B}}
\newcommand{\nvao}{o}
\newcommand{\nvai}{\infty}

\newtheorem{theorem}{Theorem}[section]
\newtheorem{definition}{Definition}[section]
\newtheorem{corollary}{Corollary}[section]
\newtheorem{identity}{Identity}[section]
\newtheorem{lemma}{Lemma}[section]
\newtheorem{result}{Result}[section]

\begin{document}
\maketitle

Letting $B$ be a blade dually representative of any geometry, I believe that we can all agree that
$B$ represents an imaginary geometry if and only if the following set is empty.
\begin{equation}\label{equ_imaginary_def}
\{x\in\V^n|p(x)\cdot B=0\}
\end{equation}
Now let $\sigma(\lambda)$ be a function defined as follows, with $r\neq 0$.
\begin{equation*}
\sigma(\lambda) = \nvao - \lambda\frac{1}{2}r^2\nvai
\end{equation*}
It is not hard to prove that $\sigma(\lambda)$ dually represents a real sphere of radius $r$ whenever
$\lambda=1$ and an imaginary sphere of radius $r$ whenever $\lambda=-1$.
It follows that $\sigma^2(\lambda)=\lambda r^2$ may be used as a test to determine
if $\sigma(\lambda)$ is real or imaginary.
\begin{align*}
\sigma^2(\lambda) &< 0\implies\mbox{$\sigma(\lambda)$ is imaginary.} \\
\sigma^2(\lambda) &> 0\implies\mbox{$\sigma(\lambda)$ is real.}
\end{align*}
Noticing that any origin-centered round of lower dimension may be dually represented by
\begin{equation*}
B=\sigma(\lambda)\prod_{k=1}^m v_k,
\end{equation*}
where $\{v_k\}_{k=1}^m$ is a sequence of $m$ pair-wise orthogonal unit-vectors taken from $\V^n$,
how might the test be generalized?  Our first task is to determine when $B$ is imaginary by the definition
established in $\eqref{equ_imaginary_def}$.  Imaginary or not, we need only consider a point $x\in\V^n$ on the
sphere represented by $B$ as we would draw $B$ on paper.  (All other points are clearly not on $B$.)
It is then clear that for all integers $k\in[1,m]$,
we have $p(x)\cdot v_k=0$, showing that $x$ is on each real dual $(n-1)$-dimensional hyper-plane $v_k$.
It follows that $p(x)\cdot B=0$ if and only if $p(x)\cdot\sigma(\lambda)=0$.  The test then generalizes as
follows.
\begin{align*}
B\tilde{B} = \sigma^2(\lambda) &< 0\implies\mbox{$B$ is imaginary.} \\
B\tilde{B} = \sigma^2(\lambda) &> 0\implies\mbox{$B$ is real.}
\end{align*}
Here, $\tilde{B}$ is the reverse of $B$.

Consider now the following identity, where $r>0$ and $v$ is a unit-length vector in $\V^n$.
\begin{align*}
  B=& \left(\nvao-rv+\frac{1}{2}(-rv)^2\nvai\right)\wedge\left(\nvao+rv+\frac{1}{2}(rv)^2\nvai\right) \\
 =&\; \nvao\wedge rv-rv\wedge\nvao+\frac{1}{2}r^2\nvao\wedge\nvai+\frac{1}{2}r^2\nvai\wedge\nvao
-\frac{1}{2}r^3v\wedge\nvai+\frac{1}{2}r^3\nvai\wedge v \\
 =&\; 2ro\wedge v+r^3\nvai\wedge v \\
 =&\; 2r(\nvao+\frac{1}{2}r^2\nvai)\wedge v
\end{align*}
It is clear that the first part of this equation is a direct real point-pair.  We will now show that
the last part of this equation is a dual imaginary circle.  According to our test, we calculate $B\tilde{B}$,
and in doing so, get $-4r^4<0$.

Now, if we were to perform this test on $BI$, a dual of $B$, then our test would indicate the realness
of $BI$ as a dual point-pair.  Indeed, $(BI)(IB)=-B\tilde{B}=4r^4>0$.

\end{document}