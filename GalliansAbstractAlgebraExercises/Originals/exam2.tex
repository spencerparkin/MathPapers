\documentclass{article}
\usepackage{amsmath}
\usepackage{amssymb}
\addtolength{\oddsidemargin}{-.575in}
\addtolength{\evensidemargin}{-.575in}
\addtolength{\textwidth}{1.0in}
\addtolength{\topmargin}{-.575in}
\addtolength{\textheight}{1.25in}
\begin{document}

\newcommand{\aut}{\mbox{Aut}}
\newcommand{\Z}{\mathbb{Z}}
\newcommand{\lcm}{\mbox{lcm}}
%\newcommand{\iff}{\Longleftrightarrow}

\section*{Problem 1}

Let $\alpha=(15)(2346)$, $\beta=(12)(456)$, and
$\gamma=(135)(246)$ be elements of $S_6$.  Write
$(\alpha\gamma)^5$ and $(\alpha\beta)^{-1}$ as products of
disjoint cycles, and write $\alpha\beta\gamma$ as a product
of transpositions.

Some work shows that $\alpha\gamma=(14263)$ so that by Theorem 5.3,
$|\alpha\gamma|=5$.  Therefore, $(\alpha\gamma)^5=(1)$.
More work shows that $\alpha\beta=(134)(25)$ so that
$(\alpha\beta)^{-1}=(431)(52)$.  Yet more work shows that
$\alpha\beta\gamma=(146532)$ so that we can rewrite this
as $\alpha\beta\gamma=(12)(13)(15)(16)(14)$.

\section*{Problem 2}

Let $\sigma=(i_1,i_2,\dots,i_m)$ be a cycle in $S_n$ and let
$\theta\in S_n$.  Show that
$\theta\sigma\theta^{-1} = (\theta(i_1),\theta(i_2),\dots,\theta(i_m))$.

This follows directly from the method of cycle multiplication.
Let $k$ be an integer where $1\leq k\leq m$ and notice that...
\begin{equation*}
(\theta(i_1,i_2,\dots,i_m)\theta^{-1})(\theta(i_k))
 = (\theta(i_1,i_2,\dots,i_m))(i_k)
 = \theta(i_{k+1})
\end{equation*}
...where $i_{k+1}=i_1$ if $k=m$.
This shows that when we calculate the product
$\theta(i_1,i_2,\dots,i_m)\theta^{-1}$ as a cycle starting
with $\theta(i_1)$, we get $(\theta(i_1),\theta(i_2),\dots,\theta(i_m))$.
What remains to be shown is that there are no more cycles in the disjoint
cycle form of the product.  Let's start another cycle in the product
using $\theta(a)$ where $a\neq i_k$ for all integers $1\leq k\leq m$.
Then $(\theta\sigma\theta^{-1})(\theta(a))=(\theta\sigma)(a)=\theta(a)$
so that the cycle closes immediately.  So there are no more cycles
in the product, and we're done.

\section*{Problem 3}

Prove that there is no permutation $\sigma$ such that
$\sigma(57)\sigma^{-1}=(157)$, for any $\sigma\in S_n$, $n\geq 7$.

Consider constructing the disjoint cycle form of the product $\sigma(57)\sigma^{-1}$.
To have the product be $(157)$, we cannot let $\sigma^{-1}(1)$ map to anything but
5 or 7, since then we would have $(\sigma(57)\sigma^{-1})(1)=1$.  If $\sigma^{-1}(1)=5$,
then we must have $\sigma(7)=5$, but this is a contradiction since $\sigma$ and
$\sigma^{-1}$ are inverses of one another.  If $\sigma^{-1}(1)=7$, then we
must have $\sigma$ fix 5.  But if $\sigma$ fixes 5, then so does $\sigma^{-1}$,
and we have $(\sigma(57)\sigma^{-1})(5)=1$ since earlier we required $\sigma^{-1}(1)=7$
making it so that $\sigma(7)=1$.  In all cases, we couldn't get $(157)$ out of the
product $\sigma(57)\sigma^{-1}$, so it's not possible.

\section*{Problem 4}

Prove that the order of $U(n)$ is even if $n>2$.

Notice that for all $n>1$, we have $n-1\in U(n)$ since $\gcd(n-1,n)=1$.
Now notice that $|n-1|=2$ since $n>2$ and $(n-1)^2=n^2-2n+1\equiv 1\pmod{n}$.
Then by Lagrange's Theorem, the order of every element of a group divides
the order of the group.  So $|n-1|=2$ divides the order of every group $U(n)$ for $n>2$.

\pagebreak
\section*{Problem 5}

Find one of the largest cyclic subgroups of $\aut(\Z_{720})$, up to isomorphism.

What we want to find is an element having the largest possible order.
If $a\in\aut(\Z_{720})$ is such an element, then $\langle a\rangle$
is one of the largest possible cyclic subgroups.  We will find $a$ in a group
isomorphic to $\aut(\Z_{720})$.  Notice that...
\begin{multline*}
\aut(\Z_{720})\approx U(720)\approx U(2^4)\oplus U(3^2)\oplus U(5)\approx
\Z_2\oplus\Z_4\oplus\Z_6\oplus\Z_4 \\
\approx\Z_2\oplus\Z_4\oplus\Z_2\oplus\Z_3\oplus\Z_4
\approx\Z_2\oplus\Z_4\oplus\Z_2\oplus\Z_{12}
\end{multline*}
...by Theorem 6.5, Theorem 8.3, and Gauss.  Using Theorem 8.1, we see
that all possible orders of the group are generated by...
\begin{equation*}
\lcm(\{1,2\},\{1,2,4\},\{1,2\},\{1,2,3,4,6,12\})
\end{equation*}
...since for cyclic groups, for every divisor of the group, there is an
element of that order.  From our choices here, it is clear that 12 is
the largest order we can get.  Then...
\begin{equation*}
\langle(0,0,0,1)\rangle\subset\Z_2\oplus\Z_4\oplus\Z_2\oplus\Z_{12}\approx\aut(\Z_{720})
\end{equation*}
...is one of the largest possible cyclic subgroups.

\section*{Problem 6}

Prove that $S_3\oplus S_4$ is not isomorphic to a subgroup of $S_6$.

Define a partition of a natural number $n$ as any set of
natural numbers whose sum is $n$.  Notice that the least
common multiple of every partition of 6 is no greater than 6.
Because we can associate every permutation in $S_6$ with a unique
parition of 6 by using the lengths of all cycles in its disjoint
cycle form, we can use Theorem 5.3 to conclude that no element
in $S_6$ has an order greater than 6.  For example,
$(123)(4)(56)\in S_6$ corresponds to the partition $\{3,1,2\}$,
and $6\geq\lcm(3,1,2)$.

Let $a\in S_3$ where $|a|=3$ and let $b\in S_4$ where $|b|=4$.
Then choose $(a,b)\in S_3\oplus S_4$ and notice that
$|(a,b)|=\lcm(|a|,|b|)=12$.  There can be no subgroup of $S_6$ having
an element of order 12.  Since isomorphisms perserve orders, there
can be no isomorphism between $S_3\oplus S_4$ and $S_6$.

In case you might want to know,
the group $S_6$ has 1 element of order 1, 75 elements of order 2,
80 elements of order 3, 180 elements of order 4, 144 elements
of order 5, and 240 elements of order 6.  A computer program gathered this.

\section*{Problem 7}

Let $H$ and $K$ be subgroups of $G$ and define $HK=\{hk : h\in H, k\in K\}$ and
$KH=\{kh : k\in K, h\in H\}$.  Prove that $HK$ is a subgroup of $G$ if and only if
$HK=KH$.

Suppose $HK$ is a subgroup of $G$.  Let $x\in HK$.  Then since $HK$ is a group,
we know that $x^{-1}\in HK$.  Further more, $x^{-1} = hk$ for some $h\in H$ and
some $k\in K$.  Then $x = (x^{-1})^{-1} = (hk)^{-1} = k^{-1}h^{-1}\in KH$.
This shows that $HK\subseteq KH$.  The proof that $KH\subseteq HK$ is similar,
and so we have $HK=KH$ as desired.

Suppose now that $HK=KH$.  Notice that $\{e\}\subseteq HK\subseteq G$.
Let $a,b\in HK$ where $a=hk$ and $b=h'k'$
for some $h,h'\in H$ and $k,k'\in K$.  We want to show that
$ab^{-1}=hkk'^{-1}h'^{-1}\in HK$.  Clearly $hkk'^{-1}\in HK$ since
$h\in H$ and $kk'^{-1}\in K$.  But since $HK=KH$, we may also say
that $hkk'^{-1}\in KH$ and write it as $k''h''$ where $k''\in K$
and $h''\in H$.  It is then clear that $k''h''h'^{-1}\in KH$ and
so it is in $HK$ also.  By the one-step subgroup test, $HK$ is
a subgroup of $G$.

\pagebreak
\section*{Problem 8}

Let $H$ be a subgroup of a group $G$.  For a fixed $a\in G$, define a conjugate
of $H$ as $aHa^{-1}=\{aha^{-1}:h\in H\}$.  Show that $aHa^{-1}$ is isomorphic to $H$.

It suffices to show that $H$ is isomorphic to $aHa^{-1}$, its conjugate with
respect to $a$.  Let $\phi:H\to aHa^{-1}$ be defined as $\phi(h)=aha^{-1}$.
We will show that this is an isomorphism from $H$ to $aHa^{-1}$.
Let $x,y\in H$ and notice that...
\begin{equation*}
x=y \iff axa^{-1} = aya^{-1} \iff \phi(x)=\phi(y)
\end{equation*}
...showing that $\phi$ is well defined and one-to-one.
Now notice that...
\begin{equation*}
\phi(xy) = axya^{-1} = axa^{-1}aya^{-1} = \phi(x)\phi(y)
\end{equation*}
...showing that $\phi$ is operation preserving.  Let $h'\in aHa^{-1}$.
Then there exists an $h\in H$ such that $h'=aha^{-1}$ by definition, and
we see that $\phi(h)=h'$ so that $\phi$ is onto.
We've now shown enough to say that $H$ and $aHa^{-1}$ are isomorphic.

\end{document}