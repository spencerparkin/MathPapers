\documentclass{birkjour}

\usepackage{tikz}
\usepackage{graphicx}
\usepackage{hyperref}

\newtheorem{thm}{Theorem}[section]
\newtheorem{cor}[thm]{Corollary}
\newtheorem{lem}[thm]{Lemma}
\newtheorem{prop}[thm]{Proposition}
\theoremstyle{definition}
\newtheorem{defn}[thm]{Definition}
\theoremstyle{remark}
\newtheorem{rem}[thm]{Remark}
\newtheorem*{ex}{Example}
\numberwithin{equation}{section}

\newcommand{\R}{\mathbb{R}}
\newcommand{\B}{\mathbb{B}}
\newcommand{\G}{\mathbb{G}}
\newcommand{\V}{\mathbb{V}}
\newcommand{\gd}{\dot{g}}
\newcommand{\gh}{\hat{g}}
\newcommand{\Gd}{\dot{G}}
\newcommand{\Gh}{\hat{G}}
\newcommand{\nvai}{\infty}
\newcommand{\nvao}{o}
\newcommand{\grade}{\mbox{grade}}

\begin{document}

\title{Versors That Give Non-Uniform Scale}

\author{Spencer T. Parkin}
\address{102 W. 500 S., \\
Salt Lake City, UT  84101} \email{spencerparkin@outlook.com}

%\subjclass{Primary 14J70; Secondary 14J29}

\dedicatory{To my dear wife Melinda.}

\begin{abstract}
Versors are found in a geometric algebra that,
when applied to elements of that algebra that are representative
of various algebraic surfaces in a constrained way, perform a non-uniform scaling
of those surfaces.
\end{abstract}

\keywords{Algebraic Surface, Conformal Model, Non-Uniform Scale, Geometric Algebra}

\maketitle

\section{Motivation}

Non-uniform scale is one of the last remaining problems of geometric algebra.

\section{The Result}

The result of this paper is simply a corollary to that of \cite{Parkin13}, but to see how,
we must first constrain the way that we represent $n$-dimensional algebraic surfaces of up to degree $m$
in the Mother Minkowski algebra of order $m$.\footnote{Recall that such representations are not unique, and so
we have the flexibility to choose our representations carefully.} What we do is let $n\leq m$, and
reserve certain sub-algebras of our mother algebra for use in specific dimensions.
To see what is meant by this, let $f:\R^n\to\R$ be a polynomial in whose zero set
we are interested.  Now define, for any integer $k\in[1,n]$, the polynomial $f_k:\R\to\R$ as
\begin{equation*}
f_k(\lambda)=f(e_1+e_2+\dots+\lambda e_k+\dots +e_{n-1}+e_n).
\end{equation*}
Having done this, we will represent the
surface of $f$ in the Mother Minkowski algebra of order
\begin{equation*}
m=\sum_{k=1}^n\deg f_k.
\end{equation*}
Now if $\G$ denotes our mother algebra and it is generated by $m$ sub-algebras $\G_i$,
each generated by the vector space $\V_i$, then we reserve $\deg f_k$
of these sub-algebras for use in dimension $k$ of our $n$ dimensions.
(We will let $\G^k$, where $k$ is an integer in $[1,n]$, denote the largest
sub-algebra of $\G$ containing all sub-algebras $\G_i$ reserved for dimension $k$.)

An example may be warrented at this point.  Let $n=3$ and consider the
polynomial given by
\begin{equation}\label{equ_example_poly}
f(x) = 3x_1^2x_2x_3^4 + 4x_1x_2^5 - 7x_3^2,
\end{equation}
where $x_k$ is notation for $x_k=x\cdot e_k$.  We will represent the
surface that is the zero set of this polynomial using an $m$-vector
in a Mother Minkowski algebra of order $m=2+5+4=11$.  The first
2 sub-algebras are reserved for dimension 1, the next 5 for dimension
2, and the last 4 for dimension 3.  The $m$-vector $B$
representing this surface is then given by
\begin{align*}
B &= 3e_{(1,2),1}\wedge e_{3,2}\wedge \nvai_{(4,5,6,7)} \wedge e_{(8,9,10),3} \wedge \nvai_{11} \\
 &+ 4e_{1,1}\wedge\nvai_2\wedge e_{(3,4,5,6,7),2}\wedge\nvai_{(8,9,10,11)} \\
 &- 7\nvai_{(1,2,3,4,5,6,7)}\wedge e_{(8,9),3}\wedge\nvai_{(10,11)}.
% Check signs.
\end{align*}
Here, notation is a challenge.  The vector $e_{i,j}$ denotes the $j^{th}$ euclidean basis vector
in the $i^{th}$ sub-algebra.  We then define
\begin{equation*}
e_{(i_1,i_2,\dots,i_r),j} = e_{i_1,j}\wedge e_{i_2,j}\wedge\dots\wedge e_{i_r,j}.
\end{equation*}
The notation for $\nvai$ is similar.

We can now say that the zero set of $f$ in equation \eqref{equ_example_poly}
is given by the set of all solutions to the equation
\begin{equation}\label{equ_surface_rep}
\bigwedge_{k=1}^m p_k(x)\cdot B = 0.
\end{equation}
Recall that $p_k(x)=\nvao_k+x_k+\frac{1}{2}x^2\nvai_k$.
Of course, we could have represented $f$ in a mother algebra of order $\deg f=7$,
but it will soon become clear why we needed our algebra $\G$ to be of order $m=11$.

Returning from the example, suppose now we have an $m$-vector $B$ representative
of any polynomial $f:\R^n\to\R$ under the constraint thus illustrated.  Seeing that
the zero set of $f$ is the set of solutions to equation \eqref{equ_surface_rep},
we make the simple observation that if $D$ is a versor taken from a sub-algebra $\G^k$,
and further, $D$ is the product of the same dilation versor $D_i$ found in each sub-algebra
$\G_i$ contained in $\G^k$, then the non-uniform scale of $f$ in the dimension of $k$ by the scale of each
$D_i$ is given by the set of solutions to the equation
\begin{equation}\label{equ_non_uniform_scale_construction}
p_1(x)\wedge p_2(x)\wedge\dots\wedge (D^{-1}p_k(x)D)\wedge\dots\wedge p_{n-1}(x)\wedge p_n(x)\cdot B=0.
\end{equation}
Now realize that for all $j\neq k$, $D$ leaves $p_j(x)$ invariant.  That is,
\begin{equation*}
D^{-1}p_j(x)D=p_j(x).
\end{equation*}
It now follows by equations (3.2) through (3.5) of \cite{Parkin13} that equation \eqref{equ_non_uniform_scale_construction}
may be rewritten as
\begin{equation*}
\bigwedge_{k=1}^n p_k(x)\cdot DBD^{-1},
\end{equation*}
showing that $D$, when applied to $B$, performs a non-uniform scaling of the surface of $f$.

\section{Closing Remarks}

Though we have now shown that versors performing the non-unform scale operation
exist, seeing that their application requires a great deal of combersome convention and notation,
a question of their practicality immediately arises.  It's certainly not practical on paper, but
perhaps such versors may find applications on the computer.

\begin{thebibliography}{9}

\bibitem{Parkin13}
S. Parkin, {\it Mother Minkowski Algebra Of Order $M$}.
Advances in Applied Clifford Algebras (2013).

\end{thebibliography}

\end{document}