\documentclass{birkjour}

\usepackage{float}
\usepackage{hyperref}

\newtheorem{thm}{Theorem}[section]
\newtheorem{cor}[thm]{Corollary}
\newtheorem{lem}[thm]{Lemma}
\newtheorem{prop}[thm]{Proposition}
\theoremstyle{definition}
\newtheorem{defn}[thm]{Definition}
\theoremstyle{remark}
\newtheorem{rem}[thm]{Remark}
\newtheorem*{ex}{Example}
\numberwithin{equation}{section}

\newcommand{\F}{\mathbb{F}}
\newcommand{\R}{\mathbb{R}}
\newcommand{\C}{\mathbb{C}}
\newcommand{\B}{\mathbb{B}}
\newcommand{\G}{\mathbb{G}}
\newcommand{\V}{\mathbb{V}}
\newcommand{\gd}{\dot{g}}
\newcommand{\gh}{\hat{g}}
\newcommand{\Gd}{\dot{G}}
\newcommand{\Gh}{\hat{G}}
\newcommand{\nvai}{\infty}
\newcommand{\nvao}{o}
\newcommand{\grade}{\mbox{grade}}
\newcommand{\rank}{\mbox{rank}}

%\received{}\accepted{}

\begin{document}

\title{Representing Geometry In Geometric Algebra}

\author{Spencer T. Parkin}
\email{spencerparkin@outlook.com}

%\subjclass{Primary 14J70; Secondary 14J29}

%\dedicatory{To my dear wife Melinda.}

\begin{abstract}
The aim of this paper is to show that versors, like blades, are natural representatives of geometry in geometric algebra.
In doing so, we extend the usual definition of how a blade represents geometry to one in which
any element of the algebra may serve as such a representative.
\end{abstract}

\keywords{Geometry, Algebraic Set, Geometric Set, Blade, Versor, Conformal Model, Homogeneous Model, Geometric Algebra, Algebraic Geometry}

\maketitle

\section{Geometric Sets}

We must begin with a review of geometric sets.  Given an $n$-dimensional space $\F^n$, we let $p:\F^n\to\V$ be a non-zero, vector-valued
function mapping points in $\F^n$ to vectors in a vector space $\V$ generating our geometric algebra $\G$.  With this in hand,
we are ready for the following definition.
\begin{defn}[Geometric Set]\label{def_geo_set}
A subset $S$ of $\F^n$ is a \emph{geometric set} if and only if there exists a set of vectors $\{v_i\}\subseteq\V$, such that
\begin{equation}\label{equ_geo_set_def}
S = G(\{v_i\}) = \bigcap_i \{x\in\F^n|p(x)\cdot v_i=0\}.
\end{equation}
\end{defn}
Notice that the subset $\{v_i\}$ of $\V$ may be of finite or infinite cardinality.  It should also be immediately clear
from Definition~\ref{def_geo_set} that the intersection of any two geometric sets is geometric.
If each expression $p(x)\cdot v_i$ is a polynomial in the components of $x$, then every geometric set is algebraic.

\begin{lem}\label{lem_geo_set_lin_indep}
If $\{E_i\}_{i=1}^r$ is any linearly independent set of elements taken from $\G$, then the set of all
solutions in $x$ to the equation
\begin{equation}\label{equ_geo_set}
0 = \sum_{i=1}^r (p(x)\cdot v_i)E_i
\end{equation}
is a geometric set.
\end{lem}
\begin{proof}
Being a linearly independent set of elements, the only linear combination of these elements that vanishes is the trivial linear combination.
It then follows that for each integer $i\in[1,r]$, we must have $p(x)\cdot v_i=0$.
\end{proof}

\begin{lem}\label{lem_geo_set_lin_dep}
If $\{E_i\}_{i=1}^r$ is any sequence of elements taken from $\G$ such that for all integers $i\in[1,r]$, we have
$\langle E_i\rangle_0=0$ and $E_i\neq 0$, then the set of all solutions in $x$ to equation \eqref{equ_geo_set}
is a geometric set.
\end{lem}
\begin{proof}
If $\{E_i\}_{i=1}^r$ is a linearly independent set, then we're done by Lemma~\ref{lem_geo_set_lin_indep}.
Supposing to the contrary, and without loss of generality, we can let $s$ be an integer with $1\leq s<r$ such that
$\{E_i\}_{i=1}^s$ is a linearly independent set, and
\begin{equation*}
\mbox{span}\{E_i\}_{i=1}^r = \mbox{span}\{E_i\}_{i=1}^s.
\end{equation*}
Now for each integer $i\in[s+1,r]$, write $E_i$ as a linear combination of the elements in $\{E_i\}_{i=1}^s$ as
\begin{equation*}
E_i = \sum_{j=1}^s\alpha_{i,j} E_j.
\end{equation*}
Having done so, we see that equation \eqref{lem_geo_set_lin_dep} becomes
\begin{align*}
0 &= \sum_{i=1}^r(p(x)\cdot v_i)E_i \\
 &= \sum_{i=1}^s(p(x)\cdot v_i)E_i + \sum_{i=s+1}^r(p(x)\cdot v_i)\sum_{j=1}^s\alpha_{i,j}E_j \\
 &= \sum_{i=1}^s\left[p(x)\cdot v_i+\sum_{j=s+1}^r\alpha_{j,i}(p(x)\cdot v_j)\right]E_i \\
 &= \sum_{i=1}^s\left[p(x)\cdot\left(v_i+\sum_{j=s+1}^r\alpha_{j,i}v_i\right)\right]E_i.
\end{align*}
We see now that the set of all solutions to equation \eqref{lem_geo_set_lin_dep} is given by
\begin{equation*}
\bigcap_{i=1}^s\left\{x\in\F^n\left|p(x)\cdot\left(v_i+\sum_{j=s+1}^r\alpha_{j,i}v_i\right)=0\right.\right\},
\end{equation*}
which is clearly a geometric set by Definition~\ref{def_geo_set}.
\end{proof}

\begin{lem}\label{lem_reduce_vec_set}
For any set of $r$ vectors $\{v_i\}_{i=1}^r$ taken from $\V$, if $S$ is the geometric set generated by
this set of vectors, then there exists a linearly independent subset of $\{v_i\}_{i=1}^r$ that also generates $S$.
\end{lem}
\begin{proof}
If $\{v_i\}_{i=1}^r$ is a linearly independent set, we're done.  Supposing otherwise, and without loss of generality,
we may let $s$ be an integer with $1\leq s<r$ such that $\{v_i\}_{i=1}^s$ is a linearly independent set, and
\begin{equation*}
\mbox{span}\{v_i\}_{i=1}^r = \mbox{span}\{v_i\}_{i=1}^s.
\end{equation*}
Clearly $G(\{v_i\}_{i=1}^s)\subseteq G(\{v_i\}_{i=1}^r)$ since $s<r$.
Now if $x\in G(\{v_i\}_{i=1}^r)$, then for all integers $i\in[1,s]$, we have $p(x)\cdot v_i=0$.
It then follows that for all integers $i\in[s+1,r]$, we have
\begin{equation*}
p(x)\cdot v_i = p(x)\cdot\sum_{j=1}^s\alpha_{i,j}v_j = 0.
\end{equation*}
Therefore, $x\in G(\{v_i\}_{i=1}^s)$.
\end{proof}

\begin{lem}\label{lem_reduce_inf_vec_set}
If $\dim\V$ is finite, then for any set of vectors $\{v_i\}$ taken from $\V$, there exists a finite subset $\{v_{k_i}\}\subset\{v_i\}$
with $0\neq\bigwedge_i v_{k_i}$ such that
\begin{equation*}
G(\{v_i\}) = G(\{v_{k_i}\}).
\end{equation*}
\end{lem}
\begin{proof}
For any set $\{v_i\}$, let $\{v_{k_i}\}$ be any finite subset such that
\begin{equation*}
\mbox{span}\{v_i\} = \mbox{span}\{v_{k_i}\}.
\end{equation*}
We now simply use the same argument made in the proof of Lemma~\ref{lem_reduce_vec_set}
and even invoke it if $\{v_{k_i}\}$ is not a linearly independent set.
\end{proof}

\begin{lem}\label{lem_solution_intersection}
If $\{E_i\}_{i=1}^r$ is any set of $r$ elements taken from our geometric algebra $\G$, then the set $A$ of all solutions in each $\alpha_i$
to the equation
\begin{equation*}
0 = \sum_{i=1}^r\alpha_i E_i
\end{equation*}
is given by
\begin{equation*}
A = \bigcap_{k=1}^r A_k,
\end{equation*}
where each $A_k$ is the set of all solutions in each $\alpha_{i,k}$ to equation $k\in[0,\dim(\V)]$, given by
\begin{equation*}
0 = \sum_{i=1}^r\alpha_{i,k}\langle E_i\rangle_k.
\end{equation*}
\end{lem}
\begin{proof}
Show it.
\end{proof}

\section{Ground Work}

Before we can show how blades and versors can represent geometric sets, we need to lay some ground work
with the following definitions, lemmas, and identities.

In this paper, we will use the following definition for the term ``versor.''
\begin{defn}[Versor]
An element $M_r\in\G$ is called a \emph{versor} if and only if there exists a set of $r$ vectors $\{m_i\}_{i=1}^r$
such that it may be written as
\begin{equation}\label{equ_M_r}
M_r = \prod_{i=1}^r m_i.
\end{equation}
\end{defn}
Note that we do not require each $m_i$ to be invertible.  If we do, we will say that the
versor is invertible.  If we require that at least one of the $m_i$ be null, we will say that
the versor is null.

It is easy to show that versors, like blades, do not have unique factorizations.  Unlike blades, however, the
size of a versor's factorization can very.  This leads us to the following definition.
\begin{defn}[Versor Rank]\label{def_versor_rank}
Given any versor $M_r\in\G$, the rank of the versor $M_r$, denoted $\rank(M_r)$, is the
smallest integer $s\in[0,r]$ such that $M_r$ may be rewritten as a geometric product of $s$ vectors.
\end{defn}

It is clear from Definition~\ref{def_versor_rank} that a lower bound on the rank of any versor
is the highest grade part appearing in its expansion.  An upper bound on the rank is given by
the size of any factorization of the versor we may have.  We will return to the concept of versor
rank once we have developed more results about versors.

For completeness, we now give a formal definition of a blade.
\begin{defn}[Blade]
An element $B_r\in\G$ is called an $r$-\emph{blade} if and only if there exists a linearly independent set of $r$
vectors $\{b_i\}_{i=1}^r$ such that
\begin{equation}\label{equ_B_r}
B_r = \bigwedge_{i=1}^r b_i.
\end{equation}
\end{defn}

\begin{lem}\label{lem_lin_indep_subblades}
Letting $B_r^{(i)}$ denote the $(r-1)$-blade
\begin{equation*}
B_r^{(i)} = \bigwedge_{\substack{j=1\\j\neq i}}^r b_i,
\end{equation*}
the set of $r$ blades $\{B_r^{(i)}\}_{i=1}^r$ is linearly independent.
\end{lem}
\begin{proof}
Supposing to the contrary, and without loss of generality, let
\begin{equation*}
B_{r-1} = B_r^{(r)} = \sum_{i=1}^{r-1}\alpha_i B_r^{(i)} = \left(\sum_{i=1}^{r-1}\alpha_i B_{r-1}^{(i)}\right)\wedge b_r.
\end{equation*}
Now notice that
\begin{equation*}
0\neq B_r = B_{r-1}\wedge b_r = B_r^{(r)}\wedge b_r = \left(\sum_{i=1}^{r-1}\alpha_i B_r^{(i)}\right)\wedge b_r = 0,
\end{equation*}
which is clearly a contradiction.
\end{proof}

We will need a result similar to Lemma~\ref{lem_lin_indep_subblades} as concerns versors.  It is as follows.
\begin{lem}\label{lem_lin_indep_subspades}
Letting $M_r^{(i)}$ denote the versor
\begin{equation*}
M_r^{(i)} = \prod_{\substack{j=1\\j\neq i}}^r m_i,
\end{equation*}
if $0\neq\bigwedge_{i=1}^r m_i$, then the set $\{M_r^{(i)}\}_{i=1}^r$ is a linearly independent set.
\end{lem}
\begin{proof}
By Lemma~\ref{lem_solution_intersection}, it suffices to show that the set $\{\langle M_r^{(i)}\rangle_{r-1}\}_{i=1}^r$ is a linearly independent set.
Now since $0\neq\bigwedge_{i=1}^r m_i$, it is clear that
\begin{equation*}
\langle M_r^{(i)}\rangle_{r-1} = \bigwedge_{\substack{j=1\\j\neq i}}^r m_i.
\end{equation*}
Seeing this, the linear independence of the set $\{\langle M_r^{(i)}\rangle_{r-1}\}_{i=1}^r$ follows immediately
from Lemma~\ref{lem_lin_indep_subblades}.
\end{proof}

We turn now to the establishment of some identities that will be important to our cause.

Letting $a$ denote a vector, and $B_r$ a blade of grade $r$ having the factorization
given in equation \eqref{equ_B_r}, recall that
\begin{equation}\label{equ_aBr_is_a_dot_Br_and_a_wedge_Br}
aB_r = a\cdot B_r + a\wedge B_r.
\end{equation}
Recalling also the commutativities of $a$ with $B_r$ in the inner and outer products as
\begin{align}
a\cdot B_r &= -(-1)^r B_r\cdot a,\label{equ_a_dot_Br_commutativity} \\
a\wedge B_r &= (-1)^r B_r\wedge a,\label{equ_a_wedge_Br_commutativity}
\end{align}
we find that
\begin{align}
a\cdot B_r &= \frac{1}{2}a\cdot B_r - \frac{1}{2}(-1)^r B_r\cdot a\nonumber \\
 &= \frac{1}{2}(aB_r - a\wedge B_r - (-1)^r(B_ra - B_r\wedge a))\nonumber \\
 &= \frac{1}{2}(aB_r-(-1)^rB_ra),\label{equ_a_dot_Br}
\end{align}
and that
\begin{align}
a\wedge B_r &= \frac{1}{2}a\wedge B_r + \frac{1}{2}(-1)^r B_r\wedge a\nonumber \\
 &= \frac{1}{2}(aB_r - a\cdot B_r + (-1)^r(B_ra - B_r\cdot a))\nonumber \\
 &= \frac{1}{2}(aB_r+(-1)^rB_ra).\label{equ_a_wedge_Br}
\end{align}
Now letting $a$ and $b$ each denote a vector, it is not hard to show that for all $r\geq 1$, we have
\begin{equation}\label{equ_a_dot_b_wedge_Br_identity}
a\cdot(b\wedge B_r) + b\wedge(a\cdot B_r) = (a\cdot b)B_r.
\end{equation}
To that end, we apply equations \eqref{equ_a_dot_Br} and \eqref{equ_a_wedge_Br} in writing
\begin{align*}
a\cdot(b\wedge B_r)
 &= \frac{1}{2}\left(a\frac{1}{2}\left(bB_r + (-1)^rB_rb\right)-(-1)^{r+1}\frac{1}{2}\left(bB_r+(-1)^rB_rb\right)a\right) \\
 &= \frac{1}{4}\left(baB_r + (-1)^raB_rb + (-1)^rbB_ra + B_rba\right), \\
b\wedge(a\cdot B_r)
 &= \frac{1}{2}\left(b\frac{1}{2}\left(aB_r-(-1)^rB_ra\right)+(-1)^{r-1}\frac{1}{2}\left(aB_r-(-1)^rB_ra\right)b\right) \\
 &= \frac{1}{4}\left(baB_r - (-1)^rbB_ra - (-1)^raB_rb + B_rab\right),
\end{align*}
from which it is easy to see that
\begin{align*}
a\cdot(b\wedge B_r)+b\wedge(a\cdot B_r) &= \frac{1}{4}(ab+ba)B_r + \frac{1}{4}B_r(ba+ab) \\
 &= \frac{1}{2}(a\cdot b)B_r + \frac{1}{2}B_r(b\cdot a) = (a\cdot b)B_r.
\end{align*}
Similarly, we must note that for all $r>1$, we have
\begin{equation}\label{equ_a_dot_b_dot_Br_identity}
a\cdot(b\cdot B_r) = -b\cdot(a\cdot B_r).
\end{equation}
To see this, we apply equation \eqref{equ_a_dot_Br} in writing
\begin{align*}
a\cdot(b\cdot B_r)
 &= \frac{1}{2}\left(a\frac{1}{2}\left(bB_r-(-1)^rB_rb\right)-(-1)^{r-1}\frac{1}{2}\left(bB_r-(-1)^rB_rb\right)a\right) \\
 &= \frac{1}{4}\left(abB_r - (-1)^raB_rb + (-1)^rbB_ra - B_rba\right),
\end{align*}
Then, by substitution, we can immediately write
\begin{equation*}
b\cdot(a\cdot B_r) = \frac{1}{4}\left(baB_r - (-1)^rbB_ra + (-1)^raB_rb - B_rab\right).
\end{equation*}
Adding these, we then see that
\begin{align*}
a\cdot (b\cdot B)+b\cdot(a\cdot B)
 &= \frac{1}{4}\left(abB_r+baB_r\right)-\frac{1}{4}\left(B_rba+B_rab\right) \\
 &= \frac{1}{4}\left(ab+ba\right)B_r-\frac{1}{4}B_r\left(ba+ab\right) \\
 &= \frac{1}{2}(a\cdot b)B_r - \frac{1}{2}B_r(b\cdot a) = 0.
\end{align*}
Note that we may have arrived at this conclusion sooner had we written
\begin{equation*}
a\cdot(b\cdot B_r) = (a\wedge b)\cdot B_r = -(b\wedge a)\cdot B_r = -b\cdot(a\cdot B_r).
\end{equation*}

We now wish to express the inner product $a\cdot B_r$ as a sum of blades.
Since the case $r=1$ is trivial, we begin by writing, for all $r>1$,
\begin{align}
a\cdot B_r
 &= a\cdot(B_{r-1}\wedge b_r)\nonumber \\
 &= (-1)^{r-1}a\cdot(b_r\wedge B_{r-1})\label{a_dot_Br_stepA} \\
 &= -(-1)^r\left(-b_r\wedge(a\cdot B_{r-1})+(a\cdot b_r)B_{r-1}\right)\label{a_dot_Br_stepB} \\
 &= -(-1)^r\left(-(-1)^r(a\cdot B_{r-1})\wedge b_r+(a\cdot b_r)B_{r-1}\right)\nonumber \\
 &= (a\cdot B_{r-1})\wedge b_r - (-1)^r(a\cdot b_r)B_{r-1}.\label{equ_a_dot_Br_recursive}
\end{align}
Here, we've gone from equation \eqref{a_dot_Br_stepA} to that of \eqref{a_dot_Br_stepB} by
applying the identity given in equation \eqref{equ_a_dot_b_wedge_Br_identity}.

Applied recursively, it is easy to see here from equation \eqref{equ_a_dot_Br_recursive} that an expansion of
$a\cdot B_r$ as a sum of blades is given by
\begin{equation}\label{equ_a_dot_Br_sum_of_blades}
a\cdot B_r = \langle B_r\rangle_0a - \sum_{i=1}^r(-1)^i(a\cdot b_i)\bigwedge_{\substack{j=1\\j\neq i}}^r b_j.
\end{equation}
One might also simply use equation \eqref{equ_a_dot_Br_recursive} to give an inductive
argument of equation \eqref{equ_a_dot_Br_sum_of_blades}.

Notice that for all $r>0$, the term $\langle B_r\rangle_0a$ vanishes in equation \eqref{equ_a_dot_Br_sum_of_blades},
yet its presence allows us the case $r=0$ if we define the summation to be zero in the vacuous case.

Having established equation \eqref{equ_a_dot_Br_sum_of_blades}, it is instructive to show that $a\cdot B_r$ is,
although it is certainly not immediately obvious, a blade of grade $r-1$.  To that end, we write, for all $r>1$,
\begin{equation*}
a\cdot B_r = (a\cdot B_{r-1})\wedge\left(b_r - \frac{a\cdot b_r}{a\cdot b_{r-1}}b_{r-1}\right),
\end{equation*}
with the understanding that if $a\cdot b_{r-1}$ is zero, we can anti-commute vector factors in equation \eqref{equ_a_dot_Br_sum_of_blades}
until this is the case, or else $a\cdot B_r$ is zero anyway.
An inductive argument can now be easily made that $a\cdot B_r$ is indeed a blade of grade $r-1$.
Notice that this proof works in any geometric algebra, regardless of the associated bilinear form.  In a euclidean geometric
algebra, an easier proof is had by writing
\begin{equation*}
a\cdot B_r = (a_{\perp} + a_{\parallel})\cdot B_r = a_{\parallel}\cdot B_r,
\end{equation*}
where $a_{\perp}$ is the orthogonal rejection $a$ from $B_r$, while $a_{\parallel}$ is the
orthogonal projection of $a$ down onto $B_r$.  The blade $B_r$ can now be orthogonalized,
with $a_{\parallel}$ as a principle factor, using the Gram-Schmidt orthogonalization process.\footnote{This process
cannot always be performed on blades taken from a non-euclidean geometric algebra.  To see this, consider rewriting
$a\wedge b$ as $a\wedge(b+\lambda a)$ where $a\cdot(b+\lambda a)=0$.  In a non-euclidean geometric algebra,
no such scalar $\lambda$ may exist due to $a$ being null.  For a description of the Gram-Schmidt process, see \cite{}.}
This factor then falls out quite easily, and we're left with a blade of grade $r-1$.

Letting $M_r$ denote a versor having the factorization given in equation \eqref{equ_M_r}, recall that
\begin{equation*}
M_r = \sum_{i=1}^r\left\langle M_r\right\rangle_i,
\end{equation*}
To be more precise, if $r$ is even,
\begin{equation}\label{equ_Mr_even}
M_r = \sum_{i=0}^{r/2}\left\langle M_r\right\rangle_{2i},
\end{equation}
while if $r$ is odd, we have
\begin{equation}\label{equ_Mr_odd}
M_r = \sum_{i=1}^{(r+1)/2}\left\langle M_r\right\rangle_{2i-1}.
\end{equation}
To see this, consider first the trivial case of $r=0$; then, for any $r>0$, the equation
\begin{equation}\label{equ_Mr_split}
M_r = M_{r-1}m_r = \langle M_{r-1}\rangle_1^r\cdot m_r + \langle M_{r-1}\rangle_1^r\wedge m_r + \langle M_{r-1}\rangle_0 m_r.
\end{equation}
Here we have extended our notation $\langle\cdot\rangle_i^j$ to mean a culling of all enclosed blades not of a grade falling
in the interval $[i,j]$.  Put another way, we have
\begin{equation*}
\langle M_r\rangle_i^j = \sum_{k=i}^j\langle M_r\rangle_k.
\end{equation*}

An inductive hypothesis can now be stated that equations \eqref{equ_Mr_even} and \eqref{equ_Mr_odd} hold for $r-1$.
If $r$ is even, then, by our inductive hypothesis, $M_{r-1}$, when expanded as a sum of blades, consists only of blades of odd grade,
and it is clear that equation \eqref{equ_Mr_split} becomes \eqref{equ_Mr_even}.  If $r$ is odd, then, by our inductive hypothesis, $M_{r-1}$, when expanded as
a sum of blades, consists only of blades of even grade, and it is clear that equation \eqref{equ_Mr_split} becomes \eqref{equ_Mr_odd}.

Now let $a$ be a vector, and convince yourself that
\begin{align}
a\cdot M_r &= -(-1)^r M_r\cdot a,\label{equ_a_dot_Mr_commutativity} \\
a\wedge M_r &= (-1)^r M_r\wedge a.\label{equ_a_wedge_Mr_commutativity}
\end{align}
Refer to equations \eqref{equ_a_dot_Br_commutativity} and \eqref{equ_a_wedge_Br_commutativity} to see this.

We now turn our attention to the following identity.
\begin{equation}\label{equ_gr_zero_part_of_Mr}
\langle M_r\rangle_0 = \langle M_{r-1}\rangle_1\cdot m_r
\end{equation}
Note that this is trivial in the case that $r$ is odd, since neither $M_r$ nor $M_{r-1}$ have parts of grade zero nor one, respectively.
Letting $r$ be even, we write
\begin{equation*}
M_r = M_{r-1}m_r = M_{r-1}\cdot m_r + M_{r-1}\wedge m_r - \langle M_r\rangle_0 m_r.
\end{equation*}
Now taking the grade zero part of both sides, we get
\begin{equation*}
\langle M_r\rangle_0 = \langle M_{r-1}\cdot m_r\rangle_0 = \langle M_{r-1}\rangle_1\cdot m_r.
\end{equation*}

We now wish to express the inner product $a\cdot M_r$ as a sum of versors.
Since the case $r=1$ is trivial, we begin by writing, for all $r>1$,
\begin{align}
a\cdot M_r &= a\cdot (M_{r-1}m_r)\nonumber \\
 &= a\cdot((\langle M_{r-1}\rangle_0 + \langle M_{r-1}\rangle_1 + \langle M_{r-1}\rangle_2^r)m_r)\nonumber \\
 &= \langle M_{r-1}\rangle_0a\cdot m_r + (\langle M_{r-1}\rangle_1\cdot m_r)a\nonumber \\
 &+ (a\cdot\langle M_{r-1}\rangle_1)m_r - (a\cdot m_r)\langle M_{r-1}\rangle_1 + a\cdot(\langle M_{r-1}\rangle_2^rm_r).\label{equ_a_dot_Mr_stepA}
\end{align}
We will return to this equation momentarily.  Until then, to ease notation, let us write $M=\langle M_{r-1}\rangle_2^r$ and see that
\begin{align}
a\cdot(Mm_r)
 &= a\cdot(M\cdot m_r + M\wedge m_r)\nonumber \\
 &= -(-1)^{r-1}a\cdot(m_r\cdot M) + (-1)^{r-1}a\cdot(m_r\wedge M)\label{equ_a_dot_Mr_stepB} \\
 &= (-1)^rm_r\cdot(a\cdot M) - (-1)^r\left[-m_r\wedge(a\cdot M)+(a\cdot m_r)M\right]\label{equ_a_dot_Mr_stepC} \\
 &= (a\cdot M)\cdot m_r + (a\cdot M)\wedge m_r - (-1)^r(a\cdot m_r)M\nonumber \\
 &= (a\cdot M)m_r - (-1)^r(a\cdot m_r)M.\label{equ_a_dot_Mr_stepD}
\end{align}
Note here our use of equations \eqref{equ_a_dot_b_dot_Br_identity} and \eqref{equ_a_dot_b_wedge_Br_identity} to
arrive at equation \eqref{equ_a_dot_Mr_stepC} from \eqref{equ_a_dot_Mr_stepB}.

Returning now to equation \eqref{equ_a_dot_Mr_stepA}, if we plug equation \eqref{equ_a_dot_Mr_stepD} into it
under the assumption that $r$ is odd, we get
\begin{equation}
a\cdot M_r = (a\cdot M_{r-1})m_r + (a\cdot m_r)M_{r-1} - \langle M_{r-1}\rangle_0am_r.
\end{equation}
And if we plug equation \eqref{equ_a_dot_Mr_stepD} into equation \eqref{equ_a_dot_Mr_stepA} under the assumption
that $r$ is even, we get
\begin{equation}
a\cdot M_r = (a\cdot M_{r-1})m_r - (a\cdot m_r)M_{r-1} + (\langle M_{r-1}\rangle_1\cdot m_r)a.
\end{equation}
It then follows, despite the parity of $r$, that
\begin{align}
a\cdot M_r &= (a\cdot M_{r-1})m_r - (-1)^r(a\cdot m_r)M_{r-1}\nonumber \\
 &- \langle M_{r-1}\rangle_0am_r + \langle M_r\rangle_0a.\label{equ_a_dot_Mr_recursive}
\end{align}
Note the use of equation \eqref{equ_gr_zero_part_of_Mr} here in our arrival at equation \eqref{equ_a_dot_Mr_recursive}.

Applied recursively, it is now easy to see from equation \eqref{equ_a_dot_Mr_recursive} that an expansion of
$a\cdot M_r$ as a sum of versors is given by
\begin{equation}\label{equ_a_dot_Mr_sum_of_versors}
a\cdot M_r = \langle M_r\rangle_0a - \sum_{i=1}^r(-1)^i(a\cdot m_i)\prod_{\substack{j=1\\j\neq i}}^rm_j.
\end{equation}
To see this, consider an inductive argument.  The cases $r=0$ and $r=1$ follow trivially by inspection.
Now make the inductive hypothesis that equation \eqref{equ_a_dot_Mr_sum_of_versors} holds for a fixed case $r-1$.
Then, applying the recursive formula \eqref{equ_a_dot_Mr_recursive} to the equation in \eqref{equ_a_dot_Mr_sum_of_versors},
adjusted for the case $a\cdot M_{r-1}$, we get equation \eqref{equ_a_dot_Mr_sum_of_versors}, thereby completing
our proof by induction.

It is very interesting now to compare this equation \eqref{equ_a_dot_Mr_sum_of_versors} with that of \eqref{equ_a_dot_Br_sum_of_blades}.
One equation is had by the other by a replacement of all outer products with geometric products, or vice-versa.

Having shown that $a\cdot B_r$ was a blade of grade $r-1$, we must consider here whether $a\cdot M_r$ can be written
as a product of $r-1$ vectors.  With that in mind, we write
\begin{align}
a\cdot M_r - \langle M_r\rangle_0a &= \sum_{i=1}^r\alpha_i M_r^{(i)}\nonumber \\
 &= \left[\sum_{i=1}^{r-1}\alpha_iM_{r-1}^{(i)}\right]\left(m_r + \alpha_r\left[\sum_{i=1}^{r-1}\alpha_iM_{r-1}^{(i)}\right]^{-1}M_{r-1}\right),\label{equ_factor_a_dot_M_r}
\end{align}
where $\alpha_i=-(-1)^i(a\cdot m_i)$.  Now, if an inverse of $a\cdot M_{r-1}-\langle M_{r-1}\rangle_0a$ does exist, then
it is probably of the form
\begin{equation*}
\left[\sum_{i=1}^{r-1}\alpha_iM_{r-1}^{(i)}\right]^{-1} = \sum_{i=1}^{r-1}\beta_i\left(M_{r-1}^{(i)}\right)^{\sim}.
\end{equation*}
Assuming a solution to this equation in each $\beta_i$ exists, we can go on to write
\begin{equation*}
\sum_{i=1}^{r-1}\beta_i\left(M_{r-1}^{(i)}\right)^{\sim}M_{r-1} = \sum_{i=1}^{r-1}\beta_i \left(\prod_{j=1}^{i-1}m_i^2\right)\tilde{V}_{i+1}m_iV_{i+1},
\end{equation*}
where $V_i$ is given by
\begin{equation*}
V_i = \sum_{j=i}^{r-1} m_i.
\end{equation*}
Looking back at equation \eqref{equ_factor_a_dot_M_r}, we can see now how a vector could be factored
out of $a\cdot M_r - \langle M_r\rangle_0a$ in terms of the geometric product.

We finally return now to the concept of versor rank.  The proof of the following lemma is pieced together
from the factorization algorithm given in \cite[p. 108]{Perwass09}.
\begin{lem}\label{lem_versor_rank_iff_lin_indep}
For every invertible versor $M_r$, we have
\begin{equation*}
\mbox{$0\neq\bigwedge_{i=1}^r m_i$ if and only if $\rank(M_r)=r$.}
\end{equation*}
\end{lem}
\begin{proof}
One direction being trivial, we only show here that if $\rank(M_r)=r$, then
the set of vectors $\{m_i\}_{i=1}^r$ is a linearly independent set.

Let $M\in\G$ be any non-zero versor of $\G$ with an unknown factorization,
and let $r$ be the largest integer for which $\langle M\rangle_r\neq 0$.
The integer $r$ being a lower-bound on the rank of $M$, if a factorization $M_r$
of $M$ of size $r$ can be found, then we have found the rank of $M=M_r$; namely, $r$.
This factorization must then be linearly-independent, because it generates $\langle M\rangle_r$,
which is clearly a blade of grade $r$.

The case $r=0$ is trivial; so letting $r>0$,
there exists a non-zero vector $m_1$ such that $m_1\wedge\langle M\rangle_r=0$;
and therefore, $m_1M$ is a versor with $\langle m_1M\rangle_{r-1}\neq 0$ with
$r-1$ being the largest integer for which this is the case.
If $r-1>0$, we repeat this process with a new vector $m_2$.  Continuing on in this fasion, we find
that, in general, we have, for a linearly independent set of $r$ vectors $\{m_i\}_{i=1}^r$,
\begin{equation*}
\left(\prod_{i=1}^r m_i\right)M=\left\langle\left(\prod_{i=1}^r m_i\right)M\right\rangle_0=\lambda\in\R,
\end{equation*}
which shows that
\begin{equation*}
M^{-1} = \lambda^{-1}\tilde{M_r}\implies M_r = \lambda(M^{-1})^{\sim} = \lambda\prod_{i=1}^r m_{r-i+1},
\end{equation*}
showing that we have found a factorization of $M_r$, as desired.
\end{proof}
Note here our requirement that $M_r$ be invertible.  It is very likely, however,
that the statement of Lemma~\ref{lem_versor_rank_iff_lin_indep} also holds true for null versors.

It is important to notice that if $M$ is a versor with rank $r$ and $v$ is a vector
such that $v\wedge\langle M\rangle_r=0$, then it is not necessarily the case
that $vM = v\cdot M$.  This is because we may have $v\wedge\langle M\rangle_{r-2}\neq 0$.
Hmmm, if we have $v\wedge\langle M\rangle_{r-2}=0$, is that enough?
The expansion of $\langle M\rangle_{r-2}$, given a known factorization, is not too hard to find.

\section{Blades And Versors As Representatives Of Geometric Sets}

At last we now have enough ground covered to begin a treatment of geometric set representation by blades and versors.
We start by showing that any element $E$ of $\G$ is representative of a geometric set as follows.

\begin{defn}\label{def_geo_set_rep_by_E}
Letting the function $\gd:\G\to P(\F^n)$ be defined as
\begin{equation*}
\gd(E) = \{x\in\F^n|p(x)\cdot E=\langle E\rangle_0p(x)\},
\end{equation*}
we call $\gd(E)$ \emph{the geometric set represented by $E$}.
\end{defn}

Notice that Definition~\ref{def_geo_set_rep_by_E} does not exactly use the concept of
the inner-product null-space as found in \cite{}.  The reason for this becomes apparent
in the following lemma.

\begin{lem}\label{lem_intersect_grade_parts}
For any element $E\in\G$, we have
\begin{equation*}
\gd(E) = \bigcap_{i=0}^{\dim\V}\gd(\langle E\rangle_i).
\end{equation*}
\end{lem}
\begin{proof}
This follows immediately from Lemma~\ref{lem_solution_intersection}.
\end{proof}

\begin{lem}
For any element $E\in\G$, the set $\gd(E)$ is a geometric set.
\end{lem}
\begin{proof}
If it can be shown, for every integer $k\in[0,\dim(\V)]$, that $\gd(\langle E\rangle_k)$ is a geometric set,
then $\gd(E)$ is a geometric set by Lemma~\ref{lem_intersect_grade_parts}.
The case $k=0$ is trivial.  Letting $k>0$, it is clear by equation \eqref{equ_a_dot_Br_sum_of_blades} that
\begin{equation*}
0 = p(x)\cdot\langle E\rangle_k = \sum_i (p(x)\cdot v_i)B_i,
\end{equation*}
where each $B_i$ is a blade of grade $k-1$.  If $k=1$, $p(x)$ factors out of the sum, and we clearly get a geometric
set.  If $k>1$, then we see that the set of all solutions $x$
to this equation gives us a geometric set by Lemma~\ref{lem_geo_set_lin_dep}.
\end{proof}

\begin{lem}\label{lem_all_geo_sets_rep_by_blades_or_versors}
For every geometric set $S$, there exists an element $E\in\G$ such that $\gd(E)=S$.
Moreover, we can always find $E$ as a blade or versor in $\G$.
\end{lem}
\begin{proof}
If $\dim\V$ is finite, then, by Lemma~\ref{lem_reduce_inf_vec_set}, any set of vectors generating the geometric
set $S$ may be reduced to a finite, linearly independent subset $\{v_i\}_{i=1}^s$.
We then have
\begin{equation*}
S=\gd\left(\bigwedge_{i=1}^s v_i\right) = \gd\left(\prod_{i=1}^s v_i\right).
\end{equation*}
To see this, consider equation \eqref{equ_a_dot_Br_sum_of_blades} with Lemma~\ref{lem_lin_indep_subblades},
and equation \eqref{equ_a_dot_Mr_sum_of_versors} with Lemma~\ref{lem_lin_indep_subspades}.

If $\dim\V$ is infinite, then we must consider blades of infinite grade, or versors of infinite rank, as the dimension
of the vector space spanned by the set of vectors $\{v_i\}$ generating $S$ may be infinite.
\end{proof}

Another way to see that $S=\gd(\prod_{i=1}^s v_i)$ is to notice that
since $\{v_i\}_{i=1}^s$
is a linearly independent set, then for every $r$-blade $B_r$ appearing in the expansion of $\prod_{i=1}^s v_i$,
where $r\leq s$, we know that $B_r$ is a subspace of $B_s=\left\langle\prod_{i=1}^s v_i\right\rangle_s$.  It then follows that
for all $r\leq s$, we have
\begin{equation*}
\gd\left(B_r\right)\subseteq\gd\left(B_s\right),
\end{equation*}
and so our result also goes through by Lemma~\ref{lem_intersect_grade_parts}.

\begin{lem}
If for every vector $v\in\V$, the expression $p(x)\cdot v$ is a polynomial in the components of $x$,
then for every algebraic set $G(\{v_i\})$, where $\{v_i\}\subseteq\V$, there exists an $s$-blade $B_s\in\G$ such that
$G(\{v_i\})=\gd(B_s)$.
\end{lem}
\begin{proof}
By the Hilbert Basis Theorem (see \cite[p. 204]{Garrity13}), there exists a finite subset $\{v_i\}_{i=1}^r\subset\{v_i\}$
such that $G(\{v_i\}_{i=1}^r)=G(\{v_i\})$.  Then, by Lemma~\ref{lem_reduce_vec_set}, a linearly independent subset $\{v_i\}_{i=1}^s$
of $\{v_i\}_{i=1}^r$ can be found such that $G(\{v_i\}_{i=1}^s)=G(\{v_i\}_{i=1}^r)$.
Lastly, we see that $G(\{v_i\}_{i=1}^s)=\gd(\bigwedge_{i=1}^s v_i)$.  Now let $B_s=\bigwedge_{i=1}^s v_i$.
\end{proof}

We will assume a finite-dimensional vector space $\V$ from here on.

Returning to Lemma~\ref{lem_all_geo_sets_rep_by_blades_or_versors}, it is telling us that if we only use blades, or only use versors, to represent geometric sets,
then we don't fail to generate any geometric set
that we could have otherwise represented using any other type of element of $\G$.

The proof of Lemma~\ref{lem_all_geo_sets_rep_by_blades_or_versors} shows that if we know
any factorization of a blade, then we can easily formulate a versor representing the same geometric set
by simply taking the vector factors together in the geometric product.\footnote{A treatment of blade factorization can be found in \cite[p. 533]{Dorst07}.}
This, however, does not work in reverse.  Not every versor factorization is linearly independent; but,
as the proof of Lemma~\ref{lem_all_geo_sets_rep_by_blades_or_versors} also shows, if we can find such a factorization, then we can likewise convert a versor
to a blade representing the same geometric set by simply taking the vector factors together in the outer product.\footnote{A treatment of versor factorization can be found in \cite[p. 107]{Perwass09}.}
If we want the associated blade, however, there is no need to factor the versor!
By Lemma~\ref{lem_versor_rank_iff_lin_indep}, we simply take the highest grade part of the versor's expansion.
To find the associated versor, there is no need to factor a blade if we know it to already have a pair-wise orthogonalize factorization.\footnote{A treatment of blade orthogonalization is given in \cite[p. 88]{Doran03}.}

% Must address direct representation.  So far, only used indirect representation.  Is it going to work?

% Meet is computed by geometric product of orthogonalized blades -- easy to show?  What does the geometric product do for us generally -- hard question.  Consider simple case of tacking on a vector to a l.i. set, where the vec is in the space spanned by that set.
% Using versors, all of group theory brought to bear on geometries.  Normal subgroups?  Special types of groups?  Isomorphisms?  Homomorphisms?

% Meet, join, geometric product, duals...?

% Reducible blades arrise when geometry is not characterized using the entire OPNS of blades.  This is
% when p(x) for all x does not expore every possible vector.  For all v, there must be an x such that v=p(x).
% May want to study the reduction of blades for algebraic sets as removing zeros of multiplicity.  Use vandermonde matrix result in algorithms book to show linearly independent set/irreducible form.

\section{Examples In The Conformal Model}

\section{Closing Remarks}

\begin{thebibliography}{9}

\bibitem{Parkin15}
S. Parkin,
\emph{An Introduction To Geometric Sets}.
Advances in Applied Clifford Algebras, Volume 25, Issue Unknown, pp. 639-655, 2015.

\bibitem{Perwass09}
C. Perwass,
\emph{Geometric Algebra with Applications in Engineering}
Springer-Verlag Berlin Heidelberg, 2009.

\bibitem{Garrity13}
T. Garrity, et. al.,
\emph{Algebraic Geometry, A Problem Solving Approach}
American Mathematical Society, Institute for Advanced Study

\bibitem{Hestenes99}
D. Hestenes,
\emph{New Foundations for Classical Mechanics}
Kluwer Academic Publishers, 1999.

\bibitem{Doran03}
C. Doran, et. al.,
\emph{Geometric Algebra for Physicists}
Cambridge University Press, 2003.

\bibitem{Dorst07}
L. Dorst, et. al.,
\emph{Geometric Algebra for Computer Science}
Morgan Kaufmann Publishers, 2007.

\end{thebibliography}

\end{document}