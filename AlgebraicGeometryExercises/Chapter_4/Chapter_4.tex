\documentclass[12pt]{article}

\usepackage{amsmath}
\usepackage{amssymb}
\usepackage{amsthm}

\title{Chapter 4 Exercises\\Algeberaic Geometry\\A Problem Solving Approach}
\author{Spencer T. Parkin}

\newtheorem{theorem}{Theorem}[section]
\newtheorem{definition}{Definition}[section]
\newtheorem{corollary}{Corollary}[section]
\newtheorem{identity}{Identity}[section]
\newtheorem{lemma}{Lemma}[section]
\newtheorem{result}{Result}[section]

%\newcommand{\gcd}{\mbox{gcd}}
\newcommand{\lcm}{\mbox{lcm}}
\newcommand{\abs}{\mbox{abs}}
\newcommand{\Z}{\mathbb{Z}}
\newcommand{\R}{\mathbb{R}}

\begin{document}
\maketitle

\section*{Exercise 4.2.9}

Let $I$ be the ideal in $k[x_1,\dots,x_n]$ generated by a set $S\subset k[x_1,\dots,x_n]$.
Show that $V(S)=V(I)$.  Thus every algebraic set is defined by an ideal.

Let $(a_1,\dots,a_n)\in V(S)$ and consider $f\in I$.  Notice that $f$ has the form
\begin{equation*}
f = \sum_{s\in S} s f_s,
\end{equation*}
where, for each $s\in S$, $f_s$ is any polynomial taken from $k[x_1,\dots,x_n]$.
From this it is clear that $f(a_1,\dots,a_n)=0\implies (a_1,\dots,a_n)\in V(I)$.

Now let $(a_1,\dots,a_n)\in V(I)$ and consider $f\in S$.  Clearly $f(a_1,\dots,a_n)=0$,
because $S\subseteq I$.  So $f\in V(I)$.

\end{document}