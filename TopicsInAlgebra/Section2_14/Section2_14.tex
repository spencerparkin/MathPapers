\documentclass[12pt]{article}

\usepackage{amsmath}
\usepackage{amssymb}
\usepackage{amsthm}

\title{Section 2.14 Exercises\\Herstein's Topics In Algebra}
\author{Spencer T. Parkin}

\newtheorem{theorem}{Theorem}[section]
\newtheorem{definition}{Definition}[section]
\newtheorem{corollary}{Corollary}[section]
\newtheorem{identity}{Identity}[section]
\newtheorem{lemma}{Lemma}[section]
\newtheorem{result}{Result}[section]

%\newcommand{\gcd}{\mbox{gcd}}
\newcommand{\lcm}{\mbox{lcm}}
\newcommand{\Z}{\mathbb{Z}}
\newcommand{\cl}{\mbox{Cl}}
\newcommand{\aut}{\mbox{Aut}}

\begin{document}
\maketitle

\section*{Thoughts On Theorem 2.14.1}

Herstein's proof of Theorem 2.14.1 is the hardest yet, but looks so much easier than Gallian's.
In trying to understand the idea behind his proof, the following may be of interest.

Let $G$ be a finite abelian group having $k$ (normal) subgroups $A_1,\dots,A_k$ such that
\begin{equation*}
G=\prod_{i=1}^k A_i,
\end{equation*}
and for every $g\in G$, there exists a unique factorization of $g$ in the form
\begin{equation*}
g=\prod_{i=1}^k a_i,
\end{equation*}
with each $a_i\in A_i$.  Now, for any integer $1\leq j\leq k$, if we consider the factor group
$G/A_j$, the interesting observation we can make is that
\begin{equation*}
G/A_j\approx\prod_{\substack{i=1\\i\neq j}}^k A_i=G_j.
\end{equation*}
By Problem 9 of Section 13, and since $G$ is abelian, we can convince ourselves that $G_j$ is an internal direct product of a subgroup of $G$,
but to convince ourselves of the isomorphism, let us begin by writing down an obvious choice of isomorphism; namely,
$\phi_j:G_j\to G/A_j$, given by
\begin{equation*}
\phi_j(x)=xA_j.
\end{equation*}
Is $\phi_j$ an isomorphism?  It's clearly operation-preserving.  Is it onto?  To see that it is, notice that
\begin{equation*}
G/A_j=\left\{\left.\left(\prod_{i=1}^k a_i\right)A_j\right|a_i\in A_i\right\}=\left\{\left.\left(\prod_{\substack{i=1\\i\neq j}}^k a_i\right)A_j\right|a_i\in A_i\right\},
\end{equation*}
since $G$ is abelian.  We now show that $\phi_j$ is one-to-one.  To that end, for any $x,y\in G$, we know that $xA_j=yA_j$ if and only if
$y^{-1}x\in A_j$.  Now realize that, since $G_j$ is an internal direct product, we may uniquely factor $y^{-1}x$ as
\begin{equation*}
y^{-1}x = \prod_{\substack{i=1\\i\neq j}}^k a_i,
\end{equation*}
with each $a_i\in A_i$, but again by Problem 9 of Section 13, we must have
\begin{equation*}
A_j\cap\prod_{\substack{i=1\\i\neq j}}^k A_i=\{e\},
\end{equation*}
since $G$ is an internal direct product.  Now since $y^{-1}x\in A_j\cap G_j$,
we must have $y^{-1}x=e$, which is what we needed to get $x=y$.

This result may generalize to $G/A\approx G'$, where
\begin{equation*}
A=\prod_{i\in I} A_i,\;\;\;\;G' = \prod_{i\not\in I} A_i,
\end{equation*}
where $I$ is some subset of the integers in $[1,k]$.

Returning to Theorem 2.14.1, perhaps there's an easier, somewhat inductive proof.
The idea is to show that any finite abelian group $G$ can be rewritten as the internal direct
product of a cyclic subgroup $A$ and some other subgroup $G'$.  We now repeat the process on $G'$.
After choosing an element $a\in G$ of maximal prime power order for some prime divisor $p$ of $|G|$,
we can let $A=\langle a\rangle$.  We must now show that $G'$ exists and is isomorphic to $G/A$.
This may not be so easy.

\end{document}