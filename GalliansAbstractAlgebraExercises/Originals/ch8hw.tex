\documentclass{article}
\usepackage{amsmath}
\usepackage{amssymb}
\title{Chapter 8 Homework}
\author{Spencer}
\addtolength{\oddsidemargin}{-.575in}
\addtolength{\evensidemargin}{-.575in}
\addtolength{\textwidth}{1.0in}
\addtolength{\topmargin}{-.575in}
\addtolength{\textheight}{1.25in}
\begin{document}
\maketitle

\newcommand{\Z}{\mathbb{Z}}
\newcommand{\lcm}{\mbox{lcm}}
%\newcommand{\gcd}{\mbox{gcd}}
\newcommand{\aut}{\mbox{Aut}}

Let $G$ be a group.  Then when $k$ is a positive integer, let the notation $G^k$ denote the
number of elements in $G$ having order $k$.  
%Notice that if
%$G$ was a cyclic group of order $n$, then $G^n=\phi(n)$, where $\phi$
%is Euler's phi function.
%By Theorem 4.4, if $k$ divides $n$, then $G^k=\phi(k)$.
%For any group $G$, $G^1=1$ since the identity is the only
%element of order 1.

\section*{Problem 4}

Show that $G\oplus H$ is Abelian if and only if $G$ and $H$ are Abelian.

Suppose $G$ and $H$ are Abelian.  Let $a=(a_G,a_H)\in G\oplus H$ and
$b=(b_G,b_H)\in G\oplus H$.  Then...
\begin{equation*}
ab = (a_G,a_H)(b_G,b_H)
 = (a_Gb_G,a_Hb_H)
 = (b_Ga_G,b_Ha_H)
 = (b_G,b_H)(a_G,a_H)
 = ba
\end{equation*}
...showing that $G\oplus H$ is Abelian.  Now suppose $G\oplus H$ is
Abelian and let $a$ and $b$ be as before.  Then...
\begin{equation*}
(a_Gb_G,a_Hb_H)
 = (a_G,a_H)(b_G,b_H)
 = (b_G,b_H)(a_G,a_H)
 = (b_Ga_G,b_Ha_H)
\end{equation*}
This implies that $a_Gb_G = b_Ga_G$ showing that $G$ is Abelian, and
that $a_Hb_H = b_Ha_H$ showing that $H$ is Abelian.  Recall that we
could choose any element from $G\oplus H$ that we wanted.

\section*{Problem 6}

Prove, by comparing orders of elements, that $\Z_8\oplus \Z_2$ is not
isomorphic to $\Z_4\oplus \Z_4$.

The possible orders of elements in $\Z_8\oplus Z_2$ come from
$\lcm(\{1,2,4,8\},\{1,2\})$, and the possible orders of elements
in $\Z_4\oplus \Z_4$ come from $\lcm(\{1,2,4\},\{1,2,4\})$.
Notice that $|(1,0)|=8$ in $\Z_8\oplus Z_2$, while there can
be no element of this order in $\Z_4\oplus \Z_4$.  Since isomorphisms
preserve orders, there is no isomorphism between these two groups.

\section*{Problem 10}

How many elements of order 9 does $\Z_3\oplus \Z_9$ have?  Don't use a
brute force attack.

We want to find all the ways that $\lcm(\{1,3\},\{1,3,9\})=9$.
We only get 9 when 9 is used as one of the arguments here.
So the answer is $\Z_3^1 \Z_9^9 + \Z_3^3 \Z_9^9 = (1)(6)+(2)(6)=18$.

\section*{Problem 14}

Suppose $G_1\approx G_2$ and $H_1\approx H_2$.  Prove that
$G_1\oplus H_1\approx G_2\oplus H_2$.

Let $\alpha:G_1\to G_2$ be the isomorphism from $G_1$ to $G_2$, and
let $\beta:H_1\to H_2$ be the isomorphism from $H_1$ to $H_2$.
Then define $\gamma:G_1\oplus H_1\to G_2\oplus H_2$ as
$\gamma((x,y))=(\alpha(x),\beta(y))$.  The one-to-one, onto, and
operation preserving properties of $\gamma$ will now fall out of
$\alpha$ and $\beta$.

Let $a=(x_1,y_1)\in G_1\oplus H_1$ and $b=(x_2,y_2)\in G_1\oplus H_1$.
\begin{multline*}
\gamma(ab) = \gamma((x_1,y_1)(x_2,y_2)) = \gamma((x_1y_1,x_2y_2))
 = (\alpha(x_1x_2),\beta(y_1y_2)) \\
 = (\alpha(x_1)\alpha(x_2),\beta(y_1)\beta(y_2))
 = (\alpha(x_1),\beta(y_1))(\alpha(x_2),\beta(y_2)) = \gamma(a)\gamma(b)
\end{multline*}
This shows that $\gamma$ is operation preserving.  Using $a$ and $b$ again,
suppose that $\gamma(a)=\gamma(b)$.  This implies that
$(\alpha(x_1),\beta(y_1))=(\alpha(x_2),\beta(y_2))$ which implies that
$\alpha(x_1)=\alpha(x_2)\implies x_1=x_2$ and $\beta(y_1)=\beta(y_2)\implies y_1=y_2$.
This then implies that $a=(x_1,y_1)=(x_2,y_2)=b$, showing that $\gamma$
is one-to-one.

Now choose $(a,b)\in G_2\oplus H_2$.  Then choose $(\alpha^{-1}(a),\beta^{-1}(b))\in G_1\oplus H_1$
so that...
\begin{equation*}
\gamma((\alpha^{-1}(a),\beta^{-1}(b))) = (\alpha(\alpha^{-1}(a)),\beta(\beta^{-1}(b)))
 = (a,b)
\end{equation*}
...showing that $\gamma$ is onto.  We've now shown enough to show that
$\gamma$ is an isomorphism from $G_1\oplus H_1$ to $G_2\oplus H_2$.
Notice that the result could be generalized to arbitrarily sized direct products.
That is, $G_1\approx H_1$, ..., $G_n\approx H_n$ implies that
$G_1\oplus\dots\oplus G_n\approx H_1\oplus\dots\oplus H_n$.  In fact,
given any direct product, you can probably swap out components of the
product for groups that are isomorphic to the group you're swapping out,
and keep all the group theoretic properties.

\section*{Problem 16}

Determine the number of elements of order 15 and the
number of cyclic subgroups of order 15 in $\Z_{30}\oplus\Z_{20}$.

By Lagrange's Theorem, the order of an element divides the order of the group.
So, using Theorem 8.1, we're looking for all the ways in which we can have...
\begin{equation*}
15 = \lcm(\{1,2,3,5,6,10,15,30\},\{1,2,4,5,10,20\})
\end{equation*}
Eliminating the arguments that can never be multiples of 15, this reduces to...
\begin{equation*}
15 = \lcm(\{1,3,5,15\},\{1,5\})
\end{equation*}
We can now break this into the cases $15=\lcm(3,5)$, and $15=\lcm(15,\{1,5\})$.
In the former case, it happens $\Z_{30}^3\Z_{20}^5=\phi(3)\phi(5)=8$ times,
and in the latter case, it happens
$\Z_{30}^{15}\Z_{20}^1+\Z_{30}^{15}\Z_{20}^5=\phi(15)\phi(1)+\phi(15)\phi(5)=40$.
So the total number of elements of order 15 in $\Z_{30}\oplus\Z_{20}$ is 48.
We can use any one of these 48 elements of order 15 to generate a cyclic subgroup
of order 15.  Each such subgroup will then have $\phi(15)=8$ generators.
Since no two such subgroups can share a generator, there are $48/8=6$ cyclic
subgroups of order 15 in $\Z_{30}\oplus\Z_{20}$.

\section*{Problem 20}

The group $S_3\oplus\Z_2$ is isomorphic to one of the groups: $\Z_{12}$,
$\Z_6\oplus\Z_2$, $A_4$, and $D_6$.  Determine which one by elimination.

Notice that $S_3\not\approx\Z_6$ so we can eliminate $\Z_6\oplus\Z_2$.
Also, $S_3\oplus\Z_2$ has no generator, so we can eliminate $\Z_{12}$.
Then, the number of elements in $A_4$ satisfying the equation $x^2=e$
is 4 by Table 5.1.  But there are at least 8 solutions to this equation
in $S_3\oplus\Z_2$.  All that remains is $D_6$, so it must be the one
isomorphic to $S_3\oplus\Z_2$.

\section*{Problem 22}

Find a subgroup of $\Z_4\oplus\Z_2$ that is not of the form $H\oplus K$,
where $H$ is a subgroup of $\Z_4$ and $K$ is a subgroup of $\Z_2$.

Notice that $\{(0,0),(1,1),(2,0),(3,1)\}$ is a proper subgroup of
$\Z_4\oplus\Z_2$ while it uses every element in $\Z_4$ and $\Z_2$.
So it can't be of the undesired form.

\section*{Problem 28}

Find a subgroup of $\Z_{12}\oplus\Z_4\oplus\Z_{15}$ that is of order 9.

Using Corollary 2 of Theorem 8.2, we see that...
\begin{equation*}
\Z_{12}\oplus\Z_4\oplus\Z_{15}\approx\Z_3\oplus\Z_4\oplus\Z_4\oplus\Z_3\oplus\Z_5
\end{equation*}
...since $\gcd(3,4)=1$ and $\gcd(3,5)=1$.
Then all elements of the form $\{(a,0,0,b,0):a,b\in\Z_3\}$ is a subgroup of order 9, and
the group $\Z_3\oplus\Z_3$ is isomorphic to it.  We recall that isomorphisms
preserve subgroups.  So we can find a subgroup of order 9 in $\Z_{12}\oplus\Z_4\oplus\Z_{15}$.
Let $\phi$ be an isomorphism from $\Z_3\oplus\Z_4\oplus\Z_4\oplus\Z_3\oplus\Z_5$ to
$\Z_{12}\oplus\Z_4\oplus\Z_{15}$.  Then $\phi(\{(a,0,0,b,0):a,b\in\Z_3\})$ is a subgroup
of order 9 in $\Z_{12}\oplus\Z_4\oplus\Z_{15}$.  (Here we used the notation
$\phi(G)=\{\phi(g):g\in G\}$.)

\section*{Problem 42}

Find an isomorphism from $\Z_{12}$ to $\Z_4\oplus\Z_3$.

We know that such an isomorphism exists by Corollary 2 of Theorem 8.2,
since $\gcd(4,3)=1$.
So let $\phi:\Z_{12}\to\Z_4\oplus\Z_3$ be one of the isomorphisms.
Since by the operation preserving property of $\phi$ we
have $\phi(x)=(\phi(1))^x$, the function $\phi$ is defined in terms
of what it maps 1 to.
Since $\Z_{12}$ is cyclic, so must $\Z_4\oplus\Z_3$ be.  So 1, being
a generator of $\Z_{12}$, must map to a generator of $\Z_4\oplus\Z_3$.
Let $\phi(1)=(1,1)$ and so define $\phi(x)=(1,1)^x$.

Let $a,b\in\Z_{12}$.  Then
$\phi(ab)=(1,1)^{ab}=(1^{ab},1^{ab})=(1^a1^b,1^a1^b)=(1^a,1^a)(1^b,1^b)=(1,1)^a(1,1)^b=\phi(a)\phi(b)$
so that $\phi$ is operation preserving.
Suppose $\phi(a)=\phi(b)$.  Then $(1,1)^a=(1,1)^b\implies a\equiv b\pmod{12}$
by Theorem 4.1.  But since, WLOG, $0\leq a\leq b<12$, we must have $a=b$,
showing that $\phi$ is one-to-one.  Then since $\phi$ is one-to-one from
a finite set to one with the same cardinality, it is also onto.  We can
now say that $\phi$ is an isomorphism of the type we were looking for.

\section*{Problem 44}

Suppose that $\phi$ is an isomorphism from $\Z_3\oplus\Z_5$ to $\Z_{15}$ and
$\phi(2,3)=2$.  Find the element in $\Z_3\oplus\Z_5$ that maps to 1.

Since $\Z_3\oplus\Z_5\approx\Z_{15}$, it is cyclic, and it has a generator.
Let $g$ be the generator in $\Z_3\oplus\Z_5$ such that $\phi(g)=1$.
Then $2=1^2=(\phi(g))^2 = \phi(g^2) = \phi((2,3))\implies g^2=(2,3)$.
Then $g=(1,4)$ is a solution.  This must be the only solution since
there can be only one generator that maps to 1 in $\Z_{15}$.

\section*{Problem 46}

Let $G=\{ax^2+bx+c:a,b,c\in\Z_3\}$.  Add elements to $G$ as you would
polynomials with integer coeficients, except use modulo 3 addition.
Prove that $G$ is isomorphic to $\Z_3\oplus\Z_3\oplus\Z_3$.  Generalize.

Let us show that $\phi(ax^2+bx+c)=(a,b,c)$ is an isomorphism from
$G$ to $\Z_3\oplus\Z_3\oplus\Z_3$.  Notice that the function is well
defined, one-to-one, and onto, since a polynomial is uniquely
determined by its coeficients.  What remains to be shown is that the
mapping is operation preserving...
\begin{align*}
\phi((a_1x^2+b_1x+c_1)+(a_2x^2+b_2x+c_2))
 &= \phi((a_1+a_2)x^2+(b_1+b_2)x+(c_1+c_2)) \\
 &= (a_1+a_2,b_1+b_2,c_1+c_2) \\
 &= (a_1,b_1,c_1)(a_2,b_2,c_2) \\
 &= \phi(a_1x^2+b_1x+c_1)\phi(a_2x^2+b_2x+c_2)
\end{align*}

As a generalization of the problem...
\begin{equation*}
\left\{\sum_{k=0}^{n-1}a_kx^k:a_k\in\Z_m\right\}\approx\bigoplus_{k=1}^n\Z_m
\end{equation*}
Notice that $n$ need not be equal to $m$, and the operation for elements of the
set on the LHS is the addition of polynomials with integer coeficients, except using
modulo $m$ addition.

\section*{Problem 50}

Express $U(165)$ as an external direct product of cyclic groups of the form $\Z_n$.
\begin{equation*}
U(165)=U(3\cdot 5\cdot 11)\approx U(3)\oplus U(5)\oplus U(11)\approx\Z_2\oplus\Z_4\oplus\Z_{10}
\end{equation*}

\section*{Problem 52}

Without doing any calculations in $\aut(\Z_{20})$, determine how many elements
of $\aut(\Z_{20})$ have order 4.  How many have order 2?

By Theorem 6.5, $\aut(\Z_{20})\approx U(20)$.
Then by Theorem 8.3, $U(20)\approx U(4)\oplus U(5)$, since $\gcd(4,5)=1$.
Then $U(4)\oplus U(5)\approx\Z_2\oplus\Z_4$ by Gauss.
So the number of elements of order 4 in $\Z_2\oplus\Z_4$ is
the number of such elements in $\aut(\Z_{20})$.
By Theorem 8.1, we want to count all the ways in which we can
have $4=\lcm(\{1,2\},\{1,2,4\})$.  This reduces to the cases
$4=\lcm(\{1,2\},4)$.  This can happen $\Z_2^1\Z_4^4+\Z_2^2\Z_4^4=
\phi(1)\phi(4)+\phi(2)\phi(4)=4$ different ways.

Counting the elements of order 2, we want $2=\lcm(\{1,2\},\{1,2\})$.
This can happen
$\Z_2^1\Z_4^2+\Z_2^2\Z_4^1+\Z_2^2\Z_4^2=\phi(1)\phi(2)+\phi(2)\phi(1)+\phi(2)\phi(2)=3$
different ways.

\section*{Problem 58}

Use the results presented in this chapter to prove that $U(144)$ is
isomorphic to $U(140)$.
\begin{equation*}
U(144)\approx U(2^4)\oplus U(3^2)\approx
\Z_2\oplus\Z_4\oplus\Z_6\approx
U(2^2)\oplus U(5)\oplus U(7)\approx U(140)
\end{equation*}
Thanks, Gauss.

\section*{Problem 64}

Show that no $U$-group has order 14.

Suppose that there exists a natural number $n$ such that $|U(n)|=14$.
Let $n=p_1^{a_1}p_2^{a_2}\dots p_k^{a_k}$ be the prime factorization of $n$.
Then by Theorem 8.3 we see that...
\begin{equation*}
U(n)\approx U(p_1^{a_1})\oplus U(p_2^{a_2})\oplus\dots\oplus U(p_k^{a_k})
\end{equation*}
Then by Carl Gauss, if $n$ is even, let $p_1=2$, and then we have...
\begin{equation*}
U(n)\approx(\Z_2\oplus\Z_{2^{a_1-2}})\oplus\Z_{p_2^{a_2-1}(p_2-1)}\oplus
\Z_{p_3^{a_3-1}(p_3-1)}\oplus\dots\oplus\Z_{p_k^{a_k-1}(p_k-1)}
\end{equation*}
...otherwise, if $n$ is odd, we have...
\begin{equation*}
U(n)\approx\Z_{p_1^{a_1-1}(p_1-1)}\oplus
\Z_{p_2^{a_2-1}(p_2-1)}\oplus\dots\oplus\Z_{p_k^{a_k-1}(p_k-1)}
\end{equation*}
But in either case, (since $2\cdot 2^{a_1-2}=2^{a_1-1}(2-1)$), we see that...
\begin{align*}
\phi(n)=|U(n)| &= p_1^{a_1-1}(p_1-1)p_2^{a_2-1}(p_2-1)\dots p_k^{a_k-1}(p_k-1) \\
 &= p_1^{a_1-1}p_2^{a_2-1}\dots p_k^{a_k-1}(p_1-1)(p_2-1)\dots(p_k-1) \\
 &= n\frac{(p_1-1)(p_2-1)\dots(p_k-1)}{p_1p_2\dots p_k}
\end{align*}
This then implies that...
\begin{equation*}
n = 14\frac{p_1p_2\dots p_k}{(p_1-1)(p_2-1)\dots(p_k-1)}
\end{equation*}
Here, if $p_1\neq 2$, we'll have an odd number in the numerator
of the quotient, and an even number in the denominator.  The denominator
must then divide 14.  But 14 cannot be written as the product of even
numbers less than 14.  So we're stuck.  Also, $\phi(2^r)=2^{r-1}$ and
14 is not a power of 2.
\end{document}