\documentclass[12pt]{article}

\usepackage{amsmath}
\usepackage{amssymb}
\usepackage{amsthm}
\usepackage{graphicx}
\usepackage{float}

\title{Master Elements}
\author{Spencer}

\newtheorem{theorem}{Theorem}[section]
\newtheorem{definition}{Definition}[section]
\newtheorem{corollary}{Corollary}[section]
\newtheorem{identity}{Identity}[section]
\newtheorem{lemma}{Lemma}[section]
\newtheorem{result}{Result}[section]

\begin{document}
\maketitle

Given a group $G=\langle S\rangle$ and any element $g\in G$, it is often desirable to
find a factorization of $g$ in terms of the elements in $S$.
If we were to generate all elements of $G$ and store them in terms of their
factorizations using $S$, then $G$ becomes a search space for the factorization
of any given element $g\in G$.  For all but the smallest of groups, however, this is completely impractical.

If we have a subnormal series of $G$, namely
\begin{equation*}
\{e\}=H_1\triangleleft  H_2\triangleleft\dots\triangleleft H_n=G,
\end{equation*}
then it may be practical to store the coset representatives of each
factor group $H_i/H_{i-1}$.  These become a search space at each tier
in the subgroup chain.  We let $k$ be the smallest integer
for which $g\in H_k$, then let $h_1\in H_k$ be the stored coset representative
such that $gH_{k-1}=h_1H_{k-1}$.  We then find that $h_1^{-1}g\in H_{k-1}$,
and repeat this process until we find
\begin{equation*}
h_r^{-1}\dots h_1^{-1}g\in H_1,
\end{equation*}
at which point we can conclude that
\begin{equation*}
g = h_1\dots h_r,
\end{equation*}
which can be rewritten in terms of the elements in $S$, provided brought them along at
each level in the chain.

Note that the generators for $H_i/H_{i-1}$ can be found in terms of the
generators $S_i$ for $H_i$ as
\begin{equation*}
H_i/H_{i-1} = \{s_1\dots s_jH_{i-1}|s_k\in S_i\} = \{(s_1H_{i-1})\dots(s_jH_{i-1})|s_k\in S_i\}.
\end{equation*}
Once the factor group $H_i/H_{i-1}$ is generated in terms of a sequence of
coset representatives, these can also be used as a transversal in Schreier's lemma
to find generators $S_{i-1}$ for $H_{i-1}$.

This approach to factorization has its own set of diffulties as well, however; not the least of
which being the difficulty in producing the subnormal chain.  Factorizations found this way
grow exponentially as the chain is descended.

So we turn now to what might be
a more practical method in circumstances where more about the group is known up front.

\begin{definition}
For any group $G$, let an element $m\in G$ be a \textbf{master element} if for
every $g\in G$, there exists a sequence $\{r_i\}_{i=1}^k\subset G$ such that
\begin{equation}\label{equ_master_ele}
g = \prod_{i=1}^k r_imr_i^{-1}.
\end{equation}
\end{definition}

A master element is often known for groups of interest, and the set of all
possible $r_i$ becomes the search space.  In the context of permutation groups,
each $r_i$ found may help us descend a stabilizer chain we needn't generate.
One of the problems with this, however, is that we don't know which $r_i$ to choose
if we can't find one that moves us to a stabilizer subgroup that stabilizers more
of the domain.  It doesn't matter if we stain in a chain; only that we get closer
to the identity element.

Calling a group \textbf{mastered} if it has a non-empty subset of master elements,
we give our first lemma.

\begin{lemma}[Gallian]
The cyclic groups are the only mastered Abelian groups.
\end{lemma}
\begin{proof}
Notice that equation \eqref{equ_master_ele} immediately reduces to
\begin{equation*}
g = m^k.
\end{equation*}
Clearly every element of every cyclic group is of this
form, and so every master element is a generator of the group.
This is not the case, however, for any non-cyclic Abelian group.
\end{proof}

This factorization method, therefore, cannot generally help us with Abelian groups.
This isn't too desparaging, however, since the factoring problem may be generally
harder in the non-Abelian cases anyway.

%How do we generate the search space?  How do we find a master element?
%If $G$ is a group of permutations on a set $S$, then maybe we must require that for all $s\in S$,
%$\mbox{orb}_G(s)=S$?  Doesn't sound right.
% I'm not sure I can prove anything about this idea.  Maybe it's dumb.

\end{document}