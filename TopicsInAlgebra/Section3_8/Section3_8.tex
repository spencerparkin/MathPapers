\documentclass[12pt]{article}

\usepackage{amsmath}
\usepackage{amssymb}
\usepackage{amsthm}

\title{Section 3.8 Exercises\\Herstein's Topics In Algebra}
\author{Spencer T. Parkin}

\newtheorem{theorem}{Theorem}[section]
\newtheorem{definition}{Definition}[section]
\newtheorem{corollary}{Corollary}[section]
\newtheorem{identity}{Identity}[section]
\newtheorem{lemma}{Lemma}[section]
\newtheorem{result}{Result}[section]

%\newcommand{\gcd}{\mbox{gcd}}
\newcommand{\lcm}{\mbox{lcm}}
\newcommand{\Z}{\mathbb{Z}}
\newcommand{\C}{\mathbb{C}}
\newcommand{\R}{\mathbb{R}}
\newcommand{\cl}{\mbox{Cl}}
\newcommand{\aut}{\mbox{Aut}}

\begin{document}
\maketitle

\section*{Thoughts}

I had some trouble with a part of the proof of Theorem 3.8.1.
For integers $a$ and $n$, the division algorithm can give us integers
$t$ and $r$ such that $a=tn+r$ with $|r|<|n|$, but Herstein asserts
that such can be found where $|r|\leq |n|/2$.  Well, what about $a=40$
and $n=7$?  Here, $t=5$ and $r=5$, and clearly $5\leq 7/2$ does not hold.

\section*{Problem 1}

Find all the units in $J[i]$.

In the field of complex numbers $\C$, multiplicative inverses are unique.
That is, letting $a+bi\in\C$, we have
\begin{equation*}
(a+bi)^{-1}=\frac{a-bi}{a^2+b^2}.
\end{equation*}
Now since $\C$ contains $J[i]$, this too must hold true.  Thus, if $x=a+bi\in J[i]$ is a unit,
we must have $d(x)|\Re(x)$ and $d(x)|\Im(x)$.  But this is only possible for $x=1,-1,i,-i$.

\section*{Problem 2}

If $a+bi$ is not a unit of $J[i]$ prove that $a^2+b^2>1$.

Perhaps Herstein forgot to exclude $0+0i$.

If $a$ and $b$ are non-zero, then by Problem 1, they're not units, and $d(a+bi)>1$.
If $a=0$, then by Problem 1, $|b|>1$.  Similarly, if $b=0$, then $|a|>1$.

\section*{Problem 3}

Find the greatest common divisor in $J[i]$ of $3+4i$ and $4-3i$, then $11+7i$ and $18-i$.

Let's consider for a moment the general problem of finding $\gcd(a+bi,c+di)$.
Since $J[i]$ is a Euclidean ring with unit element, we know by Theorem 3.7.1 that it is a principle ideal ring.
Then by Lemma 3.7.1, we know that any greatest common divisor of $a+bi$ and $c+di$ is a generator
of the ideal $A$ given by
\begin{equation*}
A=\{u(a+bi)+v(c+di)|u,v\in J[i]\}.
\end{equation*}
That generator being $x+yi$, we must have
\begin{equation*}
A=\{(x+yi)(s+ti)|s,t\in\Z\}.
\end{equation*}
Continue on...

\section*{Problem 8}

Determine all prime elements in $J[i]$.

Maybe show that for $a,b\in J[i]$, $\langle a\rangle$ and $\langle b\rangle$ are properly contained
in the ideal $I$ given by
\begin{equation*}
I = \{ua+vb|u,v\in J[i]\},
\end{equation*}
if coprime.  Now if $a$ is prime in $J[i]$, then we must have $I=J[i]$.  Where am I going with this?
I'm trying to find another way to characterize a prime.  I want to say that $a\in J[i]$ is prime
if and only if blank, where blank is something other than the definition.  By definition, $a\in J[i]$
is prime if whenever we write $a=uv$ for $u,v\in J[i]$, we have at least one of $u$ and $v$ a unit of $j[i]$.
It's hard to come somewhere straight from the definition.

\end{document}