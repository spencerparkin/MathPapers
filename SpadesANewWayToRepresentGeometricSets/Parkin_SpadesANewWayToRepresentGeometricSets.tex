\documentclass{birkjour}

\usepackage{float}
\usepackage{hyperref}

\newtheorem{thm}{Theorem}[section]
\newtheorem{cor}[thm]{Corollary}
\newtheorem{lem}[thm]{Lemma}
\newtheorem{prop}[thm]{Proposition}
\theoremstyle{definition}
\newtheorem{defn}[thm]{Definition}
\theoremstyle{remark}
\newtheorem{rem}[thm]{Remark}
\newtheorem*{ex}{Example}
\numberwithin{equation}{section}

\newcommand{\F}{\mathbb{F}}
\newcommand{\R}{\mathbb{R}}
\newcommand{\C}{\mathbb{C}}
\newcommand{\B}{\mathbb{B}}
\newcommand{\G}{\mathbb{G}}
\newcommand{\V}{\mathbb{V}}
\newcommand{\gd}{\dot{g}}
\newcommand{\gh}{\hat{g}}
\newcommand{\Gd}{\dot{G}}
\newcommand{\Gh}{\hat{G}}
\newcommand{\nvai}{\infty}
\newcommand{\nvao}{o}
\newcommand{\grade}{\mbox{grade}}
\newcommand{\rank}{\mbox{rank}}

%\received{}\accepted{}

\begin{document}

\title{Spades\\
\large A New Way To Represent Geometric Sets}

\author{Spencer T. Parkin}
\email{spencerparkin@outlook.com}

%\subjclass{Primary 14J70; Secondary 14J29}

%\dedicatory{To my dear wife Melinda.}

\begin{abstract}
Abstract...
\end{abstract}

\keywords{Key words...}

\maketitle

\section{Introduction And Motivation}

Traditionally, \emph{blades} are used, in places such as the conformal model, to represent geometric sets in geometric algebra.\footnote{A formal
treatment of geometric sets in an abstract setting is given in \cite{Parkin15}.  Geometric sets are a generalization of algebraic sets.}  Doing so,
the meet and join operations become the principle
means by which geometries are combined or intersected into new geometries.  In this paper we show that \emph{spades} may also be used
to represent geometric sets; and in so doing, the geometric product becomes the principle means by which geometries are combined or intersected
to make new geometries.

The term ``spade'' requires some explanation. See Table~\ref{tbl_terms} below.

\begin{table}[H]\label{tbl_terms}\caption{A few terms used in GA}
\begin{tabular}{p{2cm}p{9cm}}
Term & Definition \\
\hline
Blade & An outer product of zero or more linearly-independent vectors. \\
Versor & A geometric product of zero or more \emph{invertible} vectors, not necessarily forming a linearly-independent set. \\
Spade & A geometric product of zero or more vectors, not necessarily forming a linearly-independent set.\\
Null Versor & A geometric product of one or more vectors where at least one of them is null.\\
\end{tabular}
\end{table}

From these definitions it is clear that every versor is a spade, but not every spade is a versor.  The definition of a versor,
which can be found in \cite[p. 90]{Perwass09}, is well established and engrained in the literature just as it is written in Table~\ref{tbl_terms}.  Not requiring
that each vector in the factorization of a spade be invertible, this justifies the new term.

\section{Geometric Sets}

We must begin with a review of geometric sets.  Given an $n$-dimensional space $\F^n$, we let $p:\F^n\to\V$ be a vector-valued
function mapping points in $\F^n$ to vectors in a vector space $\V$ generating our geometric algebra $\G$.  With this in hand,
we are ready for the following definition.
\begin{defn}[Geometric Set]\label{def_geo_set}
A subset $S$ of $\F^n$ is a \emph{geometric set} if and only if there exists a set of $r$ vectors $\{v_i\}_{i=1}^r\subset\V$, such that
\begin{equation*}
S = \{x\in\F^n|\mbox{$p(x)\cdot v_i$ for all $i\in[1,r]$}\} = \bigcap_{i=1}^r \{x\in\F^n|p(x)\cdot v_i=0\}.
\end{equation*}
\end{defn}
If each expression $p(x)\cdot v_i$ is a polynomial in the components of $x$, then it is easy to see that, in this case, geometric sets are algebraic.
By the Hilbert Basis Theorem (see \cite[p. 204]{Garrity13}), $r$ needs not be infinite.  This is good news for us, as we'll see, since we'll eventually want to show
that every algebraic set can be represented by a blade of grade $r$.

\begin{lem}\label{lem_geo_set_lin_indep}
If $\{E_i\}_{i=1}^r$ is any linearly independent set of elements taken from $\G$, then the set of all
solutions in $\F^n$ to the equation
\begin{equation}\label{equ_geo_set}
0 = \sum_{i=1}^r (p(x)\cdot v_i)E_i
\end{equation}
is a geometric set.
\end{lem}
\begin{proof}
Being a linearly independent set of elements, the only linear combination of these elements that vanishes is the trivial linear combination.
It follows that for each integer $i\in[1,r]$, we must have $p(x)\cdot v_i=0$.
\end{proof}

\begin{lem}\label{lem_geo_set_lin_dep}
If $\{E_i\}_{i=1}^r$ is any sequence of elements taken from $\G$, then the set of all solutions in $\F^n$ to equation \eqref{equ_geo_set}
is a geometric set.
\end{lem}
\begin{proof}
If $\{E_i\}_{i=1}^r$ is a linearly independent set, then we're done by Lemma~\ref{lem_geo_set_lin_indep}.
Supposing to the contrary, and without loss of generality, we can let $s$ be an integer with $1\leq s<r$ such that
$\{E_i\}_{i=1}^s$ is a linearly independent set, and
\begin{equation*}
\mbox{span}\{E_i\}_{i=1}^r = \mbox{span}\{E_i\}_{i=1}^s.
\end{equation*}
Now for each integer $i\in[s+1,r]$, write $E_i$ as a linear combination of the elements in $\{E_i\}_{i=1}^s$ as
\begin{equation*}
E_i = \sum_{j=1}^s\alpha_{i,j} E_j.
\end{equation*}
Having done so, we see now that equation \eqref{lem_geo_set_lin_dep} becomes
\begin{align*}
0 &= \sum_{i=1}^r(p(x)\cdot v_i)E_i \\
 &= \sum_{i=1}^s(p(x)\cdot v_i)E_i + \sum_{i=s+1}^r(p(x)\cdot v_i)\sum_{j=1}^s\alpha_{i,j}E_j \\
 &= \sum_{i=1}^s\left[p(x)\cdot v_i-\sum_{j=s+1}^r\alpha_{j,i}(p(x)\cdot v_j)\right]E_i \\
 &= \sum_{i=1}^s\left[p(x)\cdot\left(v_i-\sum_{j=s+1}^r\alpha_{j,i}v_i\right)\right]E_i.
\end{align*}
We see now that the set of all solutions to equation \eqref{lem_geo_set_lin_dep} is given by
\begin{equation*}
\bigcap_{i=1}^s\left\{x\in\F^n\left|p(x)\cdot\left(v_i-\sum_{j=s+1}^r\alpha_{j,i}v_i\right)=0\right.\right\},
\end{equation*}
which is clearly a geometric set by Definition~\ref{def_geo_set}.
\end{proof}

\section{Perliminary Material}

Before we can show how blades and spades can represent geometric sets, we need to lay some ground work
with the following definitions, lemmas, and identities.

Though already given in Table~\ref{tbl_terms}, the term spade deserves its own formal definition as follows.
\begin{defn}[Spade]
An element $M_r\in\G$ is called a \emph{spade} if and only if there exists a set of $r$ vectors $\{m_i\}_{i=1}^r$
such that it may be written as
\begin{equation*}
M_r = \prod_{i=1}^r m_i.
\end{equation*}
\end{defn}
It is easy to show that spades, like blades, to not have unique factorizations.  Unlike blades, however, the
size of a spade's factorization can very.  This leads us to the following definition.
\begin{defn}[Spade Rank]
Given any spade $M_r\in\G$, the rank of the spade $M_r$, denoted $\rank(M_r)$ is the
smallest integer $s\leq r$ such that $M_r$ may be rewritten as a geometric product of $s$ vectors.\footnote{This smallest integer $s$ exists by the well-ordering principle.}
\end{defn}
It is clear that if $\{m_i\}_{i=1}^r$ is a linearly independent set of vectors, then $\rank(M_r)=r$.  The converse of this statement, however,
is not so easily proven (or disproved) and of such considerable importance to the theory of spades in geometric algebra, that, if proven, would
be elevated to the level of theorem.  For now, however, the following lemma is offered.
\begin{lem}\label{lem_spade_no_dup_in_factorization}
For any given invertible spade $M_r\in\G$, if there exist integers $1\leq i<j\leq r$ such that $m_i=m_j$, and $m_i$ is invertible, then $\rank(M_r)<r$.
\end{lem}
\begin{proof}
This is trivial in the case that $j=i+1$.  In the case that $j=i+2$, simply notice that
\begin{equation*}
m_im_{i+1}m_j = m_im_{i+1}m_i = 2(m_i\cdot m_{i+1})m_i-m_i^2 m_{i+1}.
\end{equation*}
In the case that $j>i+2$, we see that
\begin{equation*}
m_i\left(\prod_{k=i+1}^{j-1}m_k\right)m_j = m_i^2\prod_{k=i+1}^{j-1}m_im_km_i^{-1}.
\end{equation*}
\end{proof}

For completeness, we now give a formal definition of a blade.
\begin{defn}[Blade]
An element $B_r\in\G$ is called an $r$-\emph{blade} if and only if there exists a linearly independent set of $r$
vectors $\{b_i\}_{i=1}^r$ such that
\begin{equation*}
B_r = \bigwedge_{i=1}^r b_i.
\end{equation*}
\end{defn}

\begin{lem}\label{lem_lin_indep_subblades}
Letting $B_r^{(i)}$ denote the $(r-1)$-blade
\begin{equation*}
B_r^{(i)} = \bigwedge_{\substack{j=1\\j\neq i}}^r b_i,
\end{equation*}
the set of $r$ blades $\{B_r^{(i)}\}_{i=1}^r$ is linearly independent.
\end{lem}
\begin{proof}
Supposing to the contrary, and without loss of generality, let
\begin{equation*}
B_{r-1} = B_r^{(r)} = \sum_{i=1}^{r-1}\alpha_i B_r^{(i)} = \left(\sum_{i=1}^{r-1}\alpha_i B_{r-1}^{(i)}\right)\wedge b_r.
\end{equation*}
Now notice that
\begin{equation*}
0\neq B_r = B_{r-1}\wedge b_r = B_r^{(r)}\wedge b_r = \left(\sum_{i=1}^{r-1}\alpha_i B_r^{(i)}\right)\wedge b_r = 0,
\end{equation*}
which is clearly a contradiction.
\end{proof}

We will need a result similar to Lemma~\ref{lem_lin_indep_subblades} as concerning spades.  It is as follows.
\begin{lem}\label{lem_lin_indep_subspades}
Letting $M_r^{(i)}$ denote the spade
\begin{equation*}
M_r^{(i)} = \prod_{\substack{j=1\\j\neq i}}^r m_i,
\end{equation*}
if $0\neq\bigwedge_{i=1}^r m_i$, then the set $\{M_r^{(i)}\}_{i=1}^r$ is a linearly independent set.
\end{lem}
\begin{proof}
We first consider, for all $j\in[0,r]$, the $j$ equations given by
\begin{equation*}
0 = \sum_{i=1}^r\alpha_i\langle M_r^{(i)}\rangle_j.
\end{equation*}
Letting $A_k$ denote the set of all solutions in each $\alpha_i$ to equation $j$, it is clear that
the set of all solutions $A$ in each $\alpha_i$ to the equation
\begin{equation*}
0 = \sum_{i=1}^r\alpha_i M_r^{(i)}
\end{equation*}
is given by
\begin{equation*}
A = \bigcap_{i=1}^k A_k.
\end{equation*}
Thus, it suffices to show that the set $\{\langle M_r^{(i)}\rangle_{r-1}\}_{i=1}^r$ is a linearly independent set.
Now since $0\neq\bigwedge_{i=1}^r m_i$, it is clear that
\begin{equation*}
\langle M_r^{(i)}\rangle_{r-1} = \bigwedge_{\substack{j=1\\j\neq i}}^r m_i.
\end{equation*}
Seeing this, the linear independence of the set $\{\langle M_r^{(i)}\rangle_{r-1}\}_{i=1}^r$ follows immediately
from Lemma~\ref{lem_lin_indep_subblades}.
\end{proof}

\section{Blades As Representatives Of Geometric Sets}

\section{Spades As Representatives Of Geometric Sets}

\begin{thebibliography}{9}

\bibitem{Parkin15}
S. Parkin,
\emph{An Introduction To Geometric Sets}.
Advances in Applied Clifford Algebras, Volume 25, Issue Unknown, pp. 639-655, 2015.

\bibitem{Perwass09}
C. Perwass,
\emph{Geometric Algebra with Applications in Engineering}
Springer-Verlag Berlin Heidelberg, 2009.

\bibitem{Garrity13}
T. Garrity, et. al.,
\emph{Algebraic Geometry, A Problem Solving Approach}
American Mathematical Society, Institute for Advanced Study

\bibitem{Hestenes99}
D. Hestenes,
\emph{New Foundations for Classical Mechanics}
Kluwer Academic Publishers, 1999.

\bibitem{Doran03}
C. Doran, et. al.,
\emph{Geometric Algebra for Physicists}
Cambridge University Press, 2003.

\end{thebibliography}

\end{document}