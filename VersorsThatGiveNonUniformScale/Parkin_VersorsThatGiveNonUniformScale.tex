\documentclass{birkjour}

\usepackage{tikz}
\usepackage{graphicx}
\usepackage{hyperref}

\newtheorem{thm}{Theorem}[section]
\newtheorem{cor}[thm]{Corollary}
\newtheorem{lem}[thm]{Lemma}
\newtheorem{prop}[thm]{Proposition}
\theoremstyle{definition}
\newtheorem{defn}[thm]{Definition}
\theoremstyle{remark}
\newtheorem{rem}[thm]{Remark}
\newtheorem*{ex}{Example}
\numberwithin{equation}{section}

\newcommand{\R}{\mathbb{R}}
\newcommand{\B}{\mathbb{B}}
\newcommand{\G}{\mathbb{G}}
\newcommand{\V}{\mathbb{V}}
\newcommand{\gd}{\dot{g}}
\newcommand{\gh}{\hat{g}}
\newcommand{\Gd}{\dot{G}}
\newcommand{\Gh}{\hat{G}}
\newcommand{\nvai}{\infty}
\newcommand{\nvao}{o}
\newcommand{\grade}{\mbox{grade}}

\begin{document}

\title{Versors That Give Non-Uniform Scale}

\author{Spencer T. Parkin}
\address{102 W. 500 S., \\
Salt Lake City, UT  84101} \email{spencerparkin@outlook.com}

%\subjclass{Primary 14J70; Secondary 14J29}

\dedicatory{To my dear wife Melinda.}

\begin{abstract}
It is shown that for every non-uniform scale transformation, a versor
exists in a geometric algebra that can perform this transformation
on any algebraic surface.  Some implications and generalizations of this find are discussed,
as well as the possibility of acquiring all affine transformations as versors applicable
to any algebraic surface.
\end{abstract}

\keywords{Algebraic Surface, Conformal Model, Non-Uniform Scale, Geometric Algebra}

\maketitle

\section{Motivation And Review}

The question of existence of non-uniform scale versors is one of the outstanding problems of geometric algebra,
and one that stands in the way of geometric algebra competing against existing and well-proven transformation models.
As noted in the beginning of \cite{Goldman12}, $4\times 4$ matrices have long-time been a standard
in computer graphics for representing affine and projective transformations, but an
equivalent yet hopefully more capable and universally compatible model for such transformations in a more modern setting has yet to emerge
as a considerable replacement.  This paper does not purport to provide such a setting,
but it does offer a potential solution to the non-uniform scale problem.
An upcoming paper by an author of \cite{Dorst07} may provide an even better solution.

Being dependent on \cite{Parkin13}, a quick review is in order before we begin.
In \cite{Parkin13} and this paper we work in a geometric algebra that is simply
the external direct product of $m$ copies of a Minkowski algebra $\G$, which
may be denoted by $\bigoplus_{i=1}^m \G_i$.  We will be interested in the $m$-vectors
taken from $\bigwedge_{i=1}^m \G_i$ as being representative of algebraic surfaces
of up to degree $m$ by equation \eqref{equ_surface_rep} below.

\section{The Result}

The result\footnote{To understand this paper, the reader must be familiar
with \cite{Parkin13}.} of this paper is simply a corollary to that of \cite{Parkin13}, but to see how,
we must first constrain the way in which we represent $n$-dimensional algebraic surfaces of up to degree $m$
in the Mother Minkowski algebra of order $m$.\footnote{Recall that such representations are not unique, and so
we have the flexibility to choose our representations carefully.} What we do is let $n\leq m$, and
reserve certain subalgebras of our mother algebra for use in specific dimensions.
To see what is meant by this, let $f:\R^n\to\R$ be a polynomial in whose zero set
we are interested.  Now define, for each integer $k\in[1,n]$, the polynomial $f_k:\R\to\R$ as
\begin{equation*}
f_k(\lambda)=f(e_1+e_2+\dots+\lambda e_k+\dots +e_{n-1}+e_n).
\end{equation*}
Having done this, we will represent the
surface of $f$ in the Mother Minkowski algebra of order
\begin{equation*}
m=\sum_{k=1}^n\deg f_k.
\end{equation*}
Now if $\G$ denotes our mother algebra and it is generated by $m$ subalgebras $\G_i$,
each generated by the vector space $\V_i$, then we reserve $\deg f_k$
of these subalgebras for use in dimension $k$ of our $n$ dimensions.
(We will let $\G^k$, where $k$ is an integer in $[1,n]$, denote the smallest
subalgebra of $\G$ containing all subalgebras $\G_i$ reserved for dimension $k$,
and let $[\G^k]$ denote the set of indices over which $\G_i\subseteq\G^k$.)

An example may be warrented at this point.  Let $n=3$ and consider the
polynomial given by
\begin{equation}\label{equ_example_poly}
f(x) = 3x_1^2x_2x_3^4 + 4x_1x_2^3 - 7x_3^2,
\end{equation}
where $x_k$ is notation for $x_k=x\cdot e_k$.  We will represent the
surface that is the zero set of this polynomial using an $m$-vector
in a Mother Minkowski algebra of order $m=2+3+4=9$.  The first
2 subalgebras are reserved for dimension 1, the next 3 for dimension
2, and the last 4 for dimension 3.  The $m$-vector $B$
representing this surface is then given by
\begin{align}
B &= 3e_{12,1}\wedge e_{3,2}\wedge \nvai_{45} \wedge e_{6789,3}\nonumber \\
 &- 4e_{1,1}\wedge\nvai_2\wedge e_{345,2}\wedge\nvai_{6789}\nonumber \\
 &+7\nvai_{12345}\wedge e_{67,3}\wedge\nvai_{89}.\label{equ_B_example}
% Check signs.
\end{align}
Here, notation is a challenge.  The vector $e_{i,j}$ denotes the $j^{th}$ euclidean basis vector
in the $i^{th}$ subalgebra.  We then define
\begin{equation*}
e_{i_1i_2\dots i_r,j} = e_{i_1,j}\wedge e_{i_2,j}\wedge\dots\wedge e_{i_r,j}.
\end{equation*}
The notation for $\nvai$ is similar.

We can now say that the zero set of $f$ in equation \eqref{equ_example_poly}
is given by the set of all solutions to the equation
\begin{equation}\label{equ_surface_rep}
\bigwedge_{k=1}^m p_k(x)\cdot B = 0.
\end{equation}
Recall that $p_k(x)=\nvao_k+x_k+\frac{1}{2}x^2\nvai_k$.
Of course, we could have represented $f$ in a mother algebra of order $\deg f=7$,
but it will soon become clear why we needed our algebra $\G$ to be of order $m=9$.

Before moving on, notice that $[\G^1]=\{1,2\}$, $[\G^2]=\{3,4,5\}$ and
$[\G^3]=\{6,7,8,9\}$.

Returning from the example, suppose now we have an $m$-vector $B$ representing
any polynomial $f:\R^n\to\R$ under the constraint thus illustrated.  Seeing that
the zero set of $f$ is the set of solutions to equation \eqref{equ_surface_rep},
we make the simple observation that if $D$ is a versor taken from a subalgebra $\G^k$,
and further, $D$ is the product of the same origin-centered dilation versor $D_i$ found in each subalgebra
$\G_i$ contained in $\G^k$, (see \cite{Dorst07} for an explanation of dilation versors),
which is to say that $D=\prod_{i\in [\G^k]} D_i$,
then the non-uniform scale of $f$ in the dimension of $k$ by the scale of each
$D_i$ is given by the set of solutions to the equation
\begin{align}
  & \bigwedge_{i\not\in [\G^k]} p_i(x)\wedge\bigwedge_{i\in[\G^k]} D_i^{-1}p_i(x)D_i\cdot B \nonumber \\
 =\; & \bigwedge_{i\not\in [\G^k]} p_i(x)\wedge D^{-1}\left(\bigwedge_{i\in [\G^k]} p_i(x)\right)D\cdot B = 0.\label{equ_non_uniform_scale_construction}
\end{align}
Now realize that for all $i\not\in[\G^k]$, $D$ leaves $p_i(x)$ invariant.  That is,
\begin{equation*}
D^{-1}p_i(x)D=p_i(x).
\end{equation*}
It now follows by equations (3.2) through (3.5) of \cite{Parkin13} that equation \eqref{equ_non_uniform_scale_construction}
may be rewritten as
\begin{equation*}
\bigwedge_{k=1}^m p_k(x)\cdot DBD^{-1} = 0,
\end{equation*}
showing that $D$, when applied to $B$, performs a non-uniform scaling of the surface of $f$.

Putting this result into practice, let us apply a non-uniform scale transformation
to the polynomial in equation \eqref{equ_example_poly}.  Suppose we wish
to scale the surface represented by this polynomial by a factor of 2 in the $e_2$ dimension.
Letting $D_k=(\nvao_k-\nvai_k)\left(\nvao_k-\frac{1}{2}\nvai_k\right)$, the non-uniform
scale versor $D$ we
want is $D=D_3D_4D_5$.  Then, using $B$ in equation \eqref{equ_B_example}, we find that
\begin{align}
\frac{1}{2^3}DBD^{-1} &= \frac{3}{2}e_{12,1}\wedge e_{3,2}\wedge \nvai_{45}\wedge e_{6789,3}\nonumber \\
 &-\frac{4}{2^3}e_{1,1}\wedge\nvai_2\wedge e_{345,2}\wedge\nvai_{6789}\nonumber \\
 &+7\nvai_{12345}\wedge e_{67,3}\wedge \nvai_{89},\label{equ_check_result}
\end{align}
which is just what we would hope to get when checking this against the
polynomial $f\left(x_1e_1+\frac{1}{2}x_2e_2+x_3e_3\right)$.\footnote{The present author
used symbolic computation software to make the calculation in equation \eqref{equ_check_result}.
This software can be found at \url{https://github.com/spencerparkin/GAVisTool}, though it
is not recommended for general use.}  Notice that the $1/2^3$ factor on the left-hand
side homogenizes $DBD^{-1}$.

\section{Discussion}

Though we have now shown that versors performing non-unform scaling
exist, seeing that their application requires a great deal of cumbersome convention and notation,
a question of their practicality immediately arises.  It's certainly not practical on paper, but
perhaps such versors may find applications on the computer.

In any case, it might now be possible to show that any affine transformation
has an associated versor in our mother algebra that performs
this transformation on an algebraic surface.  This would be interesting,
because the set of all algebraic surfaces of a certain degree are classified by defining an
equivalence relation on this set which states that two surfaces, (in our case, $m$-vectors),
are equivalent if and only if there exists an affine transformation, (in our case, an inner
automorphism of the versor group of our mother algebra), that takes one of these
surfaces to the other.  The versor that would take any algebraic surface to the
principle representative of its equivalence class would represent an important transformation.\footnote{The
principle representatives are the origin-centered, axis-aligned surfaces having unit characteristics where possible
without loss of generality.}
It is not at all clear, however,
whether geometric algebra is the right tool for studying such equivalence classes.
As far as the mother Minkowski algebra is concerned, the only transformation we yet lack
is that of shear.

In an initial search for shear, an immediate and possibly helpful observation we can make about the result of this paper is that it
easily generalizes to the idea of a non-uniform ``X'', where ``X'' may be replaced here
by any one of the conformal transformations.  Since all such transformations may be
decomposed as one or more planar reflections and spherical inversions, the two fundamental
transformations to consider here are non-uniform reflections and non-uniform inversions.
We first note that, without loss of generality, we need only consider reflections and inversions
about the origin, since any problem can be translated into and out of this situation.
Secondly, we can rule out non-uniform inversions right away, since they can't be any more helpful
to us than non-uniform dilations.  This leaves non-uniform reflections.

A quick analysis of the 2-dimensional cases shows that a point undergoing this operation
in the horizontal dimension with a plane having a unit-normal determined by an angle $\theta$ undergoes the same
transformation illustrated by the following matrix equation.
\begin{equation*}
\left[\begin{array}{cc} -\cos 2\theta & -\sin 2\theta \\ 0 & 1 \end{array}\right]
\left[\begin{array}{c} x \\ y \end{array}\right] =
\left[\begin{array}{c} -(\cos 2\theta)x - (\sin 2\theta)y \\ y \end{array}\right]
\end{equation*}
This bears some semblance of shear but clearly misses the mark.

\section{Comparison And Possible Reconciliation}

A review of an early draft of this paper rightly pointed out the work of \cite{DoranHestenes93} as providing
all affine transformations for points in affine spaces.
In writing \cite{Parkin13}, however, the author, in his weakness, was only able to take away from \cite{DoranHestenes93}
the idea of ``gluing'' $m$ geometric algebras together to form a mother algebra
with added structure making the representation of surfaces by $m$-vectors in
that algebra a natural consequence.\footnote{Through personal communication
with David Hestenes, (an author of \cite{DoranHestenes93}), I suggested the representation scheme of \cite{Parkin13}, after which he pointed me
to the paper \cite{DoranHestenes93} as having already explored similar propositions.}  Once done, one can then leverage the known transformations
available in the geometric algebra that was replicated to form the mother algebra.
In the case of \cite{Parkin13}, it was the algebra of the conformal model; so, naturally, all surfaces
became subject to the conformal transformations.

The soon to be published paper \cite{GoldmanNotYet}, however, makes many of the ideas of \cite{DoranHestenes93} more
accessible to the average reader.  It even features a method of representing quadric
surfaces in a way similar to that found in \cite{Parkin13}, with the only drawback being
that it does not generally preserve the sandwich operation of versors on quadrics.  To solve this problem,
extend the representation scheme to the set of all algebraic surfaces of degree $m$, and gain
all of the affine transformations, the following proposition is made, even if at the cost of the
bloat created by \cite{Parkin13} and exacerbated by the present paper.\footnote{Here, ``bloat'' refers
to the excessive use of dimension.}

The paper \cite{GoldmanNotYet} uses an algebra denoted by $W\oplus W^*$ and shows that versors
in this algebra give the affine transformations on points in the vector space generating $W$.
Leveraging this algebra as was done with the conformal model in \cite{Parkin13}, we can represent
surfaces of degree $m$ as $m$-vectors in the subalgebra $\bigoplus_{i=1}^m W_i$ of
the mother algebra
\begin{equation*}
\bigoplus_{i=1}^m W_i\oplus W_i^* = \bigoplus_{i=1}^m W_i\oplus\bigoplus_{i=1}^m W_i^*.
\end{equation*}
Specifically, we would be interested in the $m$-vectors taken from $\bigwedge_{i=1}^m W_i$.

Seeing that any further
details of this idea would warrant a paper of its own, let us leave it at that.  Seeing that it would not at all be hard for
anyone familiar with \cite{Parkin13} and \cite{GoldmanNotYet} to fill in such details in the absence of such a paper,
it may not at all be warranted.

\section{Closing Remarks}

Lastly, the apparent ``coordinitis,'' (see \cite{Hestenes92}), suffered by this paper and \cite{Parkin13} should
be addressed.  Polynomials are inherently non-coordinate-free; and so, naturally, a study of their zero sets
in the framework of geometric algebra is not always free of coordinates.  Despite this,
if one begins with a coordinate-free formulation of a surface, $\bigwedge p_k(x)$ can often be
factored out of it in terms of the inner product in a coordinate-free manner.  For example, many quadrics
characterized by a scalar $\lambda$, center $c$, radius $r$ and unit-length vector $v$, are solutions in $x$ to the
coordinate-free equation
\begin{equation*}
(x-c)^2+\lambda((x-c)\cdot v)^2 - r^2 = 0,
\end{equation*}
which, when expanded, becomes
\begin{equation*}
x^2+\lambda (x\cdot v)^2 - 2x\cdot (c+\lambda (c\cdot v)v)+c^2+\lambda(c\cdot v)^2 - r^2 = 0,
\end{equation*}
out of which may be factored $p_1(x)\wedge p_2(x)$ to get
\begin{equation*}
p_1p_2\cdot(-\Omega-\lambda v_1v_2-2(c+\lambda(c\cdot v)v)_1\nvai_2-(c^2+\lambda (c\cdot v)^2-r^2)\nvai_{12})=0,
\end{equation*}
where the constant $\Omega$ is given by $\sum_{i=1}^n e_{12,i}$.
(Recall that the subscripts can act as outermorphisms, and so have algebraic properties.  See the
admittedly poor paper \cite{Parkin13_2}.)
The usefulness, if any, of these forms remains to be seen.

The study of algebraic surfaces using polynomial rings in abstract algebra, (algebraic geometry),
does not suffer from coordinitis, because it is so abstract.  There is a certain appeal to geometric algebra,
however, because its language and results, as greatly illustrated by the conformal model, are just down-right fun.

\begin{thebibliography}{10}

\bibitem{DoranHestenes93}
C. Doran et. al., {\it Lie groups as spin groups}, J. Math. Phys., Vol. 34, (1993).

\bibitem{Paeth90}
A. Paeth, {\it A Fast Algorithm for General Raster Rotation},
Graphics gems, Academic Press Professional, (1990), p. 179-195.

\bibitem{Hestenes92}
D. Hestenes, {\it Mathematical Viruses}, Clifford Algebras and their
Applications in Mathematical Physics, Vol. 47, (1992), p. 3-16.

\bibitem{Garrity13}
G. Garrity, et. al., {\it Algebraic Geometry, A Problem Solving Approach},
Student Mathematical Library, IAS/Park City Mathematical Subseries, Vol. 66, (2013).

\bibitem{Gallian12}
J. Gallian, {\it Contemporary Abstract Algebra}, Cengage Learning, 2012.

\bibitem{Dorst07}
L. Dorst, D. Fontijne and S. Mann, {\it Geometric algebra for computer
science}. Morgan Kaufmann, 2007.

\bibitem{Goldman12}
R. Goldman, X. Jia, {\it Perspective Projection in the Homogeneous and Conformal}.
Conference of Applied Geometric Algebras in Computer Science and Engineering, (2012).

\bibitem{GoldmanNotYet}
R. Goldman, S. Mann, {\it $R(4,4)$ As a Computational Framework for 3-Dimensional Computer Graphics}.
Advances in Applied Clifford Algebras, Not Yet Published.

\bibitem{Parkin13}
S. Parkin, {\it The Mother Minkowski Algebra Of Order $M$}.
Advances in Applied Clifford Algebras (2013).

\bibitem{Parkin13_2}
S. Parkin, {\it The Intersection Of Rays With Algebraic Surfaces}.
Advances in Applied Clifford Algebras (2013).

\end{thebibliography}

\end{document}