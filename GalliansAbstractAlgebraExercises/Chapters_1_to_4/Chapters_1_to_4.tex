\documentclass[12pt]{article}

\usepackage{amsmath}
\usepackage{amssymb}
\usepackage{amsthm}

\title{Chapters 1-4 Supplementary Exercises\\Gallian's Book on Abstract Algebra}
\author{Spencer T. Parkin}

\newtheorem{theorem}{Theorem}[section]
\newtheorem{definition}{Definition}[section]
\newtheorem{corollary}{Corollary}[section]
\newtheorem{identity}{Identity}[section]
\newtheorem{lemma}{Lemma}[section]
\newtheorem{result}{Result}[section]

%\newcommand{\gcd}{\mbox{gcd}}
\newcommand{\lcm}{\mbox{lcm}}
\newcommand{\abs}{\mbox{abs}}
\newcommand{\Z}{\mathbb{Z}}
\newcommand{\R}{\mathbb{R}}
\newcommand{\cl}{\mbox{cl}}

\begin{document}
\maketitle

\section*{Problem 1}

Let $G$ be a group and let $H$ be a subgroup of $G$.  For any fixed $x\in G$,
define $xHx^{-1}=\{xhx^{-1}|h\in H\}$.  Prove that $xHx^{-1}$ is a subgroup of $G$,
that if $H$ is cyclic, then $xHx^{-1}$ is cyclic, and that if $H$ is Abelian, then $xHx^{-1}$ is Abelian.

Clearly $e\in xHx^{-1}$.  Letting $a,b\in xHx^{-1}$, there exist elements $h_a,h_b\in H$
such that $a=xh_ax^{-1}$ and $b=xh_bx^{-1}$.  Now since $h_ah_b^{-1}\in H$,
we see that
\begin{equation*}
ab^{-1}=xh_ax^{-1}(xh_bx^{-1})^{-1}=xh_ax^{-1}xh_b^{-1}x^{-1}=xh_ah_b^{-1}x^{-1}\in xHx^{-1}.
\end{equation*}

Now if $H$ is cyclic, then there exists $h\in H$ such that $H=\langle h\rangle$.
We then see that
\begin{equation*}
xHx^{-1}=\{xh^kx^{-1}|k\in\Z\}=\{(xhx^{-1})^k|k\in\Z\} = \langle xhx^{-1}\rangle.
\end{equation*}

If $H$ is Abelian, then for all $a,b\in xHx^{-1}$, we have
\begin{equation*}
ab = xh_ax^{-1}xh_bx^{-1}=xh_ah_bx^{-1}=xh_bh_ax^{-1}=xh_bx^{-1}xh_ax^{-1}=ba.
\end{equation*}

\section*{Problem 2}

Let $G$ be a group and let $H$ be a subgroup of $G$.  Define
\begin{equation*}
N(H)=\{x\in G|xHx^{-1}=H\}.
\end{equation*}
Prove that $N(H)$ (called the {\it normalizer} of $H$) is a subgroup of $G$.

It is clear that $e\in N(H)$.  Now let $a,b\in N(H)$.
Then since $aHa^{-1}=H$ and $bHb^{-1}=H$, we have
\begin{equation*}
abH(ab)^{-1}=abHb^{-1}a^{-1}=aHa^{-1}=H,
\end{equation*}
showing that $ab\in N(H)$.  Now notice that since $aHa^{-1}=H$,
the function $\phi(a)=aha^{-1}$ is a bijection from $H$ to $H$.
It follows that $\phi^{-1}(a)=a^{-1}ha$ is also such a bijection,
and therefore, $a^{-1}Ha=H$, showing that $a^{-1}\in N(H)$.

\section*{Problem 3}

Let $G$ be a group.  For each $a\in G$, define $\cl(a)=\{xax^{-1}|x\in G\}$.
Prove that these subsets of $G$ partition $G$.  [$\cl(a)$ is called the {\it conjugacy
class} of $a$.]

For any $a,b\in G$, let $a\sim b$ if and only if there exists $x\in G$
such that $a=xbx^{-1}$.  We now show that this is an equivilance relation on $G$.

Notice that $a\sim a$, since $a=eae^{-1}$, giving us the reflexive property.
Then, letting $y=x^{-1}\in G$, we see that
\begin{equation*}
a\sim b\implies a=xbx^{-1}\implies b=yay^{-1}\implies b\sim a,
\end{equation*}
giving us the symmetric property.  Lastly, for $a,b,c\in G$,
let $a\sim b$ and $b\sim c$ so that for some $x,y\in G$, we have
$a=xbx^{-1}$ and $b=ycy^{-1}$.  Then we have
\begin{equation*}
a=xbx^{-1}=xycy^{-1}x^{-1}=xyc(xy)^{-1}\implies a\sim c,
\end{equation*}
since $xy\in G$, giving us the transitive property.

Seeing now that for any $a\in G$, we have
\begin{align*}
\cl(a)&=\{xax^{-1}|x\in G\}\\
 &= \{b\in G|\mbox{$\exists x\in G$ s.t. $b=xax^{-1}$}\}\\
 &= \{b\in G|b\sim a\},
\end{align*}
it follows by Theorem 0.6 that the conjugacy classes
of $G$ partition $G$.

\end{document}