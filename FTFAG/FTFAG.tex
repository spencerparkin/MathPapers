\documentclass[12pt]{article}

\usepackage{amsmath}
\usepackage{amssymb}
\usepackage{amsthm}

\title{The Fundamental Theorem\\of\\Finite Abelian Groups}
\author{Spencer T. Parkin}

\newtheorem{theorem}{Theorem}[section]
\newtheorem{definition}{Definition}[section]
\newtheorem{corollary}{Corollary}[section]
\newtheorem{identity}{Identity}[section]
\newtheorem{lemma}{Lemma}[section]
\newtheorem{result}{Result}[section]

%\newcommand{\gcd}{\mbox{gcd}}
\newcommand{\lcm}{\mbox{lcm}}
\newcommand{\Z}{\mathbb{Z}}
\newcommand{\cl}{\mbox{Cl}}
\newcommand{\aut}{\mbox{Aut}}

\begin{document}
\maketitle

This fundamental theorem is traditionally proven using a greedy-algorithm approach.
Here we attempt to give a proof based upon a divide-and-conquer approach.
The basic idea is to show that every finite abelian group, under some condition, can be non-trivially
factored into the internal direct product of two non-trivial subgroups.  This is then applied recursively
until the condition fails.  (The condition might be based upon the group having a non-trivial subgroup.)

\begin{lemma}\label{lem_factor_group}
If $G$ is a finite abelian group having a subgroup $H$, then
$G/H$ is isomorphic to a subgroup $K$ of $G$.
Moreover, $G=KH$, the internal direct product of $K$ and $H$.
\end{lemma}
\begin{proof}
If $H$ is a trivial subgroup, we have nothing to prove,
so we may assume $H$ is non-trivial.  That said, consider
the set $S$ given by
\begin{equation*}
S = \{g\in G|\mbox{$g=g'h$ where $g'\in G$, $h\in H-\{e\}$}\}.
\end{equation*}
We now claim that $K=(G-S)\cup\{e\}$ is a subgroup of $G$.
If $K=\{e\}$, we're done.  So we may assume $K\supset\{e\}$. Our group $G$ being finite, we
need only show closure.  Consider a pair of elements $a,b\in K$.
If either $a$ or $b$ is the identity, or if $a=b^{-1}$, then clearly $ab\in K$.
We may, therefore, assume that neither of these is the case.
Now suppose that $ab\in S-\{e\}$.  ($S-\{e\}$ is non-empty, because $H$ is non-trivial.)
Writing $ab=g'h$, where $g'\in G$ and $h\in H$,
it follows that $a=(g'b^{-1})h$, placing $a\in S$.  Now since $S\cap K=\{e\}$,
we must have $a=e$, which contradicts our supposition.
Therefore, $ab\not\in S-\{e\}$, and we have
\begin{equation*}
ab\in G-(S-\{e\})=(G-S)\cup\{e\}=K.
\end{equation*}

Having now established $K$ as a subgroup of $G$, consider the homomorphism
$\phi:G\to G/H$ given by $\phi(x)=xH$.  We now contend that $\phi$ is an
isomorphism from $K$ to $G/H$.

We first show that $\phi(K)=G/H$.  Let $x\in G$ and consider the
coset $xH$.  If $x\in K$, we're done.  If $x\not\in K$, then $x\in S$,
and we can write it as $x=x'h$ with $x'\in G$ and $h\in H$.
Then $xh=x'H$, and we're left with the same problem forever and ever with no end.
Fail!

What remains to be shown now is that $\phi$, when restricted to $K$,
is ono-to-one.  Let $x,y\in K$ such that $xH=yH$.
It follows that $y^{-1}x\in H$.  Now since $S\supseteq H$, we have $K\cap H=\{e\}$,
and therefore $y^{-1}x=e\implies x=y$.  We can now conclude that $\phi$
is the claimed isomorphism.

The second part of our lemma now follows from the fact that
\begin{equation*}
|HK|=\frac{|H||K|}{|H\cap K|}=|H||K|,
\end{equation*}
while we have just shown that $|K|=|G|/|H|$.
Seeing that $|HK|=|G|$, we have $G=HK$.
\end{proof}

We now jump straight to the fundamental theorem by way of Couchy's Theorem.

\begin{theorem}
Every finite abelian group $G$ is isomorphic to an external direct product
of cyclic groups.
\end{theorem}
\begin{proof}
If $G$ is of prime order, we're done.  This not being the case, let $p$
be any prime divisor of $|G|$.  By Couchy's Theorem, there exists an element $a_1\in G$
of order $p$, and therefore a cyclic subgroup $\langle a_1\rangle$ of $G$ of order $p$.
Applying Lemma~\ref{lem_factor_group}, we may write
$G=\langle a_1\rangle K_1$, where $K_1<G$ and has order $|G|/p$.
Now write $G\approx\langle a_1\rangle\times K_1$ and apply our procedure
again with $K_1$ to obtain $G\approx\langle a_1\rangle\times\langle a_2\rangle\times K_2$.
Our group $G$ being of finite order, this process must, for some integer $k$, terminate with
\begin{equation*}
G=\langle a_1\rangle\times\dots\times\langle a_k\rangle.
\end{equation*}
\end{proof}

There is something wrong here.  The number of non-isomorphic abelian groups of order $p^n$
is the number of partitions of $n$, but I'm not getting this.

\end{document}