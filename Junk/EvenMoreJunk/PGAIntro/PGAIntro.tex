\documentclass[12pt]{article}

\usepackage{amsmath}
\usepackage{amssymb}
\usepackage{amsthm}
%\usepackage{biblatex}

\title{An Introduction To\\Projective Geometry\\Using\\Geometric Algebra}
\author{Spencer T. Parkin}

\newcommand{\G}{\mathbb{G}}
\newcommand{\V}{\mathbb{V}}
\newcommand{\R}{\mathbb{R}}
\newcommand{\B}{\mathbb{B}}
\newcommand{\nvao}{o}
\newcommand{\nvai}{\infty}

\newtheorem{theorem}{Theorem}[section]
\newtheorem{definition}{Definition}[section]
\newtheorem{corollary}{Corollary}[section]
\newtheorem{identity}{Identity}[section]
\newtheorem{lemma}{Lemma}[section]
\newtheorem{result}{Result}[section]

\begin{document}
\maketitle

This paper is my attempt to build up the subject of projective
geometry using geometric algebra in my own words.  I do not
claim originality to any result in this paper.  If nothing else, this
paper simply represents a formal compilation of my notes on the
subject.  I have mainly used \cite{hestenes91}, \cite{birchfield98} and \cite{dorst07}
for research in the preparation of this paper.

I began my study of projective geometry using geometric algebra
after having already put much effort into understanding the
conformal model of geometric algebra.  I may therefore make reference
to concepts in the conformal model as we go along, but it is not a
prerequisite that the reader be familiar at all with the conformal model.

I don't know for certain, but I believe that the conformal model
might be superior to projective geometry, though to-date the latter has
been studied far more than the former, it having been around a lot longer.

\section{Representing Geometry}

Like the conformal model, we may think of geometries as subsets of the
set of all points in some $n$-dimensional Euclidean space, which we'll
denote by $\V^n$.  This also denotes an $n$-dimensional Euclidean vector space
as we adopt here the standard correlation between vectors in such a vector space
with points in an $n$-dimensional Euclidean space.

Points sets, of course, do not lend themselves easily to goemetric analysis.  So,
like the conformal model, we represent them using blades in a geometric algebra.
Why we use blades will become apparent after we define how a blade represents
a point set, because then it will become clear how the meet and join operations
of blades will allow us to do some interesting geometric operations, just as we
can in the conformal model.

For $n$-dimensional projective geometry, we use a geometric algebra generated
by an $(n+1)$-dimesnional Euclidean vector space.  If $\{e_k\}_{k=0}^{n-1}$ is any set of orthonormal basis
vectors spanning $\V^n$, let $\{e_k\}_{k=0}^n$ be a set of
orthonormal basis vectors spanning $\V^{n+1}$, which we'll use to generate our
geometric algebra $\G(\V^{n+1})$.

In projective geometry we can represent points, lines, planes, hyper-planes, and so on to higher dimensions.
Certainly results in geometry involving all of these types of geometric 
primitives can be found by simply using $\V^n$ alone, but what we'll see is that the extra dimension
in $\V^{n+1}$ will facilitate some amazingly useful constructions in $\G(\V^{n+1})$ that
make the finding of such results much easier than it would be otherwise.  Indeed, in
\cite{hestenes91}, it is shown how geometric algebra easily and naturally explains many
fundamental theorems in projective geometry.
It is my guess that interpretations of how these constructions work based on $(n+1)$-dimensional
projections into $n$-dimensional space are at least partially to blame for the title of the
subject being projective geometry.

Without further delay, we begin with a function $p:\V^n\to\V^{n+1}$ that defines
a mapping from points in our Euclidean space with vectors in our geometric algebra.
We then say that a blade $A\in\G(\V^{n+1})$ represents a piece of geometry as
the set of all points $x\in\V^n$ such that $p(x)\wedge A=0$.  From this it is clear
that all non-zero scalar multiples $A$ also represent the same piece of geometry.
This is the nature of a homogeneous representation model.

We define $p$ simply as
\begin{equation*}
p(x) = x + e_3.
\end{equation*}
Having done so, it is easy to see that for any vector $b\in\V^n$, that $p(b)$
represents the point $b$.  Now here's where it gets interesting.  Let $\{b_k\}_{k=0}^{m-1}$
be any set of $m$ points taken from $\V^n$ such that they are non-co-$(m-1)$-hyper-planar.
That is, if $m=2$, the 2 points are distinct; if $m=3$, the 3 points are non-co-linear; if
$m=4$, the 4 points are non-co-planar; if $m=5$, the points are non-co-hyper-planar, and so on.
We will now show that if $m\leq n+1$, then the blade $B=\bigwedge_{k=0}^{m-1}p(b_k)$ represents an
$(m-1)$-dimensional hyper-plane.

The case $m=n+1$ is trivial.  In that case, $B$ is a psuedo-scalar of $\G(\V^{n+1})$, and
so for all points $x\in\V^n$, we have $p(x)\wedge B=0$, showing that $B$ represents
an $n$-dimensional hyper-plane.  Let us now consider all $1<m<n+1$.  We start by
putting $B$ in a form that exposes its two primary characteristics.
\begin{equation}\label{equ_plane_canonical_form}
B = p(b_0)\wedge A = b_0\wedge A + e_n\wedge A
\end{equation}
Here we have arbitrarily chosen $a_0$ to represent what we'll call the center of the hyper-plane, if you will.
(Any point on a plane may be thought of as its center.  Let's just choose one.)  The blade $A$ here
represents the attitude of the hyper-plane, and is given by
\begin{equation*}
A = \bigwedge_{k=1}^{m-1}(b_k-b_0).
\end{equation*}
Examining now the equation $p(x)\wedge B=0$, we see that this reduces to
\begin{equation*}
x\wedge a_0\wedge A + e_n\wedge (x-a_0)\wedge A = 0,
\end{equation*}
showing that for $x$ to be part of the geometry represented by $A$, there are two
conditions that need to be met.  The first is that $x\wedge a_0\wedge A=0$, which is
a necessary yet insufficient condition that $x$ be on the hyper-plane.  The second,
however, that $(x-a_0)\wedge A=0$, is both a necessary and sufficient condition
that $x$ be on the $(m-1)$-dimensional hyper-plane.  And there we have it!

Looking back at the canonical form of a hyper-plane in this model,
equation $\eqref{equ_plane_canonical_form}$ shows how we might decompose
a given hyper-plane into its constituent characteristics.  Of course, given
a blade $B$, we couldn't hope to recover the original center and attitude
used to compose it, but we can find a center and the original attitude up to scale.
An attidue can be found as $A=e_n\cdot B$.  Then, interestingly, we can find
the center on the plane closest to the origin by requiring $b_0\cdot A=0$.
It follows that
\begin{equation*}
b_0 = (e_n\cdot(e_n\wedge B))\frac{\tilde{A}}{|A|^2}.
\end{equation*}

We have now exhausted the geometric primitives of the projective geometry model
of geometric algebra.  What remains to is to explore what types of operations we
can perform on these geometries, what results we can derive in this model, and
however else we can find the model useful in the subject of geometry.

\section{Intersecting Geometry}

Explore that here...

% decompose to get characteristic parts.  center closest to origin found.  attitude recoverable

\bibliographystyle{plain}
\bibliography{PGAIntro}

\end{document}