\documentclass[12pt]{article}

\usepackage{amsmath}
\usepackage{amssymb}
\usepackage{amsthm}

\title{Section 2.13 Exercises\\Herstein's Topics In Algebra}
\author{Spencer T. Parkin}

\newtheorem{theorem}{Theorem}[section]
\newtheorem{definition}{Definition}[section]
\newtheorem{corollary}{Corollary}[section]
\newtheorem{identity}{Identity}[section]
\newtheorem{lemma}{Lemma}[section]
\newtheorem{result}{Result}[section]

%\newcommand{\gcd}{\mbox{gcd}}
\newcommand{\lcm}{\mbox{lcm}}
\newcommand{\Z}{\mathbb{Z}}
\newcommand{\cl}{\mbox{Cl}}

\begin{document}
\maketitle

\section*{Lemma 1}

Let $\{a_i\}_{i=1}^n$ be a set of $n$ elements taken from an abelian group $G$
such that
\begin{equation*}
\gcd(a_1,\dots,a_n)=1,
\end{equation*}
which is also to say that they, collectively, are coprime.  We then have
\begin{equation*}
\left|\prod_{i=1}^n a_i\right|=\lcm(|a_1|,\dots,|a_n|).
\end{equation*}
\begin{proof}
We begin by observing that
\begin{equation*}
\left(\prod_{i=1}^n a_i\right)^k = \prod_{i=1}^n a_i^k,
\end{equation*}
since $G$ is abelian.  That the order of the product is the above least common
multiple now follows by the definition of the order of an element.
\end{proof}

\section*{Problem 5}

Let $G$ be a finite abelian group.  Prove that $G$ is isomorphic to the
direct preoduct of its Sylow subgroups.

Let's first observe that for every prime divisor $p$ of $|G|$ that
there is one and only one $p$-Sylow subgroup of $G$.  In light
of Theorem 2.12.2, this is because $G$ is abelian, and therefore, every such subgroup
is conjugate only with itself.

That said, if $\{P_i\}_{i=1}^n$ denotes the set of all Sylow subgroups of $G$,
then no prime is repeated in the associated set of primes $\{p_i\}_{i=1}^n$.
Here, each subgroup $P_i$ is a $p_i$-Sylow subgroup of $G$.
Now consider the internal direct product
\begin{equation*}
\prod_{i=1}^n P_i = \left\{\left.\prod_{i=1}^n g_i\right|g_i\in P_i\right\}.
\end{equation*}
The number of products generating this set is clearly
\begin{equation*}
\prod_{i=1}^n |P_i|,
\end{equation*}
but it remains to be seen whether this is the size of the set.  To see that
it is, consider the possibility that any two different product are equal to one another.
That is, consider the equation
\begin{equation*}
\prod_{i=1}^n g_i = \prod_{i=1}^n g_i'.
\end{equation*}
Rearranging this, we get, for any $1\leq j\leq n$,
\begin{equation*}
g_j(g_j')^{-1} = \prod_{i=1,i\neq j}^n g_i(g_i')^{-1}.
\end{equation*}
Now notice that $|g_j(g_j')^{-1}|=p_j^{\alpha_j}$.  We then see, by Lemma 1, that the
order of the right-hand side our equation is
\begin{equation*}
\lcm(p_1^{\alpha_1},\dots,p_{j-1}^{\alpha_{j-1}},p_{j+1}^{\alpha_{j+1}},\dots,p_n^{\alpha_n}) = \prod_{i=1,i\neq j}^n p_i^{\alpha_i},
\end{equation*}
yet the order of the left-hand side is $p_j^{\alpha_j}$.  We must, therefore, concede that $g_j(g_j')^{-1}=e$.
It then follows that, for all $1\leq j\leq n$, we have $g_j=g_j'$, showing that product given earlier is indeed
the size of the internal product.

We have now shown that the internal product generates $G$, and that every elemetn of $G$ has a
unique decomposition in terms of one element taken from each Sylow subgroup.  We can now apply
Theorem 2.13.1 to get the final, desired result.

\end{document}