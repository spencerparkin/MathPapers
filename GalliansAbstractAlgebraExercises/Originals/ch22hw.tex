\documentclass{article}
\usepackage{amsmath}
\usepackage{amssymb}
\title{Chapter 22 Homework}
\author{Spencer}
\addtolength{\oddsidemargin}{-.575in}
\addtolength{\evensidemargin}{-.575in}
\addtolength{\textwidth}{1.0in}
\addtolength{\topmargin}{-.575in}
\addtolength{\textheight}{1.25in}
\begin{document}
\maketitle

\newcommand{\Z}{\mathbb{Z}}
\newcommand{\R}{\mathbb{R}}
\newcommand{\N}{\mathbb{N}}
\newcommand{\Q}{\mathbb{Q}}
\newcommand{\GF}{\mbox{GF}}

\section*{Problem 6}

Prove that the rings $\Z_3[x]/\langle x^2+x+2\rangle$ and $\Z_3[x]\langle x^2+2x+2\rangle$
are isomorphic.

Both rings are fields, because $x^2+x+2$ and $x^2+2x+2$ are irreducible over $\Z_3$.
Furthermore, they are both finite fields of order $2^3$.  So by Theorem 22.1,
we have
\begin{equation*}
\Z_3[x]\langle x^2+x+2\rangle\approx\GF(2^3)\approx\Z_3[x]\langle x^2+2x+2\rangle.
\end{equation*}

\section*{Problem 8}

Determine the possible finite fields whose largest proper subfield is $\GF(2^5)$.

By Theorem 22.3, if $k>1$, then finite fields of the form $\GF(2^{5k})$ have
$\GF(2^5)$ as a proper subfield.  For $\GF(2^5)$ to be the largest proper subfield
of $\GF(2^{5k})$, we need $k\leq 5$.  So the possible fields are $\GF(2^{10})$,
$\GF(2^{15})$, $\GF(2^{20})$, and $\GF(2^{25})$.

\section*{Problem 12}

Without actually calculating $|x|$, explain why $x$ is a generator of the
cyclic group $(\Z_2[x]/\langle x^5+x^3+1\rangle)^{*}$.

It is not obvious to me why $x$ must be a generator of the cyclic group,
but I can show that it is a generator.  Since the order of the cyclic group
is $2^5$, it follows that $x$ is a generator of the group if $x^8\neq 1$ and
$x^{16}\neq 1$.  Long division shows that $x^8$ modulo $x^5+x^3+1$ is $x^4+x^3+x$,
and $x^{16}$ modulo $x^5+x^3+1$ is $x^3+x^2$, if I didn't make a mistake.

\section*{Problem 18}

Show that any finite subgroup of the multiplicative group of a field is cyclic.

Let $F$ be a field and let $G$ be a subgroup of $F^{*}$.
If we can show that $G\cup\{0\}$ is a finite field, then it follows
from Theorem 22.2 that $G$ is cyclic.  It suffices to show that $G\cup\{0\}$
is an additive subgroup of $F$.  I have no idea how to show this.

\section*{Problem 24}

If $p(x)$ is a polynomial in $\Z_p[x]$ with no multiple zeros, show that $p(x)$
divides $x^{p^n}-x$ for some $n$.

We want to find an integer $n$ so that the splitting field of $p(x)$ over
$\Z_p$ is a subfield of the splitting field for $x^{p^n}-x$ over $\Z_p$.
But notice that the splitting field for $p(x)$ over $\Z_p$ is a finite
field, and therefore has the form $\GF(p^m)$ since $\GF(p)\approx\Z_p$ is
one of its subfields.  So choose $n=m$, and we're done, because every
element in $\GF(p^n)$ will be a zero (once) of $x^{p^n}-x$, and every element
in a subset of $\GF(p^n)$ will be a zero (once) of $p(x)$.

\section*{Problem 26}

Let $f(x)$ be a cubic irreducible over $\Z_p$, where $p$ is prime.  Prove that
the splitting field of $f(x)$ over $\Z_p$ has order $p^3$ or $p^6$.

If $\Z_p[x]/\langle f(x)\rangle\approx\GF(p^3)$ contains all of the zeros
of $f(x)$, then we have found the splitting field of $f(x)$ over $\Z_p$.
If not, then only one zero of $f(x)$ will be in $\GF(p^3)$.  (If two
zeros of $f(x)$ were in $\GF(p^3)$, then the third would be also, since
two linear factors could be factored out of the cubic, giving us the third linear factor
for free.)  Let $p(x)$ be the irreducible quadratic factor of $f(x)$ in $\GF(p^3)$.
Then $\GF(p^3)/\langle p(x)\rangle\approx\GF((p^3)^2)$ is the splitting field
for $f(x)$ over $\Z_p$, because throwing in one zero of a quadratic will throw
in the other zero also.

\section*{Problem 28}

Suppose that $F$ is a field of order $2^{10}$ and $F^{*}=\langle\alpha\rangle$.
List the elements of each subfield of $F$.

\begin{align*}
\GF(2^{10}) &= \{0\}\cup\langle\alpha\rangle \\
\GF(2^5) &= \{0\}\cup\langle\alpha^{33}\rangle \\
\GF(2^2) &= \{0\}\cup\langle\alpha^{341}\rangle \\
\GF(2) &= \{0,1\}
\end{align*}

\end{document}