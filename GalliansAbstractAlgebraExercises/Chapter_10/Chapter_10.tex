\documentclass[12pt]{article}

\usepackage{amsmath}
\usepackage{amssymb}
\usepackage{amsthm}

\title{Chapter 10 Exercises\\Gallian's Book on Abstract Algebra}
\author{Spencer T. Parkin}

\newtheorem{theorem}{Theorem}[section]
\newtheorem{definition}{Definition}[section]
\newtheorem{corollary}{Corollary}[section]
\newtheorem{identity}{Identity}[section]
\newtheorem{lemma}{Lemma}[section]
\newtheorem{result}{Result}[section]

%\newcommand{\gcd}{\mbox{gcd}}
\newcommand{\lcm}{\mbox{lcm}}
\newcommand{\abs}{\mbox{abs}}
\newcommand{\Z}{\mathbb{Z}}
\newcommand{\R}{\mathbb{R}}
\newcommand{\G}{\mathbb{G}}
\newcommand{\stab}{\mbox{stab}}
\newcommand{\aut}{\mbox{Aut}}
\newcommand{\inn}{\mbox{Inn}}
\newcommand{\orb}{\mbox{orb}}

\begin{document}
\maketitle

\section*{Lemma 1}

Let $H$ be a proper subgroup of $G$.  Then
for all $g\in G-H$ and all $h\in H$, $gh\in G-H$.

Suppose $gh=h'\in H$.  Then $g=h'h^{-1}\in H$,
which is a contradiction.  Therefore, $gh\in G-H$.

\section*{Lemma 2}

Let $N$ be a normal subgroup of a group $G$.
Then for any $g\in G$ and any $n\in N$,
there exists $n'\in N$ such that $gn=n'g$ or such that $ng=gn'$.

\section*{Lemma 3}

Let $G$ be a group and let $n$ be a positive integer.
Then the number of elements in $G$ of order $n$, if any,
is divisible by $\phi(n)$, the totient of $n$.

Suppose $G$ has one or more elements of order $n$.
Let $N$ be the set $\{x\in G||x|=n\}$.  Then, for any pair
of elements $a,b\in N$, let $a\sim b$ if and only if $a\in\langle b\rangle$.
This defines an equivilance relation on $N$, since $a\in\langle a\rangle$ gives us
the reflexive property, since $a\in\langle b\rangle\implies b\in\langle a\rangle$ gives
us the symmetric property, and since $a\in\langle b\rangle$ and, for $c\in N$, $b\in\langle c\rangle$ implies
that $a\in\langle c\rangle$, giving us the transitive property.  We now note that
by Theorem 4.4, the size of each equivilance class is $\phi(n)$.
It follows that the number of elements of order $n$ is $G$ is
$s\phi(n)$, where $s$ is the number of equivilance classes.

\section*{Exercise 39}

If $K$ is a subgroup of $G$ and $N$ is a normal subgroup of $G$,
prove that $K/(K\cap N)$ is isomorphic to $KN/N$.

Notice that the normality of the subgroup $K\cap N$ in $K$ is proven
by the problem similar to Exercise 50 in Chapter 9.

We now show that $KN$ is a group.
Let $x\in KN$.  Then $x=kn$ for some $k\in K$ and $n\in N$.
But then by Lemma 2 above, $x=n'k\in NK$ for some $n'\in N$.
It follows that $KN\subseteq NK$.
Similarly, we can show that $NK\subseteq KN$, so $NK=KN$.
It then follows by Exercise 6 of the supplementary exercises
for chapters 5 through 8 that $NK$ is a group.

Is $N$ normal in $KN$?

We now let $\phi:K/(K\cap N)\to KN/N$ be a function defined as
\begin{equation*}
\phi(k(K\cap N))=kN,
\end{equation*}
and show that it is a homomorphism.  Let us first verify that this
is a well defined function.  Let $a,b\in K$ such that $a(K\cap N)=b(K\cap N)$.
Then $ab^{-1}\in K\cap N\subseteq N$, showing that $aN=bN$.

We now show that $\phi$ is operation preserving.
By the normality of $N$ and $N\cap K$, we see that
\begin{align*}
 & \phi(a(K\cap N)b(K\cap N)) \\
=\,& \phi(ab(K\cap N)) \\
 =\,& abN=aNbN \\
 =\,& \phi(a(K\cap N))(\phi(b(K\cap N)),
\end{align*}
showing that $\phi$ is operation preserving.

We now consider the kernel of $\phi$.  Notice that
\begin{align*}
\ker\phi &= \{k(K\cap N)\in K/(K\cap N)|k\in N\}, \\
 &= \{k(K\cap N)\in K/(K\cap N)|k\in K\cap N\}, \\
 &= \{K\cap N\}.
\end{align*}
It follows that $\phi$ is an isomorphism by Property 9
of Theorem 10.2.

\section*{Exercise 40}

If $M$ and $N$ are normal subgroups of $G$ and $N\leq M$, prove
that $(G/N)/(M/N)\approx G/M$.

\end{document}