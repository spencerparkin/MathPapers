\documentclass{birkjour}

\newtheorem{thm}{Theorem}[section]
\newtheorem{cor}[thm]{Corollary}
\newtheorem{lem}[thm]{Lemma}
\newtheorem{prop}[thm]{Proposition}
\theoremstyle{definition}
\newtheorem{defn}[thm]{Definition}
\theoremstyle{remark}
\newtheorem{rem}[thm]{Remark}
\newtheorem*{ex}{Example}
\numberwithin{equation}{section}

\newcommand{\R}{\mathbb{R}}
\newcommand{\B}{\mathbb{B}}
\newcommand{\G}{\mathbb{G}}
\newcommand{\V}{\mathbb{V}}
\newcommand{\gd}{\dot{g}}
\newcommand{\gh}{\hat{g}}
\newcommand{\Gd}{\dot{G}}
\newcommand{\Gh}{\hat{G}}
\newcommand{\nvai}{\infty}
\newcommand{\nvao}{o}
\newcommand{\grade}{\mbox{grade}}

\begin{document}

\title{On Two Useful Properties of\\Conformal-like Models of\\Geometric Algebra}

\author{Spencer T. Parkin}
\address{102 W. 500 S., Salt Lake City, UT  84101}
\email{spencerparkin@outlook.com}

\subjclass{Primary 14J70; Secondary 14J29}

\keywords{Algebraic Surface, Conformal Model, Conformal Transformation, Geometric Algebra}

\date{November 18, 2013}

%\dedicatory{To my dear wife Melinda.}

\begin{abstract}
Blah.
\end{abstract}

\maketitle

\section{Introduction}

This paper begins where the first section of \cite{Parkin13} leaves off.\footnote{No more than
the first section of \cite{Parkin13} need be read before reading this paper.}  In the
said section, the function $\gh$ was introduced, but its property of allowing us
to combine geometries was not given any treatment.  Here we rectify this
glossing-over of the property by giving it an in-depth look in a some-what more general context
than that of the conformal model of geometric algebra.

Another often-over-looked, and very important
property of the general model laid out in the first section of \cite{Parkin13} is that
of being able to equate different characterizations of a single geometry.  We will see
how this property follows from what we'll call the point-fitting property, and why
it is important.

\section{The Point-Fitting Property}

Letting $B\in\G$ be a $k$-blade, notice that if
there exists a set of $k$ points $\{x_i\}_{i=1}^k\subseteq\gh(B)$ such that
$\bigwedge_{i=1}^k p(x_i)\neq 0$,
then there must exist a scalar $\lambda\in\R$ such that
\begin{equation*}
B = \lambda\bigwedge_{i=1}^k p(x_i).
\end{equation*}
When there exists such a set of $k$ points for a blade $B$,
we will say that the geometry $\gh(B)$ has the point-fitting
property, and that the $k$ points $\{x_i\}_{i=1}^k$ fit $\gh(B)$,
or that $\gh(B)$ is fitted by these points.

% Adapt proofs in other paper here.

\section{The Reinterpretation Property}

Letting $A,B\in\G$ both be $k$-blades with $\gh(A)=\gh(B)$, if there
exists a scalar $\lambda\in\R$ such that $A=\lambda B$, we say that
the geometries $\gh(A)$ and $\gh(B)$ have, collectively, the reinterpretation property.\footnote{Clearly,
if $A=\lambda B$, then $\gh(A)=\gh(B)$, but, depending on how $p$ is defined, the
converse of this statement may not be generally true.}
It is not hard to show that if each of $A$ and $B$ have the ponit-fitting property,
then they share the reinterpretation property.

Under the assumption that each of $\gh(A)$ and $\gh(B)$ have the point-fitting property,
notice that if the $k$ points $\{x_i\}_{i=1}^k$ fit $\gh(A)$,
then they also fit $\gh(B)$.  Letting $A=\alpha\bigwedge_{i=1}^k p(x_i)$ and
$B=\beta\bigwedge_{i=1}^k p(x_i)$, it is clear that if $\lambda =\frac{\alpha}{\beta}$,
then $A=\lambda B$.

What this property allows us to do in the model is equate one characterization of a given
geometry with that of another.  This may well be termed a reinterpretation of the geometry.
For example, we may form a circle as the intersection of two spheres, then go on
to equate this characterization of the circle with a canonical characterization, which may
be that of the intersection of a plane and sphere centered on that plane.  Reinterpreting
the calculated geometry in terms of a canonical form, we can easily
calculate the characteristics this intersection.  Indeed, this property of the model, (the reinterpretation property),
and not the intersection property of the model itself, is what makes the ability to do intersections
in the model useful and interesting.

\begin{thebibliography}{9}

\bibitem{Parkin13}
S. Parkin, {\it The Mother Minkowski Algebra of Order $m$}.
2013
\end{thebibliography}

\end{document}