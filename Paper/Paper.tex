\documentclass{birkjour}

\usepackage{hyperref}

\newtheorem{thm}{Theorem}[section]
\newtheorem{cor}[thm]{Corollary}
\newtheorem{lem}[thm]{Lemma}
\newtheorem{prop}[thm]{Proposition}
\theoremstyle{definition}
\newtheorem{defn}[thm]{Definition}
\theoremstyle{remark}
\newtheorem{rem}[thm]{Remark}
\newtheorem*{ex}{Example}
\numberwithin{equation}{section}

\newcommand{\R}{\mathbb{R}}
\newcommand{\B}{\mathbb{B}}
\newcommand{\G}{\mathbb{G}}
\newcommand{\V}{\mathbb{V}}
\newcommand{\gd}{\dot{g}}
\newcommand{\gh}{\hat{g}}
\newcommand{\Gd}{\dot{G}}
\newcommand{\Gh}{\hat{G}}
\newcommand{\nvai}{\infty}
\newcommand{\nvao}{o}
\newcommand{\grade}{\mbox{grade}}
\newcommand{\rank}{\mbox{rank}}

%\received{}\accepted{}

\begin{document}

\title{Untitled}

\author{Spencer T. Parkin}
\email{spencerparkin@outlook.com}

%\subjclass{Primary 14J70; Secondary 14J29}

%\dedicatory{To my dear wife Melinda.}

\begin{abstract}
Abstract...
\end{abstract}

\keywords{Key words...}

\maketitle

\begin{defn}[Spade]\label{def_spade}
An element $M\in\G$ is a \emph{spade} if it can be written as the geometric
product of zero or more vectors.
\end{defn}

It follows from Definition~\ref{def_spade} that all versors are spades, but not all spades are versors.
Furthermore, while the set of all spades of $\G$ enjoys closure under the geometric product, this set,
unlike the set of versors of $\G$, does not form a group.

\begin{defn}[Spade Rank/Compact Spade Form]\label{def_spader_rank}
The \emph{rank} of a spade $M\in\G$, denoted $\rank(M)$, is the smallest number\footnote{This smallest number exists
by the well-ordering principle.  See \cite{}.} of vectors for which $M$ can
be written as a geometric product of such.  If a spade $M_r\in\G$ has factorization
\begin{equation*}
M_r=\prod_{i=1}^r m_i,
\end{equation*}
we say that it is written in \emph{compact form} if $\rank(M_r)=r$.
\end{defn}

Many identities involving a spade $M_r$ hold whether or not it is written in compact form.
It is even more important, however, to realize that every spade $M_r$ can be rewritten in a compact form
if not already in such a form.

\begin{lem}\label{lem_compact_lin_indep}
For any given spade $M_r\in\G$, $\rank(M_r)=r$ if and only if
\begin{equation*}
\langle M_r\rangle_r = \bigwedge_{i=1}^r m_i.
\end{equation*}
\end{lem}
\begin{proof}
One direction is trivial.  The other is not!
\end{proof}

\begin{lem}\label{lem_lin_indep_subblades}
For every non-zero $r$-blade $B_r\in\G$ with $r>1$, and having factorization
\begin{equation*}
B_r = \bigwedge_{i=1}^r b_i,
\end{equation*}
the set of $(r-1)$-blades $\{B_r^{(i)}\}_{i=1}^r$, where the notation $B_r^{(i)}$ is given by
\begin{equation*}
B_r^{(i)}=\bigwedge_{\substack{j=1\\j\neq i}}^r b_j,
\end{equation*}
is a linearly independent set.
\end{lem}
\begin{proof}
Supposing to the contrary, and without loss of generality, let
\begin{equation*}
B_{r-1} = B_r^{(r)} = \sum_{i=1}^{r-1}\alpha_i B_r^{(i)} = \left(\sum_{i=1}^{r-1}\alpha_i B_{r-1}^{(i)}\right)\wedge b_r.
\end{equation*}
Now notice that
\begin{equation*}
0\neq B_r = B_{r-1}\wedge b_r = B_r^{(r)}\wedge b_r = \left(\sum_{i=1}^{r-1}\alpha_i B_r^{(i)}\right)\wedge b_r = 0,
\end{equation*}
which is clearly a contradiction.
\end{proof}

\begin{lem}\label{lem_solution_intersection}
Given any spade $M_r$, the set of all solution sets $\{\alpha_i\}_{i=1}^r$ of the equation
\begin{equation*}
0 = \sum_{i=1}^r\alpha_i M_r^{(i)}
\end{equation*}
is, for all integers $j\in[0,r]$, the intersection of all sets of solution sets of the equations
\begin{equation*}
0 = \sum_{i=1}^r\alpha_i\langle M_r^{(i)}\rangle_j,
\end{equation*}
where the notation $M_r^{(i)}$ is given by
\begin{equation*}
M_r^{(i)}=\prod_{\substack{j=1\\j\neq i}}^r m_j.
\end{equation*}
\end{lem}
\begin{proof}
This is a simple consequence of there being no possibility of interference between elements of differing grade.
\end{proof}

\begin{lem}
For any given spade $M_r\in\G$ with $r>1$, if $\rank(M_r)=r$, then
the set of spades $\{M_r^{(i)}\}_{i=1}^r$ is a linearly independent set.
\end{lem}
\begin{proof}
This follows easily in consideration of lemmas \ref{lem_compact_lin_indep}, \ref{lem_lin_indep_subblades} and \ref{lem_solution_intersection}.
\end{proof}

% Make use of <*>_a^b = sum_i <*>_i notation later on.

\end{document}