\documentclass{birkjour}

\usepackage{hyperref}

\newtheorem{thm}{Theorem}[section]
\newtheorem{cor}[thm]{Corollary}
\newtheorem{lem}[thm]{Lemma}
\newtheorem{prop}[thm]{Proposition}
\theoremstyle{definition}
\newtheorem{defn}[thm]{Definition}
\theoremstyle{remark}
\newtheorem{rem}[thm]{Remark}
\newtheorem*{ex}{Example}
\numberwithin{equation}{section}

\newcommand{\R}{\mathbb{R}}
\newcommand{\B}{\mathbb{B}}
\newcommand{\G}{\mathbb{G}}
\newcommand{\V}{\mathbb{V}}
\newcommand{\gd}{\dot{g}}
\newcommand{\gh}{\hat{g}}
\newcommand{\Gd}{\dot{G}}
\newcommand{\Gh}{\hat{G}}
\newcommand{\nvai}{\infty}
\newcommand{\nvao}{o}
\newcommand{\grade}{\mbox{grade}}
\newcommand{\rank}{\mbox{rank}}

%\received{}\accepted{}

\begin{document}

\title{Untitled}

\author{Spencer T. Parkin}
\email{spencerparkin@outlook.com}

%\subjclass{Primary 14J70; Secondary 14J29}

%\dedicatory{To my dear wife Melinda.}

\begin{abstract}
Abstract...
\end{abstract}

\keywords{Key words...}

\maketitle

\begin{defn}[Spade]\label{def_spade}
An element $M\in\G$ is a \emph{spade} if it can be written as the geometric
product of zero or more vectors.
\end{defn}

It follows from Definition~\ref{def_spade} that all versors are spades, but not all spades are versors.
Furthermore, while the set of all spades of $\G$ enjoys closure under the geometric product, this set,
unlike the set of versors of $\G$, does not form a group.

\begin{defn}[Spade Rank/Compact Spade Form]\label{def_spader_rank}
The \emph{rank} of a spade $M\in\G$, denoted $\rank(M)$, is the smallest number of vectors for which $M$ can
be written as a geometric product of such.  If a spade $M_r\in\G$ has factorization
\begin{equation*}
M_r=\prod_{i=1}^r m_i,
\end{equation*}
we say that it is written in \emph{compact form} if $\rank(M_r)=r$.
\end{defn}

Many identities involving a spade $M_r$ hold whether or not it is written in compact form.

\begin{defn}[Dull/Sharp Spade]
A spade $M\in\G$ is \emph{sharp} if the grade $\rank(M)$ part of $M$ is non-zero, and \emph{dull} otherwise.
\end{defn}

If $M_r$ is a sharp blade written in compact form, it then follows that
\begin{equation*}
\langle M_r\rangle_r = \bigwedge_{i=1}^r m_i.
\end{equation*}

\begin{defn}[Ideal Spade Form]
Letting $M_r^{(i)}$ denote the spade
\begin{equation*}
M_r^{(i)}=\prod_{\substack{j=1\\j\neq i}}^r m_j,
\end{equation*}
a spade $M_r\in\G$ is written in \emph{ideal form} when the set of $r$ spades $\{M_r^{(i)}\}_{i=1}^r$
is a linearly independent set.
\end{defn}

Clearly, every spade can be written in compact form.
At this point, however, it is not clear whether every spade can be written in ideal form.
We will address this in Lemma~\ref{}.

\begin{lem}\label{lem_lin_indep_blades}
For every non-zero $r$-blade $B_r\in\G$, having factgorization
\begin{equation*}
B_r = \bigwedge_{i=1}^r b_i,
\end{equation*}
the set of $(r-1)$-blades $\{B_r^{(i)}\}_{i=1}^r$, where
\begin{equation*}
B_r^{(i)}=\bigwedge_{\substack{j=1\\j\neq i}}^r b_j,
\end{equation*}
is a linearly independent set.
\end{lem}
\begin{proof}
Supposing to the contrary, and without loss of generality, let
\begin{equation*}
B_{r-1} = B_r^{(r)} = \sum_{i=1}^{r-1}\alpha_i B_r^{(i)}.
\end{equation*}
Now notice that
\begin{equation*}
0\neq B_r = B_{r-1}\wedge b_r = B_r^{(r)}\wedge b_r = \left(\sum_{i=1}^{r-1}\alpha_i B_r^{(i)}\right)\wedge b_r = 0,
\end{equation*}
which is clearly a contradiction.
\end{proof}

\begin{lem}\label{lem_solution_intersection}
Given any spade $M_r$, the set of all solution sets $\{\alpha_i\}_{i=1}^r$ of the equation
\begin{equation*}
0 = \sum_{i=1}^r\alpha_i M_r^{(i)}
\end{equation*}
is the intersection of all sets of solution sets of the equations
\begin{equation*}
0 = \sum_{i=1}^r\alpha_i\langle M_r^{(i)}\rangle_j,
\end{equation*}
for all integers $j\in[0,r]$.
\end{lem}
\begin{proof}
This is a simple consequence of there being no possibility of cancelation between elements of differing grade.
\end{proof}

\begin{lem}\label{lem_lin_indep_sharp_blades}
Every sharp spade having rank greater than one, and written in compact form, is also written in ideal form.
\end{lem}
\begin{proof}
Let $M_r$ be a sharp spade written in compact form with $r>1$.  It follows that
$\{m_i\}_{i=1}^r$ is a linearly independent set of vectors, and so any
one of its subsets is also linearly independent.  By Lemma~\ref{lem_lin_indep_blades},
it now follows that $\{\langle M_r^{(i)}\rangle_{r-1}\}_{i=1}^r$ is a linearly independent set.
The linear independence of the set $\{M_r^{(i)}\}_{i=1}^r$ now follows by Lemma~\ref{lem_solution_intersection}.
\end{proof}

% M_r compact ==> M_{r-1} compact
% M_r sharp ==> M_{r-1} sharp?  Yes, clearly.

\begin{lem}
If $M_r$ is a dull spade written in compact form, but $M_{r-1}$ is sharp, then
$M_r$ is also written in ideal form.
\end{lem}
\begin{proof}
Being careful, it should be possible to find the expansion of $\langle M_r\rangle_{r-2}$.
\end{proof}

%\begin{thm}
%Every spade, sharp or dull, written in compact form is also written in ideal form.
%\end{thm}
%\begin{proof}
%\end{proof}

\end{document}