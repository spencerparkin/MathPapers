\documentclass[12pt]{article}

\usepackage{amsmath}
\usepackage{amssymb}
\usepackage{amsthm}

\title{Chapter 3 Exercises\\Gallian's Book on Abstract Algebra}
\author{Spencer T. Parkin}

\newtheorem{theorem}{Theorem}[section]
\newtheorem{definition}{Definition}[section]
\newtheorem{corollary}{Corollary}[section]
\newtheorem{identity}{Identity}[section]
\newtheorem{lemma}{Lemma}[section]
\newtheorem{result}{Result}[section]

%\newcommand{\gcd}{\mbox{gcd}}
\newcommand{\lcm}{\mbox{lcm}}
\newcommand{\Z}{\mathbb{Z}}
\newcommand{\R}{\mathbb{R}}

\begin{document}
\maketitle

\section*{Problem 3}

Let $Q$ and $Q^*$ be as in Exercise 2.  Find the order of each element of $Q$ and $Q^*$.

For a given $q\in Q$, let $a,b\in\Z$ be integers, with $b\neq 0$, such that $q=a/b$.
We seek the smallest positive integer $n$ for which $nq=na/b=0$.
If $a=0$, then $n=1$.  If $a\neq 0$, then there is no such integer.
The order of non-identity elements in $Q$ is therefore infinite.

For a given $q\in Q^*$, let $a,b\in\Z$ be integers, with $b>0$, such that $q=a/b$.
We seek the smallest positive integer $n$ for which $q^n=a^n/b^n=1$.
If $a=b$, then $q=1$ and $n=1$.  If $|a|<b$, then for all $n>1$, we have
$|a^n/b^n|<a/b<1$, showing that the order of $q$, in this case, is infinite.
If $|a|>b$, then for all $n>1$, we have $|a^n/b^n|>a/b>1$, show that
the order of $q$, also in this case, is infinite.

(Check this one.)

\section*{Problem 4}

Prove that in any group, an element and its inverse have the same order.

Let $G$ be a group and let $a$ be an element of $G$.  Now notice that
\begin{equation*}
(a^{-1})^{|a|} = (a^{|a|})^{-1} = e^{-1} = e.
\end{equation*}

\section*{Problem 5}

Without actually computing the orders, explain why the two elements in each
of the following pairs of elements from $Z_{30}$ must have the same
order: $\{2,28\}$, $\{8,22\}$.  Do the same for the following pairs of
elements from $U(15)$: $\{2,8\}$, $\{7,13\}$.

In the first two cases, notice that the pairs of numbers are congruent modulo 30.
Therefore, their additive powers are congruent modulo 30, so they have the same order.

In the second two cases, notice that the pairs of numbers are inverses of
one another.  So by Problem 4 above, they must have equal order.

\section*{Problem 6}

Let $x$ belong to a group.  If $x^2\neq e$ and $x^6=e$, prove that
$x^4\neq e$ and $x^5\neq e$.  What can we say about the order of $x$?

Supposing $x^4=e$, we see that $e=x^6=x^4x^2=x^2$, which is a contradiction.
Therefore, $x^4\neq e$.  Supposing $x^5=e$, we see that $e=x^6=x^5x=x\implies x^2=e$,
which reaches the same contradiction.  Therefore, $x^5\neq e$.

It is clear that $x\neq e$.  Supposing $x^3=e$, we do no contradict any of the facts
uncovered so far.  So the order of $x$ is either 3 or 6.

(Check this one if possible.)

\section*{Problem 7}

Show that if $a$ is an element of a group $G$, then $|a|\leq|G|$.

Clearly this is true in the case that $a=e$. Supposing that $a\neq e$,
we must have $|a|>1$.

Now notice that $\{a^i\}_{i=1}^{|a|}$ is a set $|a|$ elements, since for all $1\leq i<j\leq |a|$,
we have $a^i\neq a^j$.  To verify this claim, suppose $a^i=a^j$.
Then $a^{j-i}=e$, but $j-i<|a|$, which is a contradiction.

We can now say that if $|a|>|G|$, then $G$ is a proper subset of $\langle a\rangle$, but this violates the closure of
the product under $G$.

\section*{Problem 8}

Show that $U(14)=\langle 3\rangle=\langle 5\rangle$.  [Hence, $U(15)$ is cyclic.]
Is $U(14)=\langle 11\rangle$?

After manually verifying that $U(14)=\langle 3\rangle$, we can easily
show that $\langle 5\rangle=\langle 3\rangle$ by noting that $5^i\equiv 3^{5i}\pmod{14}$
and that $5$ generates the additive cyclic group $Z_6$, since $1=\gcd(5,6)$.

(How was it proven again that $k\in Z_n$ generates $Z_n$ if $1=\gcd(k,n)$?)

\section*{Problem 9}

Show that $U(20)\neq\langle k\rangle$ for any $k$ in $U(20)$.  [Hence, $U(20)$ is not cyclic.]

By inspection, there is no element of order $|U(20)|$.  I'm sure there's a theorem that would
make it easier to come to this conclusion.

\section*{Problem 10}

Prove that an Abelian group with two elements of order 2 must have a subgroup of order 4.

Let $a$ and $b$ be two elements of this group having order 2.
Clearly the elements in the set $\{e,a,b,ab\}$ are in the group.
We now show that it is a sub-group.  Closure in this set is trivial for
all but the following cases.  Note that $a^2=b^2=e$ is in the set.
Note that $ba=ab$ is in the set, as well as $aab=b$, $aba=b$,
$bab=a$ and $abb=a$, and finally, $abab=aabb=e$ is in the set.
Now notice that $e^{-1}=e$, $a^{-1}=a$, $b^{-1}=b$ and $(ab)^{-1}=b^{-1}a^{-1}=ab$.
It follows that $\{e,a,b,ab\}$ is a subgroup by Theorem 3.2.

\section*{Problem 13}

For each divisor $k>1$ of $n$, let $U_k(n)=\{x\in U(n)|x\equiv 1\pmod k\}$.
Prove that $U_k(n)$ is a subgroup of $U(n)$.

It is clear that $1\in U_k(n)$, so it is not empty; and it is clear that $U_k(n)$ is a subset of $U(n)$.
Closure is obvious.  By Theorem 3.2, what then remains to be shown, is that $a\in U_k(n)$
implies that $a^{-1}\in U_k(n)$.  Do that here...

\end{document}