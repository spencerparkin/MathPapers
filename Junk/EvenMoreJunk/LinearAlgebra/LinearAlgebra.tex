\documentclass[12pt]{article}

\usepackage{amsmath}
\usepackage{amssymb}
\usepackage{amsthm}

\title{Linear Algebra}
\author{Spencer T. Parkin}

\newcommand{\G}{\mathbb{G}}
\newcommand{\V}{\mathbb{V}}
\newcommand{\R}{\mathbb{R}}
\newcommand{\A}{\mathbb{A}}
\newcommand{\B}{\mathbb{B}}

\newtheorem{theorem}{Theorem}[section]
\newtheorem{definition}{Definition}[section]
\newtheorem{corollary}{Corollary}[section]
\newtheorem{identity}{Identity}[section]
\newtheorem{lemma}{Lemma}[section]
\newtheorem{result}{Result}[section]

\begin{document}
\maketitle

Linear algebra is the study of linear functions defined on linear spaces.
Linear spaces are more often refered to as vector spaces, suggesting
to the mind a geometric interpretation of the elements of such spaces.
Indeed, what we'll find here is that the study of linear algebra is facilitated
by the use of geometric algebra.

Talk about linear spaces and linear independence here.
% at most m vectors can be found that are linearly indep of an m-dim vec space

We now come to a formal definition of a linear function.
\begin{definition}\label{def_linear_function}
Let $\A$ and $\B$ denote any two vector spaces, not necessariliy unique, arbitrarily defined
over the field of real numbers $\R$.  Then a
linear function $f$ is a mapping from $\A$ to $\B$ that preserves both scalar-vector multiplication
and vector addition.  That is, for any scalar $\lambda\in\R$ and any two vectors $x,y\in\A$,
we have $f(\lambda x)=\lambda f(x)$ and $f(x+y)=f(x)+f(y)$.
\end{definition}
To simplify the remainder of this paper, we assume $\dim\A=\dim\B$.  It will
be possible to show that $\dim f(\A)\leq\dim\B$, so this is okay.

Given definition $\eqref{def_linear_function}$, it is natural to begin our study of such
functions by seeing how much we can learn about
them without looking at any one particular example.  Recall that if $\A$ is
an $m$-dimensional vector space, then there exists $m$ linearly
independent vectors $\{a_i\}_{k=1}^m\subset\A$ that form a basis for $\A$.
Letting $x = \sum_{i=1}^m x_ia_i$, where each $x_i\in\R$, we see that
\begin{equation}\label{equ_linear_transform}
f(x) = \sum_{i=1}^m x_i f(a_i),
\end{equation}
showing that $f$ is entirely determined by how it transforms any set of basis
vectors for $\A$.  This has immediate implications for the existance of $f^{-1}$
when we recall the definition of a non-invertible function.
\begin{definition}
A function $f:\A\to\B$ is non-invertible if there exists an element $b\in f(\A)$
along with at least two unique elements $x,y\in\A$ such that $f(x)=b$ and $f(y)=b$.
\end{definition}
This definition illustrates the non-invertibility of $f$ by the ambiguity we face in the construction
of $f^{-1}$ when we consider how to map $f^{-1}(b)$.  Does this map to $x$ or $y$?

Consider now the set $\{f(a_i)\}_{i=1}^m\subset\B$ and suppose for the moment that
it is linearly dependent.  Then, without loss of generality, we can write $f(a_m)$
as $f(a_m)=\sum_{i=1}^{m-1}\lambda_i f(a_i)$, where each $\lambda_i\in\R$.
Now let $x_m=0$ and let $y\in\A$ be $y=\sum_{i=1}^m y_i a_i$, where $y_m\neq 0$,
and for all $i<m$, we have $y_i=x_i-\lambda_i y_m$.  Clearly $x\neq y$, and we see that
\begin{align*}
f(x) &= \sum_{i=1}^{m-1} x_if(a_i)
 = \sum_{i=1}^{m-1}(y_i+\lambda_i y_m)f(a_i) \\
 &= \sum_{i=1}^{m-1}y_i f(a_i) + y_m\sum_{i=1}^{m-1}\lambda_i f(a_i)
 = \sum_{i=1}^{m-1}y_i f(a_i) + y_m f(a_m)
 = f(y),
\end{align*}
showing that $f$ is non-invertible.

Considering now the set $\{f(a_i)\}_{i=1}^m\subset\B$ to be linearly independent, we
see that...

 It follows that $f$ is invertible if and only if $f$ preserves linear independence
in the sense that if $\{a_i\}_{i=1}^m$ is a linearly independent set, then so is $\{f(a_i)\}_{i=1}^n$.

Now knowing when $f^{-1}$ exists, how do we find $f^{-1}$?  Interestingly, geometric
algebra gives us a means to answering this question.  We begin with the definition
of an outermorphism.
\begin{definition}
A function $f:\A\to\B$ is an outermorphism if it is linear and perserves the outer product.  That is,
$f$ is a linear function, and for any two vectors $x,y\in\A$, we have $f(x\wedge y)=f(x)\wedge f(y)$.
\end{definition}
Every linear function can be extended to an outermorphism if it does not already possess the
defining characteristics of such a function.  (This may need some proof.)  As an extention of a linear function, an
outermorphism preserves the original linear function so that if we find the inverse of
an outermorphism, then we have also found the inverse of the original linear function.

Taking the zero element of any geometric algebra as being without grade, we see already
that an outermorphism $f$ is invertible if and only if it is grade perserving.  Indeed,
if $f$ is invertible, then it must map $I_{\A}$ to some non-zero scalar multiple of $I_{\B}$,
where each of these denote the unit psuedo-scalars of $\G(\A)$ and $\G(\B)$, respectively.
In fact, we define the determinant of $f$ as being this scalar multiple.  Doing so, we may write
\begin{equation*}
f(I_{\A}) = (\det f)I_{\B}.
\end{equation*}

\begin{definition}
The adjoint of an outermorphism $f$ is an outermorphism $\bar{f}$ with the property that
for any two vectors $x\in\A$ and $y\in\B$, we have $f(x)\cdot y= x\cdot \bar{f}(y)$.
\end{definition}
(How do we know that this exists?)  An explicit 

\begin{equation*}
\bar{f}(x) = \sum_{i=1}^m a_i(\bar{f}(x)\cdot a_i) = \sum_{i=1}^m a_i(x\cdot f(a_i))
\end{equation*}

Looking back at equation $\eqref{equ_linear_transform}$, let us rewrite it as follows.
\begin{equation*}
f\left(\sum_{i=1}^m x_i a_i\right) = \sum_{i=1}^m x_i f(a_i)
\end{equation*}
This shows that all linear functions are simply change-of-basis transformations.

\end{document}