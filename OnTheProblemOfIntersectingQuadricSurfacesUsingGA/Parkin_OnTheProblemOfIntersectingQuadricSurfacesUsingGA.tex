\documentclass{birkjour}

%\usepackage{tikz}
%\usepackage{graphicx}
\usepackage{hyperref}

\newtheorem{thm}{Theorem}[section]
\newtheorem{cor}[thm]{Corollary}
\newtheorem{lem}[thm]{Lemma}
\newtheorem{prop}[thm]{Proposition}
\theoremstyle{definition}
\newtheorem{defn}[thm]{Definition}
\theoremstyle{remark}
\newtheorem{rem}[thm]{Remark}
\newtheorem*{ex}{Example}
\numberwithin{equation}{section}

\newcommand{\R}{\mathbb{R}}
\newcommand{\B}{\mathbb{B}}
\newcommand{\G}{\mathbb{G}}
\newcommand{\V}{\mathbb{V}}
\newcommand{\gd}{\dot{g}}
\newcommand{\gh}{\hat{g}}
\newcommand{\Gd}{\dot{G}}
\newcommand{\Gh}{\hat{G}}
\newcommand{\nvai}{\infty}
\newcommand{\nvao}{o}
\newcommand{\grade}{\mbox{grade}}

\begin{document}

\title{On The Problem Of Intersecting\\Quadric Surfaces Using\\Geometric Algebra}

\author{Spencer T. Parkin}
\address{102 W. 500 S., \\
Salt Lake City, UT  84101} \email{spencerparkin@outlook.com}

%\subjclass{Primary 14J70; Secondary 14J29}

%\dedicatory{To my dear wife Melinda.}

\begin{abstract}
Abstract goes here...
\end{abstract}

\keywords{Keywords go here...}

\maketitle

\section{Introduction}

In light of the paper \cite{}, some encouragement has been given to the present
author to develop a model of geometry, similar to the conformal model of geometric
algebra, but not limited in representation to any proper subset of the set of all
quadric surfaces.  But for such a model to achieve adequate similarity to the conformal model,
it must preserve one of more of its most desirable features; preferably, all of them.
For example, in \cite{}, the set of all conformal transformations were preserved, but
intersections were not.\footnote{A way of representing intersections using the outer
product in the model of \cite{} can be found, but its usefulness, if any, is highly questionable.}

The goal of this paper, therefore, is to
preserve the intersection property.  In other words, we want to find a model of geometry based upon
geometric algebra giving us all quadric surfaces and the ability to intersect them
as effortlessly as can be done in the conformal model.  If nothing else, the attempt
to do so in this paper will shed light on the feasibility of such an endeavor, and thereby
bring us closer to answering the question of whether it can even be done.

\section{The Intersection Property}

Let us begin by taking a closer look at exactly what the intersection property is.
Upon initial inspection, one might suppose that this property is nothing more than
the ability to represent the intersection of any two given geometries in a way consistent
with the representation of any geometry of the model, but this is not
enough.  Such a representation has no usefulness if it does not submit to an
analysis yielding the geometric characteristics of the intersection.

That having been said, we can say now that the outer product's ability to intersect geometries represented
by blades in the conformal model is really not at all interesting.  What is interesting
is the realization that we can equate one characterization of an intersection
with another, and this is the key to finding intersections in the conformal model.
The reason for this is that while one such characterization is composed as the
intersection we wish to take, the other characterization lends itself to
analysis through decomposition.

For example, suppose we wish to take the planar intersection of a conical surface.
If we know that the resulting conic section is an ellipse, then we can choose to
interpret this intersection as that of a plane and an elliptical cylinder meeting
the plane at right angles.  This latter characterization will have an easily found decomposition
yielding all features of the ellipse.\footnote{Geometries having easily decomposable
representations in our model will be referred to as canonical forms.}
Having found all such features, we can then say that we've
fully realized the given section, whereas before this we were only able to
represent it.\footnote{Note that although non-planar intersections are as easily
represented in our model of geometry as any other type of intersection,
the technique of finding intersections in this paper
may not be helpful in finding non-planar intersections for the simple reason that such intersections
have no obvious canonical form.}

The quest to find our model of geometry, (the one promised in the introductory section of
this paper), being quite difficult, the bringing of the example just given in the preceding paragraph
to fruition will become the impetus for all choices we henceforth make in finding the model.
Even if our model can do nothing more than this one example, we will consider or goal achieved.
Rest assured, however, that along the way, we will find results generally applicable to the problem at hand.

\section{Enter The Model}

So that no further delay be made, we will now let the remainder of this paper begin
exactly where the first section of \cite{} ended, assuming all results and definitions up to that point.
That said, we now introduce the function $p:\R^n\to\V$ as
\begin{align*}
p(x) &= e_0 + x \\
 &+ (x\cdot e_2)(x\cdot e_3)e_4 + (x\cdot e_1)(x\cdot e_3)e_5 + (x\cdot e_1)(x\cdot e_2)e_6 \\
 &+ (x\cdot e_1)^2e_7 + (x\cdot e_2)^2e_8 + (x\cdot e_3)^2 e_9,
\end{align*}
the vectors in $\{e_i\}_{i=0}^9$ forming an orthonormal basis for a 10-dimensional
euclidean vector space.\footnote{The reader should take care to note which results of this paper
depend upon this definition of the function $p$ and which do not.  Also note that signatures
other than the euclidean may be worth considering, but there will be no foreseeable need in this paper.}
This is sufficient to define the entire model as it is clear that for any quadric
surface, or any intersection of two or more quadric surfaces, there exists a blade
$B\in\G$ representative of this surface as $\gd(B)$.

Continuing with our example, let us now find the canonical form for an ellipse
in a plane.  Using the notation\footnote{This notation is overloaded.  When writing $x_i$, this may mean
the $i^{th}$ component of $x$, or it may mean the $i^{th}$ point in a sequence of points.  The intended
meaning will always be clear from context.} $x_i = x\cdot e_i$, an equation for such an ellipse
may be given by
\begin{equation}\label{equ_ellipse}
\frac{(x_1-h)^2}{a^2} + \frac{(x_2-k)^2}{b^2} - 1 = 0,
\end{equation}
provided $x_3=0$, which is simply an equation for the plane.
Factoring $p(x)$ out of the equation $x_3=0$, we get $p(x)\cdot e_3=0$,
and out of equation \eqref{equ_ellipse}, we get $p(x)\cdot E=0$, where the
vector $E$ is given by
\begin{equation*}
E = \left(\frac{h^2}{a^2} + \frac{k^2}{b^2} - 1\right)e_0 -
  2\frac{h}{a^2}e_1 - 2\frac{k}{b^2}e_2 + \frac{1}{a^2}e_7 + \frac{1}{b^2}e_8.
\end{equation*}
By itself, $E$ here represents an elliptical cylinder.  The ellipse is given by
the set of points $\gd(E\wedge e_3)$.  The 2-blade $E\wedge e_3$ will be the
canonical form of the ellipse that we will use to find a conic intersection that
is an ellipse.

To see that $B=E\wedge e_3$ is easily decomposable, we given the following set of
equations.
\begin{align*}
a^2 &= \frac{1}{e_3\wedge e_7\cdot B} &
h &= -\frac{a^2}{2}e_3\wedge e_1\cdot B \\
b^2 &= \frac{1}{e_3\wedge e_8\cdot B} &
k &= -\frac{b^2}{2}e_3\wedge e_2\cdot B
\end{align*}
Using these equations we can recover the ellipse.

Now let us formulate the intersection we wish to take.  We will
intersect the $x_3=0$ plane with the conical surface having points satisfying the equation
\begin{equation*}
(x_1^2+(x_2+1)^2)\cos^2\theta - x_3^2\sin^2\theta = 0,
\end{equation*}
where $2\theta$ is the angle of aperture.\footnote{Yes, this will give
us a circle in the $x_3=0$ plane and not a general ellipse, but to keep things
simpler, our primary example will consider a degenerate case of the ellipse,
and we will see that it suffices for this paper's purposes.}  We have submerged it
below the $x_3=0$ plane to get a non-trivial intersection $\gd(C\wedge e_3)$,
the vector $C$ being given by
\begin{equation*}
C = \cos^2\theta e_0 + 2\cos^2\theta e_2 + \cos^2\theta e_7 + \cos^2\theta e_8 - \sin^2\theta e_9.
\end{equation*}

\section{Making Use Of The Model}

Being now able to represent all quadric intersections, the task of
algebraically relating them remains.  For example, knowing that
$C\wedge e_3$ is an ellipse, how might we decompose it
as we would $E\wedge e_3$?  The following lemma may be able to help.
\begin{lem}\label{lma_point_factored_blade}
Letting $B\in\G$ be a blade of grade $k$, if there exist $k$ points $\{x_i\}_{i=1}^k\subset\R^n$
such that $\bigwedge_{i=1}^k p(x_i)\neq 0$, and that for all points
$x\in\{x_i\}_{i=1}^k$, we have $x\in\gh(B)$, then there exists a scalar $\lambda\in\R$
such that
\begin{equation}\label{equ_point_factored_blade}
B = \lambda\bigwedge_{i=1}^k p(x_i).
\end{equation}
\end{lem}
\begin{proof}
If $x_i\in\gh(B)$, then $p(x_i)\wedge B=0$, showing that $p(x_i)$ is in the vector
space spanned by any factorization of $B$.  Then, since $\bigwedge_{i=1}^k p(x_i)\neq 0$,
the set of vectors $\{p(x_i)\}_{i=1}^k$ is linearly independent and therefore
a basis for such a vector space.  The blades $B$ and $\bigwedge_{i=1}^k p(x_i)$
must, therefore, be equal, up to scale.
\end{proof}
For a blade $B\in\G$ having a factorization \eqref{equ_point_factored_blade},
we will refer to $B$ as an irreducible blade for reasons that will become clear shortly.

The usefulness of Lemma~\ref{lma_point_factored_blade} is realized in our next lemma.
\begin{lem}\label{lma_equating_blades}
If $A,B\in\G$ are blades of grade $k$ with $\gh(A)=\gh(B)$, and one of these is irreducible,
then there exists a scalar $\lambda\in\R$ such that $A=\lambda B$.
\end{lem}
\begin{proof}
We first establish that if one of the blades $A$ and $B$ is irreducible, then so is the other.
Assuming, without loss of generality, that $A$ is irreducible, let $\{x_i\}_{i=1}^k\subset\R^n$
be a set of $k$ points, and $\alpha\in\R$ be a scalar, such that $A=\alpha\bigwedge_{i=1}^k p(x_i)$.
Now since $\gh(A)=\gh(B)$, it is clear that for all $x\in\{x_i\}_{i=1}^k$, we have $x\in\gh(B)$,
and so it follows by Lemma~\ref{lma_point_factored_blade} that there exists a scalar $\beta\in\R$
such that $B=\beta\bigwedge_{i=1}^k p(x_i)$.

Lastly, we simply realize that if we let $\lambda=\frac{\alpha}{\beta}$, we have
$A=\lambda B$.
\end{proof}

In light of Lemma~\ref{lma_equating_blades}, the following question naturally arises.
Knowing that $\gd(E\wedge e_3)=\gd(C\wedge e_3)$, is any one of
$(E\wedge e_3)I$ and $(C\wedge e_3)I$ irreducible?\footnote{Here we're letting $I$
be the unit psuedo-scalar of our 10-dimensional geometric algebra $\G$.}  If so, we have found
a way to algebraically relate them so that an analysis by decomposition, up to scale, of $C\wedge e_3$
as $E\wedge e_3$ can move forward.  Unfortunately, it is not hard to show
that neither of these is irreducible.  To see why, consider the following equation which
expresses the form of $p(x)$ for all points in the $x_3=0$ plane.
\begin{equation*}
p(x_1e_1 + x_2e_2) = e_0 + x_1e_1 + x_2e_2 + x_1x_2e_6 + x_1^2e_7 + x_2^2e_8
\end{equation*}
Now notice that an upper-bound on the dimension of a vector space that can
be spanned by vectors of this form is clearly 6 as there are only 6 components.
But each of $E\wedge e_3$ and $C\wedge e_2$ are 2-blades, making their
duals blades of grade $10-2=8>6$.  It follows that there is no set of 8 points $\{x_i\}_{i=1}^8$
on the ellipse such that $\bigwedge_{i=1}^8 p(x_i)\neq 0$.

Not willing to give up just yet, we arrive at the following lemma.

\begin{lem}
For every $k$-blade $B\in\G$, there exists a blade $B'\in\G$ of grade $k'\leq k$
such that $\gh(B)=\gh(B')$ and $B'$ is irreducible.
\end{lem}
\begin{proof}
If $B$ is irreducible, then let $k'=k$ and $B'=B$ and we're done.
If $B$ is not irreducible, then let $j$ be the largest possible integer
for which there exists a set of $j$ points $\{x_i\}_{i=1}^j\subseteq\gh(B)$
with $\bigwedge_{i=1}^j p(x_i)\neq 0$, (clearly $j<k$), and write
\begin{equation*}
B = B_0\wedge\bigwedge_{i=1}^j p(x_i)
\end{equation*}
for some blade $B_0$ of grade $k-j$.  Now realize that for any $x\in\gh(B)$,
if $x\in\gh(B_0)$, then $x\not\in\{x_i\}_{i=1}^j$ and $p(x)\wedge\bigwedge_{i=1}^j p(x_i)\neq 0$,
which is a contradiction.  Therefore, if $x\in\gh(B)$, then, letting $k'=j$ and $B'=\bigwedge_{i=1}^jp(x_i)$,
we have $x\in\gh(B')$.  Conversely, if $x\in\gh(B')$, then clearly $x\in\gh(B)$.  It follows that
$\gh(B)=\gh(B')$.
\end{proof}

Seeing that $B'$ is potentially a reduction in grade of the blade $B$, but that it certainly does
not sacrifice the geometry represented by $B$, we will say that $B$ is reducible
in the case that $k'<k$.  In the case that $k'=k$, it is clear that $B$ is irreducible.

What we see now is that $(C\wedge e_3)I$ and $(E\wedge e_3)I$ are reducible blades,
and that equating them is possible if they can both be reduced.  Unfortunately,
an irreducible canonical form of our ellipse is not obvious, nor is a solution to the problem
of either reducing a given blade, or showing that it is already irreducible.  Each of these
problems may be about as hard  as finding, not just any, but a specific type of factorization
of the blade in question, and there may be no
way of getting around that.  That is, if we're bent on equating one blade with another.

\section{Taking a Different Approach to the Problem}

Fortunately, what may be an alternative to our original plan is that of using
$C\wedge e_3$ and $E\wedge e_3$ to generate a system of non-linear
equations in a different way.  The idea is simple.  If we know that we have a point $x\in\gd(E\wedge e_3)$, (our
canonical form), then we also have $x\in\gd(C\wedge e_3)$, (our desired intersection), and this generates
for us one or more equations in the components $x_i$ of $x$.  For example,
if we let $x=(h+a)e_1 + ke_2$, then $p(x)\cdot E\wedge e_3$ is clearly
zero, and therefore, so must be $p(x)\cdot C\wedge e_3$.
This generates one equation for us, and it is $0=(k+1)^2+(h+a)^2$.

A lemma is now in order.  For the following lemma, the asterisk symbol ``$*$'' may be
replaced by either the inner product symbol ``$\cdot$'' or the outer product symbol ``$\wedge$.''

\begin{lem}\label{lma_determine_system}
Given a blade $B\in\G$, if $\{x_i\}_{i=1}^j\subset g^*(B)$ is a
set of $j$ points with $\bigwedge_{i=1}^{j-1} p(x_i)\neq 0$ and $\bigwedge_{i=1}^j p(x_i)=0$ such
that the system of equations generated by, for all integers $i\in[1,j-1]$, $0=p(x_i)* B$,
is an under-determined system, then the system of equations generated by,
for all integers $i\in[1,j]$, $0=p(x_i)* B$, is not any more determined.\footnote{This is
also to say that the system of equations generated by all $j$ points is essentially the
same as that generated by all $j-1$ points.}
\end{lem}
\begin{proof}
For an appropriate set of $j-1$ scalars $\{\lambda_i\}_{i=1}^{j-1}\subset\R$, letting $p(x_j)=\sum_{i=1}^{j-1}\lambda_i p(x_i)$,
the equation $0=p(x_j)* B$ contributes to the system all equations generated by
\begin{equation*}
0 = \sum_{i=1}^{j-1}\lambda_ip(x_i)* B,
\end{equation*}
which is not any new information.  (Generating new equations by adding scalar multiples of existing equations
together does not make a system of equations any more determined.)
\end{proof}
What Lemma~\ref{lma_determine_system} is telling us is that if we want any hope of generating a sufficiently determined
system, then we must only consider sets of points $\{x_i\}_{i=1}^j\subset\gd(E\wedge e_3)$
for which $\bigwedge_{i=1}^j p(x_i)\neq 0$.  Realizing this, it is worth taking a moment to
consider the circumstances under which such a set of $j$ points produces a linearly
independent set of vectors $\{p(x_i)\}_{i=1}^j$.

Doing so, it is immediately clear that if $\{x_i\}_{i=1}^j$ is a linearly independent
set of points in $\R^n$, then so is the set $\{p(x_i)\}_{i=1}^j$, but we can do
a little better than this with the following lemma.
\begin{lem}\label{lma_non_co_planar}
If a given set of $j>2$ points $\{x_i\}_{i=1}^j$ are non-co-planar for a plane of
dimension $j-2$, then the set of vectors $\{p(x_i)\}_{i=1}^j$ is linearly independent.
\end{lem}
\begin{proof}
Proving the contrapositive of the lemma, let $\{\lambda_i\}_{i=1}^j$ be
a set of scalars in $\R$, not all zero, such that $0=\sum_{i=1}^j\lambda_i p(x_i)$.
It follows that $0=\sum_{i=1}^j\lambda_i(e_0+x_i)$ and therefore
$0=\sum_{i=1}^j\lambda_i$ and $0=\sum_{i=1}^j\lambda_i x_i$.
Now realize that if there exists an integer $a\in[1,j]$ such that $\lambda_a\neq 0$,
then there must exist an integer $b\in[1,j]-\{a\}$ such that $\lambda_b\neq 0$.
Without loss of generality, let $a=j$ so that $1\leq b\leq j-1$, and write
\begin{equation*}
0 = \sum_{i=1}^j\lambda_ix_i = \sum_{i=1}^{j-1}\lambda_ix_i - \left(\sum_{i=1}^{j-1}\lambda_i\right)x_j = -\sum_{i=1}^{j-1}\lambda_i(x_j-x_i),
\end{equation*}
which shows that the set of vectors $\{x_j-x_i\}_{i=1}^{j-1}$ is linearly dependent.
It now follows that the $(j-1)$-dimensional simplex determined by the points in $\{x_i\}_{i=1}^j$
has no $(j-1)$-dimensional hyper-volume.  That is,
\begin{equation*}
0 = \frac{1}{(j-1)!}\bigwedge_{i=1}^{j-1}(x_j-x_i).
\end{equation*}
But this can only be if the $j$ points are co-planar for a hyper-plane of dimension $j-2$.
\end{proof}

Lemma~\ref{lma_non_co_planar} is a good start, but there are certainly more conditions on $\{x_i\}_{i=1}^j$ to be found upon which
$\bigwedge_{i=1}^j p(x_i)\neq 0$.  The non-linearity of our function $p$ makes these conditions difficult to find, to say the least.

Returning to our example, ...

% Can we get a lemma better than lemma~\ref{lma_determine_system}?  I'd
% like to know if the system i finally get is determined.

The approach might not work, because of
\begin{align*}
0 &= p(x_1e_1+x_2e_2)\cdot C\wedge e_3 \\
 &= (p(x_1e_1+x_2e_2)\cdot C)e_3 \\
 &= \cos^2\theta(1+2x_2+x_1^2+x_2^2)e_3.
\end{align*}
No system can be determined, because it never takes the aperture into account.
It can be shown that the outer product generates the same equations.

Does this also show that no approach will ever work?

Next step: ray trace the conical surface to be sure you have the equation right.

Okay, okay, the equation for the conical surface has to be wrong, because, forgetting
all crap in this paper, its solution set in the $xy$-plane does not take into account
the aperture.  Check the equation.

\end{document}
