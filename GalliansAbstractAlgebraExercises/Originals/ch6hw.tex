\documentclass{article}
\usepackage{amsmath}
\usepackage{amssymb}
\title{Chapter 6 Homework}
\author{Spencer}
\addtolength{\oddsidemargin}{-.575in}
\addtolength{\evensidemargin}{-.575in}
\addtolength{\textwidth}{1.0in}
\addtolength{\topmargin}{-.575in}
\addtolength{\textheight}{1.25in}
\begin{document}
\maketitle

\newcommand{\aut}{\mbox{Aut}}

\section*{Problem 2}

Find $\aut(\mathbb{Z})$.

Suppose the function $\phi:\mathbb{Z}\to\mathbb{Z}$ is an automorphism
of $\mathbb{Z}$ under addition.  Let $a\in\mathbb{Z}$, then by the operation
preserving property of $\phi$, and the method used in Example 5,
we see that $\phi(a) = a\phi(1)$.  So finding all automorphisms of $\mathbb{Z}$
is a question of finding all ways of defining $\phi(1)$ such that $\phi$ is
an automorphism of $\mathbb{Z}$.  Suppose $\phi(1)=0$.  Then $\phi(a)=0$ and
this doesn't work because $\phi$ does not map $\mathbb{Z}$ onto $\mathbb{Z}$.
Suppose $|\phi(1)|>1$.  Then, again, we see that $\phi$ does not map $\mathbb{Z}$
onto $\mathbb{Z}$.  Now suppose that $\phi(1)=1$.  This obviously works since
then $\phi$ becomes the identity function.  Suppose $\phi(1)=-1$.  This makes
$\phi$ one-to-one and onto.  Now
notice that $\phi(a+b) = -(a+b) = (-a) + (-b) = \phi(a)+\phi(b)$.  So $\phi$
preserves the addition operation.  Define $\phi_k(x)=kx$.
Then $\aut(\mathbb{Z})=\{\phi_1,\phi_{-1}\}$.

\section*{Problem 3}

Let $\mathbb{R}^{+}$ be the group of positive real numbers under multiplication.
Show that the mapping $\phi(x)=\sqrt{x}$ is an automorphism of $\mathbb{R}^{+}$.

First note that the domain and codomain of $\phi$ are $\mathbb{R}^{+}$.
Now let $a,b\in\mathbb{R}^{+}$ and assume $\phi(a)=\phi(b)$.
This implies that $\sqrt{a}=\sqrt{b}\implies|a|=|b|\implies a=b$, since
we're confined to the positive reals.  So $\phi$ is one-to-one.
Now choose $a^2\in\mathbb{R}^{+}$ and we see that $\phi(a^2)=\sqrt{a^2}=|a|=a$
so that $\phi$ is onto $\mathbb{R}^{+}$.  What remains to be shown is
operation preservance.
\begin{equation*}
\phi(ab) = \sqrt{ab} = \sqrt{a}\sqrt{b} = \phi(a)\phi(b)
\end{equation*}

\section*{Problem 4}

Show that $U(8)$ is not isomorphic to $U(10)$.

Notice that the square of every element in $U(8)$ is the identity, but that
this is not the case with $U(10)$.  If $\phi$ was an isomorphism from $U(8)$
to $U(10)$, then for all $x\in U(8)$, we must have...
\begin{equation*}
\phi(e_{U(8)}) = \phi(x^2) = \phi(x)^2 = e_{U(10)}
\end{equation*}
But this clearly doesn't happen in the case that $\phi(x)=3$ since $3^2=9$ in $U(10)$.

\section*{Problem 10}

Let $G$ be a group.  Prove that the mapping $\alpha(g)=g^{-1}$ for all $g\in G$
is an automorphism if and only if $G$ is Abelian.

Suppose $\alpha$ is an automorphism of $G$, and let $a,b\in G$.  Notice that...
\begin{equation*}
\alpha(ab) = (ab)^{-1} = b^{-1}a^{-1} = \alpha(b)\alpha(a) = \alpha(ba)
\end{equation*}
...where the last equality holds by the operation preserving property of $\alpha$.
Then since $\alpha$ is one-to-one, we see that $\alpha(ab)=\alpha(ba)\implies ab=ba$,
showing that $G$ is Abelian.

Now suppose that $G$ is Abelian.  Notice that $\alpha$, as defined above, has
the correct domain and codomain for it to be an automorphism of $G$.
Then $\alpha$ must be one-to-one since no two distinct group elements have
the same inverse.  We then see that $\alpha$ is onto because inverses
come in pairs of mutual inverses.  So given any element $x$ in the codomain of $\alpha$,
we see that $\alpha(x^{-1})=(x^{-1})^{-1}=x$.  Let $a,b\in G$.  Then we see that...
\begin{equation*}
\alpha(ab) = (ab)^{-1} = b^{-1}a^{-1} = a^{-1}b^{-1} = \alpha(a)\alpha(b)
\end{equation*}
...where here we made use of the Abelian property of $G$.  So $\alpha$ is operation
preserving as well, and so $\alpha$ must be an isomorphism from $G$ to $G$, which
is what we wanted.

\section*{Problem 14}

Find $\aut(\mathbb{Z}_6)$.

Using Example 13 and Theorem 6.5, define $\phi_k(x)=kx$ and we see that...
\begin{equation*}
\aut(\mathbb{Z}_6) = \{\phi_k:k\in U(6)\} = \{\phi_1,\phi_5\}
\end{equation*}

\section*{Problem 22}

Prove or disprove that $U(20)$ and $U(24)$ are isomorphic.

In $U(24)$, the order of every non-identity element is 2.  This is not the
case with $U(20)$.  Since an isomorphism preserves the order of elements,
there cannot exist one between $U(20)$ and $U(24)$.

\section*{Problem 23}

Show that the mapping $\phi(a+bi)=a-bi$ is an automorphism of the group
of complex numbers under addition.  Show that $\phi$ preserves complex
multiplication as well.

We see that the domain and codomain of $\phi$ is $\mathbb{C}$.
Let $x,y\in\mathbb{C}$ where $\phi(x)=\phi(y)$, $x=a+bi$, and $y=c+di$.
Then the conjugate of $x$ is the conjugate of $y$.  This implies that
the real and imaginary parts of the conjugates are the same so that
$a=c$ and $-b=-d\implies b=d$.  But then this implies that the real and
imaginary parts of $x$ and $y$ are the same so that $x=y$.  So $\phi$ is
one-to-one.  Using these $x$ and $y$ again, we see that...
\begin{equation*}
\phi(x+y) = (a+c)-(b+d)i = (a-bi)+(c-di) = \phi(x)+\phi(y)
\end{equation*}
...so that $\phi$ is operation preserving.
Now choose $a+bi$ in the codomain of $\phi$.  Choose $x=a-bi$ and we see
that $\phi(x) = a-(-b)i = a+bi$.  So $\phi$ is onto $\mathbb{C}$.
We've now shown that $\phi$ is an automorphism of $\mathbb{C}$ under addition.

To show that $\phi$ preserves multiplication, we want to show that
the operations of multiplication and the taking of conjugates can be
done in either order.  Let $a=re^{ui}$ and $b=Re^{vi}$.  Then...
\begin{equation*}
\bar{ab} = rRe^{-(u+v)i} = re^{-ui}Re^{-vi} = \bar{a}\bar{b}
\end{equation*}

\section*{Problem 25}

Prove that $\mathbb{Z}$ under addition is not isomorphic to $\mathbb{Q}$ under addition.

Assume that $\phi$ is an isomorphism from $\mathbb{Z}$ to $\mathbb{Q}$.  Let $a\in\mathbb{Z}$.
Then $\phi(a) = a\phi(1)$, and we see that every isomorphism from the integers to the rationals
must be of the form $\phi(x) = qx$, where $q\in\mathbb{Q}$.  But this function can never
be onto the rationals.  For example, there does not exist $a\in\mathbb{Z}$ such that
$\phi(a)=q/2$.

\section*{Problem 28}

Let $\mathbb{R}^n=\{(a_1,a_2,\dots,a_n):a_i\in\mathbb{R}\}$.  Show that the mapping
$\phi:(a_1,a_2,\dots,a_n)\to(-a_1,-a_2,\dots,-a_n)$ is an automorphism of the group
$\mathbb{R}^n$ under componentwise addition.  Describe the action of $\phi$ geometrically.

Geometrically, I think that $\phi$ is a reflection about the origin of an $n$-dimensional
coordinate system.  It may also turn objects inside-out.

To show that $\phi$ is operation preserving, see that...
\begin{align*}
\phi((a_1,\dots,a_n)+(b_1,\dots,b_n))
 &= \phi((a_1+b_1,\dots,a_n+b_n)) \\
 &= (-a_1-b_1,\dots,-a_n-b_n) \\
 &= (-a_1,\dots,-a_n)+(-b_1,\dots,-b_n) \\
 &= \phi((a_1,\dots,a_n))+\phi((b_1,\dots,b_n))
\end{align*}

To show that $\phi$ is one-to-one, see that...
\begin{align*}
\phi((a_1,\dots,a_n))=\phi((b_1,\dots,b_n))
 &\implies (-a_1,\dots,-a_n) = (-b_1,\dots,-b_n) \\
 &\implies (a_1,\dots,a_n) = (b_1,\dots,b_n)
\end{align*}

To show that $\phi$ is onto, see that...
\begin{equation*}
\phi(x) = (a_1,\dots,a_n) \implies x = (-a_1,\dots,-a_n)
\end{equation*}

\section*{Problem 29}

Consider the following statement: The order of a subgroup divides the
order of the group.  Suppose you could prove this for finite permutation groups.
Would the statement then be true for all finite groups?

Yes.  The statement is a group theoretic property, and so is preserved by isomorphisms.
Every group can be viewed as a group of permutations because every group is isomorphic
to a group of permutations.  This is Caylay's Theorem.

\section*{Problem 30}

Suppose that $G$ is a finite Abelian group and $G$ has no element of order
2.  Show that the mapping $g\to g^2$ is an automorphism of $G$.  Show, by
example, that if $G$ is infinite, the mapping need not be an automorphism.

Let $a,b\in G$ and suppose $a^2=b^2$.  Then $a^2(b^{-1})^2 = (ab^{-1})^2=e$, by
the Abelian property of $G$.
Since there are no elements of order 2, this implies that $ab^{-1}=e\implies a=b$.
So the mapping is one-to-one.
Now since we have a one-to-one mapping from a finite set to itself, that mapping is
also onto.  (This was proved in a previous exersize.)
The mapping is clearly operation preserving.
\begin{equation*}
a^2b^2 = aabb = abab = (ab)^2
\end{equation*}
This holds by the Abelian property of $G$.

Suppose that $G=\mathbb{Z}$ under addition.  It's Abelian and has no element
of order 2, yet the mapping $g\to g^2$ is not onto $\mathbb{Z}$.

\section*{Problem 32}

Show that the mapping $a\to\log_{10}a$ is an isomorphism from $\mathbb{R}^{+}$ under
multiplication to $\mathbb{R}$ under addition.

Let $a,b\in\mathbb{R}^{+}$.  Then...
\begin{equation*}
\log_{10}a = \log_{10}b\implies 10^{\log_{10}a}=10^{\log_{10}b}\implies a=b
\end{equation*}
This shows that the mapping is one-to-one.  Then...
\begin{equation*}
\log_{10}(ab) = \log_{10}a + \log_{10}a
\end{equation*}
This property of logarithms shows that the mapping is operation preserving.  Now
choose $a=10^b$.  Then...
\begin{equation*}
\log_{10}a = \log_{10}10^b = b\log_{10}10 = b
\end{equation*}
This shows that the mapping is onto.  We've now shown everything we need to show
so that the mapping is an isomorphism from $\mathbb{R}^{+}$ to $\mathbb{R}$.

\section*{Problem 36}

Let $G=\{0,\pm 2, \pm 4, \pm 6, \dots\}$ and $H=\{0,\pm 3,\pm 6,\pm 9, \dots\}$.
Show that $G$ and $H$ are isomorphic groups under addition.  Does your
isomorphism preserve multiplication?

Define the notation $G_k$, where $k\in\mathbb{Z}$, to be the
group $G_k=\{0,\pm k, \pm 2k, \pm 3k, \dots\}$ under addition.
Then $G=G_2$ and $H=G_3$.  It now suffices to show that $\mathbb{Z}$ is isomorphic
to $G_k$ since then we would have $G\approx\mathbb{Z}\approx H\implies G\approx H$.
Consider the mapping $\phi_k:\mathbb{Z}\to G_k$ defined by $\phi_k(z)=kz$.
Let $a,b\in G_k$.  Assume $\phi_k(a)=\phi_k(b)$.  This implies that $ka=kb\implies a=b$
so that $\phi_k$ is one-to-one.  Now choose $a=b/k\in\mathbb{Z}$ and we see
that $\phi_k(a)=k(b/k) = b$ so that $\phi_k$ is onto $G_k$.
We now see that $\phi_k$ preserves addition since...
\begin{equation*}
\phi_k(a+b)=k(a+b) = ka+kb = \phi_k(a)+\phi_k(b)
\end{equation*}
An isomorphism from $G_2$ to $G_3$ would be given by $\phi_3\phi_2^{-1}(z)=(3/2)z$.
This mapping does not preserve multiplication since...
\begin{equation*}
\frac{3}{2}ab \neq \left(\frac{3}{2}\right)^2ab
\end{equation*}

\section*{Problem 40}

Show that every automorphism $\phi$ of the rational numbers $\mathbb{Q}$ under
addition to itself has the form $\phi(x)=x\phi(1)$.

Let $a/b\in\mathbb{Q}$ where $a,b\in\mathbb{Z}$.  Then...
\begin{equation*}
\phi(a/b) = \phi\underbrace{\left(\frac{1}{b}+\dots+\frac{1}{b}\right)}_a = a\phi(1/b)
\end{equation*}
Now notice that...
\begin{equation*}
\phi(1) = \phi\underbrace{\left(\frac{1}{b}+\dots+\frac{1}{b}\right)}_b = b\phi(1/b)\implies
\phi(1/b) = (1/b)\phi(1)
\end{equation*}
Putting it all together we get...
\begin{equation*}
\phi(a/b) = a\phi(1/b) = (a/b)\phi(1)
\end{equation*}

\end{document}