\documentclass[12pt]{article}

\usepackage{amsmath}
\usepackage{amssymb}
\usepackage{amsthm}

\title{Chapter 5 Exercises\\Gallian's Book on Abstract Algebra}
\author{Spencer T. Parkin}

\newtheorem{theorem}{Theorem}[section]
\newtheorem{definition}{Definition}[section]
\newtheorem{corollary}{Corollary}[section]
\newtheorem{identity}{Identity}[section]
\newtheorem{lemma}{Lemma}[section]
\newtheorem{result}{Result}[section]

%\newcommand{\gcd}{\mbox{gcd}}
\newcommand{\lcm}{\mbox{lcm}}
\newcommand{\abs}{\mbox{abs}}
\newcommand{\Z}{\mathbb{Z}}
\newcommand{\R}{\mathbb{R}}
\newcommand{\stab}{\mbox{stab}}

\begin{document}
\maketitle

\section*{Problem 10}

Show that a function from a finite set $S$ to itself is one-to-one
if and only if it is onto.  Is this true when $S$ is infinite?

If we show that a function from a finite set $S_0$ to another
$S_1$ of the same cardinality is one-to-one if and only if
it is onto, then we have shown the desired result if
we let $S_0=S_1$.

That said, let $S_0$ be a set of $n$ pigeons, and $S_1$
be a set of $n$ pigeon holes.  If $f:S_0\to S_1$ is
not onto, then at least one pigeon hole must be in use
by more than one pigeon, and therefore, $f$ is not one-to-one.
Conversely, if $f:S_0\to S_1$ is not one-to-one, then at
least one pigeon hole is in use by more than one pigeon.
But this is impossible without leaving at least one pigeon hole
out of use, so $f$ is not onto.

\section*{Problem 12}

If $\alpha$ is even, prove that $\alpha^{-1}$ is even.
If $\alpha$ is odd, prove that $\alpha^{-1}$ is odd.

Write $\alpha$ as a product of disjoint cycles.
Notice that each $n$-cycle can be written as
a product of $n-1$ 2-cycles.  Now notice that
$\alpha^{-1}$ can be written as a product of
disjoint cycles by taking each $n$-cycles for $\alpha$
and reversing the winding order of the cycle to
produce that cycle's inverse, which is also an $n$-cycle.
It follows that $\alpha^{-1}$ can be written as the
same number of disjoint 2-cycles as can be done
for $\alpha$.

\section*{Problem 13}

Prove Theorem 5.6 -- The set of even permutations in $S_n$ forms
a subgroup of $S_n$.

Note that the identity permutation is even.  Note that the product
of any two even permutations is even.  Lastly, notice that
by Problem 12 above, the inverse of any even permutation
is also in the set of even permutations.

\section*{Problem 19}

Show that if $H$ is a subgroup of $S_n$, then either every member of $H$ is an
even permutation or exactly half of the members are even.

If all members of $H$ are even, we're done.  So assume that not all
members of $H$ are even.  Let $\beta\in H$ be odd and
consider the function $\phi(\alpha)=\alpha\beta$.  Notice that $\phi$
maps the set of all even permutation of $S_n$ into the set of all
odd permutations of $S_n$.  Being a well defined function, it follows
that there are at least as many even permutations as there are
odd permutations.  On the other hand, notice that $\phi$ maps
the set of all odd permutations of $S_n$ into the set of all even
permutations of $S_n$.  Being a well defined function, it follows
that there are at least as many odd permutations as there are
even permutations.

\section*{Problem 21}

Do the odd permutations of $S_n$ form a group?  Why?

No.  We don't have closure.  And the identity is not odd.

\section*{Problem 22}

Let $\alpha$ and $\beta$ belong to $S_n$.  Prove that $\alpha^{-1}\beta^{-1}\alpha\beta$ is an even permutation.

By Problem 12, $\alpha^{-1}$ has the same parity as $\alpha$.  The same
can be said of $\beta^{-1}$ and $\beta$.  Now simply realize that for all
4 cases of parity between $\alpha$ and $\beta$, each of the two parities
appears twice in the expression, so that any odd parity gets canceled,
letting the net parity always be even.

\section*{Problem 31}

Let $G$ be a group of permutations on a set $X$.  Let $a\in X$ and
define $\stab(a)=\{a\in G|\alpha(a)=a\}$.  We call $\stab(a)$ the {\it stabilizer
of $a$ in $G$} (since it consists of all members of $G$ that leave $a$ fixed).  Prove that
$\stab(a)$ is a subgroup of $G$.

It is clear that the identity permutation is in $\stab(a)$, since it leaves all elements
of $X$ invariant.  Let $\alpha\in\stab(a)$.  Then, since $\alpha(a)=a$, we have
$\alpha^{-1}(a)=a$, showing that $\alpha^{-1}\in\stab(a)$.  Now let $\alpha,\beta\in\stab(a)$.
Then, since $(\alpha\beta)(a)=\alpha(\beta(a))=\alpha(a)=a$, we have
$\alpha\beta\in\stab(a)$.

\section*{Problem 34}

Let $H=\{\beta\in S_5|\mbox{$\beta(1)=1$ and $\beta(3)=3$}\}$.  Prove that $H$ is a subgroup of
$S_5$.  Is your argument valid when 5 is replaced by any $n\geq 3$?

Let $i$ and $j$ be distinct positive integers and let
\begin{equation*}
H=\{\beta\in S_n|\mbox{$\beta(i)=i$ and $\beta(j)=j$}\},
\end{equation*}
where $n\geq\max(i,j)$.  Now notice that
\begin{equation*}
H = \stab(i)\cap\stab(j).
\end{equation*}
It is then clear that $H$ is a subgroup of $S_n$, because the intersection
of any two subgroups is a subgroup.

\end{document}