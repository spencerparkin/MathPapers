\documentclass{article}
\usepackage{amsmath}
\usepackage{amssymb}
\title{Chapter 32 Homework}
\author{Spencer}
\addtolength{\oddsidemargin}{-.575in}
\addtolength{\evensidemargin}{-.575in}
\addtolength{\textwidth}{1.0in}
\addtolength{\topmargin}{-.575in}
\addtolength{\textheight}{1.25in}
\begin{document}
\maketitle

\newcommand{\Z}{\mathbb{Z}}
\newcommand{\R}{\mathbb{R}}
\newcommand{\N}{\mathbb{N}}
\newcommand{\Q}{\mathbb{Q}}
\newcommand{\gal}{\mbox{Gal}}

I was not able to do all problems.

%\section*{Problem 1}
%
%Let $E$ be an extension field of $\Q$.  Show that any automorphism of $E$
%acts as the identity on $\Q$.
%
%Let $\phi$ be an automorphism of $E$.  We must have $\phi(0)=0$ and
%$\phi(1)=1$.  Let $z\in\Z$.  Then we have
%\begin{equation*}
%\phi(z)=\phi(\underbrace{1+\dots+1}_z)=\underbrace{\phi(1)+\dots+\phi(1)}_z=
%\underbrace{1+\dots+1}_z=z.
%\end{equation*}
%This shows that $\phi$ fixes the integers.  Now let $a,b\in\Z$ with $b\neq 0$.
%Then $\phi(a/b)=\phi(a)/\phi(b)=a/b$, showing that $\phi$ fixes the rationals
%as well.
%
%\section*{Problem 3}
%
%Let $E$ be a field extension of the field $F$.  Show that the automorphism
%group of $E$ fixing $F$ is indeed a group.
%
%The identity automorphism is in the set.  Function composition is associative.
%The composition of automorphism functions of $E$ fixing $F$ form automorphisms
%of $E$ that fix $F$.
%The inverse of every automorphism function is its inverse as a group element.

\section*{Problem 4}

Given that the automorphism group of $\Q(\sqrt{2},\sqrt{5},\sqrt{7})$ is
isomorphic to $\Z_2\oplus\Z_2\oplus\Z_2$, determine the number of subfields
of $\Q(\sqrt{2},\sqrt{5},\sqrt{7})$ that have degree 4 over $\Q$.

Let $F=\Q$, $E=\Q(\sqrt{2},\sqrt{5},\sqrt{7})$, and $K$ be a subfield
of $E$ containing $F$.  Then by hypothesis, we have
$|\gal(E/F)|=|\Z_2\oplus\Z_2\oplus\Z_2|=8$, and we want $[K:F]=4$.
So by Theorem 32.1, we have
$[K:F]=|\gal(E/F)|/|\gal(E/K)|\implies 4=8/|\gal(E/K)|\implies 4=8/[E:K]$,
so we want to count the number of fields $K$ where $[E:K]=2$.
Well, I believe that $E$ has dimension 2 over $\Q(\sqrt{2},\sqrt{5})$,
$\Q(\sqrt{2},\sqrt{7})$, and $\Q(\sqrt{5},\sqrt{7})$.  So there are
at least 3 of them.

%\section*{Problem 5}
%
%Let $E$ be a field extension of a field $F$ and let $H$ be a subgroup of
%$\gal(E/F)$.  Show that the fixed field of $H$ is indeed a field.
%
%By definition, we have $H=\{x\in E:\mbox{$x=\phi(x)$ for all $\phi\in H$}\}$.
%Notice that if $H$ is a field, then it is a subfield of $E$.  So we can
%use a subfield test.  Let $x,y\in H$.  Then $\phi(x-y)=\phi(x)-\phi(y)=x-y$,
%showing that $x-y\in H$, and $\phi(xy)=\phi(x)\phi(y)=xy$, showing that $xy\in H$.
%So $H$ is a subring of $E$.  Unity is in $H$.  What remains to be shown is that
%every element of $H$ is a unit.  Notice that $\phi(x^{-1})=\phi(x)^{-1}=x^{-1}$,
%so that $x^{-1}\in H$, so every member of $H$ has a multiplicative inverse.

%\section*{Problem 6}
%
%Let $E$ be the splitting field of $x^4+1$ over $\Q$.  Find $\gal(E/\Q)$.
%Find all subfields of $E$.  Find the automorphisms of $E$ that have fixed
%fields $\Q(\sqrt{2})$, $\Q(\sqrt{-2})$, and $\Q(i)$.  Is there an automorphism
%of $E$ whose fixed field is $\Q$?
%
%I have no clue.

\section*{Problem 8}

Show that the Galois group of a polynomial of degree $n$ has order dividing $n!$.

Let $E$ be the splitting field of a polynomial $f(x)$ of degree $n$
over the field $F$.  The Galois group of $f(x)$ is then $\gal(E/F)$.
If $F$ is a field of characteristic 0 or a finite field, then by
Theorem 32.1, we have $[E:F]=|\gal(E/F)|$.  Clearly, $[E:F]\leq n$,
because in the worst case, $f(x)$ is irreducible over $F$, and $[E:F]=n$.
In all other cases, $[E:F]<n$, because we mod-out by an irreducible of
degree less than $n$ to extend $F$ to $E$.
It follows that $|\gal(E/F)|\leq n$, and so divides $n!$.

There is a problem here.  Mod-ing out by an irreducible does not throw
in all of the zeros of that polynomial.  We went over this in class.
Maybe we can use The Primitive Element Theorem here.
Let $a_1,a_2,\dots,a_n$ be the roots of $f(x)$.
Then $F(a_1,a_2,\dots,a_n)$ is the splitting field of $f(x)$ over $F$.
Using Theorem 21.6, there exists $a$ in the splitting field of $f(x)$
such that $F(a_1,a_2,\dots,a_n)=F(a)$.  We now want to convince ourselves
that $a$ has a degree less than or equal to $n$ over $F$.  It would
then follow that $|\gal(E/F)|=[E:F]=[F(a):F]\leq n$, but I don't know.
Is $a$ a root of $f(x)$?  I don't see why it would be.  It doesn't have to be.

\section*{Problem 16}

Suppose that $E$ is the splitting field of some polynomial over a field
$F$ of characteristic 0.  If $[E:F]$ is finite, show that there is only
a finite number of fields between $E$ and $F$.

Using Theorem 32.1, we have $|\gal(E/F)|=[E:F]<\infty$.  So there are
only a finite number of subgroups of $\gal(E/F)$.  Using Theorem 32.1 again,
the number of fields between $E$ and $F$ is finite, because there is a one-to-one
correspondence between them and the subgroups of $\gal(E/F)$.

I didn't use eveything in the hypothesis here, so maybe something is missing here.

\end{document}