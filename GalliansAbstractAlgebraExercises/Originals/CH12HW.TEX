\documentclass{article}
\usepackage{amsmath}
\usepackage{amssymb}
\title{Chapter 12 Homework}
\author{Spencer}
\addtolength{\oddsidemargin}{-.575in}
\addtolength{\evensidemargin}{-.575in}
\addtolength{\textwidth}{1.0in}
\addtolength{\topmargin}{-.575in}
\addtolength{\textheight}{1.25in}
\begin{document}
\maketitle

\newcommand{\Z}{\mathbb{Z}}
\newcommand{\R}{\mathbb{R}}

\section*{Problem 14}

Let $a$ and $b$ belong to a ring $R$ and let $m$ be an integer.
Prove that $m\cdot(ab)=(m\cdot a)b=a(m\cdot b)$.

The case $m=0$ is clear.  Suppose $m>0$.
\begin{align*}
(m\cdot a)b &= ((m-1)\cdot a) + ab \\
 &= ((m-2)\cdot a) + 2\cdot(ab) \\
 &= ((m-3)\cdot a) + 3\cdot(ab) \\
 &=\dots \\
 &= ((m-m)\cdot a) + m\cdot(ab) \\
 &= m\cdot(ab)
\end{align*}
The case of $a(m\cdot b)=m\cdot(ab)$ is proven similarly.
Now suppose $m<0$.  I assume that in this case, the notation
$m\cdot x$ reduces to $|m|\cdot(-x)$.
\begin{align*}
(m\cdot a)b &= (|m|\cdot(-a))b \\
 &= ((|m|-1)\cdot(-a))b - ab \\
 &= ((|m|-2)\cdot(-a))b - 2\cdot(ab) \\
 &= ((|m|-3)\cdot(-a))b - 3\cdot(ab) \\
 &=\dots \\
 &= ((|m|-|m|)\cdot(-a))b - |m|\cdot(ab) \\
 &= -|m|\cdot(ab) = m\cdot(ab)
\end{align*}
Again, the case of $a(m\cdot b)=m\cdot(ab)$ is proven similarly.

\section*{Problem 16}

Show that if $n$ is an integer and $a$ is an element from a ring, then
$n\cdot(-a)=-(n\cdot a)$.

The case $n=0$ is clear.  Notice that if $n<0$, then
we have $|n|\cdot a=-(|n|\cdot(-a))$, which is the same idea since we
need only have $a$ and $-a$ be additive intervses of one another.
So we will only consider $n>0$.
Notice that $n\cdot(-a)+n\cdot a=n\cdot(-a+a)$ by the Abelian property
of the ring with respect to addition.  Then $n\cdot(-a+a)=n\cdot 0=0$,
showing that $n\cdot(-a)$ is the additive inverse of $n\cdot(-a)$.

\section*{Problem 20}

Describe the elements of $M_2(\Z)$ that have multiplicative inverses.

\begin{equation*}
\{m\in M_2(\Z):\det m\neq 0\}
\end{equation*}

\section*{Problem 22}

Let $R$ be a commutative ring with unity and let $U(R)$ denote the set
of units of $R$.
Prove that $U(R)$ is a group under the multiplication of $R$.

The unity of $R$ is in $U(R)$.
Let $a,b\in U(R)$.  Then $a^{-1},b^{-1}\in U(R)$.
Since $ab^{-1}ba^{-1}=1$, we have $(ab^{-1})^{-1}=ba^{-1}$, showing
that $ab^{-1}$ is a unity of $R$ and therefore in $U(R)$.
The proof now follows from the one-step subgroup test.

\section*{Problem 24}

If $R_1,R_2,\dots,R_n$ are commutative rings with unity, show that
$U(R_1\oplus R_2\oplus\dots\oplus R_n)=U(R_1)\oplus U(R_2)\oplus\dots\oplus U(R_n)$.

We have to show that the operations of taking the direct product of groups and
collecting the units of a group are commutative.
If $x\in U(R_1\oplus R_2\oplus\dots\oplus R_n)$, then 
$x^{-1}\in U(R_1\oplus R_2\oplus\dots\oplus R_n)$, and
$xx^{-1}=(1_1,1_2,\dots,1_n)$, showing that the components of $x$ must
be units of their respective rings, so that $x\in U(R_1)\oplus U(R_2)\oplus\dots\oplus U(R_n)$.
If $x\in U(R_1)\oplus U(R_2)\oplus\dots\oplus U(R_n)$, then
$x=(x_1,x_2,\dots,x_n)$, and we can find $x^{-1}\in U(R_1)\oplus U(R_2)\oplus\dots\oplus U(R_n)$
if we let $x^{-1}=(x_1^{-1},x_2^{-1},\dots,x_n^{-1})$.
This shows that every $x\in U(R_1)\oplus U(R_2)\oplus\dots\oplus U(R_n)$ is
a unit, so it can be found in $U(R_1\oplus R_2\oplus\dots\oplus R_n)$.
I'm not entirely satisfied with this explanation, but I think it works.
(It's too long.)

\section*{Problem 26}

Determine $U(\R[x])$.

Consider the product...
\begin{equation*}
(a_m x^m + \dots + a_1 x + a_0)(b_n x^n + \dots + b_1 x + b_0) = (a_mb_nx^{m+n} + \dots + a_0b_0)
\end{equation*}
...where $a_m,b_n\neq 0$ and $m,n>0$.
Notice that the coeficients of the leading and trailing terms will always be a product of
real numbers.  The coeficients of the middle terms may use some addition so that subtraction
may give zero coeficients.
Since $\R$ is an integral domain, we must have $a_mb_n\neq 0$.  But we would need $a_mb_n=0$
if we wanted the product to be the multiplicative idenity.
So this only works when $m=n=0$ so that we could have $a_0b_0=1$.
I think that $U(\R[x])$ is the set of all constant polynomials.
It might be appropriate to say $U(\R[x])=\R$.

\section*{Problem 30}

Suppose that there is an integer $n>1$ such that $x^n=x$ for all elements
$x$ of some ring.  If $m$ is a positive integer and $a^m=0$ for some $a$,
show that $a=0$.

If $m=n$, then it's trivial.
Supposing that $m<n$ we have $a=a^n=a^{m+k}=a^ma^k=0a^k=0$, where $k$ is a positive integer.
This shows that if we can find an integer $m\leq n$ such that $a^m=0$, then $a=0$.
In the case where $m>n$, we will find such a power of $a$.
Calculate $0=a^m=a^{n+k}=a^na^k=aa^k=a^{k+1}$, where $k$ is a positive integer.
(Notice that $k+1\leq m$, but if $k+1=m$, then $k+1=n+k\implies n=1$, which can't happen,
so $k+1<m$.)
If $k+1\leq n$, then we're done.
If not, calculate $0=a^{k+1}=a^{n+k'}=a^na^{k'}=aa^{k'}=a^{k'+1}$, where $k'$ is a positive integer.
(Again, notice that $k'+1<k+1$.)
If $k'+1\leq n$, then we're done.  If not, continue in a similar manner until we're done.
The process will terminate since $m$ is finite.

\section*{Problem 32}

Let $n$ be an integer greater than 1.  In a ring in which $x^n=x$ for all $x$,
show that $ab=0$ implies $ba=0$.

\begin{equation*}
ba = (ba)^n = b(ab)^{n-1}a = 0
\end{equation*}

\section*{Problem 36}

Let $m$ and $n$ be positive integers and let $k$ be the least common multiple
of $m$ and $n$.  Show that $m\Z\cap n\Z=k\Z$.

The intersection of the set of all multiples of $n$ with the set of all multiples of $m$
will give us the set of all multiples of $m$ and $n$.  This is also the set
generated by $k$.  Did I miss something?

\section*{Problem 40}

Let $M_2(\Z)$ be the ring of all $2\times 2$ matrices over the integers
and let $R=\left\{\left[\begin{array}{cc}a&a+b\\a+b&b\end{array}\right]:a,b\in\Z\right\}$.
Prove or disprove that $R$ is a subring of $M_2(\Z)$.

The following shows that $R$ is not closed under the multiplication of $M_2(\Z)$.
\begin{equation*}
\left[\begin{array}{cc}1&1+1\\1+1&1\end{array}\right]
\left[\begin{array}{cc}1&1+2\\1+2&2\end{array}\right] =
\left[\begin{array}{cc}7&7\\5&8\end{array}\right]
\end{equation*}
Notice that $5\neq 7$.

\section*{Problem 42}

Let $R=\left\{\left[\begin{array}{cc}a&a\\b&b\end{array}\right]:a,b\in\Z\right\}$.
Prove or disprove that $R$ is a subgring of $M_2(\Z)$.

Notice that $R$ contains the additive identity of $M_2(\Z)$.
The following shows that $R$ is closed under the subtraction of $M_2(\Z)$.
\begin{equation*}
\left[\begin{array}{cc}a&a\\b&b\end{array}\right]-
\left[\begin{array}{cc}c&c\\d&d\end{array}\right]=
\left[\begin{array}{cc}a-c&a-c\\b-d&b-d\end{array}\right]
\end{equation*}
And the following shows that $R$ is closed under the multiplication of $M_2(\Z)$.
\begin{equation*}
\left[\begin{array}{cc}a&a\\b&b\end{array}\right]
\left[\begin{array}{cc}c&c\\d&d\end{array}\right]=
\left[\begin{array}{cc}ac+ad&ac+ad\\bc+bd&bc+bd\end{array}\right]
\end{equation*}

\section*{Problem 50}

Suppose that $R$ is a ring and that $a^2=a$ for all $a$ in $R$.  Show that $R$ is
commutative.

Let $a,b\in R$.  
First notice that $-a = (-a)^2 = (-a)(-a) = a^2 = a$, showing that every element
of $R$ is its own additive inverse.  It then follows that
$a+b = (a+b)^2 = (a+b)a + (a+b)b = a^2+ba+ab+b^2 = a+ba+ab+b\implies ba+ab=0\implies ba=-ab=ab$.

\end{document}
