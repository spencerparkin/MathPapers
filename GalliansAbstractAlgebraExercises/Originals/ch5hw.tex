\documentclass{article}
\usepackage{amsmath}
\usepackage{amssymb}
\title{Chapter 5 Homework}
\author{Spencer}
\addtolength{\oddsidemargin}{-.575in}
\addtolength{\evensidemargin}{-.575in}
\addtolength{\textwidth}{1.0in}
\addtolength{\topmargin}{-.575in}
\addtolength{\textheight}{1.25in}
\begin{document}
\maketitle

\newcommand{\lcm}{\mbox{lcm}}
\newcommand{\stab}{\mbox{stab}}

\section*{Problem 2}

What is the order of a $k$-cycle $(a_0a_1\dots a_{k-1})$?

Let $\alpha=(a_0a_1\dots a_{k-1})$ and choose any integer $i\in[0,k-1]$.
Notice that $\alpha(a_i)=a_{i+1\mod k}$.  Fix the natural number $j$
and suppose that $\alpha^j(a_i)=a_{i+j\mod k}$.  Then...
\begin{equation*}
\alpha^{j+1}(a_j) = \alpha(\alpha^j(a_i)) = \alpha(a_{i+j\mod k}) = a_{i+j+1\mod k}
\end{equation*}
...so that by induction we may unfix $j$.  Then since the least positive integer $j$
such that $a_i=a_{i+j\mod k}$ is $k$, we see that the least positive integer $j$
such that $a_i=\alpha^j(a_i)$ is $k$.  We can now conclude that $|\alpha|=k$.

\section*{Problem 5}

What is the order of the product of a pair of disjoint cycles of lengths 4
and 6?

Theorem 5.3 says it's $\lcm(4,6)=12$.

\section*{Problem 6}

Show that $A_8$ contains an element of order 15.

Notice that $(123)(45678)\in S_8$ and has order $\lcm(3,5)=15$.
Now notice that $(123)(45678)=(13)(12)(48)(47)(46)(45)\in A_8$.

\section*{Problem 10}

Show that a function from a finite set $S$ to itself is one-to-one if and only if
it is onto.  Is this true when $S$ is infinite?

Let $f:S\to S$ be the function in question.
Define the notation $f(S)$ to mean the set $\{f(s):s\in S\}$.

Let $f$ be onto and suppose $f$ is not one-to-one.  Then there exists two or
more elements of $S$ that have the same functional value.  This then implies
that $f(S)$ is a proper subset of $S$.  Now choose
$y\in S-f(S)$ and we see that while $y\in S$,
there does not exist $x\in S$ such that $f(x)=y$.  This contradicts the
fact that $f$ is onto.  Therefore, $f$ is one-to-one.

Let $f$ be one-to-one and suppose $f$ is not onto.  Again, this implies that
$f(S)$ is a proper subset of $S$, and this implies that $|f(S)|<|S|$ so that
by the pigeon hole principle, there must exist distinct elements $x,y\in S$ such
that $f(x)=f(y)$.  This contradicts the fact that $f$ is one-to-one.  Therefore,
$f$ is onto.

One of the reasons this theorem fails in the case of $S$ being infinite comes
from the fact that we can find one-to-one and onto mappings between a subset of
an infinite set and the infinite set.  For example, let $f(x)=\tan(x)$.  Then
$f:(-\pi/2,\pi/2)\to\mathbb{R}$ is a one-to-one and onto mapping, but when we
extend the function to $f:\mathbb{R}\to\mathbb{R}$, it fails to continue to be one-to-one.

\section*{Problem 13}

Prove that the set of even permutations of $S_n$ forms a subgroup of $S_n$.

Let the set in question be denoted by $E$.  Notice that $\epsilon\in E$ and
that $E$ is a finite set since $S_n$ is finite.
Then by Theorem 3.3, we need only show that $E$ is closed under the operation
of $S_n$.  Let $a,b\in E$.  Then $a$ and $b$ are even permutations of $S_n$.
Let $a$ be written as $2m$ 2-cycles and let $b$ be written as $2n$ 2-cycles.
Then we can write $ab$ as $2m+2n=2(m+n)$ 2-cycles when we concatinate the
2-cycle forms of $a$ and $b$.  By Theorem 5.5, all
other ways of writing $ab$ in 2-cycle form have an even number of 2-cycles.
Then since $ab\in S_n$ by the closure of $S_n$, we must have $ab\in E$.

\section*{Problem 14}

In $S_n$, let $\alpha$ be an $r$-cycle, $\beta$ be an $s$-cycle, and $\gamma$ be
a $t$-cycle.  Then $\alpha\beta$ is even if and only if $r+s$ is blank, and $\alpha\beta\gamma$
is even if and only if $r+s+t$ is blank.

Given any $r$-cycle such as $\alpha$, if $r>1$,
we can always decompose it into a product of $r-1$ 2-cycles using the method
illustrated in the proof of Theorem 5.4.
\begin{equation*}
\alpha = (a_1a_2\dots a_r) = (a_1a_r)(a_1a_{r-1})\dots(a_1a_2)
\end{equation*}
Theorem 5.5 then tells us that every docomposition of $\alpha$ into a product of
2-cycles will have the same parity as $r-1$.  So for $\alpha\beta$ to have an
even 2-cycle decomposition, we need $r-1+s-1=r+s-2$ to be even, which is so if and only
if $r+s$ is even.  This is seen when we think about concatinating the 2-cycle decompositions
of $\alpha$ and $\beta$, and again falling back on Theorem 5.5.

Using the same line of reasoning for the other blank, we see that $\alpha\beta\gamma$
is even if and only if $r-1+s-1+t-1=r+s+t-3$ is even, which is so if and only if $r+s+t$
is odd.

\section*{Problem 15}

Let $\alpha$ and $\beta$ belong to $S_n$.  Prove that $\alpha\beta$ is even if and only if
$\alpha$ and $\beta$ are both even or both odd.

Using the results of Problem 14, if $\alpha$ is an $r$-cycle and $\beta$ is an $s$-cycle,
then $\alpha\beta$ is even if and only if $r+s$ is even, which is so if and only if
$r$ and $s$ have the same parity.

\section*{Problem 16}

Associate an even permutation with the number $+1$ and an odd permutation with
the number $-1$.  Draw an analogy between stuff.

The parity of the product of two permutations can now be found by examining the
sign of the product of their associated numbers $\pm 1$.  This follows from
Theorem 5.5.

\section*{Problem 19}

Show that if $H$ is a subgroup of $S_n$, then either every member of $H$ is an
even permutation or exactly half of the members are even.

Suppose $H$ contains only even members.  This is entirely possible since
we don't run into any problems with closure under the group operation,
or problems with closure under the taking of inverses.  We also see that
the identity permutation is in $H$ since it is even.

Now suppose that not all members of $H$ are even.  Let $x\in H$ be an
odd permutation.  Then $xy$ is a new odd permutation for all $y\in H$ where
$y$ is an even permutation.  So there are at least as many odd permutations
as there are even permutations in $H$.  Using the same trick, notice that
$xy$ is a new even permutation for all $y\in H$ where $y$ is an odd permutation.
So there are at least as many even permutations as there are odd permutations in $H$.
It now follows that the number of even permutations in $H$,
and the number odd permutations in $H$ is the same.

My proof may be flawed here because I did not make use of the fact that
$H$ is a subgroup of $S_n$.  I only relied upon the fact that $H$ is a
group of permutations.

\section*{Problem 21}

Do the odd permutations in $S_n$ form a group?

No, because we fail to have closure.  The product of two odd permutations is an even permutation.

\section*{Problem 22}

Let $\alpha$ and $\beta$ belong to $S_n$.  Prove that $\alpha^{-1}\beta^{-1}\alpha\beta$
is an even permutation.

Since $\alpha^{-1}\alpha=\epsilon$ is an even permutation, we see that $\alpha$ and
$\alpha^{-1}$ have the same parity.  Let $r$ be an integer having the same parity
as $\alpha$, and let $s$ be an integer having the same parity as $\beta$.
Then $\alpha^{-1}\beta^{-1}\alpha\beta$ has the same parity as
$r+s+r+s = 2r+2s = 2(r+s)$, which must be even.

\section*{Problem 27}

Let $\beta\in S_7$ and suppose $\beta^4=(2143567)$.  Find $\beta$.

Even using Theorem 4.2, I cannot come to a solid conclusion that $|\beta|=7$.
If I could, then $\beta = \beta\beta^{|\beta|} = \beta^8 = (\beta^4)^2$,
and it's an easy calculation from there.  All I can conclude is that
$|\beta|=7\gcd(7k,4)$ for some positive integer $k$.  The only possible choices
are $|\beta|=7$, $|\beta|=14$, and $|\beta|=28$, I think.  I'm not sure how
to rule out the last two.

\section*{Problem 28}

Let $\beta=(123)(145)$.  Write $\beta^{99}$ in disjoint cycle form.

Fortunately, $\beta=(14523)$ can be written as a single cycle.
Notice that $|\beta|=5$, so we see that
$\beta^{99} = \beta^{18\cdot 5+4} = (\beta^5)^{18}\beta^4=\epsilon^{18}\beta^4=\beta^4$.
Some work then shows that $\beta^4=(13254)$.

\section*{Problem 30}

What cycle is $(a_1a_2\dots a_n)^{-1}$?

Notice that...
\begin{equation*}
\epsilon = (a_1a_2\dots a_n)(a_na_{n-1}\dots a_1)
\end{equation*}
...because the right-hand cycle sends $a_k$ to $a_{k-1}$ (wrapping subscripts), while
the left-hand cycle sends $a_{k-1}$ back to $a_k$.
So we see that...
\begin{equation*}
(a_1a_2\dots a_n)^{-1} = (a_na_{n-1}\dots a_1)
\end{equation*}

\section*{Problem 31}

Let $G$ be a group of permutations on a set $X$.  Let $a\in X$ and
define $\stab(a)=\{\alpha\in G:\alpha(a)=a\}$.  Prove that $\stab(a)$
is a subgroup of $G$.

Notice that by definition $\stab(a)\subseteq G$ and that $\stab(a)\neq\emptyset$
since $\epsilon\in\stab(a)$.  Now let $x,y\in\stab(a)$.  Then since $y$ fixes $a$,
so does $y^{-1}$.  Then since $x$ and $y^{-1}$ fix $a$, so does $xy^{-1}$.
Then since $xy^{-1}\in G$ we have $xy^{-1}\in\stab(a)$.  We can now conclude by
the one-step subgroup test that $\stab(a)$ is a subgroup of $G$.

\section*{Problem 37}

Suppose that $\beta$ is a 10-cycle.  For which integers $i$ between 2 and 10
is $\beta^i$ also a 10-cycle?

I believe that the answer is all integers $2\leq i\leq 10$ where $\gcd(i,10)=1$.
The cycle multiplication will result in a single cycle of length 10 only if
$\{0,1,2,\dots,9\}=\{0,0+i,0+2i,\dots,0+9i\}$.  We can now generalize this
result to any $r$-cycle.

\section*{Problem 38}

In $S_3$, find elements $\alpha$ and $\beta$ such that $|\alpha|=2$,
$|\beta|=2$, and $|\alpha\beta|=3$.

Choose $\alpha=(12)$ and $\beta=(23)$.  Then $|(12)(23)|=|(123)|=3$.

\section*{Problem 39}

Find group elements $\alpha$ and $\beta$ such that $|\alpha|=3$,
$|\beta|=3$, and $|\alpha\beta|=5$.

Choose $\alpha=(123)$ and $\beta=(345)$, both from $S_5$.
Then $|(123)(345)|=|(12345)|=5$.

\section*{Problem 41}

Prove that $S_n$ is non-Abelian for all $n\geq 3$.

Here we can re-use Example 1.  Let $\alpha=(123)\in S_n$
and $\beta=(13)\in S_n$.  Then...
\begin{equation*}
\alpha\beta = (23) \neq (12) = \beta\alpha
\end{equation*}
Here we're taking advantage of the cycle notation in that
numbers we fail to write down are assumed to permute to themselves.

\section*{Problem 42}

Let $\alpha$ and $\beta$ belong to $S_n$.  Prove that $\beta\alpha\beta^{-1}$ and
$\alpha$ are both even or both odd.

Since $\beta\beta^{-1}=\epsilon$, $\beta$ and $\beta^{-1}$ have the same parity.
Let $r$ be an integer with the same parity as $\alpha$ and $s$ be an integer
with the same parity as $\beta$.  Then the parity of $\beta\alpha\beta^{-1}$
is $s+r+s = 2s+r$ which has the same parity has $r$.

\section*{Problem 46}

Show that for $n\geq 3$, $Z(S_n)=\{\epsilon\}$.

Let $x\in S_n$ where $x\neq\epsilon$.  We now need to find an element $y\in S_n$
with which $x$ does not commute.  We will construct the permutation $y$.
Let $a,b,c\in\{1,2,\dots,n\}$ where each is
distinct from the others.  We can do this because $n\geq 3$.  Suppose that $x(a)=b$
and choose $y(b)=c$ so that $yx(a)=c$.  We now want to choose $y(a)$ such that
$xy(a)\neq c$.  If $x(b)=c$, choose $y(a)=c$.  If $x(c)=c$, choose $y(a)=b$.

\section*{Problem 52}

Let $G$ be a group.  Prove or disprove that $H=\{g^2:g\in G\}$ is a subgroup
of $G$.

Let $G=A_4$.  Then everything in $H$ must be decomposable into a number of
2-cycles that is divisible by 4.  This is because $g^2$ is the concatination
of two permutations of even parity.
Now choose $(123),(234)\in A_4$ and calculate
$(123)^2(234)^2 = (13)(24)$.  This is not in $H$ since its parity is 2 and 2
isn't divisible by 4.

\end{document}