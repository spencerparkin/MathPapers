\documentclass[12pt]{article}

\usepackage{amsmath}
\usepackage{amssymb}
\usepackage{amsthm}

\title{The Factorization Problem\\In\\Computational Group Theory}
\author{Spencer T. Parkin}

\newtheorem{theorem}{Theorem}[section]
\newtheorem{definition}{Definition}[section]
\newtheorem{corollary}{Corollary}[section]
\newtheorem{identity}{Identity}[section]
\newtheorem{lemma}{Lemma}[section]
\newtheorem{result}{Result}[section]

\newcommand{\stab}{\mbox{stab}}

\begin{document}
\maketitle

While this paper does not extend the frontier of the subject in question, it does
present the material in a manner suitable for novices, such as myself.  On that basis
alone I believe the paper has merit.

That being said, for any given element $g$ of a finite permutation group $G$ generated
by a set of permutation $S$, we wish to factor $g^{-1}$ in terms of $S$.
Since $g\in G=\langle S\rangle$, it is clear that for some positive integer $k$, there exists
a sequence of $k$ elements $\{s_i\}_{i=1}^k$ taken from $S$ such that
\begin{equation*}
g = s_1s_2\dots s_k,
\end{equation*}
and therefore,
\begin{equation*}
g^{-1} = s_k^{-1}s_{k-1}^{-1}\dots s_1^{-1}.
\end{equation*}
Indeed, there may be many such sequences.
How we might come up with such a sequence, however, is not immediately clear.
Our goal is to come up with a systematic way of producing such a sequence, even if it
is not of minimal length.

Associated with our group $G$ is a set $\Omega$ over which each $g\in G$ is defined.
That each $g$ is a permutation of this set $\Omega$ is to say that $g$ is a one-to-one and onto
mapping from $\Omega$ to $\Omega$.
For a given $\omega\in\Omega$,
we define the stabilizer subgroup of $G$ with respect to $\omega$ as
\begin{equation*}
\stab_{\omega}(G) = \{g\in G|g(\omega)=\omega\}.
\end{equation*}
This is the set of all permutation of $g$ that stabilize $\omega$.  It is not hard to show
that this is a subgroup of $G$.  Furthermore, it is a normal subgroup of $G$.  For any $h\in\stab_{\omega}(G)$,
and any $g\in G$, we have
\begin{equation*}
g(h(g^{-1}(\omega))) = g(g^{-1}(\omega)) = \omega.
\end{equation*}
Being normal, we can consider the factor group
\begin{equation*}
G/H = \{gH|g\in G\},
\end{equation*}
where $H=\stab_{\omega}(G)$.  We now make the observation that since $S$
generates $G$, we must have $SH=\{sH|s\in S\}$ generating $G/H$.  To see this, we write
the typical element of $G/H$ as
\begin{equation*}
gH = \left(\prod_{i=1}^k s_i\right)H = \prod_{i=1}^k (s_iH).
\end{equation*}

We now let $R$ be a set of coset representatives for $G/H$, and $[\cdot]:G\to R$ a function
mapping any element of $g$ to the coset representative in $R$ representing the coset in $G/H$ that contains $g$.
That is, $|R|=|G|/|H|=|G/H|$, and $G/H=\{rH|r\in R\}$, and for all $g\in G$, $gH=[g]H$ with $[g]\in R$.
Computationally, coming up with $R$ is really just the same problem as generating $G/H$ as we would simply use
coset representatives to represent each coset of $G/H$ anyway.  An implementation of $[g]$ might,
for each $r\in R$, check whether $g^{-1}r\in H$.

Armed with all this, and letting $g$ be any element of $G$, we're now going to begin to address our original
question by finding an element $q$ in terms of the generators of $S$ such that $gq\in H$.  Clearly, a procedure
for such a thing is recursively applicable, for $H$ is simply a permutation group associated with $\Omega-\{\omega\}$,
bringing us one step closer to the trivial group $\{e\}$.  The trick is that at each step, we must keep track of all new
generators in terms of the original generators in $S$.

So how do we go about finding $q$?  Well, if $g\in H$ (i.e., $g(\omega)=\omega$), then $q=e=ss^{-1}$ for any $s\in S$,
and we're done.  If not, consider $q=[g]^{-1}$.  Clearly $g[g]^{-1}\in H$, since $gH=[g]H$, and furthermore,
we know the factorization of $[g]^{-1}$ in terms of the generators in $S$, because we know such a factorization
for $[g]\in R$ by virtue of how we generated $R$ from $S$.

We are now almost there!  All that remains is our ability to continue this process in $H$ with $gq\in H$ and a stabilizer
subgroup of $H$ with respect to some element in $\Omega-\{\omega\}$.  For that, we need a generating set for $H$,
and this is where Scheier's Lemma comes into play.  According to Schreier, this set of generators is given by
\begin{equation*}
\{sr[sr]^{-1}|s\in S, r\in R\}.
\end{equation*}
Note that computationally, while we might boil each generator down to its permutation $g\in G$, we would need
to keep its expression (or ``word'') in terms of our original generators in $S$.

So there we have it!  In closing, it might be worth considering why the algorithm thus described does
better than the most obvious, naive and impractical approach to solving the factorization problem, which is to
simply generate all of $G$ from $S$, then search for $g^{-1}$ in $G$.  To begin, $|G|$ may be
extremely large, making its complete generation take too long and require too much memory.
That being said, the stabilizer-chain approach may have its own, similar problems.  For a group $G$
of large order, how big is the order of a factor group likely to get?  How big is a set of generators
likely to get?  I'll have to give it a try.  I believe it can also be shown that this method does not
produce solution sequences anywhere near minimal length.

\end{document}