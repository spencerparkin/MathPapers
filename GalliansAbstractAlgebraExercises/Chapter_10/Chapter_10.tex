\documentclass[12pt]{article}

\usepackage{amsmath}
\usepackage{amssymb}
\usepackage{amsthm}

\title{Chapter 10 Exercises\\Gallian's Book on Abstract Algebra}
\author{Spencer T. Parkin}

\newtheorem{theorem}{Theorem}[section]
\newtheorem{definition}{Definition}[section]
\newtheorem{corollary}{Corollary}[section]
\newtheorem{identity}{Identity}[section]
\newtheorem{lemma}{Lemma}[section]
\newtheorem{result}{Result}[section]

%\newcommand{\gcd}{\mbox{gcd}}
\newcommand{\lcm}{\mbox{lcm}}
\newcommand{\abs}{\mbox{abs}}
\newcommand{\Z}{\mathbb{Z}}
\newcommand{\R}{\mathbb{R}}
\newcommand{\G}{\mathbb{G}}
\newcommand{\stab}{\mbox{stab}}
\newcommand{\aut}{\mbox{Aut}}
\newcommand{\inn}{\mbox{Inn}}
\newcommand{\orb}{\mbox{orb}}

\begin{document}
\maketitle

\section*{Lemma 1}

Let $H$ be a proper subgroup of $G$.  Then
for all $g\in G-H$ and all $h\in H$, $gh\in G-H$.

Suppose $gh=h'\in H$.  Then $g=h'h^{-1}\in H$,
which is a contradiction.  Therefore, $gh\in G-H$.

\section*{Lemma 2}

Let $N$ be a normal subgroup of a group $G$.
Then for any $g\in G$ and any $n\in N$,
there exists $n'\in N$ such that $gn=n'g$ or such that $ng=gn'$.

\section*{Lemma 3}

Let $G$ be a group and let $n$ be a positive integer.
Then the number of elements in $G$ of order $n$, if any,
is divisible by $\phi(n)$, the totient of $n$.

Suppose $G$ has one or more elements of order $n$.
Let $N$ be the set $\{x\in G||x|=n\}$.  Then, for any pair
of elements $a,b\in N$, let $a\sim b$ if and only if $a\in\langle b\rangle$.
This defines an equivilance relation on $N$, since $a\in\langle a\rangle$ gives us
the reflexive property, since $a\in\langle b\rangle\implies b\in\langle a\rangle$ gives
us the symmetric property, and since $a\in\langle b\rangle$ and, for $c\in N$, $b\in\langle c\rangle$ implies
that $a\in\langle c\rangle$, giving us the transitive property.  We now note that
by Theorem 4.4, the size of each equivilance class is $\phi(n)$.
It follows that the number of elements of order $n$ is $G$ is
$s\phi(n)$, where $s$ is the number of equivilance classes.

Oh, I had already read this in the book as the Corollary to Theorem 4.4.

\section*{Exercise 38}

For each pair of positive integers $m$ and $n$, we can define a homomorphism
from $Z$ to $Z_m\oplus Z_n$ by $x\to (x\bmod m, x\bmod n)$.  What is the
kernel when $(m,n)=(3,4)$?  What is the kernel when $(m,n)=(6,4)$?
Generalize.

Let $\phi:Z\to Z_m\oplus Z_n$ be the homomorphism.  Seeing that
\begin{align*}
\ker\phi&=\{x\in Z|\mbox{$x\equiv 0\pmod m$ and $x\equiv 0\pmod n$}\} \\
 &=\{zm|z\in Z\}\cap\{zn|z\in Z\},
\end{align*}
it follows that
\begin{equation*}
\ker\phi=\{z\lcm(m,n)|z\in Z\}.
\end{equation*}

\section*{Exercise 39}

If $K$ is a subgroup of $G$ and $N$ is a normal subgroup of $G$,
prove that $K/(K\cap N)$ is isomorphic to $KN/N$.

Notice that the normality of the subgroup $K\cap N$ in $K$ is proven
by the problem similar to Exercise 50 in Chapter 9.

We now show that $KN$ is a group.
Let $x\in KN$.  Then $x=kn$ for some $k\in K$ and $n\in N$.
But then by Lemma 2 above, $x=n'k\in NK$ for some $n'\in N$.
It follows that $KN\subseteq NK$.
Similarly, we can show that $NK\subseteq KN$, so $NK=KN$.
It then follows by Exercise 6 of the supplementary exercises
for chapters 5 through 8 that $NK$ is a group.

Is $N$ normal in $KN$?

We now let $\phi:K/(K\cap N)\to KN/N$ be a function defined as
\begin{equation*}
\phi(k(K\cap N))=kN,
\end{equation*}
and show that it is a homomorphism.  Let us first verify that this
is a well defined function.  Let $a,b\in K$ such that $a(K\cap N)=b(K\cap N)$.
Then $ab^{-1}\in K\cap N\subseteq N$, showing that $aN=bN$.

We now show that $\phi$ is operation preserving.
By the normality of $N$ and $N\cap K$, we see that
\begin{align*}
 & \phi(a(K\cap N)b(K\cap N)) \\
=\,& \phi(ab(K\cap N)) \\
 =\,& abN=aNbN \\
 =\,& \phi(a(K\cap N))(\phi(b(K\cap N)),
\end{align*}
showing that $\phi$ is operation preserving.

We now consider the kernel of $\phi$.  Notice that
\begin{align*}
\ker\phi &= \{k(K\cap N)\in K/(K\cap N)|k\in N\}, \\
 &= \{k(K\cap N)\in K/(K\cap N)|k\in K\cap N\}, \\
 &= \{K\cap N\}.
\end{align*}
It follows that $\phi$ is an isomorphism by Property 9
of Theorem 10.2.

\section*{Exercise 40}

If $M$ and $N$ are normal subgroups of $G$ and $N\leq M$, prove
that $(G/N)/(M/N)\approx G/M$.

Notice that $M/N$ is a subgroup of $G/N$.  To see that $M/N$ is normal in $G/N$,
let $g\in G$ and let $m\in M$, and see that
\begin{equation*}
gNmN(gN)^{-1}=gmNg^{-1}N=gmg^{-1}N\in M/N,
\end{equation*}
since $gmg^{-1}\in M$ by the normality of $M$ in $G$.

Now consider the mapping $\phi:(G/N)/(M/N)\to G/M$, defined as
\begin{equation*}
\phi(xN(M/N))=yM,
\end{equation*}
where $y$ is any element in the coset $xN$.  Let us now show that
this is a well defined mapping.  Let $a,b\in G$ such that
$aN(M/N)=bN(M/N)$.  It follows that $aN(bN)^{-1}=ab^{-1}N\in M/N\implies ab^{-1}\in M$.
Now let $aN(M/N)$ map to $a'M$ and $bN(M/N)$ map to $b'M$.
Now if $a'\in aN\subseteq aM$, then $a'M=aM$.
Similarly, if $b'\in bN\subseteq bM$, then $b'M=bM$.
But now since $ab^{-1}\in M$, we see that $aM=bM$, so $a'M=b'M$.

Notice that the proof that $\phi$ is well defined also lets us simplify its usage.
That is, for any $x\in G$, we can let $xN(M/N)$ map to $xM$.  This will greatly
ease the remainder of our proof.

We now show that $\phi$ is operation preserving.  Letting $a,b\in G$, we have
\begin{align*}
 & \phi(aN(M/N)bN(M/N)) \\
 =\,& \phi(aNbN(M/N)) \\
 =\,& \phi(abN(M/N)) \\
 =\,& abM = aMbM \\
 =\,& \phi(aN(M/N))\phi(bN(M/N)).
\end{align*}

We now consider the kernel of $\phi$.  We have
\begin{align*}
\ker\phi &=\{gN(M/N)|\mbox{$g\in G$ and $\phi(gN)=M$}\} \\
 &= \{gN(M/N)|g\in M\}.
\end{align*}
Now let $a,b\in M$ and consider $aN(M/N)$ and $bN(M/N)$.
Since $a,b\in M$, we have $ab^{-1}N\in M/N$,
which, in turn, implies that $aN(bN)^{-1}\in M/N\implies aN(M/N)=bN(M/N)$.
It follows that $|\ker\phi|=1$, and therefore, $\phi$ is an isomorphism.

\section*{Exercise 47}

Suppose that for each prime $p$, $Z_p$ is the homomorphic image of a group $G$.
What can we say about $|G|$?  Give an example of such a group.

By Property 6 of Theorem 10.2, we see that $|\phi(G)|$ divides the order of $|G|$.
So, since $\phi(G)=Z_p$, we see that $p$ divides $|G|$.

An automorphism of $Z_p$ may be a trivial example.

After reading the answer in the back of the book, I'm wrong, because I did not understand
the problem statement.  For {\it every} prime $p$, $Z_p$ is {\it a} homomorphic image of {\it the}
group $G$.  So by Property 6 of Theorem 10.2, every prime $p$ divides $|G|$; and since
there are infinitely many primes, $|G|=\infty$.

\section*{Exercise 52}

Let $\alpha$ and $\beta$ be group homomorphisms from $G$ to $\overline{G}$ and let
$H=\{g\in G|\alpha(g)=\beta(g)\}$.  Prove or disprove that $H$ is a subgroup of $G$.

Clearly $e\in H$ by Property 1 of Theorem 10.1.
Now let $a,b\in H$.  We then have
\begin{equation*}
\alpha(ab^{-1})=\alpha(a)\alpha(b)^{-1}=\beta(a)\beta(b)^{-1}=\beta(ab^{-1}),
\end{equation*}
showing that $ab^{-1}\in H$.  So I think it's a subgroup of $G$.

\end{document}