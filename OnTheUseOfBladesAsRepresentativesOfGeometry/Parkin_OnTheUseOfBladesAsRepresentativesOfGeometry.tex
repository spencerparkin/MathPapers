\documentclass{birkjour}

\usepackage{tikz}
\usepackage{graphicx}
\usepackage{hyperref}

\newtheorem{thm}{Theorem}[section]
\newtheorem{cor}[thm]{Corollary}
\newtheorem{lem}[thm]{Lemma}
\newtheorem{prop}[thm]{Proposition}
\theoremstyle{definition}
\newtheorem{defn}[thm]{Definition}
\theoremstyle{remark}
\newtheorem{rem}[thm]{Remark}
\newtheorem*{ex}{Example}
\numberwithin{equation}{section}

\newcommand{\R}{\mathbb{R}}
\newcommand{\B}{\mathbb{B}}
\newcommand{\G}{\mathbb{G}}
\newcommand{\V}{\mathbb{V}}
\newcommand{\gd}{\dot{g}}
\newcommand{\gh}{\hat{g}}
\newcommand{\Gd}{\dot{G}}
\newcommand{\Gh}{\hat{G}}
\newcommand{\nvai}{\infty}
\newcommand{\nvao}{o}
\newcommand{\grade}{\mbox{grade}}

%\received{}\accepted{}

\begin{document}

\title{On The Use Of Blades As Representatives Of Geometry}

\author{Spencer T. Parkin}
\address{102 W. 500 S., \\
Salt Lake City, UT  84101} \email{spencerparkin@outlook.com}

%\subjclass{Primary 14J70; Secondary 14J29}

%\dedicatory{To my dear wife Melinda.}

\begin{abstract}
Abstract goes here...
\end{abstract}

\keywords{Key words go here...}

\maketitle

\section{Introduction And Motivation}

In many models of geometry that are based upon geometric algebra (see \cite{}), blades are used
to represent geometries.  Seeing a great deal of commonality between these models, a formal
treatment of this idea deserves to be given in an abstract setting in much the same way that,
for example, abstract algebra provides such a setting in which algebraic sets generated by ideals
of a polynomial ring can be studied.  To the best of this author's knowledge, this is the first treatment of its kind.

To keep our discussion from becoming too pedantic, the finer details upon which the major results of this paper rest are given
in the last section.  Readers needing more familiarity with geometric algebra may want to read that section first.

\section{Enter The Geometric Set}

We begin by introducting $\V$ as an $m$-dimensional vector
space generating a geometric algebra denoted by $\G$.  We leave the signature of this geometric algebra
unspecified, but in cases where a proof depends upon signature, one is given as either
euclidean or non-euclidean.\footnote{The Gram-Schmidt orthogonalization process is applicable
to all blades taken from and only from a geometric algebra having no null-vectors.  While many proofs
are simplified under the assumption that an orthogonal basis can be chosen for any given blade, no such
assumption is made in this paper for the sake of generality.}
The scalars of $\V$, (and therefore of $\G$), are taken from the field $\R$ of real numbers.\footnote{To be more abstract,
we could have used any field with characteristic 1, but there will be no foreseable advantage to doing so in this paper.}
We will let $\R^n$ denote $n$-dimensional euclidean space,\footnote{Some models of geometry find affine space to be the natural
space within which to work, but this will not be the case in this paper.} and let $\B$ denote the set of all blades taken from $\G$.
Lastly, we will let $p:\R^n\to\V$ be an unspecified, yet well-defined function that we'll use in the following definition and
throughout the remainder of this paper.\footnote{By leaving $p$ unspecified, we're abstracting away the definition of the function.
We only care that it is a well-defined function.}

\begin{defn}[Direct And Dual Representation]\label{def_blade_rep_geo}
For any blade $A\in\B$, we say that $A$ directly represents the set of all points $x\in\R^n$ such that
$p(x)\wedge A=0$, and say that $A$ dually represents the set of all points $x\in\R^n$ such that
$p(x)\cdot A=0$.  For convenience, we introduction the following functions $\gh:\B\to\R^n$ and $\gd:\B\to\R^n$
using set-builder notation.
\begin{align*}
\gh(A) &= \{x\in\R^n|p(x)\wedge A=0\} \\
\gd(A) &= \{x\in\R^n|p(x)\cdot A=0\}
\end{align*}
\end{defn}
From Definition~\ref{def_blade_rep_geo}, it is important to take away the realization that a given blade $A\in\B$ represents two geometries
simultaneously; namely, $\gh(A)$ and $\gd(A)$.  Which geometry we choose to think of $A$ as being a representative of at any given time is completely
arbitrary.

It should also be clear from Definition~\ref{def_blade_rep_geo} that the geometry represented by a blade $A$, (directly or dually), remains invariant
under any non-zero scaling of the blade $A$.  Something interesting happens, however, when we take the dual of $A$, as Lemma~\ref{lem_dual_rep} will show.

\begin{lem}[Dual Relationship Between Representations]\label{lem_dual_rep}
For any subset $R$ of $\R^n$, if there exists $A\in\B$ such that $\gh(A)=R$, then $\gd(AI)=R$, where
$I$ is the unit psuedo-scalar of $\G$.  Similarly, if there exists $A\in\B$ such that $\gd(A)=R$, then $\gh(AI)=R$.
\end{lem}
\begin{proof}
The first of these two statements is proven by
\begin{equation*}
0=p(x)\wedge A=-(p(x)\cdot AI)I\iff p(x)\cdot AI=0,
\end{equation*}
while the second is proven by
\begin{equation*}
p(x)\cdot A=0\iff 0=(p(x)\cdot A)I=p(x)\wedge AI.
\end{equation*}
(See identities \eqref{equ_dual_of_v_dot_A} and \eqref{equ_dual_of_v_dot_dual_A} of Section~\ref{sec_useful_identities} and
recall the zero-product property of the geometric product.)
\end{proof}

In other words, Lemma~\ref{lem_dual_rep} is telling us that for a single given geometry, the algebraic relationship between a
blade directly (dually) representative of that geometry, and a blade dually (directly) representative of that geometry, is simply
that, up to scale, they are duals of one another.

Of course, there will also be a geometric relationship between the geometry that is directly represented by a single given
blade $A\in\B$, and the geometry that is dually represented by $A$, but this depends upon the definition of our function
$p$, which we choose, in this paper, to leave open to speculation.

With Lemma~\ref{lem_dual_rep} in hand, geometric algebra's equivalent of an algebraic set may be given as follows.\footnote{If $p$ is
defined appropriately, geometric sets are algebraic sets.}

\begin{defn}[Geometric Set]\label{def_geo_set}
A subset $R\subset\R^n$ for which there exists a blade $A\in\B$ such that $\gh(A)=R$ is what we'll refer to as a {\it geometric set}.
\end{defn}

By Lemma~\ref{lem_dual_rep}, it is easy to see that Definition~\ref{def_geo_set} is equivalent to a version of itself that replaces $\gh$ with $\gd$.
Therefore, for any geometric set $R$, we can claim the existence of a blade $A\in\B$ such that $\gh(A)=R$ or $\gd(A)=R$.

We will let $G$ denote the set of all geometric sets.  For any set $R\in G$, we will let $R^*$ denote the dual geometric
set of $R$.  Specifically, if $A\in\B$ is a blade such that $R=\gh(A)$, then $R^*=\gd(A)$.  Equivalently, if $R=\gd(A)$, then $R^*=\gh(A)$.

\section{A Fundamental Theorem Of Geometric Set Representation}

% need to cover existence and uniqueness of irreducible blades...introduce this in terms of converse of obvious statement: A=LB ==> gh(A)=gh(B).  (liken to nullzensnats)
% state as theorem.
% give algorithm for finding irreducible blade.

It was mentioned earlier that for any pair of blades $A,B\in\B$ with $\lambda\in\R$ such that $A=\lambda B$, we have
$\gh(A)=\gh(B)$.  There often being a need to algebraically relate two independently made formulations of the same geometric
set, an investigation of the converse of this statement leads us to an important theorem about
geometric set representation.

\begin{defn}[Irreducible/Reducible Blade]
Given an $r$-blade $A\in\B$, if there does not exist an $s$-blade $B\in\B$ with $s<r$ such
that $\gh(A)=\gh(B)$, then $A$ is what we'll refer to as an {\it irreducible} blade.  A blade that is
not irredicuble is {\it reducible}.
\end{defn}

Define vacuous blade here?

\begin{lem}
If $A\in\B$ is an irreducible $r$-blade, then there exists a set of $r$ points $\{x_i\}_{i=1}^r\subset\R^n$
and a scalar $\alpha\in\R$ such that
\begin{equation*}
A = \alpha\bigwedge_{i=1}^r p(x_i).
\end{equation*}
\end{lem}
\begin{proof}
This is not so obvious at all.  Maybe it's not true.
\end{proof}

\begin{lem}
If $A\in\B$ is an irreducible blade, and $B\in\B$ is any blade such that $\gh(A)=\gh(B)$, then
for all $v\in\V$, we have $v\wedge A=0\implies v\wedge B=0$.
\end{lem}

\begin{thm}
For every geometric set $R\in G$, there exists, up to scale, a unique, irreducible blade $A\in\B$ such that
$\gh(A)=R$.
\end{thm}

\section{The Intersection Property Of Geometric Sets}

Before a theorem about intersections of geometric sets can happen, we have to establish a statement
about when the outer product of two blades is non-zero in terms of the geometric sets represented by
those blades.  This may not be possible without a definition of $p$.

\begin{thm}[Intersection Of Geometric Sets]
If $R,S\in G$ are geometric sets such that $R^*\cap S^*=\emptyset$,
then $R\cap S$ is a geometric set.
\end{thm}
\begin{proof}
Even with the FTGSR, this is too far of a stretch, because we don't know when we get a linearly
independent set from a set of points in space.
\end{proof}

When is the intersection of two geometric sets a geometric set?  Given $R,S\in G$, does there
exist a blade $A\in\B$ such that $\gh(A)=R\cap S$?

% need to cover intersections and union-like operations.

% can take anything from CGAIntro.pdf?

\section{Geometric Set Transformations}

% cover transformations by versors and how that relates to understanding the action of a versor on p?
% p must have desirable property.  (form preserving versors)  no, versor has the property of preserving p.

% are these geo sets forming a group using inner automorphisms?
% can make topology out of intersections and unions of geometric sets?  no, probably not.

\section{Concluding Remarks}

% may want to ask what else might be missing; i don't know
% don't babble.

\section{Useful Identities And Lemmas}\label{sec_useful_identities}

In this section we give a number of useful algebraic identities and lemmas that would
otherwise distract us from the flow of the paper if given in the main body.\footnote{This
section is not intended as a complete or comprehensive review of geometric algebra.
See \cite{} for such a review.}

Letting $v\in\V$ and $A\in\B$, recall that
\begin{equation}\label{equ_v_geoprod_A_identity}
vA = v\cdot A+v\wedge A.
\end{equation}
Also recall that
\begin{align}
v\wedge A &= \frac{1}{2}(vA+(-1)^rAv),\label{equ_v_wedge_A_identity} \\
v\cdot A &= \frac{1}{2}(vA-(-1)^rAv),\label{equ_v_dot_A_identity}
\end{align}
where $r=\grade(A)$.
Realizing that $\grade(I)=m$, and that by \eqref{equ_v_geoprod_A_identity}, we have $vI=v\cdot I$, we can use
equation \eqref{equ_v_dot_A_identity} to establish the commutativity of vectors in $\V$ with the unit psuedo-scalar $I$ as
\begin{equation}\label{equ_v_commute_I}
vI = -(-1)^mIv.
\end{equation}
Using equation \eqref{equ_v_commute_I} in conjunction with equation \eqref{equ_v_dot_A_identity}, we find that
\begin{equation}\label{equ_dual_of_v_dot_A}
(v\cdot A)I = v\wedge AI.
\end{equation}
(In verifying this identity, it helps to realize that for any integer $k$, we have $(-1)^k=(-1)^{-k}$.)
Replacing $A$ in equation \eqref{equ_dual_of_v_dot_A} with $AI$, we find that
\begin{equation}\label{equ_dual_of_v_dot_dual_A}
v\wedge A = -(v\cdot AI)I.
\end{equation}

Recall that in the case that $v\wedge A$ vanishes, we can say that $v$ is
in the vector space spanned by any factorization of $A$.  Furthermore,
if $A$ is an $r$-blade, at most $r$ linearly independent vectors can be
found where for each such vector $v$, we have $v\wedge A=0$.

\begin{lem}
If $A\in\B$ is a non-zero blade of grade $r>0$ and $\{v_i\}_{i=1}^r$ is a set
of $r$ linearly independent vectors such that for all $v\in\{v_i\}_{i=1}^r$,
we have $v\wedge A=0$, then there exists $\lambda\in\R$ such that
\begin{equation*}
A = \lambda\bigwedge_{i=1}^r v_i.
\end{equation*}
\end{lem}

Returning to the product $v\cdot A$ given in equation \eqref{equ_v_dot_A_identity},
an alternative expansion is given by
\begin{equation}\label{equ_v_dot_A_sum}
v\cdot A = -\sum_{i=1}^r (-1)^i(v\cdot b_i)A_i,
\end{equation}
where $A$ is factored as $\bigwedge_{i=1}^r a_i$, and we define $A_i$ as
\begin{equation*}
A_i = \bigwedge_{j=1,j\neq i}^r a_i.
\end{equation*}
This leads to the following recursive formulation.
\begin{equation*}
v\cdot A=(v\cdot a_1)A_1-a_1\wedge(v\cdot A_1)
\end{equation*}
If a blade $B\in\B$ has grade $s$ and factorization $\bigwedge_{i=1}^s b_i$, then
we can express the product $A\cdot B$ recursively as
\begin{equation*}
A\cdot B = \left\{\begin{array}{ll}
A_r\cdot (a_r\cdot B) & \mbox{if $r\leq s$,} \\
(A\cdot b_1)\cdot B_1 & \mbox{if $r\geq s$.}
\end{array}\right.
\end{equation*}

Interestingly, though it is not at all obvious from equation \eqref{equ_v_dot_A_sum}
or \eqref{equ_v_dot_A_identity}, the product
$v\cdot A$ is a blade.  It is clearly homogeneous of grade $r-1$, but it is not
immediately clear that it is a blade.  To see that it is a blade, let $\alpha=\prod_{i=1}^r v\cdot a_i$,
and let $\alpha_i$ be given by
\begin{equation*}
\alpha_i=\prod_{j=1,j\neq i}v\cdot a_j.
\end{equation*}
(If $\alpha=0$, the following argument can be reduced to a smaller yet similar case; so
we assume $\alpha\neq 0$.)
Then, letting $v_i=\alpha_1a_1-(-1)^i\alpha_ia_i$, notice that for all integers $1<i\leq r$,
we have
\begin{equation*}
v_i\wedge (v\cdot A)=\alpha A-\alpha A=0.
\end{equation*}
Seeing now that the linear independence of the set of $r-1$ vectors
$\{v_i\}_{i=2}^r$ follows from that of the set of $s$ vectors $\{a_i\}_{i=1}^r$,
we can invoke Lemma~\ref{} in claiming that for some non-zero scalar $\lambda\in\R$,
we have
\begin{equation*}
v\cdot A = \lambda\bigwedge_{i=2}^r v_i,
\end{equation*}
showing that $v\cdot A$ is indeed a blade of grade $r-1$.

\begin{lem}\label{lem_A_i_lin_indep}
For a non-zero $r$-blade $A$ factored as $\bigwedge_{i=1}^r a_i$, the set of
$(r-1)$-blades $\{A_i\}_{i=1}^r$ is linearly independent.
\end{lem}
\begin{proof}
Suppose there exists a non-trivial set of $r$ scalars $\{\alpha_i\}_{i=1}^r$
such that $0 = \sum_{i=1}^r\alpha_i A_i$.
Then, without loss of generality, suppose that $\alpha_r\neq 0$, and rearrange
our equation as $-\alpha_rA_r=\sum_{i=1}^{r-1}\alpha_iA_i$.  Now notice
that while $a_r\wedge-\alpha_rA_r\neq 0$, we have $a_r\wedge\sum_{i=1}^{r-1}\alpha_iA_i=0$,
which is a contradiction.
\end{proof}

Note that a perhaps more elegant proof of Lemma~\ref{lem_A_i_lin_indep} could have been given
under the assumption
of a euclidean signature; in which case, the Gram-Schmidt orthogonalization process would have
allowed us to choose, without loss of generality, an orthogonal factorization of the blade $A$.
Doing so, $A$ becomes the versor $\prod_{i=1}^r a_i$, and we may write
\begin{equation*}
0 = \sum_{i=1}^r \alpha_iA_i \iff
0=\left(\sum_{i=1}^r\alpha_iA_i\right)A = -\sum_{i=1}^r(-1)^{r-i}A_i^2\alpha_ia_i.
\end{equation*}


% may want (a.b)=(VaV^-1).(VbV^-1) lemma

\begin{thebibliography}{9}

\end{thebibliography}

\end{document}