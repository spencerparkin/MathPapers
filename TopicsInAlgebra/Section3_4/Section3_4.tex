\documentclass[12pt]{article}

\usepackage{amsmath}
\usepackage{amssymb}
\usepackage{amsthm}

\title{Section 3.4 Exercises\\Herstein's Topics In Algebra}
\author{Spencer T. Parkin}

\newtheorem{theorem}{Theorem}[section]
\newtheorem{definition}{Definition}[section]
\newtheorem{corollary}{Corollary}[section]
\newtheorem{identity}{Identity}[section]
\newtheorem{lemma}{Lemma}[section]
\newtheorem{result}{Result}[section]

%\newcommand{\gcd}{\mbox{gcd}}
\newcommand{\lcm}{\mbox{lcm}}
\newcommand{\Z}{\mathbb{Z}}
\newcommand{\cl}{\mbox{Cl}}
\newcommand{\aut}{\mbox{Aut}}

\begin{document}
\maketitle

\section*{Thoughts}

It's interesting to note that, by definition, an ideal need not be a subring of a ring.
But maybe this is always the case?  Let $U$ be an ideal of a ring $R$.
$U$ is already an additive subgroup of $R$.  To be a subring, we must show that $U$ is a ring.
I think all that is required at this point is closure under multiplication.  Let $a,b\in U$.  Then for any $r\in R$,
we have $rab\in U$ since $(ra)b\in U$, and $abr\in U$, since $a(br)\in U$.  So we have closure.

Prove: if $\phi$ is a homomorphism from a ring $R$ with unit element 1 onto a ring $R'$ with
unit element $1'$, and $R'$ is an integral domain, then $\phi(1)=1'$ and $R$ is an integral domain.
(Note: Problem 20 says my conditions here are more than sufficient for showing $\phi(1)=1'$.)

Suppose there exists $1'\neq b\in R'$ such that $ba=a$ for all $0'\neq a\in R'$.
Then $1'a=ba\implies (1'-b)a=0'\implies 1'-b=0'\implies 1'=b$; so the multiplicative
identity $1'$ in $R'$ is unique.  Now note that since $\phi(1)\phi(a)=\phi(a)$ for all
$a\in R$, $\phi(1)$ acts as an identity in $R'$; but since there is only one such element in $R'$,
we must have $\phi(1)=1'$.

To see that $R$ is an integral domain, notice that for all $a,b\in R$, we have
\begin{equation*}
0=ab\implies 0'=\phi(ab)=\phi(a)\phi(b)\implies\mbox{$\phi(a)=0'$ or $\phi(b)=0'$},
\end{equation*}
which, in turn, implies that $a=0$ or $b=0$.

\section*{Problem 2}

If $F$ is a field, prove its only ideals are $\{0\}$ and $F$ itself.

We first note that for every homomorphism $\phi$ of a ring $R$, we find an ideal of $R$; namely, $\ker\phi$.
And then for every ideal $I$ of $R$, we find a homomorphism of $R$; namely, $\phi(x)=x+I$.
So there is a one-to-one correspondence between ideals of $R$ and homomorphisms of $R$.

By Problem 3, any homomorphism of $F$ is trivial.  So if $\phi$ is
such a homomorphism, it is either $\phi(x)=0$ or $\phi(x)=x$.
We then find the set of all ideals of $F$ as the kernels of these homomorphisms; which
are $F$ and $\{0\}$, respectively.

\section*{Problem 3}

Prove that any homomorphism of a field is either an isomorphism or takes each element into $0$.

Let $\phi$ be a homomorphism of a field $F$.  If $\phi(x)=0$, we're done; so assumes this is not the case.
We can, therefore, claim that there are non-additive-identity elements in $\phi(F)$.  Let $a\in F$ such
that $\phi(a)$ is such an element.  Now see that $\phi(a)=\phi(a\cdot 1)=\phi(a)\phi(1)$, showing that
$\phi(1)$ in $\phi(F)$ acts as a multiplicative identity element in the ring that is the homomorphic image of $F$.
We then find that for any $a\in F$,
\begin{equation*}
\phi(1)=\phi(aa^{-1})=\phi(a)\phi(a^{-1})\implies\phi(a)^{-1}=\phi(a^{-1}).
\end{equation*}
We can now conclude that $\phi(F)$ is a division ring, and its commutativity would certainly
follow from that of $F$.  So $\phi(F)$ is a field, and therefore an integral domain.
Lastly, for any pair of elements $a,b\in F$ such that $\phi(a)=\phi(b)$,
we have
\begin{equation*}
\phi(1)=\phi(a)\phi(b)^{-1}=\phi(ab^{-1}).
\end{equation*}
It then follows that $ab^{-1}=1$, since $\phi(F)$ is an integral domain.
We can now say that $\phi$ is an isomorphism.

\end{document}