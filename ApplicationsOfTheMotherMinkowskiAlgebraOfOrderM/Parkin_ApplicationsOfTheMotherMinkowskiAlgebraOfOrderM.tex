\documentclass{birkjour}

\usepackage{tikz}
\usepackage{graphicx}
\usepackage{hyperref}

\newtheorem{thm}{Theorem}[section]
\newtheorem{cor}[thm]{Corollary}
\newtheorem{lem}[thm]{Lemma}
\newtheorem{prop}[thm]{Proposition}
\theoremstyle{definition}
\newtheorem{defn}[thm]{Definition}
\theoremstyle{remark}
\newtheorem{rem}[thm]{Remark}
\newtheorem*{ex}{Example}
\numberwithin{equation}{section}

\newcommand{\R}{\mathbb{R}}
\newcommand{\B}{\mathbb{B}}
\newcommand{\G}{\mathbb{G}}
\newcommand{\V}{\mathbb{V}}
\newcommand{\gd}{\dot{g}}
\newcommand{\gh}{\hat{g}}
\newcommand{\Gd}{\dot{G}}
\newcommand{\Gh}{\hat{G}}
\newcommand{\nvai}{\infty}
\newcommand{\nvao}{o}
\newcommand{\grade}{\mbox{grade}}

\begin{document}

%-------------------------------------------------------------------------
% editorial commands: to be inserted by the editorial office
%
%\firstpage{1} \volume{228} \Copyrightyear{2004} \DOI{003-0001}
%
%
%\seriesextra{Just an add-on}
%\seriesextraline{This is the Concrete Title of this Book\br H.E. R and S.T.C. W, Eds.}
%
% for journals:
%
%\firstpage{1}
%\issuenumber{1}
%\Volumeandyear{1 (2004)}
%\Copyrightyear{2004}
%\DOI{003-xxxx-y}
%\Signet
%\commby{inhouse}
%\submitted{March 14, 2003}
%\received{March 16, 2000}
%\revised{June 1, 2000}
%\accepted{July 22, 2000}
%
%
%
%---------------------------------------------------------------------------

\title{Applications Of\\The Mother Minkowski Algebra Of Order $m$}

\author{Spencer T. Parkin}
\address{102 W. 500 S., Salt Lake City, UT  84101}
\email{spencerparkin@outlook.com}

\subjclass{Primary 14J70; Secondary 14J29}

%\date{October 22, 2013}

%\dedicatory{To my dear wife Melinda.}

\begin{abstract}
Abstract goes here...
\end{abstract}

\maketitle

\section{Introduction}

That we may begin with new material immediately, this paper assumes full knowledge
of the paper \cite{}, continuing where this paper ended, and regiving equations from
\cite{} only where necessary.

\section{A Refinements Of Notation}

Notation is important as it facilitates understanding and ease of algebraic manipulation.
We are going to keep the notation of \cite{}, but think of it somewhat differently.
In this paper we are going to work exclusively in $\G_1^n$ and transform multivectors
from this sub-algebra to their counter-parts in $\G_{i>1}^n$ through a use of sub-script notation.
To denote the sub-algebra in which an element $B$ resides, (for any integer $i\in[1,m]$,
we will never use a single letter to denote an element that cannot be placed in $\G_i$),
we will use a sub-script, writing $B_i$, to show that $B\in\G_i$.  If the sub-script is
absent, we can assume $B=B_1$.

As shown in the appendix of \cite{}, there is an outermorphism that lets us transform
elements in one of these sub-algebras to its counter-part in any other.  And indeed,
to make our notation more concrete, we may let the subscription of an
otherwise unscripted letter or expression denote the application of this outermorphism
in the case $i>1$, $i$ being the subscript in question.
Doing so, we see that for any two vectors $a,b\in\V_1$ and any integer $i\in[1,m]$, we have
the following properties.
\begin{align*}
(a)_i &= a_i \\
(a+b)_i &= a_i + b_i \\
(a\wedge b)_i &= a_i\wedge b_i \\
a\cdot b &= a_i\cdot b_i
\end{align*}
The last property here applies to outermorphisms generally, but is especially obvious
with the particular outermorphism we're using here.
It should also be mentioned that for all $i\neq j$, we have $a_i\cdot a_j=0$, yet $a_i\wedge a_j\neq 0$,
a property that has been specifically exploited in the use of the mother algebra to
represent algebraic surfaces.
It should be noted that if $i>1$, we have $(a_i)_i=-a$.

Thinking of these subscripts as an outermorphism may aide us in making algebraic manipulation
and in the way we think about what we're doing.

\section{Intersecting Rays With Algebraic Surfaces}

In \cite{} it was shown that an $m$-vector $B$ with $\nvai\cdot B=0$ may be representative of any
algebraic surface of up to degree $m$ as the set of all points $x\in\R_1^n$ such that $F(x)=0$,
where $F:\R_1^n\to\R$ is given by
\begin{equation*}
F(x) = \bigwedge_{i=1}^m p_i(x)\cdot B,
\end{equation*}
the function $p_i:\R_1^n\to\V_i$ being given by
\begin{equation*}
p_i(x) = \nvao_i + x_i + \frac{1}{2}x_i^2\nvai_i,
\end{equation*}
having its origins in the paper \cite{}.

What we wish to do in this section is, given a point $x\in\R_1^n$ and a
direction $v\in\R_1^n$, find the scalar $\lambda\in\R$, if any, such that
$F(x+\lambda v)=0$.  Attempting to do so, we easily find\footnote{Notice that
since $\nvai\cdot B=0$, we can pretend that we have $p_i(x)=\nvao_i+x_i$,
ignoring the $\frac{1}{2}x_i^2\nvai_i$ term.} that
\begin{equation*}
F(x+\lambda v) = \bigwedge_{i=1}^m (p_i(x)+\lambda v_i)\cdot B,
\end{equation*}
the expansion of which gives us a polynomial of degree $m$ in the scalar $\lambda$.
We can then show that this polynomial can be written in terms of the directional derivative of $F$
in the direction of $v$ as
\begin{align*}
F(x+\lambda v) &= F(x) + \sum_{i=1}^m \lambda^i\nabla_v^i F(x) \\
 &= F(x) + \sum_{i=1}^m \lambda^i v_j\cdot\nabla^i F_j(x),
\end{align*}
where $j$ is any integer in $[1,m]$.
Here, $\nabla_v^i F$ is the $i^{th}$ order directional derivative of $F$ in the direction of $v$.
It is the directional derivative of the directional derivative, and so on, $i$ times.  Similarly,
$\nabla^i F_j$ is the $i^{th}$ order gradient of $F$ using the sub-algebra $\G_j$.
For $i=1$, we would write
\begin{equation*}
\nabla F_j(x) = \sum_{i=1}^n e_{i,j}\nabla_{e_{i,j}}F(x),
\end{equation*}
being careful to differentiate, (no pun intended), between the two different uses
of subscript notation here.  One is used to iterate over an orthonormal basis
$\{e_{i,j}\}_{i=1}^n$ of the vector space $\V_j$,
while the other denotes the sub-algebra $\G_j$ in which we are working.

What we see now is that if all orders of the gradient of $F$ are available to us,
we can use them to find the coefficients of the polynomial whose roots we need
to find in the problem intersecting a ray with the surface.  Interestingly, this
also shows how one might numerically integrate the function $F$.

(This result has to exist somewhere already.  Find it.)

\end{document}