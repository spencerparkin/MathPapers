\documentclass{birkjour}

\usepackage{amsmath}
\usepackage{amssymb}
\usepackage{amsthm}
\usepackage{graphicx}
\usepackage{float}
\usepackage{url}

\newtheorem{thm}{Theorem}[section]
\newtheorem{cor}[thm]{Corollary}
\newtheorem{lem}[thm]{Lemma}
\newtheorem{prop}[thm]{Proposition}
\theoremstyle{definition}
\newtheorem{defn}[thm]{Definition}
\theoremstyle{remark}
\newtheorem{rem}[thm]{Remark}
\newtheorem*{ex}{Example}
\numberwithin{equation}{section}

\newcommand{\G}{\mathbb{G}}
\newcommand{\V}{\mathbb{V}}
\newcommand{\R}{\mathbb{R}}
\newcommand{\Z}{\mathbb{Z}}
\newcommand{\Poly}{\mathbb{P}}
\newcommand{\Gi}{\dot{G}}
\newcommand{\Go}{\hat{G}}

\begin{document}

\title{A Model Of Algebraic Sets\\Using\\Geometric Algebra}

\author{Spencer T. Parkin}
\email{spencer.parkin@gmail.com}

\numberwithin{equation}{section}

\subjclass{Primary 14J70; Secondary 14J29}

\keywords{Quadric Surface, Quartic Surface, Geometric Algebra, Quadric Model, Conformal Model}

%\dedicatory{To Melinda and Naomi}

\begin{abstract}
This paper is an attempt to study algebraic sets using
the techniques of geometric algebra (GA).  Refering to
algebraic sets as geometries, conditions are
found under which the outer product performs the
union and intersection operation of geometries as
represented by blades of a GA.
This model of algebraic sets using GA
is simply a generalization of the conformal model of
geometric algebra (CGA) to a model capable of representing
geometries that are each the zero set of one or more
polynomials of any form.
\end{abstract}

\maketitle

\section{Introduction And Motivation}

Letting $\R^n$ denote an $n$-dimensional Euclidean space, we are
going to let $\Poly^n$ denote the set of all polynomials $f:\R^n\to\R$
of any degree.  Then, letting $P\subseteq\Poly^n$ be any set of such
polynomials, we define the zero set $Z$ of $P$, denoted $Z(P)$,
as the set given by
\begin{equation}\label{equ_zero_set_of_polys}
Z(P) = \{x\in\R^n|\mbox{$f(x)=0$ for all $f\in P$}\}.
\end{equation}
Every subset $S$ of $\R^n$ for which there exists a set $P\subseteq\Poly^n$
such that $S=Z(P)$ is what we refer to as an algebraic set.  It is well known
that for any algebraic set $S\subseteq\R^n$, there is always such a subset $P$
of $\Poly^n$ of finite cardinality.

Given the definition in equation \eqref{equ_zero_set_of_polys}, it is easy to show that
the subsets $S$ of $\R^n$ that are the geometries of the conformal model of geometric algebra
(CGA), and other similar models of geometry based upon geometric algebra (GA), are simply algebraic sets.
The goal of this paper is to show that there exists a model of geometry, based upon GA,
where every possible algebraic set has a representative in the form of an element of that GA.
A desire to come up with such a model of GA is motivated by the admittedly fanciful
dream of the German mathematician Leibniz, referred to in \cite{} and claimed to
have already been realized in the daunting book \cite{}.  In any case, it would seem that a generalization
of CGA or any CGA-like model to one that hosts the set of all algebraic sets in $\R^n$ would bring
us closer to such a goal.  Work to this end has already been done in \cite{Parkin13a,Parkin13b} which
has brought us, the reader and writer, to the present paper.

\section{Blades As Algebraic Sets}

We begin with an examination of $\Poly^n$ as a linear space, observing that
it is of countably infinite dimension.  A set of basis vectors for this space may be taken
as the set of all unit-monomials in anywhere from 1 to $n$ of the $n$ variable components
of an arbitrary point in $\R^n$.\footnote{The
number $c_k$ of such monomials homogeneous of a degree $k$ is given by
\begin{equation*}
c_k=\sum_{i=1}^n\binom{n}{i}p(k,i),
\end{equation*}
where $p(k,i)$ is the number of partitions of size $i$ of the integer $k$.  The
combinatorics of the matter are not important, but they are pointed out here as an added
measure of the reality of the set of monomials we are talking about.}
Considering now any mapping from the set $\Z^+$ of
positive integers to this said set $\{g_i\}_{i=1}^\infty$ of unit-monomials,\footnote{For example,
if $n=2$, then we might take $g_1(x,y)=1$, $g_2(x,y)=x$, $g_3(x,y)=y$, $g_3(x,y)=x^2$,
$g_4(x,y)=xy$, $g_5(x,y)=y^2$, $g_6(x,y)=x^3$, $g_7(x,y)=x^2y$, and so on.}
and letting $\V^\infty$ denote a simple Euclidean\footnote{A GA of Euclidean signature
keeps things simpler, but the model of this paper can be made to work with any GA having
a non-degenerate, non-Euclidean signature.}
vector space of countably infinite
dimension, if we define the function $p:\R^n\to\V^\infty$ as
\begin{equation}\label{equ_form_vector}
p(x) = \sum_{i=1}^\infty g_i(x)e_i,
\end{equation}
where $\{e_i\}_{i=1}^\infty$ is any orthonormal basis for $\V^\infty$, then for
any polynomial $f\in\Poly^n$, there must exist a unique\footnote{Let $v_1,v_2\in\V^\infty$
be two such vectors.  Then $0=f(x)-f(x)=p(x)\cdot (v_1-v_2)$.  Now notice that there
does not exist $x\in\R^n$ such that $p(x)=0$.  It follows that $v_1=v_2$.}
vector $v\in\V^\infty$ such that
\begin{equation}\label{equ_form_coef_split}
f(x) = p(x)\cdot v.
\end{equation}
It now follows that for any subset $P$ of $\Poly^n$, there exists a blade $B\in\G(\V^\infty)$,
such that
\begin{equation}
Z(P) = \{x\in\R^n|p(x)\cdot B=0\}.
\end{equation}
Clearly, we need only consider such finite sets $P=\{f_i\}_{i=1}^k$ that are linearly independent
sets of $k$ polynomials.  Then, for each polynomial $f_i$, there is an associated vector $v_i$ by
equation \eqref{equ_form_coef_split}, and the set $\{v_i\}_{i=1}^k$ must clearly be a linearly independent
set of $k$ vectors.
We may then take $B$ to be the $k$-blade
\begin{equation}
B=\bigwedge_{i=1}^k v_i,
\end{equation}
seeing that the equation $p(x)\cdot B=0$ becomes
\begin{equation}\label{equ_intersection_of_geometries}
0 = -\sum_{i=1}^k (-1)^i(p(x)\cdot v_i)B_i,
\end{equation}
where $B_i$ denotes the product $B$ with $v_i$ removed.  Realize
that $\{B_i\}_{i=1}^k$ is a linearly independent set of $(k-1)$-blades,
and therefore $p(x)\cdot B=0$ if and only if $p(x)\cdot v_i=0$ for all $i\in\Z_k$.
For convenience, we will let $\Z_k$ denote the set of $k$ integers in $[1,k]\cap\Z^+$.

Already we have fulfilled the promise of the introductory section of this paper,
but we will continue now with one further development that is motivated by a desire
to pair our current ability to intersect geometries with an ability to take their union.

We start by letting $\{\V_i^\infty\}_{i=1}^\infty$ be a countably infinite set of
vector spaces isomorphic to $\V^\infty$.  With the exception of the zero vector,
we consider these vector spaces as pair-wise disjoint, so that for any pair of non-zero
vectors $v_i\in\V_i^\infty$ and $v_j\in\V_j^\infty$ with $i\neq j$, we have $v_i\cdot v_j=0$.
Let $\V$ simply denote the vector space spanned by the set of vectors $\cup_{i\in\Z^+}\{e_{ij}\}_{j=1}^\infty$,
where for each $i\in\Z^+$, the set $\{e_{ij}\}_{j=1}^\infty$ is an orthonormal basis for
the vector space $\V_i^\infty$.
It is clear that the dimension of $\V$,
like that of $\V^\infty$ above, is countably infinite as the union of countably many countable
sets is countable.\footnote{This may come as some comfort to the reader. If you already thought that
the dimension of $\V^\infty$ was ridiculously huge, the dimension of the vector space $\V$ is no bigger.
If it would help, however, it would be reasonable to choose a positive integer $k\in\Z^+$ and
have for all $i>k$, $g_i=0$.  This would bring $\V$ to finite dimension while
enabling us to represent any desired subset of the set of all algebraic sets.}

\begin{defn}[The $q$ Function]\label{def_q_func}
Given a blade $B\in\G(\V)$, the function $q$ is a mapping from
the set of all blades in $\G(\V)$ to subsets of $\Z^+$ so that $i\in q(B)$ if and only if
there exists a vector $v\in\V_i^\infty$ such that $v\wedge B=0$.
\end{defn}

With Definition~\ref{def_q_func} in place, we may now proceed with the following two definitions.

\begin{defn}[The Algebraic Set $\Gi$]\label{def_gi}
Given a blade $B\in\G(\V)$, the set $\Gi$ of $B$, denoted $\Gi(B)$, is the set
\begin{equation}
\Gi(B) = \left\{x\in\R^n\left|\bigwedge_{i\in q(B)} p_i(x)\cdot B=0\right.\right\},
\end{equation}
where for each $i\in\Z^+$, we define $p_i$ similar to equation \eqref{equ_form_vector} as
\begin{equation}\label{equ_pi_func}
p_i(x) = \sum_{j=1}^\infty g_j(x)e_{ij}.
\end{equation}
\end{defn}
For any product of the form $\prod_{i\in S}$ or $\bigwedge_{i\in S}$, where $S$ is
a set of positive integers, we take the terms in the product to be in ascending order
with respect to the index $i$.  While of course the outer product is non-commutative,
we will be more interested in the vector sub-spaces represented by blades in this
paper rather than the handedness of those blades.

\begin{defn}[The Algebraic Set $\Go$]\label{def_go}
Given a blade $B\in\G(\V)$, the set $\Go$ of $B$, denoted $\Go(B)$, is the set
\begin{equation}
\Go(B) = \left\{x\in\R^n\left|\bigwedge_{i\in q(B)}p_i(x)\wedge B=0\right.\right\}.
\end{equation}
\end{defn}

We now proceed to investigate the consequences of Definition~\ref{def_gi}
and that of Definition~\ref{def_go}.

\section{The Algebraic Set $\Gi$}

As promised, there now exist conditions under which we are able to take the
union of any two geometries represented by blades in $\G(\V)$.  The result is
as follows.

\begin{lem}[The Union Of Geometries]\label{lma_union_geos_for_gi}
For any two blades $A,B\in\G(\V)$ with the property that $q(A)\cap q(B)$ is empty,
we have
\begin{equation}
\Gi(A)\cup\Gi(B)=\Gi(A\wedge B).
\end{equation}
\end{lem}
\begin{proof}
Without loss of generality,
we may write $C=A\wedge B$ as $C=\bigwedge_{i=1}^k C_k$, where for
each $i\in\Z_k$, we have $C_i\in\G(V_i^\infty)$.
(Notice that $q(C)=\Z_k$.)  It is now clear that
\begin{align}
\bigwedge_{i\in\Z_k} p_i(x)\cdot C &= \pm\bigwedge_{i\in\Z_{k-1}} p_i(x)\cdot
(p_k(x)\cdot C_k)\wedge\bigwedge_{i\in\Z_{k-1}} C_i \\
 &= \pm\bigwedge_{i\in\Z_k}p_i(x)\cdot C_i.\label{equ_union_product}
\end{align}
(Notice that the outer product in \eqref{equ_union_product} becomes
a scalar product in the case that every $C_i$ is a vector.)
It is now clear that $\Gi(C)$ is indeed the union of $\Gi(A)$ and $\Gi(B)$.
\end{proof}
It is not hard to show that for any blade $B\in\G(\V)$, there exists
a rotor $R$, such that $B'$, given by $B'=RBR^{-1}$, has the property
$\Gi(B')=\Gi(B)$ while $q(B')$ is mapped to any other set of integers
of size $|q(B)|$.  This fact can be used to adjust any given pair
of blades $A,B\in\G(\V)$, if needed, so that $q(A)\cap q(B)$ is empty.
Admittedly, the need for any such adjustment prior to a union operation
seems to detract from our dream of an algebra where the geometric elements
would combine effortlessly in products that perform desired geometric operations.
Unfortunately, things don't get any better as the next lemma shows.

\begin{lem}[The Intersection Of Geometries]\label{lma_intersect_geos}
For any two blades $A,B\in\G(\V)$ such that $q(A)=q(B)$
with $|q(A)|=|q(B)|=1$, we have
\begin{equation}
\Gi(A)\cap\Gi(B)=\Gi(A\wedge B).
\end{equation}
\end{lem}
\begin{proof}
Revisit the conversation of this paper surrounding equation \eqref{equ_intersection_of_geometries}.
\end{proof}
In this case, a simple rotor adjustment to one of $A$ and $B$ will only help
in the case that $|q(A)|=|q(B)|=1$.  If either one of $|q(A)|$ or $|q(B)|$
is greater than one, however, more adjustments are needed before an intersection can
be taken.  The following equation illustrates why.
\begin{equation}\label{equ_intersection_expansion}
\bigcup_{i\in q(A)}\Gi(A_i)\cap\bigcup_{i\in q(B)}\Gi(B_i) =
\bigcup_{\substack{i\in q(A)\\j\in q(B)}} \Gi(A_i)\cap \Gi(B_j),
\end{equation}
Here we have considered $A$ as the blade $\bigwedge_{i\in q(A)}A_i$, where
for each $i\in q(A)$, we have $A_i\in\G(\V_i^\infty)$.  We have similiarly
considered $B$ in terms of the blades $\{B_i\}_{i\in q(B)}$.  It is now
easy to see that, by the right-hand side of equation \eqref{equ_intersection_expansion},
there exists\footnote{For each $(i,j)\in q(A)\times q(B)$, choose blades $A_i'$ and
$B_j'$ such that $\Gi(A_i')=\Gi(A_i)$, that $\Gi(B_i')=\Gi(B_i)$, and that
Lemma~\ref{lma_intersect_geos} may be applied
to get $\Gi(A_i')\cap\Gi(B_j')=\Gi(A_i'\wedge B_j')$.  (This may be done using rotor
adjustments of $A_i$ and $B_j$.)
Furthermore, make these choices so that for all $(i,j)\neq (k,l)$, Lemma~\ref{lma_union_geos_for_gi}
may be applied to get
\begin{equation*}
\Gi(A_i'\wedge B_j')\cup\Gi(A_k'\wedge B_l')=\Gi(A_i'\wedge B_j'\wedge A_k'\wedge B_l').
\end{equation*}
The blade $C$ may now be taken as an outer product of all $A_i'\wedge B_j'$
over all $(i,j)\in q(A)\times q(B)$.} a blade $C$
with the property that $\Gi(C)=\Gi(A)\cap\Gi(B)$, but we do not necessarily
have $\mbox{grade}(C)=\mbox{grade}(A\wedge B)$, showing that a rotor
adjustment, which is grade preserving, to one or both of $A$ and $B$, cannot
be of general help.  We need the factorizations of $A$ and $B$ to formulate $C$.
One possible saving grace taunts us with the following lemma.

\begin{lem}\label{lma_geo_reduction}
For any blade $B\in\G(\V)$, there exists a blade $B'\in\G(\V)$ with $|q(B')|=1$ such
that $\Gi(B')=\Gi(B)$.
\end{lem}
\begin{proof}
It is clear that $\Gi(B)$ is an algebraic set as finite unions and arbitrary
intersections of such sets are algebraic.  Now simply see that the set of all
algebraic sets is covered by the set
\begin{equation}
\{\Gi(B')|\mbox{$B'\in\G(\V)$ and $q(B')=\{1\}$}\}.
\end{equation}
\end{proof}
Lemma~\ref{lma_geo_reduction} proves the existance of a blade with a desired
property, but does not give us
any clue to a means of calculating it.  Providing such a means in geometric algebra,
as we'll see, is possible, but doesn't seem to come naturally.

Letting $A,B\in\G(\V)$ be two blades with $q(A)\cap q(B)$ empty,
we know that $A\wedge B$ is the union of two geometries.  Suppose
now that $|q(A)|=|q(B)|=1$ and that we want a blade $C\in\G(\V)$
with $\Gi(C)=\Gi(A\wedge B)$, where $|q(C)|=1$.
Well, writing the $k$-blade $A$ as $\bigwedge_{i=1}^k a_i$
and the $l$-blade $B$ as $\bigwedge_{i=1}^l b_i$, we see that
\begin{equation}
\bigcap_{i=1}^k\Gi(a_i)\cup\bigcap_{i=1}^l\Gi(b_i) = \bigcap_{i=1}^k\bigcap_{j=1}^l \Gi(a_i\wedge b_j),
\end{equation}
showing that
\begin{equation}
C=\bigwedge_{i=1}^k\bigwedge_{j=1}^l r(a_i\wedge b_j),
\end{equation}
where the function $r:\G(\V)\to\V$ is a linear function that maps
a basis 2-blade to the appropriate basis vector by the associations between
basis vectors and unit-monomials implied by equation \eqref{equ_pi_func}.
For example, we have $e_{ij}$ associated with $g_j$ and $e_{kl}$ with $g_l$,
therefore, we have, for $i\neq k$, $r(e_{ij}\wedge e_{kl})$ mapped to the basis vector $e_m$ for
the integer $m$ where $g_jg_l=g_m$.  This isn't pretty, but it is a well defined
function.

Such a mapping $r$ could easily be extended to map blades of any grade,
but it cannot be extended to an outermorphism so that $C=r(A\wedge B)$.
Notice that $\mbox{grade}(C)$ is not necessarily $\mbox{grade}(A\wedge B)$.
It would seem that geometric algebra doesn't go to work for us in this case without the need
to add more machinary.

% examine Gi(A ^ B) for general A,B and give boolean expression (using set unions/intersections)
% do same for Go(A ^ B) in that section.

\section{The Algebraic Set $\Go$}

Interestingly, the condition of Lemma~\ref{lma_union_geos_for_gi} is the same
as the following lemma.
\begin{lem}[The Union Of Geometries]\label{lma_union_geos_for_go}
For any two blades $A,B\in\G(\V)$ with the property that $q(A)\cap q(B)$ is empty,
we have
\begin{equation}
\Go(A)\cup\Go(B)=\Go(A\wedge B).
\end{equation}
\end{lem}
\begin{proof}
Without loss of generality, we may write $C=A\wedge B$ as $C=\bigwedge_{i=1}^k C_k$,
where for each $i\in\Z_k$, we have $C_i\in\G(\V_i^\infty)$.  (Notice again that $q(C)=\Z_k$.)
It is now clear that
\begin{equation}
\bigwedge_{i\in\Z_k}p_i(x)\wedge C=\pm\bigwedge_{i\in\Z_k}p_i(x)\wedge C_i,
\end{equation}
and therefore, $\Go(C)$ is indeed the union of $\Go(A)$ and $\Go(B)$.
\end{proof}
Applying the condition of Lemma~\ref{lma_intersect_geos}, we get the
following lemma.
\begin{lem}\label{lma_combine_geos}
For any two blades $A,B\in\G(\V)$ such that $q(A)=q(B)$
with $|q(A)|=|q(B)|=1$, we have
\begin{equation}
\Go(A)\cup\G(B)\subseteq\Go(A\wedge B).
\end{equation}
\end{lem}
\begin{proof}
Letting $i\in\Z^+$ be the integer such that $\{i\}=q(A)=q(B)$, realize that while
\begin{equation}
\mbox{$p_i(x)\wedge A=0$ or $p_i(x)\wedge B=0\implies p_i(x)\wedge A\wedge B=0$}
\end{equation}
is a true statement, the converse of this statement is not true.
\end{proof}
Lemma~\ref{lma_combine_geos} gives us the exciting prospect of fitting
geometries to a set of points as the next lemma shows.
\begin{lem}[Point-Fitting Geometries]\label{lma_point_fitting}
Letting $i\in\Z^+$ be a fixed integer, $S\subset\R^n$ a finite set
of points, and $B=\bigwedge_{x\in S} p_i(x)$, if we have $B\neq 0$,
then for all $x\in S$, we have $x\in\Go(B)$; in other words, $S\subseteq\Go(B)$.
\end{lem}

Lemma~\ref{lma_combine_geos} is unsatisfactory, however, because it does not tell
us exactly what geometry we get out of $A\wedge B$ in terms of $A$ and $B$ as is
the case with Lemma~\ref{lma_intersect_geos}.
Interestingly, we can apply Lemma~\ref{lma_intersect_geos} to solve this
problem, if we use it in addition to the following lemma.
\begin{lem}[Dual Geometries]\label{lma_dual_geos}
Let $i\in\Z^+$ be a fixed integer and let $B\in\G(\V_i^\infty)$
be a blade.  Now let $I_i$ denote the unit-psuedo-scalar of
any vector sub-space $\V_i$ of $\V_i^\infty$ of finite dimension
such that $B\in\G(\V_i)$ and $B$ is not a psuedo-scalar.
It then follows that
\begin{equation}
\Go(B)=\Gi(BI_i).
\end{equation}
\end{lem}
\begin{proof}
Simply note that
\begin{equation}
\mbox{$p_i(x)\wedge B=0$ iff $(p_i(x)\cdot BI_i)I_i=0$ iff $p_i(x)\cdot BI_i=0$.}
\end{equation}
\end{proof}
The solution to the dilemma presented by Lemma~\ref{lma_combine_geos}
can now be addressed as follows.  Let $i\in\Z_k$ be a fixed integer,
and let us assume that for every $c\in\V_i^\infty$, we know what
geometry we get from $\Gi(c)$.  It now follows by Lemma~\ref{lma_intersect_geos}
that for any blade $C\in\V_i^\infty$, we know what geometry we get from $\Gi(C)$.
Then, for any two blades $A$ and $B$ of Lemma~\ref{lma_combine_geos},
we can apply Lemma~\ref{lma_dual_geos} to show that
\begin{equation}
\Go(A\wedge B)=\Gi((A\wedge B)I_i).
\end{equation}
It follows now that we know what geometry we get from $A\wedge B$
in terms of its dual $\pm(A\wedge B)I_i$.  So the answer is that we can understand
the geometry of such blades through a geometric interpretation of their duals.

To give a concrete example of this, it is not at all obvious that the
outer product of the points of CGA can produce the rounds and flats of
that model until we relate those blades with their duals, the geometry
of which may be thought of as the set of all possible intersections of rounds and flats.

Returning to the idea of point-fitting in Lemma~\ref{lma_point_fitting}, one way to understand
what geometry we get in our generalized model of algebraic sets
from the fitting of all points in a set $S$ is to analyze a dual of the
blade $B$.  This may require finding a factorization of $BI_i$.

\section{Transforming The Geometries of $\G(\V)$}

Admittedly, up this point in the paper, nothing insightful or new about
algebraic sets has been revealed through our use of GA to model
such sets.  Perhaps the only advantage of using such a model
is that, by virtue of using blades to represent algebraic sets,
this lends itself well to the use of versors as transformations
applicable to any geometry of the model.  As can be seen
in CGA, this reveals an interesting relationship
between transformations as versors and geometries as blades.
It is that many geometries (blades) are in fact transformations (versors).
When reinterpreted as a geometry, the transformation performed by a given versor
often has geometric significance with respect to that geometry.
For example, spheres of CGA, as versors, perform inversions about
a sphere.  The planes of CGA, as versors, perform reflections about
a plane.

% show that every algebraic set is the outer product of points.
% for vectors, this is easy.  for blades, we need to prove
% linear independence.

% then, knowing how to transform any points lets us
% understand how any algebraic set is transformed.
% the problem of knowing how a given versor transform
% any algebraic set is reduced to knowing how any
% versor transforms a p_i(x) -- a point.

\section{Comments And Criticisms}

Unfortunately, the conclusion that must be reached at the end of this paper is
that there does not appear to be any tangible benefit to the now
presented model of algebraic sets using geometric algebra in terms
of the union and intersection operations.  The outer product does not
appear to calculate for us anything beyond what we can already
do by simply multiplying polynomial equations together, or storing
them in a set.

The only possible redeeming quality of the approach given is perhaps
that, by virtue of using blades to represent algebraic sets, it lends
itself well to the use of versors in the desire to transform such sets.

% Possible citations:
% http://mathdl.maa.org/images/upload_library/22/Ford/DesmondFearnleySander.pdf
% Invariant algebras and geometric reasoning, Hongbo Li
% http://www.xtec.cat/~rgonzal1/proceedings.pdf

\bibliographystyle{amsplain}
\bibliography{Parkin_AModelOfAlgebraicSetsUsingGA}

\end{document}