\documentclass{article}
\usepackage{amsmath}
\usepackage{amssymb}
\title{Chapter 18 Homework}
\author{Spencer}
\addtolength{\oddsidemargin}{-.575in}
\addtolength{\evensidemargin}{-.575in}
\addtolength{\textwidth}{1.0in}
\addtolength{\topmargin}{-.575in}
\addtolength{\textheight}{1.25in}
\begin{document}
\maketitle

\newcommand{\Z}{\mathbb{Z}}
\newcommand{\R}{\mathbb{R}}
\newcommand{\N}{\mathbb{N}}
\newcommand{\Q}{\mathbb{Q}}
\newcommand{\lcm}{\mbox{lcm}}

\section*{Problem 6}

Let $D$ be an integral domain.  Define $a\sim b$ if $a$ and $b$ are associates.
Show that this defines an equivalence relation on $D$.

For all $a\in D$, we have $a\sim a$ since $a=1a$ and 1 is a unit.
For some $a,b\in D$, suppose $a\sim b$.  Then there is a unit $u\in D$
such that $a=ub\implies u^{-1}a=b$.  But $u^{-1}$ is also a unit,
so $b\sim a$.  Now suppose that for some $a,b,c\in D$, that $a\sim b$ and $b\sim c$.
Then there exist units $u_1,u_2\in D$ such that $a=u_1b$ and $b=u_2c$.
It follows that $a=u_1u_2c$ and clearly $u_1u_2$ is a unit since $u_1u_2u_2^{-1}u_1^{-1}=1$.

\section*{Problem 8}

Let $D$ be a Euclidean domain with measure $d$.  Show that if $a$ and $b$ are
associates in $D$, then $d(a)=d(b)$.

Let $u\in D$ be a unit such that $a=ub$ and $b=u^{-1}a$.
Then $d(a)=d(ub)\leq d(u^{-1}ub)=d(b)$ and
$d(b)=d(u^{-1}a)\leq d(uu^{-1}a)=d(a)$, showing that $d(a)=d(b)$.

\section*{Problem 14}

Show that $1-i$ is an irreducible in $\Z[i]$.

Suppose $1-i=xy$ for some $x,y\in\Z[i]$.
Then $2=1^2+1^2=N(1-i)=N(xy)=N(x)N(y)$ implies, without loss of generality,
that $N(x)=1$ and $N(y)=2$.  It follows that $x$ is a unit since $N(x)=1$.

\pagebreak
\section*{Problem 22}

In $\Z[\sqrt{5}]$, prove that both $2$ and $1+\sqrt{5}$ are irreducible but not prime.

If we let $x=2+\sqrt{5}$ and $y=-4+2\sqrt{4}$, then $2=xy$ and clearly $2|xy$.
Suppose $2|x$.  Then there exist $a,b\in\Z$ such that
$x=2(a+b\sqrt{5})\implies 1=2b\implies b=1/2\not\in\Z$.  By contradiction,
$2$ does not divide $x$.  Now suppose $2|y$.  Then there exist $a,b\in\Z$ such that
$y=2(a+b\sqrt{5})\implies -4=2a\implies a=-1/2\not\in\Z$.  By contradiction,
$2$ does not divide $y$.  We can now say that $2$ is not prime in $\Z[\sqrt{5}]$.

Let $x,y\in D$ such that $2=xy$.  Then $4=N(2)=N(xy)=N(x)N(y)$.
Suppose now that $N(x)=2$.  Then there exist $a,b\in\Z$ such that
$\pm 2=a^2-5b^2$ which implies that $a^2\equiv 2\pmod{5}$ or
$a^2\equiv 3\pmod{5}$.  But neither of these has a solution in $\Z_5$,
so the equations $\pm 2=a^2-5b^2$ have no solution in $\Z$.  By contradiction, $N(x)\neq 2$,
and we see that, without loss of generality, $N(x)=1$ and $N(y)=4$.
It follows that $x$ is a unit, and so 2 is irreducible in $\Z[\sqrt{5}]$.

If $1+\sqrt{5}$ is prime, then it must divide 2, since
$(1+\sqrt{5})(1-\sqrt{5})=-4=-2\cdot 2$.  So there must exist $a,b\in\Z$
such that $2=(1+\sqrt{5})(a+b\sqrt{5})$ which implies that $2=a+5b$ and $a+b=0$,
which is impossible.  So $1+\sqrt{5}$ is not prime.

Now let $x,y\in D$ such that $1+\sqrt{5}=xy$.  Then $4=N(1+\sqrt{5})=N(xy)=N(x)N(y)$.
We already showed above that there does not exist an element of $\Z[\sqrt{5}]$
whose norm is 2, so without loss of generality, we must have $N(x)=1$ and $N(y)=4$.
It follows that $x$ is a unit and so $1+\sqrt{5}$ is irreducible.

\section*{Problem 28}

Determine the units in $\Z[i]$.

We're looking for all $a,b\in\Z$ such that $N(a+bi)=a^2+b^2=1$.  From this it is
clear that the only units are $\pm 1$ and $\pm i$.

\section*{Problem 30}

Show that $3x^2+4x+3\in\Z_5[x]$ factors as $(3x+2)(x+4)$ and
$(4x+1)(2x+3)$.  Explain why this does not contradict the corollary
of Theorem 18.3.

That the polynomial factors in the two given ways is clear.
Now notice that $-(3x+2)=-3x-2=2x+3$ and $-(x+4)=-x-4=4x+1$.
The unique factorization theorem guarentees uniques up to sign and
order of factors.

\section*{Problem 36}

Find the inverse of $1+\sqrt{2}$ in $\Z[\sqrt{2}]$.  What is the multiplicative
order of $1+\sqrt{2}$?

The multiplicative inverse of $1+\sqrt{2}$ is $-1+\sqrt{2}$.
Let $n$ be the multiplicative order of $1+\sqrt{2}$.
Then $(1+\sqrt{2})(1+\sqrt{2})^{n-1}=1$ implies that...
\begin{equation*}
\sum_{k=0}^{n-1}\binom{n-1}{k}(\sqrt{2})^k=(1+\sqrt{2})^{n-1}=-1+\sqrt{2}
\end{equation*}
...but there is no positive integer $n$ such that the series above
will give us a sum of the form $-1+b\sqrt{2}$ for some $b\in\Z$.
So the multiplicative order of $1+\sqrt{2}$ is infinity.
I'm not sure if this is the argument the book was looking for.  Perhaps
there is an easier way.  Or perhaps I screwed up.

\section*{Problem 38}

Let $R=\Z\oplus\Z\oplus\dots$ (the collection of all sequences of integers under
componentwise addition and multiplication).  Show that $R$ has ideals $I_1,I_2,I_3,\dots$
with the property that $I_1\subset I_2\subset I_3\subset\dots$.

Notice that...
\begin{equation*}
\langle (1,0,0,0,\dots)\rangle\subset
\langle (1,1,0,0,\dots)\rangle\subset
\langle (1,1,1,0,\dots)\rangle\subset\dots
\end{equation*}
So there is no ascending chain condition on $R$.

\end{document}