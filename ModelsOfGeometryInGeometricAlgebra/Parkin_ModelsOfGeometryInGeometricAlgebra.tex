\documentclass{birkjour}

%\usepackage{tikz}
%\usepackage{graphicx}
\usepackage{hyperref}
\usepackage{algpseudocode}

\newtheorem{thm}{Theorem}[section]
\newtheorem{cor}[thm]{Corollary}
\newtheorem{lem}[thm]{Lemma}
\newtheorem{prop}[thm]{Proposition}
\theoremstyle{definition}
\newtheorem{defn}[thm]{Definition}
\theoremstyle{remark}
\newtheorem{rem}[thm]{Remark}
\newtheorem*{ex}{Example}
\numberwithin{equation}{section}

\newcommand{\R}{\mathbb{R}}
\newcommand{\B}{\mathbb{B}}
\newcommand{\G}{\mathbb{G}}
\newcommand{\V}{\mathbb{V}}
\newcommand{\Z}{\mathbb{Z}}
\newcommand{\gd}{\dot{g}}
\newcommand{\gh}{\hat{g}}
\newcommand{\Gd}{\dot{G}}
\newcommand{\Gh}{\hat{G}}
\newcommand{\nvai}{\infty}
\newcommand{\nvao}{o}
\newcommand{\grade}{\mbox{grade}}

\begin{document}

\title{Models Of Geometry In Geometric Algebra}

\author{Spencer T. Parkin}
\address{102 W. 500 S., \\
Salt Lake City, UT  84101} \email{spencerparkin@outlook.com}

\subjclass{Primary 14J27; Secondary 14J29}

%\dedicatory{To my dear wife Melinda.}

\begin{abstract}
Abstract goes here...
\end{abstract}

%\keywords{Geometric Algebra, Conformal Model}

\maketitle

\section{Introduction And Motivation}

Many of the pieces of geometry we study can be described as the roots
of a given polynomial.  Consequently, the geometries generated by many different
polynomials have been studied; and, naturally, this has led to a general theory
of geometries that can be described as the zero set of one or more polynomials; namely, algebraic
geometry.  Ideals of polynomial rings being the generators of algebraic sets,
abstract algebra became the basic framework of the theory.\footnote{Modern
algebraic geometry has grown far beyond algebraic sets as the primary object of study;
but originally, this is what algebraic geometry was about.}

Similarly, various models of geometry have been developed in the framework
of geometric algebra, but they all have one thing in common: they're based
upon the idea of the algebraic set.  It stands to reason, then, that it may be
worth trying to find a unified theory of such models in geometric algebra.
In other words, it may be worth studying such models in a more abstract setting.
That is the focus of this paper.

\section{From Polynomials To Blades}

Let $F$ be a field of characteristic 1, and let $f\in F[x_1,\dots,x_n]$ be a polynomial of
arbitrary degree in $n$ variables.  Being interested in the set of all $x\in F(x_1,\dots,x_n)$ such that
$f(x)=0$, how might we translate the description of this set into the language of geometric algebra?
Letting $\G$ denote a geometric algebra generated by an infinitely-dimensional vector space $\V$ whose
scalars are taken from $F$, (the set of euclidean vectors $\{e_i\}_{i=1}^\infty$ generate $\V$),
we introduce an appropriately defined function $p:\V\to F$ and
simply factor it out of the equation $f(x)=0$ to obtain
\begin{equation*}
p(x)\cdot v = 0,
\end{equation*}
where $v\in\V$.  Defined appropriately, for every $f\in F[x_1,\dots,x_n]$, there would
exist a unique $v\in\V$ such that $p(x)\cdot v=0$ if and only if $f(x)=0$.  The existence
of such a function $p$, and the establishment of the ensuing claim, therefore, must constitute our first result.
\begin{lem}\label{lem_vector_polynomial_correspondence}
Letting $p:\V\to F$ be defined as
\begin{equation}\label{equ_definition_of_p}
p(x)=\sum_{i=1}^\infty m_i(x)e_i,
\end{equation}
where the polynomial sequence $\{m_i\}_{i=1}^\infty\subset F[x_1,\dots,x_n]$ enumerates all possible unit monomials
in the variables $x_1,\dots,x_n$, there exists, for every polynomial $f\in F[x_1,\dots,x_n]$,
a unique vector $v\in\V$, such that for all $x\in F$, we have $f(x)=p(x)\cdot v$.
\end{lem}
\begin{proof}
It is clear that there must exist a sequence of scalars $\{\alpha_i\}_{i=1}^\infty\subset F$
such that
\begin{equation*}
f(x)=\sum_{i=1}^\infty \alpha_i m_i(x).
\end{equation*}
Letting $v=\sum_{i=1}^\infty \alpha_i e_i$, we find that $v$ factors out of this
equation as $f(x)=p(x)\cdot v$, as desired.  To show uniqueness, suppose $v\neq w\in\V$
satisfies the equation $f(x)=p(x)\cdot w$ with $w=\sum_{i=1}^\infty\beta_i e_i$.
Then, since $v\neq w$, there exists a positive integer $i$ such that $\alpha_i=v\cdot e_i\neq w\cdot e_i=\beta_i$,
and therefore, $\alpha_i m_i(x)\neq \beta_i m_i(x)$, which is a contradiction.

\end{proof}
At this point it is important to say that we should not get caught up in the way that $p$
is or may be defined.  It really doesn't matter.  What does matter is that $p$ is defined in such
a way as to satisfy the property of Lemma~\ref{lem_vector_polynomial_correspondence} (i.e., that
there is a one-to-one correspondence between polynomials in $F[x_1,\dots,x_n]$ and vectors in $\V$.)
We therefore shall not make any further use of equation \eqref{equ_definition_of_p} in the remainder of this paper.

Our interest, however, does not stop at the zero set of a single polynomial.  For a set of
$r$ polynomials $\{f_j\}_{j=1}^r\subset F[x_1,\dots,x_n]$, we want the set of all
$x\in F(x_1,\dots,x_n)$ such that for all $f\in\{f_j\}_{i=1}^r$, we have $f(x)=0$.
Interestingly, geometric algebra provides a convenient description of such a set.

% address euclidean signature

% get to blades.

\begin{thebibliography}{9}

\bibitem{Parkin13}
S. Parkin, {\it Mother Minkowski Algebra Of Order $M$}.
Advances in Applied Clifford Algebras (2013).

\end{thebibliography}

\end{document}