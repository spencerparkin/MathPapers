\documentclass[12pt]{article}

\usepackage{amsmath}
\usepackage{amssymb}
\usepackage{amsthm}

\title{Chapter 9 Exercises\\Gallian's Book on Abstract Algebra}
\author{Spencer T. Parkin}

\newtheorem{theorem}{Theorem}[section]
\newtheorem{definition}{Definition}[section]
\newtheorem{corollary}{Corollary}[section]
\newtheorem{identity}{Identity}[section]
\newtheorem{lemma}{Lemma}[section]
\newtheorem{result}{Result}[section]

%\newcommand{\gcd}{\mbox{gcd}}
\newcommand{\lcm}{\mbox{lcm}}
\newcommand{\abs}{\mbox{abs}}
\newcommand{\Z}{\mathbb{Z}}
\newcommand{\R}{\mathbb{R}}
\newcommand{\G}{\mathbb{G}}
\newcommand{\stab}{\mbox{stab}}
\newcommand{\aut}{\mbox{Aut}}
\newcommand{\inn}{\mbox{Inn}}
\newcommand{\orb}{\mbox{orb}}

\begin{document}
\maketitle

\section*{Exercise 61}

Suppose that $H$ is a normal subgroup of a finite group $G$.
If $G/H$ has an element of order $n$, show that $H$ has an element of order $n$.
Show, by example, that the assumption that $G$ is finite is necessary.

The case $n=1$ is trivial, so let $n>1$.  Let $a\in G$ such that $|aH|=n$.  Clearly $a\neq e$.
It follows that the mapping $\phi:H\to H$, given by $\phi(h)=a^nh$ is a non-trivial permutation of
the elements of $H$ and so $\phi$ is a member of the group of permutations of $H$.
We then see that $a^{|\phi|n}=e$.  But it is easy to see that for all
integers $i\in [1,|\phi|n-1]$, we have $a^i\neq e$.  So $|a^{|\phi|}|=n$.

\end{document}