\documentclass[12pt]{article}

\usepackage{amsmath}
\usepackage{amssymb}
\usepackage{amsthm}
\usepackage{graphicx}
\usepackage{float}

\title{Notes On Euler Angles}
\author{Spencer T. Parkin}

\newcommand{\G}{\mathbb{G}}
\newcommand{\V}{\mathbb{V}}
\newcommand{\R}{\mathbb{R}}
\newcommand{\B}{\mathbb{B}}
\newcommand{\nvao}{o}
\newcommand{\nvai}{\infty}

\newtheorem{theorem}{Theorem}[section]
\newtheorem{definition}{Definition}[section]
\newtheorem{corollary}{Corollary}[section]
\newtheorem{identity}{Identity}[section]
\newtheorem{lemma}{Lemma}[section]
\newtheorem{result}{Result}[section]

\begin{document}
\maketitle

\section*{The Application Of Euler Angles}

Let $R_x$, $R_y$ and $R_z$ each be unit-rotors that rotate about the $x$, $y$ and $z$ axes, respectively.
Then it is easy to imagine the rotation generated by the rotor $R_zR_yR_x$.
A point is rotated about the $x$-axis, then the $y$, and then the $z$.  Each rotation is about
an axis from the same fixed coordinate frame.  We can similarly deduce the effect of the rotation generated by the
rotor $R_xR_yR_z$.  However, like $R_zR_yR_x$, we are also free to think of the rotation generated by
$R_xR_yR_z$ in the order of first rotating about the $x$-axis, then a (not the) $y$-axis, and then
a (not the) $z$-axis, if, to be a bit more specific, we think about such a $y$-axis as a rotation of the $y$-axis.
Similarly, we must think about such a $z$-axis as a rotation of the $z$-axis.  This can all be explained
algebraically as follows.  Consider a vector $v$ to be rotated by $R_xR_yR_z$.  Instead of applying $R_z$
first, which is what we would naturally want to do to get the rotation $v'$ of $v$ as
\begin{equation}\label{equ_zyx}
v' = R_xR_yR_z v \tilde{R}_z\tilde{R}_y\tilde{R}_x,
\end{equation}
we will apply $R_x$ first, and write
\begin{equation}\label{equ_euler_xyz}
v' = R(R_x R_y \tilde{R}_x)R_x v \tilde{R}_x(R_x \tilde{R}_y \tilde{R}_x)\tilde{R},
\end{equation}
where the rotor $R$ is given by
\begin{equation*}
R = ((R_xR_y\tilde{R}_x)R_x)R_z((R_xR_y\tilde{R}_x)R_x)^{-1}.
\end{equation*}
Here we must realize that we can apply rotors to rotors to rotate the axis of rotation
about which a rotor rotates.  (Pardon the tongue twister.)  The pattern here is seen as an application of $R_x$ first,
then the application of $R_y$ with what was first applied to $v$, (namely $R_y$), applied to it,
then the application of $R_z$ with what was then applied to $v$, (namely $R$), applied to it.

Visualizing the rotation of the vector $v$ in equation \eqref{equ_euler_xyz}, we see that
with each rotation, we must also apply that rotation to the $xyz$-axis frame about which we do subsequent rotations.
Indeed, this is how Euler rotations are applied.  Interestingly, what we find is that when we
simplify equation \eqref{equ_euler_xyz}, we arrive at equation \eqref{equ_zyx}.
It follows now that the application of the Euler angles first in the $x$, then $y$, then $z$-axis,
is accomplished by a rotor of the form $R_xR_yR_z$.

\section*{Euler Angle Decomposition}

In this section we concern ourselves with the extrapolation of a set of Euler angles for
a given rotation axis order from a given unit-rotor $R$.  If one were to take a purely algebraic
approach to this problem, they would quickly arrive at an undetermined system of non-linear equations,
which seems quite intractable.  A different and perhaps better approach is to take what we know
about the application of Euler angles to come up with a solution to the problem.
Our approach here will be in terms of visualizing the original and final coordinate frames,
and to work by using the first two Euler angles to align a single axis.  The remaining Euler angle
will then rotate the two frames into complete alignment.  Clearly the axis we choose to align must
coincide with the same axis about which the final Euler angle will rotate.

We begin by making sure we have a well defined idea about what it means to rotate
about an axis by an angle.  Given an axis $a$ (a unit-length vector) and an angle $\theta$,
we will define the unit-rotor $R(a,\theta)$ as
\begin{equation*}
R(a,\theta) = \exp\left(-\frac{\theta}{2}aI\right) = \cos\left(\frac{\theta}{2}\right)-aI\sin\left(\frac{\theta}{2}\right),
\end{equation*}
and its application to a vector $v$ to get $v'$ as
\begin{equation*}
v' = Rv\tilde{R},
\end{equation*}
where $I$ is the unit-pseudo scalar.  Defined this way, if the axis vector $a$ is pointing towards us,
we can imagine positive angles of rotation as moving points counter-clock-wise.  It is important to
realize these fine details if we are to succeed in the Euler decomposition of a rotor.

Having now a handle on rotations about arbitrary axes, let $R$ be the unit-rotor we wish to decompose
into a set of Euler angles $\theta_x$, $\theta_y$ and $\theta_z$, applied in that order.  It then becomes
clear that our first order of business is to find the angle $\theta_x$ such that $e_2$ (the $y$-axis) is
rotated by an amount sufficient to align it orthogonal to $e_1'=Re_1\tilde{R}$.  This is in preparation for the application
of the angle $\theta_y$, which will align the rotation of $e_3$ by $\theta_x$ about $e_1$ into $e_3$.
(We will let $e_2'=Re_2\tilde{R}$ and $e_3'=Re_3\tilde{R}$.)
Setting up our equation, we have
\begin{equation*}
R(e_1,\theta_x)e_2\tilde{R}(e_1,\theta_x)\cdot e_1' = 0.
\end{equation*}
(No, we need to get $e_2$ orthogonal to $e_3'$.  This is wrong.)

Simplifying this with the eventual use of the double-angle formulas, we get
\begin{equation*}
(e_1'\cdot e_2)\cos\theta_x = -(e_1'\cdot e_3)\sin\theta_x.
\end{equation*}
Once a solution to this can be found, we then realize that
\begin{equation*}
\theta_y = \pm\cos^{-1}(e_3'\cdot e_3),
\end{equation*}
the sign of which is determined by the sign of the pseudo-scalar $e_3'\wedge e_3\wedge e_2$.
We then find that
\begin{equation*}
\theta_z = \pm\cos^{-1}(e_2'\cdot e_2),
\end{equation*}
the sign of which is determined by the sign of the pseudo-scalar $e_2'\wedge e_2\wedge e_3$.

Now, there are quite a few cases still to consider in all of these steps, but this is the basic path of reasoning.
It should also be noted that Euler angles are not unique, even if we constrain all angles to be within
the range $[0,2\pi]$.

\end{document}